    In~\cite{Potter} we have shown that dry-erase markers have a
    use in science. In this report we address the use of dry-erase
    markers specifically to the science and engineering performed at
    the DOE national laboratories. Let us begin with a description
    of a couple of basic dry-erase marker drawings. We use the
    circle and the square to make our point.

    \ifthenelse{\boolean{reportSAND}}   {
	\section{The Circle}
    }{
	\subsection{The Circle}
    }
	Figure~\ref{fig1} shows one of the basic shapes. It appears
	in many dry-erase drawings and has many important properties.

	\begin{figure}[ht]
	    \centering
	    \begin{picture}(50,50)(0,0)
		\put(25,25){\circle{50}}
		\put(25,25){\circle{1}}
	    \end{picture}
	    \caption[The circle]{A circle is one of the basic
		shapes.  It has many important properties. Note,
		for example, that every point along its
		circumference is at exactly the same distance
		from the center. A truly important property
		for a cicrle.}
	    \label{fig1}
	\end{figure}

	Whole books have been written about the circle and similar,
	lesser shapes. So, it would be presumptuous to attempt
	to even list some of its properties and uses in dry-erase
	marker drawings. Simply admire the shape in Figure~\ref{fig1}
	and its power will soon engulf you in ideas and beautiful
	feelings of completeness.


    \ifthenelse{\boolean{reportSAND}}   {
	\subsection{Small Circles}
    }{
	\subsubsection{Small Circles}
    }
	We also need an example of a subsubsection to make sure
	titles and table of content entries work for those.

	And while we are at it, we might as well check
	paragraph spacing. Although, we will do that again in
	\ifthenelse{\boolean{reportSAND}}   {
	    Chapter~\ref{sec:long}\footnote{In strict article mode,
	    section numbers cannot be referenced because section
	    numbering is turned of. If you use strict mode, do not
	    reference your sections by number. For reports you can
	    reference chapters}.
	}{
	    a later section\footnote{In strict article mode,
	    section numbers cannot be referenced because section
	    numbering is turned of. If you use strict mode, do not
	    reference your sections by number. For reports you can
	    reference chapters}.
	}

    \ifthenelse{\boolean{reportSAND}}   {
	\section{The Square}
    }{
	\subsection{The Square}
    }
	Figure~\ref{fig2} illustrates another famous shape.
	We will show some of its uses in later chapters, but we
	wanted to introduce it here because this is a good place
	for another figure.

	\begin{figure}[ht]
	    \centering
	    \begin{picture}(50,50)(0,0)
		\put(0,0){\framebox(50,50){}}
		\put(25,25){\circle{1}}
	    \end{picture}
	    \caption[The square]{A square is another of the basic
		shapes. It is not quite as powerful as the circle. It
		has some similarities (note that the four corners all have
		the same distance to the center), and has many fine
		uses in everyday dry-erase marker drawing.}
	    \label{fig2}
	\end{figure}

    \ifthenelse{\boolean{reportSAND}}   {
	\section{The Dot}
    }{
	\subsection{The Dot}
    }
	We also need a figure with a short caption. In order to
	do this, we introduce the dot. Figure~\ref{fig3} shows the
	shape of a dot.

	\begin{figure}[ht]
	    \centering
	    \begin{picture}(50,50)(0,0)
		\put(25,25){\circle{1}}
	    \end{picture}
	    \caption[The dot]{A simple dot}
	    \label{fig3}
	\end{figure}

    \ifthenelse{\boolean{reportSAND}}   {
	\section{Tables and Such}
    }{
	\subsection{Tables and Such}
    }
	In order to test our class file, we also need to have some
	tables. One should be enough for our purposes, so here it
	is: Table~\ref{tab1}. On second thought, we need another
	one to test the list of tables with multiple entries. So,
	we introduce Table~\ref{tab2}.

	\begin{table}[ht]
	    \centering
	    \caption[Shapes]{This superb table lists a few
		of the more important shapes and some of
		their properties. Be aware that this condensed list
		can by no means describe all the properties or
		shapes drawable by dry-erase markers.}
	    \bigskip

	    \begin{tabular}{|l|c|l|c|}
		\hline \hline
		Name  & Number of & Importance & Shape \\
		      & corners   &            &       \\
		\hline
		circle & 0        & high       & $\bigcirc$ \\
		square & 4        & medium     & $\diamond$ \\
		triangle & 3      & low        & $\triangle$ \\
		\hline
	    \end{tabular}
	    \label{tab1}
	\end{table}

	\begin{table}[ht]
	    \centering
	    \caption{A magic square}
	    \bigskip

	    \begin{tabular}{|c|c|c|c|}
		\hline
		    1 & 15 & 14 & 4 \\ \hline
		    12 & 6 & 7 & 9 \\ \hline
		    8 & 10 & 11 & 5 \\ \hline
		    13 & 3 & 2 & 16 \\ \hline
	    \end{tabular}
	    \label{tab2}
	\end{table}
