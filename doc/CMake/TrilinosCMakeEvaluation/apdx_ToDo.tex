

%
\section{Trilinos/CMake ToDo}
%

In this section, we describe the specific tasks/requirements that must
be completed in order to complete this initial Trilinos/CMake
evaluation to facilitate an informed decision about whether Trilinos
will drop autotools for the build and switch over to CMake after
Trilinos 9.0 is released.  Also, as an another decision, if in
addition the Perl-based test harness will be replaced with CTest/CDash
and when.  We can drop autotools for building Trilinos and we can keep
the current Perl-based test harness for awhile (or forever) or we can
switch over to CTest/CDash and drop the current Perl-based test
harness at the same time or we can maintain both test harness for some
time.  There are many possibilities.

\begin{enumerate}

{}\item Decide on package autonomy

Two options:

  \begin{itemize}

  {}\item Option 1: Require that all packages be built from the top
  Trilinos level $=>$ This will significantly reduce the amount of
  effort it takes to create and maintain packages.  NOTE: Packages
  that want to build independently can add whatever special support for
  building separately.  NOTE: You could pass include directories and
  libraries to cmake for individual packages.

  {}\item Option 2: Require that all packages be able to build
  separately not from the top level. $=>$ This forces every package
  have to take on this requirement.

  \end{itemize}

Suggested approach: Have each package just read and write global
variables for PACKAGE\_INCLUDE\_DIRS and PACKAGE\_LIBS set up at the
global level.  Then, packages can be configured and built
independently by passing in the needed cache variables.  This is
essentially what has been done but we have not tested building
packages independently yet.

{}\item Refactor and break about the TRILINOS\_ADD\_EXECUTABLE(...) 
script to separate add\_target(...) and add\_test(...) commands.
Features that these new scripts must support and related issues
include:

  \begin{enumerate}

  {}\item Allow test target executable to be defined independently
  from its test [Done]

  {}\item All executables should have the *.exe extension, even on
  Linux as this makes it easy to grep for executables and this is the
  current Trilinos convention ... Investigate what happens on Windows
  ... see if there is an global extension to add ... [Somewhat done
  but more testing on Windows is needed]

  {}\item Allow (require) the names of the test executables to be
  exactly the same as with current autotools system (to allow use of
  test/definition files with no changes). [Done]

  {}\item Include or exclude specific machines for running a test
  (both as a group of tests or individually) ... I.e. HOST and XHOST
  [Done]

  {}\item Do not assume tests are in a sub-directory but allow for this
  (i.e. the DIRECTORY argument is optional). [Done]

  {}\item Allow for multiple sets of input arguments for each test
  executable

    \begin{itemize}

    {}\item If unique test name is not given, allow for automatic
    numeric post-fix [Done]

    \end{itemize}

   [Done]

  {}\item Build in MPI information (serial and/or MPI, number of
  processors, etc.)

    \begin{itemize}

    {}\item The ``Comm'' (i.e. 'mpi' or 'serial') [Done]

    {}\item Selection/definition of test based on the number of allowed
    MPI processors for the test and the global machine max number of
    processes (Hint: See what the Perl-based 'runtests' script does).
    I.e. compare the input NUM\_MPI\_PROCS to the global cache variable
    MPIEXEC\_MAX\_NUMPROCS and see if the test passes.  TODO: Search
    through the test/definition files to see how COMM = MPI(...) is
    actually being used and talk with Jim W.\ about this.
	
    \end{itemize}

   [Done]

  {}\item Allow pass-through of all of the set\_test\_properties(...) 
  command for WILL\_FAIL, PASS\_REGULAR\_EXPRESSION, and
  FAIL\_REGULAR\_EXPRESSION [Done].

  {}\item (Low priority) Put in support for the compareTestOutput
  program and other post-processing programs that look at test output.
  TODO: Grep through the current test/definition files to see where
  compareTestOutput.  Also, talk with Jim W. about making this feature
  more flexible and simpler to use any post-processing program to
  determine success or failure.

  {}\item (Low priority) Support adding keywords for groups of tests
  or individual tests.  Supporting this for real requires keyword
  support in ctest which it does not have this but we could emulate
  this by tacking on the keywords to the end of each test name in the
  group.

  {}\item (Low priority) Specify current working directory where the
  test should be run from (e.g. DESCEND\_INTO\_DIRECTORY option).
  TODO: Search and see where DESCEND\_INTO\_DIR is being used in
  current test/definition files.  Workaround: If you define the test in
  a CMakeList.txt file in the actually directory, ctest will cd into
  that directory when it runs the test.  We don't need
  DESCEND\_INTO\_DIRECTORY [WONTFIX]

  \end{enumerate}

Here is what an example of the refactored macros might look like:

\begin{verbatim}

TRILINOS_ADD_EXECUTABLE(ArrayRCP_test
  DIRECTORY MemoryManagement
  NAME ArrayRCP_test # Name of the test (defalts to executable name)
  DIRECTORY MemoryManagement
  XHOST S859352.sandia.gov # Run on all hosts except this one
  KEYWORDS unit framework install
  PASS_REGULAR_EXPRESSION "All Tests PASSED"
  COMM serial mpi
  NUM_MPI_PROCS 1
  ARGS "--verbose --n=1" "--n=2" "--n=4"
  )

# Above, add_test(...) would define the tests ArrayRCP_test_0,
# ArrayRCP_test_1, and ArrayRCP_test_2.

TRILINOS_ADD_TEST(ArrayRCP_test
  NAME ArrayRCP_test_big
  DIRECTORY MemoryManagement
  DESEND_INTO_DIR
  KEYWORDS performance
  COMM serial mpi
  NUM_MPI_PROCS 6-16
  ARGS "--n=500" "--n=1000"
  )

# Above, add_test(...) would define the tests ArrayRCP_test_big_0,
# ArrayRCP_test_big_1

\end{verbatim}

It is critically important that the TRILINOS\_ADD\_TEST(...) macro be
very flexible and full featured as it needs to replace the current
Perl-based test harness PACKAGE/test/defintion file entries.  Also, it
will be difficult to change once a lot of tests have been
defined. [Done]

{}\item Bring the current Trilinos/CMake support in the existing
CMake-ified packages up to spec with the above requirements and to
achieve:

  \begin{enumerate}

  {}\item Put in custom targets 'runtest-serial' and 'runtest-mpi' to
  invoke the Perl 'runtest' script to run tests.  These targets must
  be supported in each package and at the global Trilinos
  level. [Done]

  {}\item Remove all duplication in header file paths, library paths
  and libraries (Ross B.) [Done]

  {}\item Build all test and example executables with exact same names
  in the same locations as is being done with existing autotools
  system (Ross B.) [Done]

  {}\item Implement automatic dependency tracking facility for
  required and optional dependencies and recursive enables and
  disables (Ross B.) [Done]

  {}\item Add full ctest testing support for all current tests and
  examples the same as is defined and being performed with the current
  PACKAGES/test/definition files (Ross) ... Not done yet but not hard
  ...

  {}\item Test installing Trilinos headers and libraries and create
  same structure as existing Trilinos system (Ross B.)

  {}\item Get a CDash dashboard up and running controlled by Trilinos
  developers and get nightly testing with a cron script
  going. (Esteban G.)

  {}\item Add Windows, MAC, and Linux nightly testing for
  Trilinos/CMake (Esteban G.)

  {}\item (Lower priority just because it is a slam dunk with no risk)
  Implement and demonstrate the generation of export makefiles in the
  same exact way the current autotools system
  does\footnote{Specifically requested by Mike Heroux and Roger
  Pawlowski} (Ross B.) ... This is *not* going to be hard ...

  {}\item (Low priority) Make the default 'make' target just build
  libraries and add 'tests' and 'examples' targets just like the
  current autotools system does (Ross B.)

  {}\item (Low priority) Implement two-state configuration and
  building to support circular dependencies between tests/examples and
  up-stream libraries.  Test this in a superficial way?

  \end{enumerate}

{}\item Expand CMake/CTest support to the packages rtop, epetraext,
thyra, amesos, ifpack, aztecoo, stratimikos, CTrilinos, and
ForTrilinos (optionally). (Ross B.)

This list of packages will bring a number of new issues into play
(e.g. different capitalization of package directories, Fortran 2003
support, etc.)  and will help to test the circular test/example
library dependencies issue.

{}\item Test out and demonstrate CMake features:

  \begin{enumerate}

  {}\item CTest time-out feature killing MPI tests (see if OpenMPI
  brings down orphaned processes or not) (Ross B.) [Done]

  {}\item Building MS Visual Studio projects on MS Windows (Esteban
  G.) [Done]

  {}\item Shared library support on Linux and other platforms (Ross B.)

    \begin{itemize}

    {}\item Look into rebuilds for changing *.cpp files and impact on
    relinking.  CMake should not be relinking executables when just a
    *.o file changes in a shared library.  If CMake is relinking, we
    need to find out if this is really needed or if this is a bug in
    CMake.

    {}\item Look into a 'libs' target to avoid re-links.

    \end{itemize}

  {}\item Full dependency tracking on various platforms (Ross B.)

  {}\item Using CPack to create binary serial distribution for MS
  Windows (e.g.\ installed headers and libraries, and/or Visual Studio
  project files, etc.)\footnote{Specifically requested by Mike
  Heroux.} (Esteban B.)

  \end{enumerate}

\end{enumerate}

Once all of the above tasks are finished, this initial evaluation of
CMake/CTest/CDash etc. for Trilinos will be complete.

Shortly after the completion of the evaluation and the corresponding
summary and conclusions are stated below, the Trilinos leaders will
decide on a course of action regarding replacing the current autotools
and Perl-based test harness with CMake/CTest/CDash/CPack and on a time
table for doing so.
