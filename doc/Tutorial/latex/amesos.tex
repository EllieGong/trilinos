%@HEADER
% ************************************************************************
% 
%          Trilinos: An Object-Oriented Solver Framework
%              Copyright (2001) Sandia Corporation
% 
% Under terms of Contract DE-AC04-94AL85000, there is a non-exclusive
% license for use of this work by or on behalf of the U.S. Government.
% 
% This program is free software; you can redistribute it and/or modify
% it under the terms of the GNU General Public License as published by
% the Free Software Foundation; either version 2, or (at your option)
% any later version.
%   
% This program is distributed in the hope that it will be useful, but
% WITHOUT ANY WARRANTY; without even the implied warranty of
% MERCHANTABILITY or FITNESS FOR A PARTICULAR PURPOSE.  See the GNU
% General Public License for more details.
%   
% You should have received a copy of the GNU General Public License
% along with this program; if not, write to the Free Software
% Foundation, Inc., 675 Mass Ave, Cambridge, MA 02139, USA.
% 
% Questions? Contact Michael A. Heroux (maherou@sandia.gov)
% 
% ************************************************************************
%@HEADER

\chapter{Interfacing Direct Solvers with Amesos}
\label{chap:amesos}

The Amesos package provides an object-oriented interface to several
direct sparse solvers. Amesos will solve (using a direct factorization
method) the linear systems of equations
\begin{equation}
\label{eq:amesos_ls}
A X = B
\end{equation}
where $A$ is stored as an Epetra\_RowMatrix object, and $X$ and $B$ are
Epetra\_MultiVector objects.

The Amesos package has been designed to face some of the challenges of
direct solution of linear systems. In fact, many solvers have been
proposed in the last years, and often each of them requires different
input formats for the linear system matrix. Unfortunately, it is not
uncommon that the interface changes between revisions. Amesos aims to
solve those problems, furnishing a clean, consistent interface to many
direct solvers.

Using Amesos, users can interface their codes with a (large) variety of
direct linear solvers, sequential or parallel, simply by a code
instruction of type
\begin{verbatim}
AmesosProblem.Solver();
\end{verbatim}
Amesos will take care of redistributing data among the processors, if
necessary.

This Chapter starts with few notes on the installation of the third-part
packages required by Amesos. Then, the Chapter will present the use of
Amesos objects, to interface with the following packages:
\begin{itemize}
\item UMFPACK, version 4.1 (in Section~\ref{sec:umfpack});
\item SuperLUdist, version 2.0 (in Section~\ref{sec:superludist}).
\end{itemize}

%%%
%%%
%%%

\section{Installation of Trilinos third-part Packages}
\label{sec:3pl}

Amesos is an interface to other packages, mainly developed outside the
Trilinos framework\footnote{Currently, SuperLU is included in the
  Trilinos framework.}. In order to use those packages, the user should
check carefully copyright and licensing of those third party codes.
Please refer to the web page for the package that you are interested in
for details. Amesos supports a variety of direct solvers for linear
systems of equations, and you are likely to use Amesos with only few of
them. We suggest to define the shell variable \verb!TRILINOS_3PL!  to
define the directory used to stored third-part packages. For instance,
under BASH you may have a line of type
\begin{verbatim}
export TRILINOS_3PL=/home/msala/Trilinos3PL
\end{verbatim}
in your \verb!.bashrc! file. Then, you may decide to create a directory
to hold include files and libraries. For instance, to run LINUX with
MPI:
\begin{verbatim}
$ mkdir ${TRILINOS_3PL}/LINUX_MPI
$ mkdir ${TRILINOS_3PL}/LINUX_MPI/include
$ mkdir ${TRILINOS_3PL}/LINUX_MPI/lib
\end{verbatim}
(Note that this will reflect the directory structure used by Trilinos,
see Section~\ref{sec:installing}). While installing a package, you can
now copy all include files and libraries in these directories.

Using this setting, you can configure Amesos with a command of type
\begin{verbatim}
$ cd ${TRILINOS_HOME}/packages/amesos
$ ./configure --prefix=${TRILINOS_HOME}/LINUX_MPI \
  --enable-mpi --with-mpi-compilers \
  --enable-amesos-umfpack \
  --enable-amesos-superludist \
  --with-amesos-superludistlib="${TRILINOS_3PL}/SuperLU_DIST_2.0/libsuperlu_LINUX.a"
\end{verbatim}
(This command is followed by \verb!make! and \verb!make install!, as
usual.)  This will enable UMFPACK and SuperLUdist, which are the two
packages covered in this Chapter.

For more details about the configuration options of Amesos, please refer
to Amesos documentation.

\begin{remark}
  Amesos can handle multiple right-hand side using Epetra\_MultiVectors.
\end{remark}

%%%
%%%
%%%

\section{UMFPACK}
\label{sec:umfpack}

File \TriExe{amesos/ex1.cpp} shows how to use Amesos to solve a linear
system with UMFPACK\footnote{UMFPACK is a set of routines solving sparse
  linear systems via LU factorization. It is copyrighted by Timothy
  A.~Davis. More information can be obtained at the web page 
  {\tt http://www.cise.ufl.edu/research/sparse/umfpack}. }.

Suppose that \verb!A!, \verb!x! and \verb!b! are an Epetra\_RowMatrix
and two Epetra\_MultiVector, respectively, or compatible dimensions.
Amesos objects for the solution of linear systems requires an
Epetra\_LinearProblem object, plus another object,
\verb!AMESOS::Parameter::List!, used to specify the parameters.
\begin{verbatim}
Epetra_LinearProblem Problem(&A,&x,&b);
AMESOS::Parameter::List params;
\end{verbatim}
Then, only few lines are required: We can define an Amesos object and
solve the problem,
\begin{verbatim}
Amesos_Umfpack UmfpackProblem(Problem,params);
UmfpackProblem.Solve();
\end{verbatim}
or, alternatively, it is possible to specify when symbolical
factorization, numerical factorization and solution occur,
\begin{verbatim}
Amesos_Umfpack UmfpackProblem(Problem,params); 
UmfpackProblem.SymbolicFactorization();
UmfpackProblem.NumericFactorization();
UmfpackProblem.Solve();
\end{verbatim}
Note that {\sl exactly} the same code can be run with more than one
processor. In this case, being UMFPACK a serial solver, Amesos will take
care to gather all required data on a processor, solve sequentially
the linear system, and then broadcast the solution.

%%%
%%%
%%%

\section{SuperLUdist}
\label{sec:superludist}

Solving using SuperLUdist\footnote{SuperLU\_DIST is a parallel extension
  to the  serial SuperLU library.   It  is targeted for the  distributed
  memory parallel  machines. It is written  by XXX and YYY. Please refer
  to  the  web site {\tt  http://www.nersc.gov/~xiaoye/SuperLU} for more
  information.}  is not   much  different    from  what presented     in
Section~\ref{sec:umfpack}.    Instead of   declaring  an Amesos\_Umfpack
object, one can proceed as follows:
\begin{verbatim}
Amesos_Superludist * SuperludistProblem = 
 new Amesos_Superludist(Problem,params);
\end{verbatim}
followed by a call to Solve(), possibly preceded by
SymbolicFactorization() and NumericFactorization().

\begin{remark}
  We have declared a pointer to and Amesos\_Superludist object because
  the destructor of this object contains some MPI
  calls. As in example \TriExe{amesos/ex2.cpp} the destructor is called
  at the end of the main function (after a call to
  \verb!MPI_Finalize()!, we have to delete this object using the C++
  statement
\begin{verbatim}
delete SuperludisProblem;
\end{verbatim}
  before the call to \verb!MPI_Finalize()!.
\end{remark}

%%%
%%%
%%%

\section{A Generic Interface to Various Direct Solvers}
\label{sec:amesos_generic}

All Amesos objects are derived from the Amesos\_BaseClass object. Using
the capabilities of C++, one may decide to code a generic interface to a
direct solver as follows:
\begin{verbatim}
// parameter vector for Amesos
AMESOS::Parameter::List ParamList;

// prepare the linear solver
Amesos_BaseSolver * AmesosProblem;

switch( choice ) {
case ML_SOLVE_WITH_AMESOS_UMFPACK:
    AmesosProblem = new Amesos_Umfpack( *LinearProblem, ParamList );
    break;
  case ML_SOLVE_WITH_AMESOS_SUPERLUDIST:
    AmesosProblem = new Amesos_Superludist( *LinearProblem, ParamList );
    break;
  default:
    cerr << ``Error'' << endl;
}
\end{verbatim}

Now, factorization and solution are the same for all the packages:
\begin{verbatim}  
AmesosProblem->SymbolicFactorization();
AmesosProblem->NumericFactorization();
AmesosProblem = (void *) AmesosProblem ;
\end{verbatim}


