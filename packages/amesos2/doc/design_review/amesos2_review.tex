%\documentclass{beamer}
\documentclass[xcolor=dvipsnames]{beamer}

\usecolortheme[named=brown]{structure}
\useoutertheme{infolines}
%\usefonttheme{serif}
\usefonttheme[stillsansseriftext]{serif}
%\usetheme{Singapore}
\usetheme[height=7mm]{Madrid}
\setbeamertemplate{navigation symbols}{}

%-------------------------------------------------------------------------------
% get epsf.tex file, for encapsulated postscript files:
\input epsf
%-------------------------------------------------------------------------------
% macro for Postscript figures the easy way
% usage:  \postscript{file.ps}{scale}
% where scale is 1.0 for 100%, 0.5 for 50% reduction, etc.
%

\newcommand{\postscript}[2]
{\setlength{\epsfxsize}{#2\hsize}
\centerline{\epsfbox{#1}}}
%-------------------------------------------------------------------------------

\usepackage{graphics}
\usepackage{color}
\usepackage{epsfig}
\usepackage{microtype}
\usepackage{listings}
\lstset{
  language=C++,
  basicstyle=\small}

\renewcommand{\thefootnote}{}

\title[Amesos2]{Amesos2: Common Interface to Direct Solvers}
\author[Bavier, Boman, Rajamanickam]{Eric Bavier and Erik Boman and Siva Rajamanickam}
\institute[]{
Sandia National Laboratories
}
\date[]{July 6, 2011}


\begin{document}

\begin{frame}[plain]
  \titlepage
  \footnote{\tiny{Sandia is a multiprogram laboratory operated by Sandia Corporation, a wholly owned subsidiary of Lockheed Martin, for the United States Department of Energy'’s National Nuclear Security Administration under contract DE-AC04-94AL85000.}}
\end{frame}

\section{Introduction}

\begin{frame}
  \frametitle{Amesos2 Motivation}
  
  \begin{itemize}
  \item Uniform Interface to common third-party direct solvers.
    \medskip
  \item Support a variety of scalar and ordinal datatypes
    \medskip
  \item Support Epetra and Tpetra data structures and allow a design that can
    extend to other matrix types (for eg.: PetSc matrix, compressed column
    matrix)
    \medskip
  \item Revisit Amesos design choices and redesign to easily support more
    solvers and matrix types.
  \end{itemize}
    
\end{frame}

\section{Design}

\begin{frame}
  \frametitle{Use-case Considerations}
  %% - Multiple datatypes (scalar/ordinal as well as matrix/vector)
  %% - Novice Users
  %% - ``Expert'' Users
\end{frame}

\section{Examples}

\begin{frame}[fragile]          % ``fragile'' for slides with verbatim
  \frametitle{Use Case 1: Simple Solve.}
  A, X, B all known when creating the solver and the user requires one direct
  solve.
  \begin{lstlisting}
RCP<MAT> A; RCP<MV> X; RCP<MV> B;
// initialize A and B
RCP<Solve<MAT,MV> > solver = Amesos::create(A,X,B);
solver->solve(); // solution placed in X
  \end{lstlisting}
\end{frame}

\begin{frame}[fragile]
  \frametitle{Typical Usage: Preorder, Symbolic, Numeric, Solve}
  A, X, B all known when creating the solver and the user requires one or
  multiple solves. Different steps of the factorization could be called in
  different places.
  \begin{lstlisting}
RCP<MAT> A; RCP<MV> X; RCP<MV> B;
// initialize A and B
RCP<Solve<MAT,MV> > solver = Amesos::create(A,X,B);
solver->preOrdering();
solver->symbolicFactorization();
solver->numericFactorization();
solver->solve();
  \end{lstlisting}
\end{frame}

\begin{frame}
  \frametitle{Solver Interface and Creating Solver}
\end{frame}

\begin{frame}
  \frametitle{Amesos2 Parameters vs Solver Parameters}
\end{frame}

\begin{frame}
  \frametitle{Amesos2 Parameters vs Solver Parameters}
\end{frame}

\begin{frame}
  \frametitle{Solvers supported in Amesos2}
  These solvers will be supported by the Sept. release:
  \begin{itemize}
  \item KLU2 (in progress)
  \item SuperLU
  \item SuperLU\_MT
  \item SuperLU\_dist (in progress)
  \end{itemize}
We are also considering adding LAPACK (for almost dense matrices in CrsMatrix 
format) and Pardiso.
\end{frame}

\begin{frame}
  \frametitle{Support for New Solvers}
\end{frame}

\begin{frame}
  \frametitle{Support for New Matrices and Multivectors}
\end{frame}

\begin{frame}[fragile]
  \frametitle{Expert Usage}
  \framesubtitle{Releasing $A$ after numeric factorization}
  For preconditoners or smoothers when there is one symbolic factorization,
  multiple numeric factorization and multiple solves.
  \begin{lstlisting}
RCP<MAT> A;
// Get A from somewhere
RCP<Solver<MAT,MV> > solver = Amesos::create("SuperLU",A);
Teuchos::ParameterList params("Amesos2");
params.sublist("SuperLU").set("IterRefine","DOUBLE");
params.sublist("SuperLU").set("ColPerm","MMD_AT_PLUS_A");
solver->setParameters(params);
solver->symbolicFactorization().numericFactorization();
A = Teuchos::null;          // no longer need A
solver.setA(Teuchos::null); // tell solver to release A
RCP<MV> X; RCP<MV> B;
// do some other work, finally get B's values
solver->solve(X,B);         // solution placed in X
// do some more work and get new values in B
solver->solve(X,B);
  \end{lstlisting}
\end{frame}

\begin{frame}[fragile]
  \frametitle{Expert Usage}
  \framesubtitle{Reusing the solver for a different matrix}
  Amesos does not support this expert-only usecase: Use the same solver instance  for solving different linear systems.
  \begin{lstlisting}
RCP<MAT> A1, A2;
// initialize A1, A2, and B
RCP<Solver<MAT,MV> > solver = Amesos::create(A1,X,B);
solver->solve(); // solution in X
solver->setA(A2);
solver->solve(); // refactorizes A2 first
  \end{lstlisting}
\end{frame}


\begin{frame}
  \frametitle{Internal Design: Solver Hierarchy}
\end{frame}

\begin{frame}
  \frametitle{Internal Design: MatrixAdapter Hierarchy}
\end{frame}

\begin{frame}
  \frametitle{Future Work}
  \begin{itemize}
  \item Need better support to do ordering: Zoltan2 will provide both
    graph partitoning based and minimum degree based orderings.
  \item  Currently we use SuperLU's internal orderings. KLU2 uses the orderings
    from Amesos.
  \item Need to support more solvers and matrix types.
  \end{itemize}
\end{frame}

\end{document}
