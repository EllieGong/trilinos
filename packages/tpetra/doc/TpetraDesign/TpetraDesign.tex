\documentclass[10pt,relax]{TpetraDesign}
% ---------------------------------------------------------------------------- %
%
% Set the title, author, and date
%
\title{The Design and Evolution of Tpetra}
\SANDsubtitle{}

\author{Paul M. Sexton and Michael A. Heroux \\
       Sandia National Laboratories\\
       P.O. Box 5800\\
       Albuquerque, NM 87185-1110
     }

% There is a "Printed" date on the title page of a SAND report, so
% the generic \date should generally be empty.
\date{}


\SANDnum{SAND2005-xxxx} \SANDprintDate{August 2005}
\SANDauthor{Paul M. Sexton \\
Computational Mathematics and Algorithms Department \\
 \\
Michael A. Heroux \\
Computational Mathematics and Algorithms Department \\
 \\
Sandia National Laboratories \\
P.O. Box 5800 \\
Albuquerque, NM 87185-1110}



\SANDreleaseType{Unlimited Release}


\begin{document}
\maketitle

\begin{abstract}
This paper describes the design and implementation of Tpetra,
a C++ library for basic linear algebra computations on high-performance
distributed node systems. We first provide a brief overview of
the package and its uses. We then discuss in detail both how
Tpetra is designed and implemented, but also the considerations
and decisions that led to those designs being chosen.
\end{abstract}

\clearpage
\section*{Acknowledgement}
The authors would like to acknowledge the support of the ASCI and LDRD programs
that funded development of Trilinos.

\newpage

\section{Introduction}


Many numerical algorithms for the solution of problems in
scientific and engineering applications use vectors and
matrices as basic elements.   Generally, the exact details
of how these mathematical objects are stored, or how computations
are performed using them, is not of primary importance for the
algorithm designer.  Because of this, using abstract object models
that describe the attributes and methods associated with each
object class, and defining the relationship between classes, is an
effective way to separate algorithm design and implementation from
vector and matrix design.

The practical importance of this separation comes from the fact
that there are many different data structures used to store
vectors and matrices and correspondingly many different ways
to implement basic matrix and vector operations.  The variety
of vector matrix data structures and implementations is, to
some extent, a result of arbitrary choices made by application
designers.  However, often there are sound reasons for choosing
particular data structures, either because of the intended
computer architecture or because of the specifics of the
application.  Abstraction of the vector and matrix objects
leverages the investment in sophisticated algorithm implementations
across many different vector and matrix implementations and
provides a convenient generic programming framework for new
algorithm development.

In this paper we present a detailed object model for vectors
and matrices that are intended for use in a parallel, distributed
memory computing environment.  We present a brief overview of
the classes, followed by a UML object model, in the next section.
In subsequent sections we present a collection of pure virtual
classes, two C++ implementations of this model and a Java
implementation.

\end{document}
