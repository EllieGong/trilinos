\documentclass[pdf,ps2pdf,11pt]{SANDreport}
\usepackage{pslatex}

%Local stuff
\usepackage{graphicx}
\usepackage{latexsym}
%
% Command to print 1/2 in math mode real nice
%
\newcommand{\myonehalf}{{}^1 \!\!  /  \! {}_2}

%
% Command to print over/under left aligned in math mode
%
\newcommand{\myoverunderleft}[2]{ \begin{array}{l} #1 \\ \scriptstyle #2 \end{array} }

%
% Command to number equations 1.a, 1.b etc.
%
\newcounter{saveeqn}
\newcommand{\alpheqn}{\setcounter{saveeqn}{\value{equation}}
\stepcounter{saveeqn}\setcounter{equation}{0}
\renewcommand{\theequation}
	{\mbox{\arabic{saveeqn}.\alph{equation}}}}
\newcommand{\reseteqn}{\setcounter{equation}{\value{saveeqn}}
\renewcommand{\theequation}{\arabic{equation}}}

%
% Shorthand macros for setting up a matrix or vector
%
\newcommand{\bmat}[1]{\left[ \begin{array}{#1}}
\newcommand{\emat}{\end{array} \right]}

%
% Command for a good looking \Re in math enviornment
%
\newcommand{\RE}{\mbox{\textbf{I}}\hspace{-0.6ex}\mbox{\textbf{R}}}
\newcommand{\CPLX}{\mbox{\textbf{C}}\hspace{-1.1ex}\mbox{l}\hspace{0.4ex}}

%
% Commands for Jacobians
%
\newcommand{\Jac}[2]{\displaystyle{\frac{\partial #1}{\partial #2}}}
\newcommand{\jac}[2]{\partial #1 / \partial #2}

%
% Commands for Hessians
%
\newcommand{\Hess}[2]{\displaystyle{\frac{\partial^2 #1}{\partial #2^2}}}
\newcommand{\hess}[2]{\partial^2 #1 / \partial #2^2}
\newcommand{\HessTwo}[3]{\displaystyle{\frac{\partial^2 #1}{\partial #2 \partial #3}}}
\newcommand{\hessTwo}[3]{\partial^2 #1 / (\partial #2 \partial #3)}



%\newcommand{\Hess2}[3]{\displaystyle{\frac{\partial^2 #1}{\partial #2 \partial #3}}}
%\newcommand{\myHess2}{\frac{a}{b}}
%\newcommand{\hess2}[3]{\partial^2 #1 / (\partial #2 \partial #3)}

%
% Shorthand macros for setting up a single tab indent
%
\newcommand{\bifthen}{\begin{tabbing} xxxx\=xxxx\=xxxx\=xxxx\=xxxx\=xxxx\= \kill}
\newcommand{\eifthen}{\end{tabbing}}

%
% Shorthand for inserting four spaces
%
\newcommand{\tb}{\hspace{4ex}}

%
% Commands for beginning and ending single spacing
%
\newcommand{\bsinglespace}{\renewcommand{\baselinestretch}{1.2}\small\normalsize}
\newcommand{\esinglespace}{}

%
% Command for text font
%

\newcommand{\ttt}[1]{\texttt{#1}}


\newtheorem{dumb_fact}{Dumb Fact}[section]

% If you want to relax some of the SAND98-0730 requirements, use the "relax"
% option. It adds spaces and boldface in the table of contents, and does not
% force the page layout sizes.
% e.g. \documentclass[relax,12pt]{SANDreport}
%
% You can also use the "strict" option, which applies even more of the
% SAND98-0730 guidelines. It gets rid of section numbers which are often
% useful; e.g. \documentclass[strict]{SANDreport}

% ---------------------------------------------------------------------------- %
%
% Set the title, author, and date
%

\title{
\center
An Overview of MOOCHO \\[2ex]
The Multifunctional Object-Oriented arCHitecture for Optimization
}
\author{
Roscoe A. Bartlett \\ Optimization/Uncertainty Estim \\ \\ Sandia National
Laboratories\footnote{ Sandia is a multiprogram laboratory operated by Sandia
Corporation, a Lockheed-Martin Company, for the United States Department of
Energy under Contract DE-AC04-94AL85000.}, Albuquerque NM 87185 USA, \\ }
\date{}

% ---------------------------------------------------------------------------- %
% Set some things we need for SAND reports. These are mandatory
%
\SANDnum{SAND2006-xxx}
\SANDprintDate{??? 2005}
\SANDauthor{
Roscoe A. Bartlett \\ Optimization/Uncertainty Estim \\ \\
}

% ---------------------------------------------------------------------------- %
% The following definitions are optional. The values shown are the default
% ones provided by SANDreport.cls
%
\SANDreleaseType{Unlimited Release}
%\SANDreleaseType{Not approved for general release}

% ---------------------------------------------------------------------------- %
% The following definition does not have a default value and will not
% print anything, if not defined
%
%\SANDsupersed{SAND1901-0001}{January 1901}

% ---------------------------------------------------------------------------- %
%
% Start the document
%
\begin{document}

\maketitle

% ------------------------------------------------------------------------ %
% An Abstract is required for SAND reports
%

%
\begin{abstract}
%

MOOCHO (Multifunctional Object-Oriented arCHitecture for Optimization) is a
C++ Trilinos package of object-oriented software for solving equality and
inequality constrained nonlinear programs (NLPs) using large-scale
gradient-based optimization methods.  The primary focus of MOOCHO up to this
point has been the development of active-set and interior-point successive
quadratic programming (SQP) methods.  MOOCHO was initially developed (under
the name rSQP++) to support primarily reduced-space SQP (rSQP) but other
related types of optimization algorithms can also be developed.  Using MOOCHO,
it is possible to specialize all of the linear-algebra computations and also
modify many other parts of the algorithm externally (without modifying default
library source code).  One of the most unique features of the MOOCHO framework
is that it supports completely abstract linear algebra which allows
sophisticated implementations on parallel distributed-memory supercomputers
but is not tied to any particular linear algebra library (although adapters to
a few linear algebra libraries are available).  In addition, MOOCHO contains
adapters to support massively parallel simulation-constrained optimization
through Thyra interfaces.  Access to a great deal of linear solver technology
in Trilinos is available through the ``Facade'' classes in the Stramikimos
package.

This document provides a high-level overview of MOOCHO that describes the
motivation for MOOCHO, the basic mathematical notation used in MOOCHO, the
algorithms that MOOCHO implements, and what types of optimization problmes are
appropriate to be solved by MOOCHO.  More detailed documentaion on how to
install MOOCHO, how to define NLPs, and how to run MOOCHO algorithms is
provided in a companion document [???].

%
\end{abstract}
%

% ------------------------------------------------------------------------ %
% An Acknowledgement section is optional but important, if someone made
% contributions or helped beyond the normal part of a work assignment.
% Use \section* since we don't want it in the table of context
%
\clearpage
\section*{Acknowledgment}
The authors would like to thank ...

The format of this report is based on information found
in~\cite{Sand98-0730}.

% ------------------------------------------------------------------------ %
% The table of contents and list of figures and tables
% Comment out \listoffigures and \listoftables if there are no
% figures or tables. Make sure this starts on an odd numbered page
%
\clearpage
\tableofcontents
\listoffigures
%\listoftables

% ---------------------------------------------------------------------- %
% An optional preface or Foreword
%\clearpage
%\section{Preface}
%Although muggles usually have only limited experience with
%magic, and many even dispute its existence, it is worthwhile
%to be open minded and explore the possibilities.

% ---------------------------------------------------------------------- %
% An optional executive summary
%\clearpage
%\section{Summary}
%Once a certain level of mistrust and scepticism has
%been overcome, magic finds many uses in todays science
%and engineering. In this report we explain some of the
%fundamental spells and instruments of magic and wizardry. We
%then conclude with a few examples on how they can be used
%in daily activities at national Laboratories.

% ---------------------------------------------------------------------- %
% An optional glossary. We don't want it to be numbered
%\clearpage
%\section*{Nomenclature}
%\addcontentsline{toc}{section}{Nomenclature}
%\begin{itemize}
%\item[alohomora]
%spell to open locked doors and containers
%\end{itemize}

% ---------------------------------------------------------------------- %
% This is where the body of the report begins; usually with an Introduction
%
\SANDmain % Start the main part of the report

%
\section{Introduction}
%

MOOCHO is an object-oriented C++ software package building gradient-based
algorithms for large-scale nonlinear programing.  MOOCHO is designed to allow
the incorporation of many different algorithms and to allow external
configuration of specialized linear-algebra objects such as vectors, matrices
and linear solvers.  Data-structure independence has been recognized as an
important feature missing in current optimization software
\cite{ref:wright_1999}.

While the MOOCHO framework can be used to implement many different types of
optimization methods (e.g.~Generalized Reduced Gradient (GR) [???], Augmented
Lagrangian (AL) [???], Successive Quadratic Programming (SQP) [???] etc.) the
main focus has been SQP methods.  Successive quadratic programming (SQP)
related methods are attractive mainly because they generally require the
fewest number of function and gradient evaluations to solve a problem as
compared to other optimization methods {}\cite{ref:schmid_accel_1993}.
Another attractive property of SQP methods is that they can be adapted to
effectively exploit the structure of the underlying NLP
{}\cite{ref:varvarezos_1994}.  A variation of SQP, known as reduced-space SQP
(rSQP), works well for NLPs where there are few degrees of freedom (see
Section {}\ref{moocho:sec:nlp_formulation}) and many constraints.
Quasi-Newton methods for approximating the reduced Hessian of the Lagrangian
are also very efficient for NLPs with few degrees of freedom.  Another
advantage of rSQP is that a decomposition for the equality constraints can be
used which only requires solves with a basis of the Jacobian of the
constraints (see Section \ref{moocho:sec:rSQP}) and therefore can utilize very
specialized application-specific data structures and linear solvers.
Therefore, rSQP methods can be tailored to exploit the structure of
simulation-constrained optimization problems and can show excellent parallel
algorithmic scalability.

There is a distiction to be made between a user of MOOCHO and a developer of
MOOCHO, though it may it be narrow one in some cases.  Here we define a user
as anyone who uses MOOCHO to solve an optimization problem using a
pre-existing MOOCHO algorithm.  A MOOCHO user can vary from someone who uses a
predeveloped interface to a modeling environment like AMPL
{}\cite{ref:ampl_1993} to someone who uses MOOCHO to solve a discrietized
PDE-constrained optimization problem on a massively parallel computer using
specialized application-specific data structures and linear solvers
{}\cite{ref:biros_1999}.  While the first type of user does not need to write
any C++ code and does not even need to know what C++ is, the latter type of
sophisitcated user has to write a fair amount of C++ code.  There are also
many different types of use cases of MOOCHO that lie in between these two
extremes.  This user's guide seeks to address, at least to some degree, the
needs of this entire range of users.  Because of this, there will be a fair
amount of discussion of the object-oriented design of the relavent parts of
MOOCHO.

%
% RAB: I have edited to here on 8/4/2006
%

In the next section (Section \ref{moocho:sec:sqp_background}), the
basic mathematical structure of SQP methods is presented.  This
presentation is intended to establish the nomenclature of MOOCHO for
users and developers.  This nomenclature is key to being able to
understand and modify the MOOCHO algorithms.  Appendix
\ref{app:moocho_nomenclature_summary} contains a summary of this
notation.  The basic software design of MOOCHO that both users and
developers must understand is described in Section
\ref{moocho:sec:basic_software_design}.  This is followed in Section
\ref{moocho:sec:nlp_and_lin_alg_itfc} by a basic description of the
linear algebra and NLP interfaces for MOOCHO.  These interfaces
provide the foundation for allowing the types of specialized data
structures and linear solvers that an advanced user would use with
MOOCHO.  Section \ref{moocho:sec:solve_explicit_nlps} discusses a
software-based use of MOOCHO for general NLPs where explicit gradient
entries are computed.  Apart from using a predeveloped interface to
MOOCHO (e.g.~AMPL), this is the simplest use case for MOOCHO.  This
section includes a complete example NLP with numerious C++ code
excepts.  This discussion is followed up in Section
\ref{moocho:sec:solve_specialized_nlps} by an example NLP that
specializes all of the linear algebra and NLP interfaces, uses
application specific linear solvers, and runs on a distributed-memory
parallel computer using MPI.  This example represents the most
advanced use case for MOOCHO and provides the needed foundation for
even the most advanced interface to a sophisticated application.
Section
\ref{moocho:sec:algo_configurations} describes the algorithm
configuration classes that are used to build MOOCHO algorithms and
includes a fairly detailed discussion of a default configuration
called ``MamaJama''.  Details of the input and output files for MOOCHO
(for the ``MamaJama'' configuration and an example NLP) are discussed
in Section \ref{rsqp:sec:detailed_example}. This section describes
the example printouts that are included in Appendix
\ref{app:ex_moocho_printout}.  Finally, Appendix
\ref{app:moocho_install} describes the installation for the base
distribution of MOOCHO which is a first step to using MOOCHO.

%
\section{Mathematical Background}
\label{moocho:sec:sqp_background}
%

%
\subsection{Nonlinear Program (NLP) Formulation}
\label{moocho:sec:nlp_formulation}
%

MOOCHO can be used to solve NLPs of the general form: 

{\bsinglespace
\begin{eqnarray}
\mbox{min}  &  & f(x)                     \label{moocho:eqn:nlp:obj} \\
\mbox{s.t.} &  & c(x) = 0                 \label{moocho:eqn:nlp:equ} \\
            &  & x_L \leq x    \leq x_U   \label{moocho:eqn:nlp:bnds}
\end{eqnarray}
\begin{tabbing}
\hspace{4ex}where:\hspace{5ex}\= \\
\>	$x, x_L, x_U \:\in\:\mathcal{X}$ \\
\>	$f(x) : \:\mathcal{X} \rightarrow \RE$ \\
\>	$c(x) : \:\mathcal{X} \rightarrow \mathcal{C}$ \\
\>	$\mathcal{X} \:\subseteq\:\RE\:^n$ \\
\>	$\mathcal{C} \:\subseteq\:\RE\:^m$.
\end{tabbing}
\esinglespace}

Above, we have been very careful to define vector spaces for the
relevant vectors and nonlinear operators.  In general, only vectors
from the same vector space are compatible and can participate in
linear-algebra operations.  Mathematically, the only requirement for
the compatibility of real-valued vector spaces should be that the
dimensions match up and that the same inner products are used.
However, having the same dimensionality will not be sufficient to
allow the compatibility of vectors from different vector spaces in the
implementation.  The vector spaces become very important later when
the NLP interfaces and the implementation of MOOCHO is discussed in
more detail in Section \ref{moocho:sec:nlp_and_lin_alg_itfc} and in
\cite{ref:moochodevguide}.

We assume that the operators $f(x)$ and $c_j(x)$ for $j = 1 \ldots m$
in (\ref{moocho:eqn:nlp:obj})--(\ref{moocho:eqn:nlp:equ}) are
nonlinear functions with at least second-order continuous derivatives.
The rSQP algorithms described later only require first-order
information for $f(x)$ and $c_j(x)$ in the form of a vector $\nabla
f(x)$ and a matrix $\nabla c(x)$ respectively.  The bound inequality
constraints in (\ref{moocho:eqn:nlp:bnds}) may have lower bounds equal
to $-\infty$ and/or upper bounds equal to $+\infty$.  The absences of
some of these bounds can be exploited by many SQP algorithms.

It is very desirable for the functions $f(x)$ and $c(x)$ to at least
be defined (i.e.~no NaN or Inf return values) everywhere in the set
defined by the relaxed variable bounds $x_L - \delta \leq x \leq x_U +
\delta$.  Here, $\delta$ (see the method
\texttt{max\_var\_bounds\_viol()} in the Doxygen documentation for the
\texttt{\textit{NLP}} interface) is a relaxation (i.e.~wiggle room)
that the user can set to allow the optimization algorithm to compute
$f(x)$ and $c(x)$ outside the strict variable bounds $x_L \le x \le
x_U$ in order to compute finite differences and the like.  The SQP
algorithms in MOOCHO will never evaluate $f(x)$ and $c(x)$ outside the
above relaxed variable bounds.  This gives users a measure of control
in how the optimization algorithms interact with the NLP model.

%
% ToDo: Move this somewhere else (modeling for optimization).
%
% The user should be aware of this when formulating an
%NLP to be solved.  For example, suppose there is a term $log(x_5)$ in one of the constraints.  In order
%to keep this term well defined and bounded one would like to set a lower bound like $x_5 \ge x_L_5 = 10^{-6}$.
%However, for accurate finite differencing one would like to allow $\delta \approx 10^{-5}$ but this may
%cause $x_5$ to be negative ($x_5 = x_L_5 - \delta = - 0.9 \times 10^{-5}$) which would cause the $log(x_5)$
%term to be NaN.  To account for this, the user would be advised to set the lower bound as
%$x_5 \ge x_L_5 = 10^{-4}$.  If this lower bound was active at the solution of the NLP, the user would then
%be advised to look at the Lagrange multiplier for this variable bound (see $\nu$ below) to see how sensitive
%the object function is to this variable and, then perhaps decrease this lower bound some and decrease
%$\delta$ also.  However, setting $\delta = 0$ will not cause the SQP algorithms to fail, but it will
%limit the types of testing and other algorithmic options that can be performed using finite differencing.

The Lagrangian function $L(\lambda, \nu_L, \nu_U)$ and the Lagrange multipliers ($\lambda$, $\nu_L$, $\nu_U$) for this
NLP are defined by

{\bsinglespace
\begin{eqnarray}
L(x,\lambda,\nu_L,\nu_U)
 & = & \left\{ f(x) + \lambda^T c(x) + \nu_L^T ( x_L - x ) + \nu_U^T ( x - x_U ) \right\} \:\in\:\RE
\label{moocho:eqn:L_def}
\end{eqnarray}
%
\begin{equation}
\nabla_{x} L(x,\lambda,\nu) = \left\{ \nabla f(x) + \nabla c(x) \lambda + \nu
   \right\} \:\in\:\mathcal{X}
\label{moocho:eqn:GL_def}
\end{equation}
%
\begin{equation}
\nabla_{xx}^2 L(x,\lambda) = \left\{ \nabla^2 f(x) + \sum^m_{j=1} \lambda_{(j)} \nabla^2 c_j(x) \right\}
	\:\in\: \mathcal{X}|\mathcal{X} \label{moocho:eqn:HL_def}
\end{equation}
%
\begin{tabbing}
\hspace{4ex}where:\hspace{5ex}\= \\
\>	$\nabla f(x) : \:\mathcal{X} \rightarrow \mathcal{X}$ \\
\>	$\nabla c(x) = {\bmat{cccc} \nabla c_1 (x) & \nabla c_2 (x) & \ldots & \nabla c_m (x)  \emat}
         : \:\mathcal{X} \rightarrow \mathcal{X}|\mathcal{C}$ \\
\>	$\nabla^2 f(x) : \:\mathcal{X} \rightarrow \mathcal{X}|\mathcal{X}$ \\
\>	$\nabla^2 c_j(x) : \:\mathcal{X} \rightarrow \mathcal{X}|\mathcal{X} \; \mbox{, for}\:j = 1 \ldots m$ \\
\>	$\lambda \:\in\:\mathcal{C}$ \\
\>	$\nu \equiv \nu_U - \nu_L \:\in\:\mathcal{X}$.
\end{tabbing}

Above, we use the notation $\lambda_{(j)}$ with the subscript in
parentheses to denote the $j^{\mbox{th}}$ component of the vector
$\lambda$ and to differentiate this from a simple math accent.  Also,
$\nabla c(x) : \mathcal{X} \rightarrow \mathcal{X}|\mathcal{C}$ is
used to denote a nonlinear operator (the gradient of the equality
constraints $\nabla c(x)$ in this case) that maps from the vector
space $\mathcal{X}$ to a matrix space $\mathcal{X}|\mathcal{C}$ where
the columns and rows in this matrix space lie in the vector spaces
$\mathcal{X}$ and $\mathcal{C}$ respectively.  The returned matrix
object $A = \nabla c \in \mathcal{X}|\mathcal{C}$ defines a linear
operator where $q = A p$ maps vector from $p \in \mathcal{C}$ to $q
\in \mathcal{X}$.  The transposed matrix object $A^T$ defines a linear
operator where $q = A^T p$ maps vector from $p \in \mathcal{X}$ to $q
\in \mathcal{C}$.

Note how the vector and matrix spaces in the above
expressions match up.  For example, the vectors and matrices in
(\ref{moocho:eqn:GL_def}) can be replaced by their vector and matrix
spaces as

\[
 \left\{ \nabla f(x) + \nabla c(x) \lambda + \nu \right\}
\Rightarrow
 \left\{ \mathcal{X} + (\mathcal{X}|\mathcal{C}) \mathcal{C} + (\mathcal{X}|\mathcal{H}) \mathcal{H}
 + \mathcal{X} \right\}
\Rightarrow \mathcal{X}.
\]

The compatibility of vectors and matrices in linear-algebra operations
is determined by the compatibility of the associated vector spaces.
At all times, we must know to which vector or matrix space a
linear-algebra quantity belongs.

Given the definition of the Lagrangian and its derivatives in (\ref{moocho:eqn:L_def})--(\ref{moocho:eqn:HL_def}),
the first- and second-order necessary KKT optimality conditions \cite{ref:nash_sofer_1996} for a
solution $(x^*, \lambda^*, \nu^*_L, \nu^*_U)$ to (\ref{moocho:eqn:nlp:obj})--(\ref{moocho:eqn:nlp:bnds}) are given in
(\ref{moocho:eqn:kkt:lin_dep_grads})--(\ref{moocho:eqn:kkt:HL_psd}).  There are four different categories of optimality
conditions shown here: linear dependence of gradients (\ref{moocho:eqn:kkt:lin_dep_grads}), feasibility
(\ref{moocho:eqn:kkt:equ_feas})--(\ref{moocho:eqn:kkt:bnds_feas}), non-negativity of Lagrange multipliers
for inequalities (\ref{moocho:eqn:kkt:nonneg_bnds_mult}),
complementarity (\ref{moocho:eqn:kkt:xl_comp})--(\ref{moocho:eqn:kkt:xu_comp}), and curvature
(\ref{moocho:eqn:kkt:HL_psd}).

{\bsinglespace
\begin{equation}
\nabla_{x} L(x^*,\lambda^*,\nu^*) = \nabla f(x^*) + \nabla c(x^*) \lambda^* + \nu^* = 0
\label{moocho:eqn:kkt:lin_dep_grads}
\end{equation}
%
\begin{equation}
c(x^*) = 0
\label{moocho:eqn:kkt:equ_feas}
\end{equation}
%
\begin{equation}
x_L \leq x^* \leq x_U
\label{moocho:eqn:kkt:bnds_feas}
\end{equation}
%
\begin{equation}
(\nu_L)^*, (\nu_U)^* \geq 0
\label{moocho:eqn:kkt:nonneg_bnds_mult}
\end{equation}
%
\begin{equation}
(\nu_L)^*_{(i)} ( (x_L)_{(i)} - (x^*)_{(i)} ) = 0, \;\; \mbox{for} \; i = 1 \ldots n
\label{moocho:eqn:kkt:xl_comp}
\end{equation}
%
\begin{equation}
(\nu_U)^*_{(i)} ( (x^*)_{(i)} - (x_U)_{(i)} ) = 0, \;\; \mbox{for} \; i = 1 \ldots n
\label{moocho:eqn:kkt:xu_comp}
\end{equation}
%
\begin{equation}
d^T \: \nabla_{xx}^2 L(x^*,\lambda^*) \: d \geq 0, \;\; \mbox{for all feasible directions $d \:\in\:\mathcal{X}$}.
\label{moocho:eqn:kkt:HL_psd}
\end{equation}
\esinglespace}

Sufficient conditions for optimality require that stronger assumptions
be made about the NLP (e.g.~constraint qualification on $c(x)$ and
perhaps conditions on third-order curvature in case $d^T
\: \nabla_{xx}^2 L(x^*,\lambda^*) \: d = 0$ in
(\ref{moocho:eqn:kkt:HL_psd})).

To solve a NLP, an SQP algorithm must first be supplied an initial
guess for the unknown variables $x_0$ and in some cases also the
Lagrange multipliers $\lambda_0$ and $\nu_0$.  The optimization
algorithms implemented in MOOCHO generally require that $x_0$ satisfy
the variable bounds in (\ref{moocho:eqn:nlp:bnds}), and if not, then
the elements of $x_0$ are forced in bounds.  The matrix $\nabla c(x)$
is abstracted behind a set of object-oriented interfaces.  An rSQP
algorithm only needs to perform matrix-vector multiplication with
$\nabla c(x)$ and solve for a square, nonsingular basis of $\nabla
c(x)$ through a \texttt{\textit{BasisSystem}} interface.  The
implementation of $\nabla c(x)$ is completely abstracted away from the
optimization algorithm.  A simpler interface to NLPs has also been
developed where the matrix $\nabla c(x)$ is never represented even
implicitly (i.e.~no matrix-vector products) and only specific
quantities are supplied to the rSQP algorithm (see the ``Tailored
Approach'' in \cite{ref:schmid_rsqp_1994} and the ``direct
sensitivity'' NLP interface in \cite{ref:moochodevguide}).

%
\subsection{Successive Quadratic Programming (SQP)}
\label{moocho:sec:SQP}
%

A popular class of methods for solving NLPs is successive quadratic
programming (SQP) \cite{ref:boggs_tolle_1996}.  An SQP method is
equivalent, in many cases, to applying Newton's method to solve the
optimality conditions represented by
(\ref{moocho:eqn:kkt:lin_dep_grads})--(\ref{moocho:eqn:kkt:equ_feas}).
At each Newton iteration $k$ for
(\ref{moocho:eqn:kkt:lin_dep_grads})--(\ref{moocho:eqn:kkt:equ_feas}),
the linear subproblem (also known as the KKT system) takes the form

{\bsinglespace
\begin{equation}
{\bmat{cc}
	W    & A \\
	A^T  &
\emat}
{\bmat{c}
	d \\
	d_{\lambda}
\emat}
=
-
{\bmat{c}
	\nabla_x L \\
	c
\emat}
\label{rsqp:eqn:full_kkt_sys}
\end{equation}
\begin{tabbing}
\hspace{4ex}where:\hspace{5ex}\= \\
\>	$d = x_{k+1} - x_k \:\in\:\mathcal{X}$ \\
\>	$d_{\lambda} = \lambda_{k+1} - \lambda_k \:\in\:\mathcal{C}$ \\
\>	$W \approx \nabla_{xx}^2 L(x_k,\lambda_k) \:\in\:\mathcal{X}|\mathcal{X}$ \\
\>	$A = \nabla c(x_k) \:\in\:\mathcal{X}|\mathcal{C}$ \\
\>	$c = c(x_k) \:\in\:\mathcal{C}$.
\end{tabbing}
\esinglespace}

The Newton matrix in (\ref{rsqp:eqn:full_kkt_sys}) is known as the KKT matrix.
By substituting $d_{\lambda} = \lambda_{k+1} - \lambda_k$ into (\ref{rsqp:eqn:full_kkt_sys})
and simplifying, this linear system becomes equivalent to the optimality conditions of
the following QP

{\bsinglespace
\begin{eqnarray}
\mbox{min}  &  & g^T d + \myonehalf d^T W d                   \label{moocho:eqn:qp_newton:obj} \\
\mbox{s.t.} &  & A^T d + c = 0                                \label{moocho:eqn:qp_newton:equ}
\end{eqnarray}
\begin{tabbing}
\hspace{4ex}where:\hspace{5ex}\= \\
\>	$g = \nabla f(x_k) \:\in\:\mathcal{X}.$
\end{tabbing}
\esinglespace}

The advantage of the QP formulation over the Newton linear system formulation is that
inequality constraints can be directly added to the QP and a relaxation can be defined
which yields the following QP

{\bsinglespace
\begin{eqnarray}
\mbox{min}  &  & g^T d + \myonehalf d^T W d + M(\eta)                   \label{moocho:eqn:qp:obj} \\
\mbox{s.t.} &  & A^T d + (1-\eta) c = 0                                 \label{moocho:eqn:qp:equ} \\
            &  & x_L - x_k \leq d \leq x_U -x_k                         \label{moocho:eqn:qp:bnds} \\
            &  & 0 \leq \eta \leq 1                                     \label{moocho:eqn:qp:eta_bnd}
\end{eqnarray}
\begin{tabbing}
\hspace{4ex}where:\hspace{5ex}\= \\
\>	$M(\eta) \:\in\:\RE \rightarrow \RE$.
\end{tabbing}
\esinglespace}

Near the solution of the NLP,
the set of active constraints for
(\ref{moocho:eqn:qp:obj})--(\ref{moocho:eqn:qp:eta_bnd}) will be the same
as the optimal active-set for the NLP in
(\ref{moocho:eqn:nlp:obj})--(\ref{moocho:eqn:nlp:bnds})
\cite[Theorem 18.1]{ref:nocedal_wright_1999}.

The relaxation of the QP shown in
(\ref{moocho:eqn:qp:obj})--(\ref{moocho:eqn:qp:eta_bnd}) is only one
form of a relaxation but has the essential properties.  Note that the
solution $\eta = 1$ and $d = 0$ is always feasible by construction.
The penalty function $M(\eta)$ is either a linear or quadratic term
where if $\frac{\partial M(\eta)}{\partial \eta}|_{\eta = 0}$ is
sufficiently large then an unrelaxed solution (i.e.~$\eta = 0$) will
be obtained if a feasible region for the original QP exists.  For
example, the penalty term may take a form such as $M(\eta) =
(\tilde{M}) \eta$ or $M(\eta) = (\tilde{M})(\eta + \myonehalf \eta^2)$
where $\tilde{M}$ is a large constant often called ``big M''.

Once a new estimate of the solution ($x_{k+1}$, $\lambda_{k+1}$,
$\nu_{k+1}$) is computed, the error in the optimality conditions
(\ref{moocho:eqn:kkt:lin_dep_grads})--(\ref{moocho:eqn:kkt:bnds_feas})
is checked.  If these KKT errors are within some specified tolerance,
the algorithm is terminated with the optimal solution.  If the KKT
error is too large, the NLP functions and gradients are then computed
at the new point $x_{k+1}$ and another QP subproblem
(\ref{moocho:eqn:qp:obj})--(\ref{moocho:eqn:qp:eta_bnd}) is solved
which generates another step $d$ and so on.  This algorithm is
continued until a solution is found or the algorithm runs into trouble
(there can be many causes for algorithm failure), or it is prematurely
terminated because it is taking too long (i.e.~maxumum number of
iterations or runtime is exceeded).

The iterates generated from $x_{k+1} = x_k + d$ are generally only
guaranteed to converge to a local solution to the first-order KKT
conditions when close to the solution.  Therefore, globalization
methods are used to insure (given a few, sometimes strong, assumptions
are satisfied) the SQP algorithm will converge to a local solution
from remote starting points.  One popular class of globalization
methods are line search methods.  In a line search method, once the
step $d$ is computed from the QP subproblem, a line search procedure
is used to find a step length $\alpha$ such that $x_{k+1} = x_k +
\alpha d$ gives {\em sufficient reduction} in the value of a
{\em merit function} $\phi(x_{k+1}) < \phi(x_k)$.  A merit function
is used to balance a trade-off between minimizing the objective
function $f(x)$ and reducing the error in the constraints $c(x)$.  A
commonly used merit function is the $\ell_1$ defined by
(\ref{moocho:eqn:phi_L1}) where $\mu$ is a penalty parameter that is
adjusted to insure descent along the SQP step $x_k + \alpha d$ for
$\alpha > 0$.

{\bsinglespace
\begin{equation}
\phi_{\ell_1}(x) = f(x) + \mu ||c(x)||_1
\label{moocho:eqn:phi_L1}
\end{equation}
\esinglespace}

An alternative line search based on a ``Filter'' has also been implemented which generally
performs better and does not require the maintenance of a penalty parameter $\mu$.
Other globalization methods such as trust region (using a merit function or the filter)
can also be applied to SQP.

Because SQP is essentially equivalent to applying Newton's method to
the optimality conditions, it can be shown to be quadratically
convergent near the solution of the NLP
\cite{ref:nocedal_overton_1985}.  It is this fast rate of convergence
that makes SQP the method of choice for many applications.  However,
there are many theoretical and practical details that need to be
considered.  One difficulty is that in order to achieve quadratic
convergence the exact Hessian of the Lagrangian $W$ is needed, which
requires exact second-order information $\nabla^2 f(x)$ and $\nabla^2
c_j(x)$, $j = 1 \ldots m$.  For many NLP applications, second
derivatives are not readily available and it is too expensive and/or
inaccurate to compute them numerically.  Other difficulties with SQP
include how to deal with an indefinite Hessian $W$.  Also, for large
problems, the full QP subproblem in
(\ref{moocho:eqn:qp:obj})--(\ref{moocho:eqn:qp:eta_bnd}) can be
extremely expensive to solve directly.  These and other difficulties
have motivated the research of large-scale decomposition methods for
SQP.  One class of these methods is reduced-space (or reduced Hessian)
SQP, or rSQP for short.

%
\subsection{Reduced-Space Successive Quadratic Programming (rSQP)}
\label{moocho:sec:rSQP}
%

In a reduced-space SQP (rSQP) method, the full-space QP subproblem
(\ref{moocho:eqn:qp:obj})--(\ref{moocho:eqn:qp:eta_bnd}) is decomposed
into two smaller subproblems that, in many cases, are easier to solve.
To see how this is done, first a null-space decomposition
\cite[Section 18.3]{ref:nocedal_wright_1999}
is computed for some linearly independent set of the
linearized equality constraints $A_d \:\in\:\mathcal{X}|\mathcal{C}_d$
where $c_d(x)\:\in\:\mathcal{C}_d\:\in\:\RE\:^{r}$ are the decomposed
and $c_u(x)\:\in\:\mathcal{C}_u\:\in\:\RE\:^{(m-r)}$ are the
undecomposed equality constraints and

{\bsinglespace
\begin{equation}
c(x) =
{\bmat{c} c_d(x) \\ c_u(x) \emat} \:\in\:\mathcal{C}_d \times \mathcal{C}_u
\; \Longrightarrow \;
\nabla c(x_k) = {\bmat{cc} \nabla c_d(x_k) & \nabla c_u(x_k) \emat}
= {\bmat{cc} A_d & A_u \emat} \:\in\:\mathcal{X}|(\mathcal{C}_d \times \mathcal{C}_u).
\label{moocho:eqn:lin_indep_constr}
\end{equation}
\esinglespace}
%
Above, the vector space $\mathcal{C} = \mathcal{C}_d \times \mathcal{C}_u$ denotes a concatenated vector space
(also known as a product of vector spaces) with a dimension which is the sum of the constituent vector spaces
$|\mathcal{C}| = |\mathcal{C}_d| + |\mathcal{C}_u| = r + (m - r) = m$.
This decomposition is defined by a null-space matrix $Z$ and a matrix $Y$ with the following properties:

{\bsinglespace
\begin{equation}
\begin{array}{ll}
Z \:\in\:\mathcal{X}|\mathcal{Z}
	& \mbox{s.t.} \; (A_d)^T Z = 0 \\
Y \:\in\:\mathcal{X}|\mathcal{Y}
	& \mbox{s.t.} \; {\bmat{cc} Y & Z \emat} \; \mbox{is nonsingular}
\end{array}
\label{moocho:eqn:Z_Y_def}
\end{equation}
\begin{tabbing}
\hspace{4ex}where:\hspace{5ex}\= \\
\>	$\mathcal{Z} \:\subseteq\:\RE\:^{(n-r)}$ \\
\>	$\; \mathcal{Y} \:\subseteq\:\RE\:^{r}$.
\end{tabbing}
\esinglespace}
%
It is important to distinguish the spaces $\mathcal{Z}$ and $\mathcal{Y}$ from the the
matrices $Z$ and $Y$.  The null-space matrix $Z\:\in\:\mathcal{X}|\mathcal{Z}$
is a linear operator that maps vectors from the space $u\:\in\:\mathcal{Z}$ to vectors in the space
of the unknowns $v = Z u \:\in\:\mathcal{X}$.  The matrix $Y\:\in\:\mathcal{X}|\mathcal{Y}$
is a linear operator that maps vectors from the space $u\:\in\:\mathcal{Y}$ to vectors in the space
of the unknowns $v = Y u \:\in\:\mathcal{X}$.

In many presentations of reduced-space SQP, the matrix $Y$ is referred to as the ``range-space''
matrix since several popular choices of this matrix form a basis for the range space
of $A_d$.  However, note that the matrix $Y$ need not be a true basis matrix for the range-space of
$A_d$ in order to satisfy the nonsingularity property in (\ref{moocho:eqn:Z_Y_def}).
For this reason, here the matrix $Y$ will be referred to as the ``quasi-range-space'' matrix to make this
distinction.

By using (\ref{moocho:eqn:Z_Y_def}), the search direction $d$
can be broken down into $d = (1-\eta) Y p_y + Z p_z$, where $p_y \:\in\:\mathcal{Y}$ and $p_z \:\in\:\mathcal{Z}$ are the
known as the quasi-normal (or quasi-range space) and tangential (or null space) steps respectively.
By substituting $d = (1-\eta) Y p_y + Z p_z$ into
(\ref{moocho:eqn:qp:obj})--(\ref{moocho:eqn:qp:eta_bnd}) we obtain the quasi-normal
(\ref{moocho:eqn:range_space_step}) and tangential (\ref{moocho:eqn:rsqp:obj})--(\ref{moocho:eqn:rsqp:inequ})
subproblems.  In (\ref{moocho:eqn:rsqp:obj}), $\zeta \leq 1 $ is a damping parameter which can be used to insure
descent of the merit function $\phi(x_{k+1}+\alpha d)$.\\[1ex]

{\bsinglespace
\begin{center}\textbf{Quasi-Normal (Quasi-Range-Space) Subproblem}\end{center}
\begin{equation}
p_y = - R^{-1} c_d \:\in\:\mathcal{Y}
\label{moocho:eqn:range_space_step}
\end{equation}
\hspace{4ex}where: $R \equiv [(A_d)^T Y]  \:\in\:\mathcal{C}_d|\mathcal{Y}$
	(nonsingular via (\ref{moocho:eqn:Z_Y_def})). \\[2ex]

\begin{center}\textbf{Tangential (Range-Space) Subproblem (Relaxed)}\end{center}
\begin{eqnarray}
\mbox{min}  &  & (g^r + \zeta w)^T p_z + \myonehalf p_z^T [Z^T W Z] p_z + M(\eta)
                 \label{moocho:eqn:rsqp:obj} \\
\mbox{s.t.} &  & U_z p_z + (1-\eta) u = 0 
                 \label{moocho:eqn:rsqp:equ} \\
            &  & b_L \leq Z p_z - (Y p_y) \eta \leq b_U	
                 \label{moocho:eqn:rsqp:inequ}
\end{eqnarray}
\begin{tabbing}
\hspace{4ex}where:\hspace{5ex}\= \\
\>	$g^r \equiv Z^T g \:\in\:\mathcal{Z}$ \\
\>	$w \equiv Z^T W Y p_y \:\in\:\mathcal{Z}$ \\
\>	$\zeta \:\in\:\RE$ \\
\>	$U_z \equiv [(A_u)^T Z] \:\in\:\mathcal{C}_u|\mathcal{Z}$ \\
\>	$U_y \equiv [(A_u)^T Y] \:\in\:\mathcal{C}_u|\mathcal{Y}$ \\
\>  $u   \equiv U_y p_y + c_u \:\in\:\mathcal{C}_u$ \\
\>	$b_L \equiv x_L - x_k - Y p_y \:\in\:\mathcal{X}$ \\
\>	$b_U \equiv x_U - x_k - Y p_y \:\in\:\mathcal{X}$.
\end{tabbing}
\esinglespace}

By using this decomposition, the Lagrange multipliers $\lambda_d$ for
the decomposed equality constraints ($(A_d)^T d + c_d = 0$) do not
need to be computed in order to produce steps $d = (1-\eta) Y p_y + Z
p_z$.  However, these multipliers can be used to determine the penalty
parameter $\mu$ for the merit function \cite[page
544]{ref:nocedal_wright_1999} or to compute the Lagrangian function.
Alternatively, a multiplier free method for computing $\mu$ has been
developed and tested with good results \cite{ref:schmid_rsqp_1994}.
In any case, it is useful to compute these multipliers at the solution
of the NLP since they give the sensitivity of the objective function
to those constraints \cite[page 436]{ref:nash_sofer_1996}.  An
expression for computing $\lambda_d$ can be derived by applying
(\ref{moocho:eqn:Z_Y_def}) to $Y^T \nabla L(x,\lambda,\nu)$ to yield

{\bsinglespace
\begin{equation}
\lambda_d = - R^{-T} \left( Y^T(g + \nu) + U_y^T \lambda_u \right)
    \:\in\:\mathcal{C}_d.
\label{moocho:eqn:lambda_d}
\end{equation}
\esinglespace}

There are many details that need to be worked out in order to
implement an rSQP algorithm and there are opportunities for a lot of
variability.  There are some significant decisions that need to be
made: how to compute the null-space decomposition that defines the
matrices $Z$, $Y$, $R$, $U_z$ and $U_y$, and how the reduced Hessian
$Z^T W Z$ and the cross term $w$ in (\ref{moocho:eqn:rsqp:obj}) are
calculated (or approximated).

There are several different ways to compute decomposition matrices $Z$
and $Y$ that satisfy (\ref{moocho:eqn:Z_Y_def})
\cite{ref:schmid_accel_1993}.  For small-scale rSQP, an orthonormal
$Z$ and $Y$ ($Z^T Y = 0$, $Z^T Z = I$, $Y^T Y = I$) can be computed
using a QR factorization of $A_d$ \cite{ref:nocedal_overton_1985}.
This decomposition gives rise to rSQP algorithms with many desirable
properties.  However, using a QR factorization when $A_d$ is of very
large dimension is prohibitively expensive.  Therefore, other choices
for $Z$ and $Y$ have been investigated that are more appropriate for
large-scale rSQP.  Methods that are more computationally tractable are
based on a variable-reduction decomposition
\cite{ref:schmid_accel_1993}.  In a variable-reduction decomposition,
the variables are partitioned into dependent $x_D$ and independent
$x_I$ sets

{\bsinglespace
\begin{eqnarray}
x_D & & \:\in\:\mathcal{X}_D \\
x_I & & \:\in\:\mathcal{X}_I \\
x = {\bmat{c} x_D \\ x_I \emat} & & \:\in\: \mathcal{X}_D \times \mathcal{X}_I
\label{rsqp:eqn:x_D_I} \\
\end{eqnarray}
\begin{tabbing}
\hspace{4ex}where:\hspace{5ex}\= \\
\>	$\mathcal{X}_D\:\subseteq\:\RE^{r}$\\
\>	$\mathcal{X}_I\:\subseteq\:\RE^{n-r}$
\end{tabbing}
\esinglespace}

such that the Jacobian of the constraints $A^T$ is partitioned as
shown in (\ref{moocho:eqn:basis_partitioning}) where $C$ is a square,
nonsingular matrix known as the basis matrix.  The variables $x_D$ and
$x_I$ are also called the state and design (or controls) variables
\cite{GBiros_OGhattas_1999a} in some applications or the basic and
nonbasic variables \cite{ref:murtagh_minos_1995} in others.  What is
important about this partitioning of variables is that the $x_D$
variables define the selection of the basis matrix $C$, nothing more.
Some types of optimization algorithms give more significance to this
partitioning of variables (for example, in MINOS
\cite{ref:murtagh_minos_1995} the basic variables are also variables
that are not at an active bound) however no extra significance can be
attributed here.

This basis selection is
used to define a variable-reduction null-space matrix $Z$ in
(\ref{moocho:eqn:vr:Z}) which also determines $U_z$ in
(\ref{moocho:eqn:vr:Uz}).

{\bsinglespace
\begin{center}\textbf{Variable-Reduction Partitioning}\end{center}
\begin{equation}
A^T =
{\bmat{c}
(A_d)^T \\
(A_u)^T
\emat}
=
{\bmat{cc}
C & N \\
E & F
\emat}
\label{moocho:eqn:basis_partitioning}
\end{equation} 
\begin{tabbing}
\hspace{4ex}where:\hspace{5ex}\= \\
\>	$C \:\in\:\mathcal{C}_d|\mathcal{X}_D$ \hspace{4ex} (nonsingular)\\
\>	$N \:\in\:\mathcal{C}_d|\mathcal{X}_I$ \\
\>	$E \:\in\:\mathcal{C}_u|\mathcal{X}_D$ \\
\>	$F \:\in\:\mathcal{C}_u|\mathcal{X}_I$.
\end{tabbing}

\begin{center}\textbf{Variable-Reduction Null-Space Matrix}\end{center}
\begin{eqnarray}
Z & \equiv & {\bmat{c} - C^{-1} N \\ I \emat}       \label{moocho:eqn:vr:Z}  \\
U_z & = & F - E \: C^{-1} N                         \label{moocho:eqn:vr:Uz} 
\end{eqnarray}
\esinglespace}

There are many choices for the quasi-range-space matrix $Y$ that
satisfy (\ref{moocho:eqn:Z_Y_def}).  Two relatively computationally
inexpensive choices are the coordinate and orthogonal decompositions
shown below.

{\bsinglespace
\begin{center}\textbf{Coordinate Variable-Reduction Null-Space Decomposition}\end{center}
\begin{eqnarray}
Y & \equiv & {\bmat{c} I \\ 0 \emat}    \label{moocho:eqn:vr_coor:Y} \\
R & = & C                               \label{moocho:eqn:vr_coor:R} \\
U_y & = & E                             \label{moocho:eqn:vr_coor:Uy}
\end{eqnarray}

\begin{center}\textbf{Orthogonal Variable-Reduction Null-Space Decomposition}\end{center}
\begin{eqnarray}
Y & \equiv & {\bmat{c} I \\ N^T C^{-T} \emat}       \label{moocho:eqn:vr_ortho:Y} \\
R & = & C (I + C^{-1 }N N^T C^{-T})                 \label{moocho:eqn:vr_ortho:R} \\
U_y & = & E - F N^T C^{-T}                          \label{moocho:eqn:vr_ortho:Uy}
\end{eqnarray}
\esinglespace}

The orthogonal decomposition ($Z^T Y = 0$, $Z^T Z \neq I$, $Y^T Y \neq
I$) defined in (\ref{moocho:eqn:vr:Z})--(\ref{moocho:eqn:vr:Uz}) and
(\ref{moocho:eqn:vr_ortho:Y})--(\ref{moocho:eqn:vr_ortho:Uy}) is more
numerically stable than the coordinate decomposition and has other
desirable properties in the context of rSQP
\cite{ref:schmid_accel_1993}.  However, the amount of dense linear
algebra required to compute the factorizations needed to solve for
linear systems with $R$ (\ref{moocho:eqn:vr_ortho:R}) is $O((n-r)^2
r)$ floating point operations (flops) which can dominate the cost of
the algorithm for larger $(n-r)$.  Therefore, for larger $(n-r)$, the
coordinate decomposition ($Z^T Y \neq 0$, $Z^T Z \neq I$, $Y^T Y \neq
I$) defined in (\ref{moocho:eqn:vr:Z})--(\ref{moocho:eqn:vr:Uz}) and
(\ref{moocho:eqn:vr_coor:Y})--(\ref{moocho:eqn:vr_coor:Uy}) is
preferred because it is cheaper but the downside is that it is also
more susceptible to problems associated with a poor selection of
dependent variables and ill-conditioning in the basis matrix $C$ that
can result in greatly degraded performance and even failure of an rSQP
algorithm.  See the option \texttt{quasi\-\_range\-\_space\-\_matrix}
in Section \ref{moocho:sec:solver_options}.

Another important decision is how to compute the reduced Hessian $Z^T
W Z$.  For many NLPs, second derivative information is not available
to compute the Hessian of the Lagrangian $W$ directly.  In these
cases, first derivative information can be used to approximate $B
\approx Z^T W Z$ using quasi-Newton methods (e.g.~BFGS)
\cite{ref:nocedal_overton_1985}.  When $(n-r)$ is small, $B$ is small
and cheap to update.  Under the proper conditions the resulting
quasi-Newton, rSQP algorithm has a superlinear rate of local
convergence (even using $w$ = 0 in (\ref{moocho:eqn:rsqp:obj}))
\cite{ref:biegler_et_al_1995}.  Even when $(n-r)$ is large,
limited-memory quasi-Newton methods can still be used, but the price
one pays is in only being able to achieve a linear rate of convergence
(with a small rate constant hopefully).  For some classes of NLPs,
good approximations of the Hessian $W$ are available and may have
specialized properties (i.e.~structure) that makes computing the exact
reduced Hessian $B = Z^T W Z$ computationally feasible (i.e.~see NMPC
in \cite{RABartlett_2001}).  See the options
\texttt{exact\_reduced\_hessian} and \texttt{quasi\_newton} in Section
\ref{moocho:sec:solver_options}.  Other options include solving for
system with the exact reduced Hessian $B = Z^T W Z$ iteratively which
only requires matrix-vector products with $W$ which can be computed
efficiently using automatic differentiation (for instance) in some
cases \cite{ref:adolc_1996}.

In addition to variations that affect the convergence behavior of the
rSQP algorithm, such as null-space decompositions, approximations used
for the reduced Hessian and many different types of merit functions
and globalization methods, there are also many different
implementation options.  For example, linear systems such as
(\ref{moocho:eqn:range_space_step}) can be solved using direct or
iterative solvers and the reduced QP subproblem in
(\ref{moocho:eqn:rsqp:obj})--(\ref{moocho:eqn:rsqp:inequ}) can be
solved using a variety of methods (active set vs. interior point) and
software \cite{ref:schmid_qpkwik_1994}.

%
\subsection{General Inequalities, Slack Variables and Basis Permutations}
\label{moocho:sec:nlp_with_slacks}
%

Up to this point, only simple variable bounds in
(\ref{moocho:eqn:nlp:bnds}) have been considered and the SQP and rSQP
algorithms have been presented in this context.  However, the actual
underlying NLP may include general inequalities and take the form

{\bsinglespace
\begin{eqnarray}
\mbox{min}  &  & \breve{f}(\breve{x})                                     \label{moocho:eqn:nlporig:obj} \\
\mbox{s.t.} &  & \breve{c}(\breve{x}) = 0                                 \label{moocho:eqn:nlporig:equ} \\
            &  & \breve{h}_L \leq \breve{h}(\breve{x}) \leq \breve{h}_U   \label{moocho:eqn:nlporig:inequ} \\
            &  & \breve{x}_L \leq \breve{x}            \leq \breve{x}_U   \label{moocho:eqn:nlporig:bnds}
\end{eqnarray}
\begin{tabbing}
\hspace{4ex}where:\hspace{5ex}\= \\
\>	$\breve{x}, \breve{x}_L, \breve{x}_U \:\in\:\breve{\mathcal{X}}$ \\
\>	$\breve{f}(x) : \:\breve{\mathcal{X}} \rightarrow \RE$ \\
\>	$\breve{c}(x) : \:\breve{\mathcal{X}} \rightarrow \breve{\mathcal{C}}$ \\
\>	$\breve{h}(x) : \:\breve{\mathcal{X}} \rightarrow \breve{\mathcal{H}}$ \\
\>	$\breve{h}_L, \breve{h}_L \:\in\:\breve{\mathcal{H}}$ \\
\>	$\breve{\mathcal{X}} \:\in\:\RE\:^{\breve{n}}$ \\
\>	$\breve{\mathcal{C}} \:\in\:\RE\:^{\breve{m}}$ \\
\>	$\breve{\mathcal{H}} \:\in\:\RE\:^{\breve{m}_I}$.
\end{tabbing}
\esinglespace}

NLPs with general inequalities are converted into the standard form by
the addition of slack variables $\breve{s}$ (see
(\ref{moocho:eqn:nlpmap:c})).  After the addition of the slack
variables, the concatenated variables and constraints are then
permuted (using permutation matrices $Q_x$ and $Q_c$) according to the
current basis selection into the ordering in
(\ref{moocho:eqn:nlp:obj})--(\ref{moocho:eqn:nlp:bnds}).  The exact
mapping from
(\ref{moocho:eqn:nlporig:obj})--(\ref{moocho:eqn:nlporig:bnds}) to
(\ref{moocho:eqn:nlp:obj})--(\ref{moocho:eqn:nlp:bnds}) is given below

{\bsinglespace
\begin{eqnarray}
x & = & Q_x {\bmat{c} \breve{x} \\ \breve{s} \emat} \label{moocho:eqn:nlpmap:x} \\
x_L & = & Q_x {\bmat{c} \breve{x}^L \\ \breve{h}^L \emat} \label{moocho:eqn:nlpmap:xl} \\
x_U & = & Q_x {\bmat{c} \breve{x}_u \\ \breve{h}_u \emat} \label{moocho:eqn:nlpmap:xu} \\
c(x) & = & Q_c {\bmat{c} \breve{c}(\breve{x}) \\ \breve{h}(\breve{x}) - \breve{s} \emat}
	\label{moocho:eqn:nlpmap:c}
\end{eqnarray}
\esinglespace}

Here we consider the implications of the above transformation in the context of
rSQP algorithms.

Note if $Q_x = I$ and $Q_c = I$ that the matrix $\nabla c$ takes the form:

\begin{equation}
\nabla c = {\bmat{cc} \nabla \breve{c} & \nabla \breve{h} \\ & -I \emat}
	\label{moocho:eqn:Gc_orig}
\end{equation}

One question to ask is how the Lagrange multipliers for
the original constraints can be extracted from the optimal solution
$(x,\lambda,\nu)$ that satisfies the optimality conditions
in (\ref{moocho:eqn:kkt:lin_dep_grads})--(\ref{moocho:eqn:kkt:HL_psd})?
First, consider the linear dependence of gradients optimality condition
for the NLP formulation in (\ref{moocho:eqn:nlporig:obj})--(\ref{moocho:eqn:nlporig:bnds})

\begin{equation}
\nabla_{\breve{x}} \breve{L}(\breve{x}^*,\breve{\lambda}^*,\breve{\lambda_I}^*,\breve{\nu}^*)
= \nabla \breve{f}(\breve{x}^*) + \nabla \breve{c}(\breve{x}^*) \breve{\lambda}^*
+ \nabla \breve{h}(\breve{x}^*) \breve{\lambda_I}^* + \breve{\nu}^* = 0.
\label{moocho:eqn:kktorig:lin_dep_grads}
\end{equation}

To see how the Lagrange multiples $\lambda^*$ and $\nu^*$ can be used to compute
$\breve{\lambda}^*$, $\breve{\lambda_I}^*$ and $\breve{\nu}^*$ one simply has
to substitute (\ref{moocho:eqn:nlpmap:x}) and (\ref{moocho:eqn:nlpmap:c})
with $Q_x = I$ and $Q_c = I$ into (\ref{moocho:eqn:kkt:lin_dep_grads})
and expand as follows

\begin{eqnarray}
\nabla_{x} L(x,\lambda,\nu)
 & = & \nabla f + \nabla c \lambda + \nu \nonumber \\
 & = & {\bmat{c} \nabla \breve{f} \\ 0 \emat}
     + {\bmat{cc} \nabla \breve{c} & \nabla \breve{h} \\ & -I \emat}
       {\bmat{c} \lambda_{\breve{c}} \\ \lambda_{\breve{h}} \emat}
     + {\bmat{c} \nu_{\breve{x}} \\ \nu_{\breve{s}} \emat}
	\nonumber \\
 & = & {\bmat{c}
	\nabla \breve{f} + \nabla \breve{c} \lambda_{\breve{c}} + \nabla \breve{h} \lambda_{\breve{h}} + \nu_{\breve{x}} \\
	-\lambda_{\breve{h}} + \nu_{\breve{s}}
	\emat}.
	\label{moocho:eqn:kkt:compare_lagr}
\end{eqnarray}

By comparing (\ref{moocho:eqn:kktorig:lin_dep_grads}) and (\ref{moocho:eqn:kkt:compare_lagr})
it is clear that the mapping is
$\breve{\lambda} = \lambda_{\breve{c}}$, $\breve{\lambda}_I = \lambda_{\breve{h}} = \nu_{\breve{s}}$
and $\breve{\nu} = \nu_{\breve{x}}$.  For arbitrary $Q_x$ and $Q_c$ it is also easy to
perform the mapping of the solution.  What is interesting about
(\ref{moocho:eqn:kkt:compare_lagr}) is that it says that for general inequalities
$\breve{h}_j(\breve{x})$ that are not active at the solution (i.e.~$(\nu_{\breve{s}})_{(j)} = 0$), the
Lagrange multiplier for the converted equality constraint $(\lambda_{\breve{h}})_{(j)}$ will be zero.
This means that these converted inequalities can be eliminated from the problem and not impact the
solution (which is what we would have expected).  Zero multiplier values means that constraints
will not impact the optimality conditions or the Hessian of the Lagrangian.

The basis selection shown in (\ref{moocho:eqn:lin_indep_constr}) and
(\ref{rsqp:eqn:x_D_I}) is determined by the permutation matrices $Q_x$ and $Q_c$
and these permutation matrices can be partitioned as follows:

\begin{eqnarray}
Q_{x} & = & {\bmat{c} Q_{xD} \\ Q_{xI} \emat} \label{moocho:eqn:Qx} \\
Q_{c} & = & {\bmat{c} Q_{cD} \\ Q_{cU} \emat} \label{moocho:eqn:Qc}.
\end{eqnarray}

A valid basis selection can always be determined by simply including all of the slacks
$\breve{s}$ in the full basis and then finding a sub-basis for $\nabla \breve{c}$.  To show
how this can be done, suppose that $\nabla \breve{c}$ is full rank and
the permutation matrix $(\breve{Q}^x)^T = {\bmat{cc} (\breve{Q}_{xD})^T
& (\breve{Q}_{xI})^T \emat}$ selects a basis
$\breve{C} = (\nabla \breve{c})^T (\breve{Q}_{xD})^T$.
Then the following basis selection for the transformed NLP (with $Q_c = I$)
could always be used
regardless of the properties or implementation of $\nabla \breve{h}$

\begin{eqnarray}
Q_x & = & {\bmat{ccc}
 \breve{Q}_{xD}  &                &     \\
                 &                & I   \\
                 & \breve{Q}_{xI} &
\emat} \\
C & = & {\bmat{cc} ( \breve{Q}_{xD} \nabla \breve{c} )^T \\ ( \breve{Q}_{xD} \nabla \breve{h} )^T & -I \emat}
	\label{moocho:eqn:C_with_slacks} \\
N & = & {\bmat{c} ( \breve{Q}_{xI} \nabla \breve{c} )^T \\ ( \breve{Q}_{xI} \nabla \breve{h} )^T \emat}.
\end{eqnarray}

Notice that basis matrix in (\ref{moocho:eqn:C_with_slacks}) is lower block
triangular with non-singular blocks on the diagonal.  It is therefore straightforward to solve
for linear systems with this basis matrix.  In fact, the direct sensitivity matrix
$D = C^{-1} N$ takes the form

\begin{equation}
D = - {\bmat{cc}
	( \breve{Q}_{xD} \nabla \breve{c} )^{-T} ( \breve{Q}_{xI} \nabla \breve{c} )^{T} \\
    ( \breve{Q}_{xD} \nabla \breve{h} )^T ( \breve{Q}_{xD} \nabla \breve{c} )^{-T} ( \breve{Q}_{xI} \nabla \breve{c} )^{T}
      & -( \breve{Q}_{xI} \nabla \breve{h} )^T
\emat}.
\label{moocho:eqn:D_with_slacks}
\end{equation}

The structure of (\ref{moocho:eqn:D_with_slacks}) is significant in
the context of active-set QP solvers that solve the reduced QP
subproblem in
(\ref{moocho:eqn:rsqp:obj})--(\ref{moocho:eqn:rsqp:inequ}) using a
variable-reduction null-space decomposition.  A row of $D$
corresponding to a general inequality constraint only has to be
computed if the slack for the constraint is at a bound.  Also note
that the above transformation does not increase the number of degrees
of freedom of the NLP since $n-m = \breve{n}-\breve{m}$.  All of this
means that adding general inequalities to a NLP imparts little extra
cost for the rSQP algorithm as long as these constraints are not
active.

For reasons of stability and algorithm efficiency, it may be desirable
to keep at least some of the slack variables out of the basis and this
can be accommodated also but is more complex to describe.

Most of the steps in an SQP algorithm do not need to know that there
are general inequalities in the underlying NLP formulation but some
steps do (i.e.~globalization methods and basis selection).  Therefore,
those steps in an SQP algorithm that need access to this information
are allowed to access the underlying NLP in a limited manner (see the
Doxygen documentation for the class
\texttt{NLP\-Interface\-Pack\-::\textit{NLP}}).

%
\section{Overview of Software Architecture, Solvers, and Examples}
%

%
\section{Defining Optimization Problems}
%

%
\subsection{Defining general serial NLPs with explicit derivative entries}
%

%
\subsection{Defining simulation-constrained parallel NLPs through Thyra}
%

%
\section{Solving Optimization Problems with MOOCHO}
%

%
\subsection{Algorithm configurations for MOOCHO}
%

%
\subsection{Running MOOCHO algorithms}
%

%
\subsection{Summary}
%

% ---------------------------------------------------------------------- %
% References
%
\clearpage
\bibliographystyle{plain}
\bibliography{references}
\addcontentsline{toc}{section}{References}

% ---------------------------------------------------------------------- %
% Appendices should be stand-alone for SAND reports. If there is only
% one appendix, put \setcounter{secnumdepth}{0} after \appendix
%
\appendix
%\input{apdx_ThyraOperatorVectorClassDecl}

%\begin{SANDdistribution}
%\end{SANDdistribution}

\end{document}
