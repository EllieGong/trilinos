In the field of scientific computing there are many specialized programs designed for specific applications 
in areas such as biology, chemistry, and physics. These applications are often very powerful and extraordinarily 
useful in their respective domains. However, some suffer from a common problem: a non-intuitive, poorly-designed user interface. 
The purpose of Optika is to address this problem and provide a simple, viable solution. Using
only a list of parameters passed to it, Optika can dynamically generate a GUI. This allows the user to specify 
parameters' values in a fashion that is much more intuitive than the traditional ``input decks'' used by some 
parameterized scientific applications. By leveraging the power of Optika, these scientific applications 
will become more accessible and thus allow their designers to reach a much wider audience while requiring 
minimal extra development effort.
