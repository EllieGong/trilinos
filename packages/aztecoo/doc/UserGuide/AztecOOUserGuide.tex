 \documentclass[12pt,relax]{AztecOOUserGuide}
 \usepackage{fancyvrb}
\SANDsubtitle{}
\title{\AztecOOTM{} User Guide\footnote{For \AztecOOTM{} Version 3.6 in \TrilinosTM{} Release
8.0}}

\author{Michael A. Heroux}

    % There is a "Printed" date on the title page of a SAND report, so
    % the generic \date should generally be empty.
    \date{} % Remove ``\today'' in final version


\SANDnum{SAND2004-3796} \SANDprintDate{Updated August 2007}
\SANDauthor{Michael A. Heroux \\
    Computational Mathematics and Algorithms Department \\
     Sandia National Laboratories \\
    P.~O.~Box 5800 \\
    Albuquerque, NM 87185-1110}


\SANDreleaseType{Unlimited Release}


% New commands
\newcommand{\Az}  {{\bf Aztec}}
\newcommand\R     {{\rm \bf R}}
\newcommand\I     {{\rm \bf I}}
\newcommand\C     {{\rm \bf C}}
\newcommand{\sign}{{\it sign}}
\newcommand{\diag}{{\it diag}}
%
% define boxes for describing variables, etc
%

\def\optionbox#1#2{\noindent$\hphantom{hix}${\parbox[t]{2.10in}{\it #1}}{\parbox[t]{3.9in}{#2}} \\[1.1em]}

\def\choicebox#1#2{\noindent$\hphantom{hixthere}$\parbox[t]{2.10in}{\sf #1}\parbox[t]{3.5in}{#2}\\[0.8em]}

\def\structbox#1#2{\noindent$\hphantom{hix}${\parbox[t]{2.10in}{\it #1}}{\parbox[t]{3.9in}{#2}} \\[.02cm]}

\def\in{\hskip .2in \=}
\def\hsp{\hskip .4in \=}
\def\sp{\hskip .18in \=}
\def\sh{\hskip .18in }
\def\bb{\hskip .034in }
\def\lil{\hskip .1in }


\def\protobox#1{\vspace{2em}{\flushleft{\bf Prototype}
\hrulefill}\flushleft{\fbox{\parbox[t]{6in}{\vspace{1em}{\sf #1}\vspace{1em}}}}}

\newcommand{\comm}{Epetra\_Comm}
\newcommand{\serialcomm}{Epetra\_SerialComm}
\newcommand{\mpicomm}{Epetra\_MpiComm}
\newcommand{\mpismpcomm}{Epetra\_MpiSmpComm}
\newcommand{\lbcomm}{Epetra\_LbComm}
\newcommand{\map}{Epetra\_Map}
\newcommand{\blockmap}{Epetra\_BlockMap}
\newcommand{\serialdensevector}{Epetra\_SerialDenseVector}
\newcommand{\intserialdensevector}{Epetra\_IntSerialDenseVector}
\renewcommand{\vector}{Epetra\_Vector}
\newcommand{\multivector}{Epetra\_MultiVector}
\newcommand{\crsgraph}{Epetra\_CrsGraph}
\newcommand{\crsmatrix}{Epetra\_CrsMatrix}
\newcommand{\vbrmatrix}{Epetra\_VbrMatrix}
\newcommand{\rowmatrix}{Epetra\_RowMatrix}
\newcommand{\distobject}{Epetra\_DistObject}
\newcommand{\linearproblem}{Epetra\_LinearProblem}
\newcommand{\operator}{Epetra\_Operator}
\newcommand{\fecrsmatrix}{Epetra\_FE\-Crs\-Matrix}
\newcommand{\fevbrmatrix}{Epetra\_FE\-Vbr\-Matrix}
\newcommand{\msrmatrix}{Epetra\_MsrMatrix}
\newcommand{\statustest}{AztecOO\_StatusTest}
\newcommand{\oooperator}{AztecOO\_Operator}


\begin{document}
\maketitle

\begin{abstract}

The \TrilinosTM{} Project is an effort to facilitate the design, development,
integration and ongoing support of mathematical software libraries.
\AztecOOTM{} is a package within Trilinos that enables the use of the Aztec solver
library~\cite{Aztec2.1} with \EpetraTM{}~\cite{Epetra-home-page} objects.
AztecOO provides access to Aztec preconditioners and solvers by
implementing the Aztec ``matrix-free'' interface using Epetra.
While Aztec is written in C and procedure-oriented,
AztecOO is written in C++ and is object-oriented.

In addition to providing access to Aztec capabilities, AztecOO also
provides some signficant new functionality.  In particular it provides
an extensible status testing capability that allows expression of
sophisticated stopping criteria as is needed in production use of
iterative solvers.  AztecOO also provides mechanisms for using
Ifpack~\cite{Ifpack-home-page}, ML~\cite{ML-home-page} and AztecOO
itself as preconditioners.
\end{abstract}


\section*{Acknowledgement}
The authors would like to acknowledge the support of the ASCI and LDRD
programs that funded development of AztecOO and the authors of Aztec
2.1 upon which AztecOO is built: Ray Tuminaro, Mike Heroux, Scott
Hutchinson and John Shadid.

\clearpage
\tableofcontents
\listoffigures
%\listoftables

\clearpage

\section{Introduction}
\label{Section:Introduction}

AztecOO is a collection of C++ classes that support the construction
and use of objects for solving linear systems of equations of the form
\begin{equation}
\label{e:axb}
Ax = b
\end{equation}
via
preconditioned Krylov methods, as provided in Aztec.

This user guide is intended to introduce a new user to the basic features of AztecOO.
%(i) introduce a new user to the basic features of AztecOO,
%(ii) discuss the details of AztecOO constructors, and
%(iii) illustrate the use of AztecOO's advanced uses.
This document is not intended as a reference
manual.  Detailed descriptions of AztecOO classes and methods can be
found online at the Trilinos Project home page~\cite{Trilinos-home-page}.


\subsection{Overview of Major AztecOO Classes and Features}

AztecOO contains a variety of classes to support the solution of
linear systems of equations of the form $Ax=b$ using preconditioned
iterative methods.  AztecOO also fully contains Aztec, so any application
that is using Aztec can use the AztecOO library in place of Aztec.

The primary AztecOO class is of the same name, AztecOO.  An AztecOO
class instance acts as a manager of Aztec, accepting user data as
Epetra objects.  If an AztecOO object is instantiated using Epetra
objects, all of Aztec's preconditioners and Krylov methods can be
applied to the Epetra-defined problem.  However, AztecOO provides a
variety of mechanisms to override default Aztec capabilities.  Users
can construct and use preconditioners from Ifpack or ML, or can use
another instance of the AztecOO class as a preconditioner.  Users can
also override the default convergence tests in Aztec and use any
combination of available status tests in \statustest{} classes,
or define their own.

\subsection{A Special Note for Aztec Users}

AztecOO completely contains the full functionality and interface of
Aztec 2.1~\cite{Aztec2.1}.  Therefore, any code that calls Aztec 2.1 functions
can use AztecOO include files and library in place of Aztec 2.1.   Any file
that includes Aztec header files must be recompiled and the executable file must
be re-linked with the AztecOO library, and the Aztec library must not be linked.
In addition, the Epetra~\cite{Epetra-home-page},
LAPACK~\cite{lapack} and BLAS~\cite{BLAS1, BLAS2, BLAS3}
libraries must also be available to the linker.
If \InlineCommand{--enable-aztecoo-azlu} is
provided to the \InlineCommand{configure} command,
the Y12M~\cite{y12m} library must also be provided.
Finally, it is likely that you
will need the linker to be aware of C++ libraries.  Often the C++ compiler itself
is the best program to use for linking, since it is aware of what system libraries
are needed for linking.

Trilinos (and AztecOO as part of Trilinos) use a much different build
process than Aztec 2.1.  Trilinos uses
Autoconf~\cite{Autoconf} and Automake~\cite{Automake} to support a configure/make
process.  If you are familar
with configure/make procedures, you should find Trilinos  fairly easy install.
A Trilinos Installation Guide is available from
the Trilinos website~\cite{Trilinos-home-page}.  If you are familiar with the ad hoc
build process of Aztec 2.1 and are unfamiliar with configure/make, you may find
building Trilinos and AztecOO challenging.  We encourage you to carefully read the
Trilinos Installation Guide, since it will help you with understanding configure/make
procedures in general, and the use of the processes in Trilinos specifically.

Unlike Aztec, AztecOO does not provide its own copies of BLAS, LAPACK or Y12M libraries.
This change is part of a general policy in Trilinos to provide interfaces to third-party
libraries but not code.  This change is generally considered good practice from a software
engineering perspective for a number of reasons, but can be a hindrance to Aztec users
making a transition to AztecOO.  Optimized BLAS libraries are available for most computer
systems.  We recommend you obtain one of these libraries from the Internet, if it is not
already installed on your computer system.  LAPACK is also available in an optimized form
for some systems, but this is less critical.  LAPACK can also be obtained from the Internet,
or may already be installed on your system.  Y12M is an old sparse direct solver also available
from the Internet.  It is used by Aztec's domain decomposition preconditioners as a local
subdomain solver.  Most Aztec users do not require it.  By default, it is not required for
building the AztecOO library.  You may enable it by passing the argument
\InlineCommand{--enable-aztecoo-azlu} to the configure command.

\subsection{Use of Epetra}

AztecOO relies on Epetra for both concrete and abstract classes that
describe matrix, vector and linear operator objects.
Although  concrete classes are needed to construct matrices, AztecOO
itself uses these matrices via two Epetra abstract classes.  By using
abstract interfaces, we can support any of the predefined classes that
implement the abstract interfaces and allow users to define
new implementations.  This allows AztecOO
to be easily extended.

\subsubsection{Epetra Abstract Classes}

The two primary abstract Epetra classes used by AztecOO are:
\begin{enumerate}
\item {\bf \rowmatrix{}: } Supports the use of any class that is
conceptually a linear operator with access to coefficient data.
Although row-orientation is assumed, this class works equally well
with column oriented data, since transpose operations are supported.
This class provides an interface to access matrix data.  In addition,
it extends the \operator{} interface, so any class that implements
\rowmatrix{} also implmenents \operator{}, described next.  There are
four primary classes in Epetra that implement \rowmatrix{}, namely
the \crsmatrix{}, \vbrmatrix{}, \fecrsmatrix{} and \fevbrmatrix{}
classes.  AztecOO also provides an implementation of \crsmatrix{}
called \msrmatrix{}.  \msrmatrix{} allows users whose application is
already forming Aztec DMSR matrix structs to encapsulate the DMSR
matrix in a class that implements \rowmatrix.  The encapsulation does
not copy the data in the DMSR matrix struct.  This feature is
important for people making a transition from Aztec to AztecOO.
\item {\bf \operator{}: } Supports the use of any class that is
conceptually a linear operator.  There are only a handful of methods
in this class, the most important of which are the Apply() and
ApplyInverse() (ApplyInverse() can be defined as nonexistent).
A large number of Epetra classes implement the
\operator{} interface, including the \crsmatrix{}, \vbrmatrix{},
\fecrsmatrix{} and \fevbrmatrix{} since these classes implement
\rowmatrix{} and \rowmatrix{} extends \operator{}.  In addition, ML
and IFPACK both implement the \operator{} interface, so they can be
used as preconditioners for AztecOO.  A class called \oooperator{}
also implements the \operator{} interface using an existing AztecOO
class instance.  This allows AztecOO to be used as a preconditioner
for itself.

\end{enumerate}

\subsubsection{Epetra Concrete Classes}

Given the above abstract classes, we need some concrete classes in
order to construct explicit objects.  Specifically,
the following concrete Epetra classes are used:
\begin{enumerate}
\item {\bf \vector{}: } Supports construction and use of distributed
vectors of double-precision numbers.  Once constructed, \vector{}
objects can be used in multiple ways.  Common operations such as
norms, dot products and vector updates are supported by methods in
this class.  Additional functionality is available via several
extension techniques discussed later.
\item{\bf \multivector{}: } An \multivector{} object is a collection
of \vector{} objects (although \vector{} is actually implemented as a
specialization of \multivector{}).  Specifically, an \multivector{}
object is a collection of vectors with the same size and
distribution.  This kind of object is useful for block algorithms and
applications that manage multiple vectors simultaneously.  \vector{}
and \multivector{} objects are understood by all Trilinos packages.
\item{\bf \crsgraph{}:} Supports the construction and use of adjacency
graphs.  These graphs are used to describe the pattern of Epetra
sparse matrix classes and provide pattern-based information to
load-balancing interfaces.  The graphs are also used to implement
overlapping subdomain algorithms and a variety of other parallel
algorithms.
\item{\bf \crsmatrix{}:} Supports construction and use of distributed
sparse matrix objects.  Once constructed, an \crsmatrix{} object can
be used with any Trilinos solvers or preconditioners.  This class also
supports common matrix and matrix-vector operations such as matrix
scaling, matrix norms and matrix-vector multiplication.
\item{\bf \fecrsmatrix{}:} This class inherits from \crsmatrix,
providing an interface to construct the matrix from element stiffness
matrices.  Once constructed, this matrix can be used as an \crsmatrix{}.
\item{\bf \vbrmatrix{}:} Although less frequently used than the
\crsmatrix{} class, this class supports the construction of sparse
matrices whose entries are dense matrices.  This type of matrix is
frequently found in applications where multiple degrees of freedom are
tracked at each mesh point.  When properly used, the \vbrmatrix{}
class can offer substantial machine performance and algorithmic
robustness improvements.
\item{\bf \fevbrmatrix{}:} This class inherits from \vbrmatrix,
providing an interface to construct the block matrix from element stiffness
matrices.  Once constructed, this matrix can be used as an \vbrmatrix{}.
\item{\bf \msrmatrix{}:} Although not part of Epetra, we list this
class here because it is a viable option for existing Aztec users.
An \msrmatrix{} object is constructed by passing in an existing
AZ\_MATRIX struct defining a DMSR matrix as described in the Aztec 2.1
User's Guide~\cite{Aztec2.1}. Given the matrix data in this form, the
\msrmatrix{} class implements the \rowmatrix{} interface using the
DMSR matrix with little extra storage required.  This class is useful
if an existing Aztec user has already constructed a DMSR matrix.
\item{\bf \linearproblem{}:} An \linearproblem{} object is an
aggregate object that encapsulates the problem $Ax=b$.  It contains a
pointer to an \rowmatrix{} or \operator{} representing $A$, and an
\multivector{} for $x$ and another for $b$.  (NOTE:  Although the
interface supports multiple right-hand-sides and solution vectors,
AztecOO presently supports the solution of only one right-hand-side.)
Although use of the \linearproblem{} is not required for passing $A$,
$x$ and $b$ to AztecOO, it is highly recommended.  Use of
\linearproblem{} guarantees that the matrix, left-hand-side and
right-hand-side are compatible.  Also, the \linearproblem{} class
provides methods for scaling the linear problem using any of a variety
of diagonal scaling methods.
\end{enumerate}




\section{A First Example}
\label{Section:FirstExample}
Before proceeding with additional descriptive information, we
introduce a simple example in this section in order to explicitly
illustrate a straight-forward use of AztecOO.  This specific example
constructs a tridiagonal matrix and a random RHS vector.  Then it
constructs an AztecOO object, sets a few parameters for the solver and
then solves the problem.

The example code is listed in Figure~\ref{Figure:FirstExample}.
In the remainder of this section we proceed with a
description of the code.
\begin{figure}
\begin{center}
\begin{Verbatim}[frame=single,numbers=left,fontfamily=courier,fontsize=\tiny]
// @HEADER
// ***********************************************************************
// 
//        AztecOO: An Object-Oriented Aztec Linear Solver Package 
//                 Copyright (2002) Sandia Corporation
// 
// Under terms of Contract DE-AC04-94AL85000, there is a non-exclusive
// license for use of this work by or on behalf of the U.S. Government.
// 
// This library is free software; you can redistribute it and/or modify
// it under the terms of the GNU Lesser General Public License as
// published by the Free Software Foundation; either version 2.1 of the
// License, or (at your option) any later version.
//  
// This library is distributed in the hope that it will be useful, but
// WITHOUT ANY WARRANTY; without even the implied warranty of
// MERCHANTABILITY or FITNESS FOR A PARTICULAR PURPOSE.  See the GNU
// Lesser General Public License for more details.
//  
// You should have received a copy of the GNU Lesser General Public
// License along with this library; if not, write to the Free Software
// Foundation, Inc., 59 Temple Place, Suite 330, Boston, MA 02111-1307
// USA
// Questions? Contact Michael A. Heroux (maherou@sandia.gov) 
// 
// ***********************************************************************
// @HEADER

#include "AztecOO_config.h"
#include "AztecOO_Version.h"
#ifdef HAVE_MPI
#include "mpi.h"
#include "Epetra_MpiComm.h"
#else
#include "Epetra_SerialComm.h"
#endif
#include "Epetra_Map.h"
#include "Epetra_Vector.h"
#include "Epetra_CrsMatrix.h"
#include "AztecOO.h"

int main(int argc, char *argv[]) {

#ifdef HAVE_MPI
  MPI_Init(&argc,&argv);
  Epetra_MpiComm Comm( MPI_COMM_WORLD );
#else
  Epetra_SerialComm Comm;
#endif
  if (Comm.MyPID()==0)
    cout << AztecOO_Version() << endl << endl;

  cout << Comm <<endl;

  int NumMyElements = 100;
  // Construct a Map that puts same number of equations on each processor
  Epetra_Map Map(-1, NumMyElements, 0, Comm);
  int NumGlobalElements = Map.NumGlobalElements();

  // Create a Epetra_Matrix
  Epetra_CrsMatrix A(Copy, Map, 3);
  
  // Add  rows one-at-a-time
  double negOne = -1.0;
  double posTwo = 2.0;
  for (int i=0; i<NumMyElements; i++) {
    int GlobalRow = A.GRID(i); int RowLess1 = GlobalRow - 1; int RowPlus1 = GlobalRow + 1;

    if (RowLess1!=-1) A.InsertGlobalValues(GlobalRow, 1, &negOne, &RowLess1);
    if (RowPlus1!=NumGlobalElements) A.InsertGlobalValues(GlobalRow, 1, &negOne, &RowPlus1);
    A.InsertGlobalValues(GlobalRow, 1, &posTwo, &GlobalRow);
  }
  
  // Finish up
  A.FillComplete();

  // Create x and b vectors
  Epetra_Vector x(Map);
  Epetra_Vector b(Map);
  b.Random(); // Fill b with random values

  // Create Linear Problem
  Epetra_LinearProblem problem(&A, &x, &b);
  // Create AztecOO instance
  AztecOO solver(problem);

  solver.SetAztecOption(AZ_precond, AZ_Jacobi);
  solver.Iterate(100, 1.0E-8);

  cout << "Solver performed " << solver.NumIters() << " iterations." << endl
       << "Norm of true residual = " << solver.TrueResidual() << endl;
#ifdef HAVE_MPI
  MPI_Finalize() ;
#endif
  return 0;
}

\end{Verbatim}

\end{center}
\caption{\label{Figure:FirstExample} Simple AztecOO/Epetra Example}
\end{figure}

\subsection{Explanation of Figure~\ref{Figure:FirstExample}}

\paragraph{Line 1}
Include the AztecOO\_config.h file.  This file contains macros
definitions that were defined during the configuration process.  In
particular, the macro HAVE\_MPI will be defined or undefined in this
file, depending on whether AztecOO was built in MPI mode or not.  We
will use HAVE\_MPI below to determine if our example code should be
compiled with MPI support or not.
\paragraph{Lines 2--7}
Include the appropriate implementation of the \comm{} class.  If
AztecOO was built in MPI mode, the macro ``HAVE\_MPI'' will be defined
and this example will be built with MPI support.  If not, then the
example will be built in serial mode.  Note that these lines of code,
lines 15--20 and lines 60--62 are the only difference between a serial
and distributed memory version of the example.
\paragraph{Lines 8--11}
Include the other necessary Epetra and AztecOO header files.
It is a good practice to explicitly include header files for all
classes you explicitly use, and only those header files.
\paragraph{Line 13}
Start of main program.
\paragraph{Lines 15--21}
Depending on whether or not AztecOO was built in MPI
mode\footnote{Whether or not Trilinos, and AztecOO as a Trilinos package, is built
in MPI mode is determined by how the Trilinos (or AztecOO) {\tt
configure} script in invoked.  If no MPI-related arguments are passed
to the configure script, then packages are built in serial mode only.
If one or more MPI options are invoked, then packages are built with
MPI support (in addition to serial support).}, MPI will be initialized
and an \mpicomm{} object will be constructed, or an \serialcomm{}
object will be constructed.  Please note that, in principle, the serial
version of this example would work, even if AztecOO were built in MPI
mode.  Serial mode is always available.  Line 21 prints the \comm{}
object to cout.
\paragraph{Line 23}
Define the local problem dimension.  NumMyElements will be used to
define an \map{} that has 100 elements on each processor.  In turn the
map will be used to construct vectors with 100 entries on each
processor and matrices with 100 rows on each processor.
\paragraph{Lines 24--25}
Constructs an \map{} object that has NumMyElements elements spread
across the parallel (or serial) machine.  The first argument indicates
that we are not specifying the global number of elements, but allowing
the \map{} constructor to compute it as the sum of
NumMyElements defined on each calling processor.  The second argument
is the number of elements assigned to the calling processors.  The
third argument (a
``0'') indicates that our global indices are zero-based.  Fortran
users would typically pass in a ``1'' here.  The fourth argument is the
Comm object we just built.
\paragraph{Line 26}
Once an \map{} object is constructed, we can query it for how many total
elements are in the map\footnote{For readers
who are not familiar with a single-program, multiple data (SPMD)
programming model, it may be useful to read a bit about it.  Typing
``SPMD tutorial'' into a web search engine should be a sufficient
starting point.}.  In this way, the remainder of our code can operate
independent of how data is distributed.
\paragraph{Line 29}
Instantiates (creates) an \crsmatrix{}.  The first argument tells the
constructor whether or not data passed in to this object should be
copied (user values and indices will be copied to internal storage) or
viewed (user values and indices will be pointed to by this object and
the user {\it must} guarantee the integrity of that data).  View mode
is available across many Epetra classes.  In general, this is a very
dangerous practice.  However, in certain very important situations, it
is essential to have this mode.  This is especially true when using
Epetra with Fortran, or when accepting matrix data from other parts of
application where it is too expensive to replicate the data storage.

The second argument is the \map{} object we just constructed.  The
third argument is an advisory value telling the constructor
approximately how many nonzero values will be defined for each row of
the matrix\footnote{getting this value wrong does not
affect the correctness of results, but may affect performance and
efficient use of memory}.  We are constructing a tridiagonal matrix,
so the value ``3'' is appropriate.

At this point the matrix is an empty ``bucket'' ready to receive
matrix values and indices.  Also, at this point, most of the methods
in the \crsmatrix{} cannot be called successfully for this object.
\paragraph{Lines 31--43}
These lines insert values and indices into the matrix we just
instantiated.  Our matrix is tridiagonal with a value of 2 at each
diagonal and -1 on the immediate off-diagonals.  We do not go into
detail about the methods called here.  The reader should look at the
Epetra User Guide~\cite{Epetra-User-Guide} or the online reference
material at the Trilinos home page~\cite{Trilinos-home-page}.
\paragraph{Lines 45--48}
Once the matrix is constructed, we create our vectors b and x using
the same map that determined the layout of the matrix rows.  We also
fill b with random values (line 48).
\paragraph{Lines 50--53}
Now that A, x and b are formed, we can define a linear problem
instance.  This object will in turn be used to define an AztecOO
instance.  Note that it is possible to construct AztecOO objects in
other ways, but we strongly recommend use of the constructor shown in
line 53.  Note that when the AztecOO object is constructed, the
parameter and option values listed in
Section~\ref{Section:OptionsAndParams} will be set to their default
values.  These defaults can be changed by calling the SetAztecOption(),
SetAztecParam() and SetParameters() methods (see section \ref{Section:OptionsAndParams}).
\paragraph{Lines 55--56}
Once the solver object is instantiated, we change the value of
AZ\_precond to AZ\_Jacobi.  Note that the key/value pairs passed in to
this method can be any valid pair as defined in
Section~\ref{Section:OptionsAndParams}.  Next we call the Iterate()
method, passing in the maximum number of iterations that can be
performed and a tolerance that should be used to test for
convergence.  Depending on the values of the Aztec parameters and
options, this method will attempt to solve the problem using the
prescribed preconditioner (if any) and the specified iterative
method.  It will also print intermediate results if the user has
requested them.  Upon exit from this method, the problem will
hopefully be solved and the solution will be in x which in turn is
part of the linear problem instance.  Also upon exit, a number of
methods can be called to determine the results of the iterations.
\paragraph{Lines 58--59}
Print results from calling the solver.
\paragraph{Lines 60--62}
If our code was compile in MPI mode, we need to call MPI\_Finalize()
for proper clean up.
\paragraph{Line 64} Program exit.

\section{Aztec Options and Parameters\label{Section:OptionsAndParams}}

Because AztecOO is partly a wrapper around Aztec, much of the
control and selection of solver options and parameters is done via three
method calls that set Aztec options and parameters, namely
SetAztecOption(), SetAztecParam() and SetParameters()
 (see \ref{parameterlist}).  Most of the
options and parameters are identical to those found in Aztec 2.1.
However, there are a few new options and parameters.  Below we
list all options and parameters, including the default value and
description of each.

\subsection{Aztec Options\label{optionI}}

The following list of key/value pairs can be used with the
SetAztecOption() method to change the behavior of the Iterate() method:
\vspace{2em}
{\flushleft{\bf Specifications} \hrulefill}
\nopagebreak \\[0.5em]
%
\optionbox{options[{\sf AZ\_solver}]}{Specifies solution
  algorithm. DEFAULT: \sf AZ\_gmres.}
\choicebox{AZ\_cg}{Conjugate gradient (Applicable to symmetric
positive definite matrices, sometimes usable with mildly
non-symmetric matrices).}

\choicebox{AZ\_cg\_condnum}{Conjugate gradient with condition number
estimation. (Similar to AZ\_cg. Additionally computes extreme
eigenvalue estimates using the generated Lanczos matrix).}

\choicebox{AZ\_gmres}{Restarted generalized minimal residual.}
\choicebox{AZ\_gmres\_condnum}{Restarted GMRES with condition number
estimation. (Similar to AZ\_gmres. Additionally computes extreme
eigenvalues using the generated Hessenberg matrix.)}
\choicebox{AZ\_cgs}{Conjugate gradient squared.}
\choicebox{AZ\_tfqmr}{Transpose-free quasi-minimal residual.}
\choicebox{AZ\_bicgstab}{Bi-conjugate gradient with
  stabilization.}
\choicebox{AZ\_lu}{Sparse direct solver (single processor only).
{\bf Note: This option is available only when --enable-aztecoo-azlu is
specified on the AztecOO configure script invocation command}}
%
%\optionbox{options[{\sf AZ\_scaling}]}{Specifies scaling algorithm.
%  The entire matrix is scaled (overwriting the old
%  matrix). Additionally, the right hand side, the initial guess and
%  the final computed solution are scaled if necessary. For
%  symmetric scaling, this transforms $ A x = b$ into
%  $ S A S y = S b $ as opposed to $ S A x = S b $ when symmetric
%  scaling is not used. NOTE: The residual within \Az{} is now
%  given by $ S (b - A x) $. Thus, residual printing and convergence
%  checking are effected by scaling.  DEFAULT: \sf
%  AZ\_none.}
%%
%\choicebox{AZ\_none}{No scaling.}
%\choicebox{AZ\_Jacobi}{Point Jacobi scaling.}
%\choicebox{AZ\_BJacobi}{Block Jacobi scaling where the block
%  size corresponds to the VBR blocks.  Point Jacobi scaling is
%  performed when using the MSR format.}
%\choicebox{AZ\_row\_sum}{Scale each row so the magnitude of its
%  elements sum to 1.}
%\choicebox{AZ\_sym\_diag}{Symmetric scaling so diagonal elements
%  are 1.}
%\choicebox{AZ\_sym\_row\_sum}{Symmetric scaling using the matrix
%  row sums.}
%
\optionbox{options[{\sf AZ\_precond}]}{Specifies preconditioner.
  DEFAULT: \sf AZ\_none.}
\choicebox{AZ\_none}{No preconditioning.}
\choicebox{AZ\_Jacobi}{$k$ step Jacobi (block Jacobi for DVBR matrices
  where each block corresponds to a VBR block). The number of
  Jacobi steps, $k$, is set via {\it options}[{\sf AZ\_poly\_ord}].}
\choicebox{AZ\_Neumann}{Neumann series polynomial
  where the polynomial order is set via
  {\it options}[{\sf AZ\_poly\_ord}].}
\choicebox{AZ\_ls}{Least-squares polynomial
  where the polynomial order is set via
  {\it options}[{\sf AZ\_poly\_ord}].}
\choicebox{AZ\_sym\_GS}{Non-overlapping domain decomposition
  (additive Schwarz)
  $k$ step symmetric Gauss-Siedel.
  In particular, a symmetric Gauss-Siedel domain decomposition
  procedure is used where each processor independently
  performs one step of
  symmetric Gauss-Siedel on its local matrix, followed by communication
  to update boundary values before the next local symmetric
  Gauss-Siedel step. The number of steps, $k$, is set via
  {\it options}[{\sf AZ\_poly\_ord}].}
\choicebox{AZ\_dom\_decomp}{Domain decomposition preconditioner
  (additive Schwarz). That is, each processor augments
  its submatrix according to {\it options}[{\sf AZ\_overlap}]
  and approximately ``solves'' the resulting subsystem
  using the solver specified by \\
  $\hphantom{using the solr}$
  {\it options}[{\sf AZ\_subdomain\_solve}].\\
  Note: {\it options}[{\sf AZ\_reorder}] determines whether
  matrix equations are reordered (RCM) before ``solving'' submatrix problem.}
\optionbox{options[{\sf\small AZ\_subdomain\_solve}]}{Specifies the solver
  to use on each subdomain when {\it options}[{\sf AZ\_precond}] is set
  to {\sf AZ\_dom\_decomp} DEFAULT: \sf AZ\_ilut.}
\choicebox{AZ\_lu}{Approximately solve processor's submatrix via
  a sparse LU factorization in conjunction with a drop tolerance
  {\it params}[{\sf AZ\_drop}]. The current sparse
  lu factorization is provided by the package Y12M~\cite{y12m}.
  {\bf Note: This option is available only when --enable-aztecoo-azlu is
specified on the AztecOO configure script invocation command}}
\choicebox{AZ\_ilut}{Similar to {\sf AZ\_lu} using
  Saad's {\sf ILUT} instead of LU \cite{ilut}. The drop
  tolerance is given by {\it params}[{\sf AZ\_drop}]
  while the fill-in is given by {\it params}[{\sf AZ\_ilut\_fill}]. }
\choicebox{AZ\_ilu}{Similar to {\sf AZ\_lu} using
  {\sf ilu(k)} instead of LU with k determined by
  {\it options}[{\sf AZ\_graph\_fill}]}
\choicebox{AZ\_rilu}{Similar to {\sf AZ\_ilu} using
  {\sf rilu(k,$\omega$)} instead of {\sf ilu(k)}
  with $\omega$ ($0 \ge \omega \ge 1$) given by {\it params}[{\sf AZ\_omega}]
  \cite{milu}.}
\choicebox{AZ\_bilu}{Similar to {\sf AZ\_ilu} using block
  {\sf ilu(k)} instead of {\sf ilu(k)} where each block corresponds
  to a VBR block.}
\choicebox{AZ\_icc}{Similar to {\sf AZ\_ilu} using
  {\sf icc(k)} instead of {\sf ilu(k)} \cite{icc}.}
%
\optionbox{options[{\sf AZ\_conv}]}{Determines the residual expression used
  in convergence checks and printing.  DEFAULT: {\sf AZ\_r0}. Note
that this feature is overridden if the user registers an
AztecOO\_StatusTest object with the AztecOO solver instance.
  The iterative solver terminates if the corresponding residual expression
  is less than {\it params}[{\sf AZ\_tol}]:}
\choicebox{AZ\_r0}{$\|r\|_2 / \|r^{(0)}\|_2 $}
\choicebox{AZ\_rhs}{$\|r\|_2 / \|b\|_2 $}
\choicebox{AZ\_Anorm}{$\|r\|_2 / \|A\|_{\infty} $}
\choicebox{AZ\_noscaled}{$\|r\|_2$}
\choicebox{AZ\_sol}{$\|r\|_{\infty}
  /(\|A\|_{\infty} * \|x\|_1 + \|b\|_{\infty}) $}
\choicebox{AZ\_weighted}{$\|r\|_{WRMS} $\\
  where $\| \cdot \|_{WRMS} = \sqrt{(1/n) \sum_{i=1}^n (r_i/w_i)^2}$,
  $n$ is the total number of unknowns, $w$ is a weight
  vector provided by the
  user  via {\it params}[{\sf AZ\_weights}] and
  $r^{(0)}$ is the initial residual. {\bf Note:  AZ\_weighted is not
available in AztecOO.}}
%
\optionbox{options[{\sf AZ\_diagnostics}]}{Similar to AZ\_output.
 Specifies diagnostic information that can be used to tune solver performance.
 In particular, condition estimates for preconditioners, performance ratings and preconditioner memory use.
  DEFAULT: {\sf AZ\_all}.}
\choicebox{AZ\_all}{Print all possible diagnostics.}
\choicebox{AZ\_none}{Print no diagnostics.  Note that if AZ\_output
is set to AZ\_none, diagnostics will not be printed either.}
%
\optionbox{options[{\sf AZ\_output}]}{Specifies information (residual
  expressions - see {\it options}[{\sf AZ\_conv}]) to be printed.
  DEFAULT: \sf 1.}
\choicebox{AZ\_all}{Print out the matrix and indexing vectors for
  each processor. Print out all intermediate residual expressions.}
\choicebox{AZ\_none}{No intermediate results are printed.}
\choicebox{AZ\_warnings}{Only Aztec warnings are printed.}
\choicebox{AZ\_last}{Print out only the final residual expression.}
\choicebox{AZ\_summary}{Print header containing problem description,
iterative method and preconditioner description.}
\choicebox{$>$0}{Print residual expression every {\it
    options[{\sf AZ\_output}]\/} iterations.}
%
\optionbox{options[{\sf AZ\_pre\_calc}]} {Indicates whether to use
  factorization information from previous calls to Iterate().
  DEFAULT: {\sf AZ\_calc} if the user is using the native Aztec
preconditioners selected via options[{\sf AZ\_precond}].  If the user has
registered an \operator{} object using the SetPrecOperator()
method, then this operator will be used as the preconditioner and it
is assumed that the preconditioner is already constructed.}
\choicebox{AZ\_calc}{Use no information from previous Iterate() calls.}
\choicebox{AZ\_recalc}{Use preprocessing information from a
  previous call but recalculate preconditioning factors. This is
  primarily intended for factorization software which performs a
  symbolic stage.}
\choicebox{AZ\_reuse}{Use preconditioner from a previous
  Iterate() call, do not recalculate preconditioning factors.
  Also, use scaling factors from previous call to scale the
  right hand side, initial guess and the final solution.}
%
%
\optionbox{options[{\sf AZ\_graph\_fill}]}{The level of graph fill-in (k)
  for incomplete factorizations: ilu(k), icc(k), bilu(k).
  DEFAULT: 0}
%
\optionbox{options[{\sf AZ\_max\_iter}]}{Maximum number of iterations. DEFAULT:
  500, unless an statustest{} object has been registered using the
SetStatusTest() method in which case this option is ignored.}
%
\optionbox{options[{\sf AZ\_poly\_ord}]}{The polynomial order when using
  polynomial preconditioning.  Also, the number of steps when using Jacobi or
  symmetric Gauss-Seidel preconditioning.  DEFAULT: 3 for polynomial
preconditioners, 1 for Jacobi and Gauss-Seidel preconditioners.}
%
\optionbox{options[{\sf AZ\_overlap}]}{Determines the submatrices factored with
  the domain decomposition algorithms (see {\it options}[{\sf AZ\_precond}]).
  DEFAULT: 0.}
%
%\choicebox{AZ\_none}{Factor the local submatrix defined on this processor
%  by discarding column entries that correspond to external elements.}
%
\choicebox{AZ\_diag}{Factor the local submatrix defined on this processor
  augmented by a diagonal (block diagonal for VBR format) matrix. This diagonal
  matrix corresponds to the diagonal entries of the matrix rows (found on other
  processors) associated with external elements.  This can be viewed as taking
  one Jacobi step to update the external elements and then performing domain
  decomposition with {\sf AZ\_none} on the residual equations.}
%
\choicebox{k}{Augment each processor's local submatrix with
  rows from other processors. The new rows are obtained in k
  steps (k $\ge$ 0). Specifically at each augmentation step,
  rows corresponding to external unknowns are obtained. These
  external unknowns are defined by nonzero columns in the
  current augmented matrix not containing a corresponding
  row on this processor. After the k steps, all columns
  associated with external
  unknowns are discarded to obtain a square matrix.
  The resulting procedure is an overlapped additive Schwarz
  procedure.}
%
\optionbox{options[{\sf AZ\_type\_overlap}]}{Determines how overlapping
    subdomain results are combined when different processors
    have computed different values for the same unknown.
    DEFAULT: \sf AZ\_standard.}
\choicebox{AZ\_standard}{The resulting value of an unknown is
    determined by the processor owning that unknown. Information
    from other processors about that unknown is discarded.}
\choicebox{AZ\_symmetric}{Add together the results obtained from different
    processors corresponding to the same unknown. This keeps the
    preconditioner symmetric if a symmetric technique was used on
    each subdomain.}
%
\optionbox{options[{\sf AZ\_kspace}]}{Krylov subspace size for
  restarted GMRES.\\
  DEFAULT: 30.}
%
\optionbox{options[{\sf AZ\_reorder}]}{Determines whether RCM reordering
  will be done in conjunction with domain decomposition incomplete
  factorizations. 1 indicates RCM reordering is used. 0 indicates that
  equations are not reordered.  DEFAULT:~1.}
%
\optionbox{options[{\sf AZ\_keep\_info}]}{Determines whether matrix
  factorization information will be kept after this solve (for example
  to solve the same system with another right hand side, see
  {\it options}[{\sf AZ\_pre\_calc}]).  1 indicates factorization
  information is kept.  0 indicates that factorization information is
  discarded.  DEFAULT: 0.}
%
\optionbox{options[{\sf AZ\_orthog}]}{GMRES orthogonalization scheme.\\
  DEFAULT: {\sf AZ\_classic}.}
\choicebox{AZ\_classic}{2 steps of classical Gram-Schmidt
orthogonalization.}

\choicebox{AZ\_modified}{A single step of Modified Gram-Schmidt
orthogonalization.}
%
\optionbox{options[{\sf AZ\_aux\_vec}]}{Determines $\tilde r$ (a required
  vector within some iterative methods). The convergence behavior varies
  slightly depending on how this is set.  DEFAULT: \sf AZ\_resid.}
\choicebox{AZ\_resid}{$\tilde r$ is set to the initial residual vector.}
\choicebox{AZ\_rand}{$\tilde r$ is set to random numbers between -1 and 1.
  NOTE: When using this option, the convergence depends on the number of
  processors (i.e. the iterates obtained with x processors differ from the
  iterates obtained with y processors if x $\ne$ y).}  $\hphantom{h}$
\subsection{\Az{} parameters\label{optionD}}

The double precision array {\it params\/} is set up by the AztecOO
solver instance and is of
length {\sf AZ\_PARAMS\_SIZE}. Because of this, we do not support
{\it options}[{\sf AZ\_conv}] = {\sf AZ\_weighted}).  This type of
functionality is still possible by defining an implementation of the
AztecOO\_StatusTest abstract class.

Below we list the key/value pairs that can be used with the
SetAztecParam() method: \vspace{2em}
{\flushleft{\bf Specifications} \hrulefill} \nopagebreak \\[0.5em]
%
\optionbox{params[{\sf AZ\_tol}]}{Specifies tolerance value used in
   conjunction with convergence tests. DEFAULT: $10^{-6}$.}
\optionbox{params[{\sf AZ\_drop}]}{Specifies drop tolerance used in
   conjunction with LU  or ILUT preconditioners (see description
   below for ILUT). \\ DEFAULT: 0.0.}
\optionbox{params[{\sf AZ\_ilut\_fill}]}{ ILUT uses two criteria for
   determining the number of nonzeros in the resulting approximate
   factorizations. For examples, setting {\it params}[{\sf AZ\_ilut\_fill}]
   $ = 1.3 $, requires that the ILUT factors contain no more than
   approximately 1.3 times the number of nonzeros of the original matrix.
   Additionally, ILUT drops all elements in the resulting factors that are
   less than {\it params}[{\sf AZ\_drop}]. Thus, when
   {\it params}[{\sf AZ\_drop}] is set to zero, nothing is dropped and the
   size of the matrix factors is governed only by {\it params}[{\sf AZ\_ilut\_fill}].
   However, positive values of {\it params}[{\sf AZ\_drop}] may result in
   matrix factors containing significantly fewer nonzeros. \cite{ilut} \\
   DEFAULT: 1.}
\optionbox{params[{\sf AZ\_omega}]}{Damping or relaxation parameter used
   for RILU. When {\it params}[{\sf AZ\_omega}] is set to zero, RILU
   corresponds to ILU(k). When it is set to one, RILU corresponds to
   MILU(k) where k is given by {\it options}[{\sf AZ\_graph\_fill}].
   \cite{milu}\\ DEFAULT: 1.}
\optionbox{params[{\sf AZ\_weights}]}{
   When {\it options}[{\sf AZ\_conv}] = AZ\_weighted, the {\it i\/}'th local
   component of the weight vector is stored in the location
   {\it params}[{\sf AZ\_weights}+i].}
\optionbox{params[{\sf AZ\_rthresh}]}{Parameter used to modify the
relative magnitude of the diagonal entries of the matrix that is
used to compute any of the incomplete factorization preconditioners.
When {\it params}[{\sf AZ\_rthresh}] is set to zero, no relative perturbation is performed.
See Section~\ref{Sect:DiagonalPerturbations} and the discussion of
{\it params}[{\sf AZ\_athresh}] for more information.\\
DEFAULT: 0.0}
\optionbox{params[{\sf AZ\_athresh}]}{Parameter used
to modify the absolute magnitude of the diagonal entries of the
matrix that is used to compute any of the incomplete factorization
preconditioners. When {\it params}[{\sf AZ\_athresh}] is set to
zero, no absolute perturbation is performed. See
Section~\ref{Sect:DiagonalPerturbations} and the discussion of
{\it params}[{\sf AZ\_rthresh}] for more information.\\
DEFAULT: 0.0}

\subsection{ParameterList Interface\label{parameterlist}}
The AztecOO method SetParameters() allows options and parameters to be set
using values held in a Teuchos::ParameterList container. The SetParameters()
method is available only if the option \InlineCommand{--enable-aztecoo-teuchos}
is provided to the \InlineCommand{configure} command.

This method extracts any mixture of options and parameters from a
Teuchos::\-Para\-meterList object and uses them to set values in AztecOO's
internal
options and params arrays. SetParameters() may be called repeatedly, and does
not reset default values or previously-set values unless those values are
contained in the current Teuchos::ParameterList argument. (Of course, if the
method SetAztecDefaults() is called after SetParameters() has been called,
any parameters that were set by SetParameters() will be lost.)

A Teuchos::ParameterList is a collection of named Teuchos::ParameterEntry
objects. (Please refer to documentation for the Teuchos package for information
on using the Teuchos::ParameterList.) AztecOO recognizes names which mirror the
macros defined in az\_aztec\_defs.h and described above in subsections
\ref{optionI} and \ref{optionD}.
In addition, it recognizes case insensitive versions of those names, with or
without the prepended `AZ\_'. As an example, the following are equivalent and
valid: ``AZ\_solver", ``SOLVER", ``Solver". To set an entry in the Aztec options
array, the type of the ParameterEntry value may be either a string or an int
in some cases. For example, if selecting the solver, the following are equivalent
and valid: AZ\_gmres (which is an int), ``AZ\_gmres" (which is a string) or
``GMRES" (case-insensitive, 'AZ\_' is optional).

To set an entry in the Aztec params array, the type of the
ParameterEntry value must be double.

By default, the SetParameters() method will silently ignore parameters which have
unrecognized names or invalid types. Users may provide an optional argument
specifying that warnings be printed for unused parameters. Alternatively,
users may iterate the ParameterList afterwards and check the isUsed
attribute on the ParameterEntry objects.

Here is a brief example of setting parameters using the ParameterList interface:
\begin{verbatim}
  Teuchos::ParameterList parameterlist;
  parameterlist.set("solver", AZ_gmres); //AZ_gmres stored as int
  parameterlist.set("precond", "Jacobi"); //"Jacobi" -> AZ_Jacobi
  parameterlist.set("AZ_tol", 1.e-10);

  AztecOO azoo;
  azoo.SetParameters(parameterlist);
\end{verbatim}

\subsection{Return status\label{status}}

The double precision array {\it status} of length {\sf AZ\_STATUS\_SIZE}
returned from Iterate()\footnote{ All integer information returned from
  Iterate() is cast into double precision and stored in {\it status}.}.
The contents of {\it status} are described below.  \vspace{2em}
{\flushleft{\bf Specifications} \hrulefill} \nopagebreak \\[0.5em]
%
\optionbox{status[{\sf AZ\_its}]}{Number of iterations taken by the
   iterative method.}
\optionbox{status[{\sf AZ\_why}]}{Reason why Iterate() terminated.}
      \choicebox{AZ\_normal}{User requested convergence criteria is
                 satisfied.}
      \choicebox{AZ\_param}{User requested option is not available.}
      \choicebox{AZ\_breakdown}{Numerical breakdown occurred.}
      \choicebox{AZ\_loss}{Numerical loss of precision occurred.}
      \choicebox{AZ\_ill\_cond}{The Hessenberg matrix within GMRES is
        ill-conditioned. This could be caused by a number of reasons.
        For example, the preconditioning matrix could be nearly singular
        due to an unstable factorization (note: pivoting is not implemented
        in any of the incomplete factorizations). Ill-conditioned Hessenberg
        matrices could also arise from a singular application
        matrix. In this case, GMRES tries to compute a least-squares solution.}
      \choicebox{AZ\_maxits}{Maximum iterations taken without convergence.}
\optionbox{status[{\sf AZ\_r}]}{The true residual norm corresponding to
   the choice {\it options}[{\sf AZ\_conv}] (this norm is calculated
   using the computed solution).}
\optionbox{status[{\sf AZ\_scaled\_r}]}{The true residual ratio expression
   as defined by  {\it options}[{\sf AZ\_conv}].}
\optionbox{status[{\sf AZ\_rec\_r}]}{Norm corresponding to
   {\it options}[{\sf AZ\_conv}] of final residual or estimated final
   residual (recursively computed by iterative method). Note: When using
   the 2-norm, {\bf tfqmr} computes an estimate of the residual norm
   instead of computing the residual.}
\optionbox{status[{\sf AZ\_solve\_time}]}{Utilization time in Aztec to solve system.}
\optionbox{status[{\sf AZ\_Aztec\_version}]}{Version number of Aztec.}
%
 When AztecOO returns abnormally, the user may elect to restart using
 the current computed solution as an initial guess.

\section{Choosing a Krylov Method}
\label{Sect:KrylovMethods}

Choosing a Krylov method is determined by many factors, some of
which are very problem-specific.  However, there are some general
guidelines worth mentioning.  Selection of a Krylov solver is done
by setting the value of options[{\sf AZ\_solver}] as described in
Section~\ref{Section:OptionsAndParams}.

\subsection{Symmetric (Positive Definite) Problems}

AztecOO provides two choices for solving symmetric positive definite
(SPD) problems, both of which are conjugate gradient (CG) methods. A
third option exists for some rare situations.
\begin{enumerate}
\item (AZ\_cg) The first is CG (AZ\_cg), which is the standard preconditioned CG
method.  Note that, strictly speaking, CG assume the matrix is
symmetric and positive semi-definite.  It does sometimes work for
mildly non-symmetric problems.  Preconditioners for CG must also be
symmetric.  Thus, it is usually ineffective to use incomplete LU
type of preconditioners with CG.

\item (AZ\_cg\_condnum) The second solver for symmetric problems is
also a preconditioned CG solver, but in addition to computing the
solution, this solver will construct the tridiagonal Lanczos matrix
that provides good estimates of the extreme eigenvalues of the
preconditioned\footnote{If you want to use AZ\_cg\_condnum to
estimate the extreme eigenvalues of your original matrix, you should
not use a preconditioner or scaling method when calling this solver.
 In general, changing the preconditioner will also change the extreme eigenvalue estimates}
 matrix.  This solver is more expensive than AZ\_cg, so it should
 only be used when this information is needed.

\item (AZ\_gmres and AZ\_gmres\_condnum)  Although there are
specialized Krylov methods for symmetric indefinite problems,
AztecOO does not implement them.  However, if such a solver is
needed, AZ\_gmres can be very effective.  Note that, if a problem
has many eigenvalues on either side of the imaginary access in the
complex plane, then the size of the restart value (options[{\sf
AZ\_kspace}] should be large.  For very difficult problems, it is
best to set this value equal to the number of iterations performed.

\end{enumerate}

\subsection{Non-symmetric Problems}

For SPD problems, CG is both optimal algorithmically and has fixed
cost per iteration.  For non-symmetric problems there is no such
solver.  It is well-known that GMRES without restart is an optimal
solver.  In practice it is very robust, especially with a good
preconditioner.  However, its cost can be prohibitive, both in
memory and computation time.  As a result, literally dozens of
non-optimal, but lower cost iterative methods have been developed
for non-symmetric problems.  AztecOO has two implementations of
GMRES and several lower cost sub-optimal methods.

Of these options, only two are commonly used, and a third
occasionally:
\begin{enumerate}
  \item (AZ\_gmres) Because AztecOO and its predecessor Aztec have
  often been applied to very difficult non-symmetric systems, and
  because GMRES is such an effective parallel Krylov method, GMRES is by far
  the most common non-symmetric solver used in AztecOO.  The
  effectiveness of GMRES can be strongly affected by the choice of
  restart value and orthogonalization method.  We discuss these
  issues below.
  \item (AZ\_gmres\_condnum) This version of GMRES retains an
  unfactored copy of the generated Hessenberg matrix and computes
  its eigenvalues using LAPACK.  The extreme eigenvalues of this
  Hessenberg
  matrix are often a good approximation to the extreme eigenvalues
  of the preconditioned matrix.
  \item (AZ\_bicgstab) Among the many sub-optimal non-symmetric
  Krylov methods, BiCGSTAB has proven to be the most robust for a
  reasonable cost.  For problems that are well-conditioned, BiCGSTAB
  can often be a much cheaper alternative to GMRES.  However, it is
  still ineffective for difficult problems.  Furthermore, unlike
  GMRES where the restart value can be increased, there are
  no parameters that can be used to make it more robust.
\end{enumerate}

\subsection{More about GMRES}

As mentioned above, GMRES is by far the most robust general-purpose
Krylov method available.  Part of its robustness comes from the
ability to tune two parameters, namely options[{\sf AZ\_kspace}] and
options[{\sf AZ\_orthog}]. options[{\sf AZ\_kspace}] determines the
number of Krylov vectors that will be used as part of the
least-squares problem to generate the next approximate solution.
Generally setting this value larger improves the robustness,
decreases iteration count, but increases costs. This value is set to
30 by default, but for challenging problems one should set it (much)
higher, especially if memory is available on the computer.  For very
difficult problems, we recommend setting options[{\sf AZ\_kspace}]
equal to the maximum number of iterations.

The parameter options[{\sf AZ\_orthog}] can be used to select the
type of Gram-Schmidt algorithm to used.  We provide two options:
\begin{enumerate}
  \item Two steps of classical Gram-Schmidt (Double CGS).

  \item One step of modified Gram-Schmidt. (Single MGS).
\end{enumerate}

For many years, (single) MGS was the preferred option for GMRES
because of its superior numerical accuracy over single CGS. However,
in the past several years it has been recognized that double CGS is
more effective than single MGS, as effective as double MGS and has
superior parallel performance.  As a result, AztecOO uses double CGS
by default.  However, there may be instances where single MGS would
be sufficient for robustness and it can have a lower cost in some
situations.

\section{Diagonal Perturbations and Incomplete Factorizations}
\label{Sect:DiagonalPerturbations}
 One of the new features in AztecOO
that was not part of Aztec 2.1 is the ability compute incomplete
factorizations of perturbed systems. One attribute of coupled
multi-physics problems is that incomplete factorizations can be
difficult to compute, even if the original matrix $A$ is
well-conditioned.  A few sources of difficulty are:
\begin{enumerate}
\item Zero diagonal entries.  In this case, unless fill-in occurs prior
to dividing by the zero diagonal, or we perform some type of pivoting, the
factorization will fail or produce unusable factors.  In some
instances even when fill-in does occur, the diagonal value may be
too small to produce a usable factorization.
\item Singular principle sub-matrices.  In this case, boundary
conditions are missing or insufficient to determine a nonsingular
upper left sub-matrix.
\item Singularity due to domain partitioning.  When executing in
parallel using additive Schwarz methods, we observe situations where
an incomplete factorization for the entire domain exists but one or
more factorizations for the subdomains do not.
\end{enumerate}

One straightforward technique to address poorly conditioned factors
is to introduce diagonal perturbations.  In this
situation, the incomplete factorization is performed on a matrix that
is identical to $A$ except that diagonal entries
are perturbed, usually to increase diagonal dominance.
This idea was introduced by
Manteuffel~\cite{Manteuffel:80} as a means for computing incomplete
Cholesky decompositions for symmetric positive definite systems
and extended to nonsymmetric matrices by van der
Vorst~\cite{vanderVorst:81},
Saad~\cite{Saad:94a} and Chow~\cite{Chow:97}.  It is used for
block entry matrices in a package called BPKIT~\cite{ChowHeroux:98}.

The incomplete factorization preconditioners in AztecOO (and in
IFPACK~\cite{Ifpack-home-page}) are sensitive to two parameters:
\begin{itemize}
  \item {\it params}[{\sf AZ\_athresh}]: Absolute threshold $\alpha$.
  \item {\it params}[{\sf AZ\_rthresh}]: Relative threshold $\rho$.
\end{itemize}

Given these two values, the preconditioner is computed for a matrix
$B$ such that off-diagonal entries of $B$ are the same as the
original matrix $A$ but the diagonals are replaced with:

\begin{equation}\label{diagaonalperturbations}
    b_{ii} \leftarrow \textrm{sign$(a_{ii})$}\alpha + (1+\rho)a_{ii}
\end{equation}

where sign() is the sign of the value, either $+1$ or $-1$.

Since Krylov methods such as GMRES are invariant under scaling, and
a very large diagonal
perturbation essentially makes the off-diagonal elements irrelevant,
one way to view diagonal perturbation is as establishing a continuum
between an accurate but poorly conditioned incomplete factorization
and less accurate but perfectly conditioned Jacobi diagonal scaling.
Given this continuum, the strategy is then to choose a minimal perturbation
that sufficiently stabilizes the factorization.


\subsection{Perturbation Strategies}

As mentioned above, we often have difficulty computing usable incomplete
factorizations for our problems.  The most common source of problems
is that the factorization may encounter a small or zero pivot,
in which case the factorization can fail, or even if the factorization
succeeds, the factors may be so poorly conditioned that use of them in
the iterative phase produces meaningless results.  Before we can fix
this problem, we must be able to detect it.  To this end, we use a
simple but effective condition number estimate for $(LU)^{-1}$.

\subsubsection{Estimating Preconditioner Condition Numbers}

The condition of a matrix $B$, called $cond_p(B)$, is defined as
$cond_p(B)
= \|B\|_p\|B^{-1}\|_p$ in some appropriate norm $p$.  $cond_p(B)$
gives some indication of how many accurate floating point
digits can be expected from operations involving the matrix and its
inverse.  A condition number approaching the accuracy of a given
floating point number system, about 15 decimal digits in IEEE double
precision, means that any results involving $B$ or $B^{-1}$ may be
meaningless.

The $\infty$-norm of a vector $y$ is defined as the maximum of the
absolute values of the vector entries, and the $\infty$-norm of a
matrix C is defined as
$\|C\|_\infty = \max_{\|y\|_\infty = 1} \|Cy\|_\infty$.
A crude lower bound for the $cond_\infty(C)$ is
$\|C^{-1}e\|_\infty$ where $e = (1, 1, \ldots, 1)^T$.  It is a
lower bound because $cond_\infty(C) = \|C\|_\infty\|C^{-1}\|_\infty
\ge \|C^{-1}\|_\infty \ge |C^{-1}e\|_\infty$.

For our purposes, we want to estimate $cond_\infty(LU)$, where $L$ and
$U$ are our incomplete factors.  Chow~\cite{Chow:97} demonstrates that
$\|(LU)^{-1}e\|_\infty$ provides an effective estimate for
$cond_\infty(LU)$.  Furthermore, since finding $z$ such that $LUz = y$
is a basic kernel for applying the preconditioner, computing this
estimate of $cond_\infty(LU)$ is performed by setting $y = e$, calling
the solve kernel to compute $z$ and then
computing $\|z\|_\infty$.

AztecOO provides the ability to query the condition number of the
preconditioner, if one is available.  This value is obtained by
calling the Condest() method on an AztecOO object instance.


\subsubsection{{\it A priori} Diagonal Perturbations}

Given the above method to estimate the conditioning of the incomplete factors,
if we detect that our factorization is too ill-conditioned
we can improve the conditioning by perturbing the matrix diagonal and
restarting the factorization using
this more diagonally dominant matrix.  In order to apply perturbation,
prior to starting
the factorization, we compute a diagonal perturbation of our matrix
$A$ in Eq.~\ref{e:axb} and perform the factorization on this perturbed
matrix.  The overhead cost of perturbing the diagonal is minimal since
the first step in computing the incomplete factors is to copy the
matrix $A$ into the memory space for the incomplete factors.  We
simply compute the perturbed diagonal at this point.  The actual
perturbation values we use are discussed below.


\subsection{Strategies for Managing Condition Numbers}

Without any prior knowledge of a problem, the first step to take when
computing a preconditioner is to compute the original factors without
any diagonal perturbation.  This usually gives the most accurate
factorization and, if the condition estimate of the factors is not too
big, will lead to the best convergence.  If the condition estimate of
the original factors is larger than machine precision, say greater
than 1.0e15, then it is possible that the factorization will destroy
convergence of the iterative solver.  This will be evident if the
iterative solver starts to diverge, stagnates, or aborts because
it detects ill-conditioning.  In these cases, diagonal perturbations
may be effective.  If the condition estimate of the preconditioner is
well below machine precision (less than 1.0e13) and one is not achieving
convergence, then diagonal perturbation will probably not be useful.
Instead, one should try to construct a more accurate factorization by
increasing fill.

\subsubsection{Strategies for {\bf\it a priori} Diagonal Perturbations}

The goal when applying {\it a priori} perturbations is to find a close to minimal
perturbation that reduces the condition estimate below machine
precision (roughly 1.0e16).  In some practical settings,
we use the strategy outlined in
Figure~\ref{f:aprioristrategy}.
{\small
\begin{figure}
\vspace{.05in}
\hrule
\vspace{.1in}
\begin{enumerate}
\item Set the absolute threshold $\alpha = 0.0$ and the relative
threshold $\rho = 0.0$ (equivalent to no perturbation).
\item Define perturbed diagonal entries as $d_{i} = \sign(d_i)\alpha +
d_i(1+\rho)$ and compute the incomplete factors $L$ and $U$.
\item Compute $condest = \|(LU)^{-1}e\|_\infty$ where $e = (1, 1,
\ldots, 1)^T$.
\item If failure ($condest > 10^{15}$ or convergence is poor), set $\alpha =
10^{-5}$, $\rho = 0.0$.  Repeat Steps 2 and 3.
\item If failure, set $\alpha =
10^{-5}$, $\rho = 0.01$. Repeat Steps 2 and 3.
\item If failure, set $\alpha =
10^{-2}$, $\rho = 0.0$. Repeat Steps 2 and 3.
\item If failure, set $\alpha =
10^{-2}$, $\rho = 0.01$. Repeat Steps 2 and 3.
\item If still failing, continue alternate increases in
   the two threshold values.
\end{enumerate}
\vspace{.1in}
\hrule
\caption{Simple {\it a priori} Threshold Strategy}
\label{f:aprioristrategy}
\end{figure}
} Essentially, we replace the diagonal values $(d_1, d_2, \ldots,
d_n)$ with $d_i = \sign(d_i)\alpha + d_i(1+\rho)$, $i=1, 2, \ldots,
n$, where $n$ is the matrix dimension and $\sign(d_i)$ returns the
sign of the diagonal entry.  This has the effect of forcing the
diagonal values to have minimal magnitude of $\alpha$ and to
increase each by an amount proportional to $\rho$, and still keep
the sign of the original diagonal entry.


\section {Optimal reuse of AztecOO for Repeated Solves}

In many practical situations, a linear solver is being called
repeatedly to solve problems that have similar numeric properties and
similar or identical nonzero structure.  In these cases, it is often
beneficial to reuse in later solves some or all of the work performed
in formulating the preconditioner for an earlier solve.  The Aztec
option options[{\sf AZ\_pre\_calc}] allows the user to specify that the
preconditioner from the previous solve should be retained for
subsequent solves.

\subsection{Reuse and the SetPrecOperator() Method}

A user can overrides the Aztec preconditioner options by
forming their own preconditioner that conforms to the \operator{}
interface.  In particular, users may construct ML and Ifpack
preconditioners for use with AztecOO.  Once an \operator{}-compliant
preconditioner is constructed, it can be registered with an AztecOO
object as the preconditioner.  In this situation, the preconditioner
will never be reset by AztecOO.  Instead, the user has this responsibility.

Here are the rules for what preconditioner gets used:
\begin{itemize}
\item If you instantiate an AztecOO instance with "A" as an
\operator{}, no internal Aztec preconditioners are available to you
unless you later register (before calling the Iterate() method) an
\rowmatrix{} using the SetPrecMatrix() method, in which case the
internal Aztec preconditioner will be used according to the settings
of the options and params vectors.  Instead of calling
SetPrecMatrix(), you can instead set use of a user-provided
preconditioner by registering an \operator{}, which should already
be fully constructed, using the SetPrecOperator() method.

\item If you instantiate an AztecOO instance with "A" as a matrix,
the internal Aztec preconditioners will automatically be used in
accord with the associated options and params lists.

\item If, subsequently you call SetPrecOperator() with an
\operator{} that is ready to use, then the internal preconditioner
will be deleted and your operator will be used.

\item It is possible to alternate back and forth between an internal
Aztec preconditioner and a user-provided preconditioner by calling
SetPrecMatrix() to get the internal preconditioner and calling
SetPrecOperator() to get a user-provided preconditioner.

\item It is also possible to call SetPrecOperator() after a previous
call to SetPrecOperator() in order to change to a new user-defined
preconditioner. Also, it is possible to call SetPrecMatrix() after a
previous call to SetPrecMatrix(), or after Iterate() has been
called, in order to change which internal preconditioner will be
used.  Of course some change in the options or params vectors must
be changed in order to actually obtain a different preconditioner.
\end{itemize}

\section{Accessing AztecOO from Thyra and Python}

In addition to accessing AztecOO via its C++ interfaces, bindings
exist that allow use via the Trilinos abstract interfaces in Thyra.
Thyra is a package of abstract C++ interfaces for accessing one or
more concrete solver capabilities via a single solver interface.

AztecOO also is accessible via the PyTrilinos package using Python.
Python provides an excellent interactive prototyping environment, as
well as a flexible application development environment.  AztecOO,
Epetra, ML, IFPACK and several other Trilinos packages are available
to Python users via the PyTrilinos package.

Detailed discussions of Thyra and PyTrilinos can be found by going
to the Trilinos home page~\cite{Trilinos-home-page}.

\clearpage
\bibliographystyle{plain}
%\bibliography{../../../../doc/CommonFiles/TrilinosBibliography}
\bibliography{AztecOOUserGuide}
\addcontentsline{toc}{section}{References}

\appendix
\section{AztecOO \InlineCommand{configure} Options}

Most often AztecOO's \InlineCommand{configure} command will be invoked automatically as
part of the Trilinos-level \InlineCommand{configure} command.  Regardless of how
AztecOO's \InlineCommand{configure} command is invoked, the following options can be
used to customize the configure process.  These options can also be listed  by executing
\DisplayCommand{./configure --help}
in the main AztecOO directory.  In fact, obtaining the options this way is preferred, since
option may have changed since the publiction of this document.  However, for convenience, we
list the configuration options presently available:

\begin{Verbatim}
`configure' configures aztecoo 3.3d to adapt to many kinds of systems.

Usage: ../../../packages/aztecoo/configure [OPTION]... [VAR=VALUE]...

To assign environment variables (e.g., CC, CFLAGS...), specify them as
VAR=VALUE.  See below for descriptions of some of the useful variables.

Defaults for the options are specified in brackets.

Configuration:
  -h, --help              display this help and exit
      --help=short        display options specific to this package
      --help=recursive    display the short help of all the included packages
  -V, --version           display version information and exit
  -q, --quiet, --silent   do not print `checking...' messages
      --cache-file=FILE   cache test results in FILE [disabled]
  -C, --config-cache      alias for `--cache-file=config.cache'
  -n, --no-create         do not create output files
      --srcdir=DIR        find the sources in DIR [configure dir or `..']

Installation directories:
  --prefix=PREFIX         install architecture-independent files in PREFIX
                          [/usr/local]
  --exec-prefix=EPREFIX   install architecture-dependent files in EPREFIX
                          [PREFIX]

By default, `make install' will install all the files in
`/usr/local/bin', `/usr/local/lib' etc.  You can specify
an installation prefix other than `/usr/local' using `--prefix',
for instance `--prefix=$HOME'.

For better control, use the options below.

Fine tuning of the installation directories:
  --bindir=DIR           user executables [EPREFIX/bin]
  --sbindir=DIR          system admin executables [EPREFIX/sbin]
  --libexecdir=DIR       program executables [EPREFIX/libexec]
  --datadir=DIR          read-only architecture-independent data [PREFIX/share]
  --sysconfdir=DIR       read-only single-machine data [PREFIX/etc]
  --sharedstatedir=DIR   modifiable architecture-independent data [PREFIX/com]
  --localstatedir=DIR    modifiable single-machine data [PREFIX/var]
  --libdir=DIR           object code libraries [EPREFIX/lib]
  --includedir=DIR       C header files [PREFIX/include]
  --oldincludedir=DIR    C header files for non-gcc [/usr/include]
  --infodir=DIR          info documentation [PREFIX/info]
  --mandir=DIR           man documentation [PREFIX/man]

Program names:
  --program-prefix=PREFIX            prepend PREFIX to installed program names
  --program-suffix=SUFFIX            append SUFFIX to installed program names
  --program-transform-name=PROGRAM   run sed PROGRAM on installed program names

System types:
  --build=BUILD     configure for building on BUILD [guessed]
  --host=HOST       cross-compile to build programs to run on HOST [BUILD]
  --target=TARGET   configure for building compilers for TARGET [HOST]

Optional Features:
  --disable-FEATURE       do not include FEATURE (same as --enable-FEATURE=no)
  --enable-FEATURE[=ARG]  include FEATURE [ARG=yes]
  --enable-maintainer-mode  enable make rules and dependencies not useful
                          (and sometimes confusing) to the casual installer
  --enable-mpi            MPI support
  --disable-dependency-tracking  speeds up one-time build
  --enable-dependency-tracking   do not reject slow dependency extractors
  --enable-aztecoo-azlu   Enable az-lu preconditioner. Default is no. Requires
                          y12m.
  --enable-aztecoo-teuchos
                          Enable AztecOO Teuchos::ParameterList interface.
                          Default is no. Requires Teuchos package.
  --enable-tests          Build tests for all Trilinos packages (not all
                          packages are sensitive to this option) (default is
                          yes)

  --enable-aztecoo-tests  Build AztecOO tests (default is yes if
                          --disable-tests is not specified)
  --enable-examples       Build examples for all Trilinos packages (not all
                          packages are sensitive to this option) (default is
                          yes)
  --enable-aztecoo-examples
                          Build AztecOO examples (default is yes if
                          --disable-examples is not specified)
  --enable-libcheck       Check for some third-party libraries including BLAS
                          and LAPACK. (Cannot be disabled unless tests and
                          examples are also disabled.) (default is yes)

Optional Packages:
  --with-PACKAGE[=ARG]    use PACKAGE [ARG=yes]
  --without-PACKAGE       do not use PACKAGE (same as --with-PACKAGE=no)
  --with-install=INSTALL_PROGRAM
                          Use the installation program INSTALL_PROGRAM rather
                          the default that is provided. For example
                          --with-install="/path/install -p"
  --with-mpi-compilers=PATH
                          use MPI compilers mpicc, mpif77, and mpicxx, mpic++
                          or mpiCC in the specified path or in the default
                          path if no path is specified. Enables MPI
  --with-mpi=MPIROOT      use MPI root directory (enables MPI)
  --with-mpi-libs="LIBS"  MPI libraries ["-lmpi"]
  --with-mpi-incdir=DIR   MPI include directory [MPIROOT/include] Do not
                          use -I
  --with-mpi-libdir=DIR   MPI library directory [MPIROOT/lib] Do not use
                          -L
  --with-ccflags          additional CCFLAGS flags to be added: will prepend
                          to CCFLAGS
  --with-cxxflags         additional CXXFLAGS flags to be added: will
                          prepend to CXXFLAGS
  --with-cflags           additional CFLAGS flags to be added: will prepend
                          to CFLAGS
  --with-fflags           additional FFLAGS flags to be added: will prepend
                          to FFLAGS
  --with-libs             List additional libraries here. For example,
                          --with-libs=-lsuperlu or
                          --with-libs=/path/libsuperlu.a
  --with-ldflags          additional LDFLAGS flags to be added: will prepend
                          to LDFLAGS
  --with-ar               override archiver command (default is "ar cru")
  --with-libdirs          OBSOLETE use --with-ldflags instead. (ex.
                          --with-ldflags="-L<DIR> -L<DIR2>")
  --with-incdirs          additional directories containing include files:
                          will prepend to search here for includes, use -Idir
                          format
  --with-lapacklib        name of library containing LAPACK: will search lib
                          directories for -lname
  --with-blaslib          name of library containing BLAS: will search lib
                          directories for -lname
  --with-blas=<lib>       use BLAS library <lib>
  --with-lapack=<lib>     use LAPACK library <lib>

Some influential environment variables:
  CC          C compiler command
  CFLAGS      C compiler flags
  LDFLAGS     linker flags, e.g. -L<lib dir> if you have libraries in a
              nonstandard directory <lib dir>
  CPPFLAGS    C/C++ preprocessor flags, e.g. -I<include dir> if you have
              headers in a nonstandard directory <include dir>
  CXX         C++ compiler command
  CXXFLAGS    C++ compiler flags
  F77         Fortran 77 compiler command
  FFLAGS      Fortran 77 compiler flags
  CXXCPP      C++ preprocessor

Use these variables to override the choices made by `configure' or to help
it to find libraries and programs with nonstandard names/locations.

Report bugs to <maherou@sandia.gov>.
\end{Verbatim}

%\section{Troubleshooting}
%\label{Section:Troubleshooting}
%\subsection{Conditional Code, Incorrect Results and Stalled Programs}
%Many methods in AztecOO distributed classes (those derive from the
%\distobject{} class) require all processors to
%participate in the method call.  For example, to compute the update of
%an \vector{}, all processors that own a portion of the vector must
%call the Update() method.  When calling a Norm2() method or something
%similar, all processors, regardless of whether they own any portion of
%a vector, must participate in the Norm2() call.
%Figure~\ref{Figure:HungCode} illustrates
%several versions of a code segment that computes the 2-norm of a residual
%and prints it from processor 0.  Only the final version is correct.

\end{document}
