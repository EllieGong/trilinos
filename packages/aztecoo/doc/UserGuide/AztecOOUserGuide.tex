 \documentclass[12pt,relax]{AztecOOUserGuide}
 \usepackage{fancyvrb}
\SANDsubtitle{}
\title{\AztecOOTM{} User Guide}

\author{Michael A. Heroux \\
	\\
	    Sandia National Laboratories\\
	    P.O. Box 5800\\
	    Albuquerque, NM 87185-1110 
	 }

    % There is a "Printed" date on the title page of a SAND report, so
    % the generic \date should generally be empty.
    \date{} % Remove ``\today'' in final version


\SANDnum{SAND2003-xxxx}
\SANDprintDate{Printed September 2003}
\SANDauthor{Michael A. Heroux \\
	Computational Mathematics and Algorithms Department \\
	 Sandia National Laboratories \\
	P.~O.~Box 5800 \\
	Albuquerque, NM 87185-1110}


\SANDreleaseType{Unlimited Release}


% New commands

\newcommand{\comm}{Epetra\_Comm}
\newcommand{\serialcomm}{Epetra\_SerialComm}
\newcommand{\mpicomm}{Epetra\_MpiComm}
\newcommand{\mpismpcomm}{Epetra\_MpiSmpComm}
\newcommand{\lbcomm}{Epetra\_LbComm}
\newcommand{\map}{Epetra\_Map}
\newcommand{\blockmap}{Epetra\_BlockMap}
\newcommand{\serialdensevector}{Epetra\_SerialDenseVector}
\newcommand{\intserialdensevector}{Epetra\_IntSerialDenseVector}
\renewcommand{\vector}{Epetra\_Vector}
\newcommand{\multivector}{Epetra\_MultiVector}
\newcommand{\crsgraph}{Epetra\_CrsGraph}
\newcommand{\crsmatrix}{Epetra\_CrsMatrix}
\newcommand{\vbrmatrix}{Epetra\_VbrMatrix}
\newcommand{\rowmatrix}{Epetra\_RowMatrix}
\newcommand{\distobject}{Epetra\_DistObject}
\newcommand{\linearproblem}{Epetra\_LinearProblem}
\newcommand{\operator}{Epetra\_Operator}
\newcommand{\fecrsmatrix}{Epetra\_FECrsMatrix}
\newcommand{\fevbrmatrix}{Epetra\_FEVbrMatrix}
\newcommand{\msrmatrix}{Epetra\_MsrMatrix}
\newcommand{\statustest}{AztecOO\_StatusTest}
\newcommand{\oooperator}{AztecOO\_Operator}


\begin{document}
\maketitle

\begin{abstract}

The \TrilinosTM{} Project is an effort to facilitate the design, development,
integration and ongoing support of mathematical software libraries.
\AztecOOTM{} is a package within Trilinos that enables the use of the Aztec solver 
library~\cite{Aztec2.1} with \EpetraTM{}~\cite{Epetra-Users-Guide} objects.  
AztecOO provides access to Aztec preconditioners and solvers by
implementing the Aztec ``matrix-free'' interface using Epetra.
While Aztec is written in C and procedure-oriented, 
AztecOO is written in C++ and is object-oriented.

In addition to providing access to Aztec capabilities, AztecOO also
provides some signficant new functionality.  In particular it provides
an extensible status testing capability that allows expression of
sophisticated stopping criteria as is needed in production use of
iterative solvers.  AztecOO also provides mechanisms for using 
Ifpack~\cite{Ifpack-Ref-Guide}, ML~\cite{ML-home-page} and AztecOO
itself as preconditioners.
\end{abstract}


\section*{Acknowledgement}
The authors would like to acknowledge the support of the ASCI and LDRD 
programs that funded development of AztecOO and the authors of Aztec
2.1 upon which AztecOO is built: Ray Tuminaro, Mike Heroux, Scott
Hutchinson and John Shadid.  

\clearpage
\tableofcontents
\listoffigures
\listoftables

\clearpage


NOTE: Add section on repeated use of AztecOO objects.


\section{Introduction}
\label{Section:Introduction}

AztecOO is a collection of C++ classes that support the construction
and use of objects for solving linear systems of equations via
preconditioned Krylov methods, as provided in Aztec.

This user guide is intended to (i) introduce a new
user to the basic features of AztecOO, (ii) discuss the details of AztecOO
constructors, and (iii) illustrate the use of
AztecOO's advanced uses.  This document is not intended as a reference
manual.  Detailed descriptions of AztecOO classes and methods can be
found online at the Trilinos Project home page~\cite{Trilinos-home-page}.


\subsection{Overview of Major AztecOO Classes and Features}

AztecOO contains a variety of classes to support the solution of
linear systems of equations of the form $Ax=b$ using preconditioned
iterative methods.  AztecOO also fully contains Aztec, so any application
that is using Aztec can use the AztecOO library in place of Aztec.  

The primary AztecOO class is of the same name, AztecOO.  An AztecOO
class instance acts as a manager of Aztec, accepting user data as
Epetra objects.  If an AztecOO object is instantiated using Epetra
objects, all of Aztec's preconditioners and Krylov methods can be
applied the Epetra-defined problem.  However, AztecOO provides a
variety of mechanisms to override default Aztec capabilities.  Users
can construct and use preconditioners from Ifpack or ML, or can use
another instance of the AztecOO class as a preconditioner.  Users can
also override the default convergence tests in Aztec and use any
combination of available status tests in \statustest{} classes,
or define their own.

\subsection{Use of Epetra}

AztecOO relies on Epetra for both concrete and abstract classes that
describe matrix, vector and linear operator objects.  
Although  concrete classes are needed to construct matrices, AztecOO
itself uses these matrices via two Epetra abstract classes.  By using
abstract interfaces, we can support any of the predefined classes that
implement the abstract interfaces andallows users to define
new implementations.  This allows AztecOO
to be easily extended.  

\subsubsection{Epetra Abstract Classes}

The two primary abstract Epetra classes used by AztecOO are:
\begin{enumerate}
\item {\bf \rowmatrix{}: } Supports the use of any class that is
conceptually a linear operator with access to coefficient data.
Although row-orientation is assumed, this class works equally well
with column oriented data, since transpose operations are supported.  
This class provides an interface to access matrix data.  In addition,
it extends the \operator{} interface, so any class that implements
\rowmatrix{} also implmenents \operator{}, described next.  There are
four primary classes in Epetra that implement \rowmatrix{}, namely 
the \crsmatrix{}, \vbrmatrix{}, \fecrsmatrix{} and \fevbrmatrix{}
classes.  AztecOO also provides an implementation of \crsmatrix{}
called \msrmatrix{}.  \msrmatrix{} allows users whose application is
already forming Aztec DMSR matrix structs to encapsulate the DMSR
matrix in a class that implements \rowmatrix.  The encapsulation does
not copy the data in the DMSR matrix struct.  This feature is
important for people making a transition from Aztec to AztecOO.
\item {\bf \operator{}: } Supports the use of any class that is
conceptually a linear operator.  There are only a handful of methods
in this class, the most important of which are the Apply() and
ApplyInverse() (ApplyInverse() can be defined as nonexistent).  
A large number of Epetra classes implement the
\operator{} interface, including the \crsmatrix{}, \vbrmatrix{},
\fecrsmatrix{} and \fevbrmatrix{} since these classes implement
\rowmatrix{} and \rowmatrix{} extends \operator{}.  In addition, ML
and IFPACK both implement the \operator{} interface, so they can be
used as preconditioners for AztecOO.  A class called \oooperator{}
also implements the \operator{} interface using an existing AztecOO
class instance.  This allows AztecOO to be used as a preconditioner
for itself.

\end{enumerate}

\subsubsection{Epetra Concrete Classes}

Given the above abstract classes, we need some concrete classes in
order to construct explicit objects.  Specifically,
the following concrete Epetra classes are used:
\begin{enumerate}
\item {\bf \vector{}: } Supports construction and use of distributed
vectors of double-precision numbers.  Once constructed, \vector{}
objects can be used in multiple ways.  Common operations such as
norms, dot products and vector updates are supported by methods in
this class.  Additional functionality is available via several
extension techniques discussed later.
\item{\bf \multivector{}: } An \multivector{} object is a collection
of \vector{} objects (although \vector{} is actually implemented as a
specialization of \multivector{}).  Specifically, an \multivector{}
object is a collection of vectors with the same size and
distribution.  This kind of object is useful for block algorithms and
applications that manage multiple vectors simultaneously.  \vector{}
and \multivector{} objects are understood by all Trilinos packages.
\item{\bf \crsgraph{}:} Supports the construction and use of adjacency
graphs.  These graphs are used to describe the pattern of Epetra
sparse matrix classes and provide pattern-based information to
load-balancing interfaces.  The graphs are also used to implement
overlapping subdomain algorithms and a variety of other parallel
algorithms.
\item{\bf \crsmatrix{}:} Supports construction and use of distributed
sparse matrix objects.  Once constructed, an \crsmatrix{} object can
be used with any Trilinos solvers or preconditioners.  This class also
supports common matrix and matrix-vector operations such as matrix
scaling, matrix norms and matrix-vector multiplication.
\item{\bf \vbrmatrix{}:} Although less frequently used than the
\crsmatrix{} class, this class supports the construction of sparse
matrices whose entries are dense matrices.  This type of matrix is
frequently found in applications where multiple degrees of freedom are
tracked at each mesh point.  When properly used, the \vbrmatrix{}
class can offer substantial machine performance and algorithmic
robustness improvements.
\item{\bf \linearproblem{}:} An \linearproblem{} object is an
aggregate object that encapsulates the problem $Ax=b$.  It contains a
pointer to an \rowmatrix{} or \operator{} representing $A$, and an 
\multivector{} for $x$ and another for $b$.  (NOTE:  Although the
interface supports multiple right-hand-sides and solution vectors,
AztecOO presently supports the solution of only one right-hand-side.)
Although use of the \linearproblem{} is not required for passing $A$,
$x$ and $b$ to AztecOO, it is highly recommended.  Use of
\linearproblem{} guarantees that the matrix, left-hand-side and
right-hand-side are compatible.  Also, the \linearproblem{} class
provides methods for scaling the linear problem using any of a variety
of diagonal scaling methods.
\end{enumerate}




\section{A First Example}
\label{Section:FirstExample}
Before proceeding with additional descriptive information, we
introduce a simple example in this section in order to explicitly
illustrate a straight-forward use of AztecOO.  Although this example only
starts to tap the capabilities of AztecOO, we hope the reader finds it
useful.

The example code, listed in Figure~\ref{Figure:FirstExample}
and~\ref{Figure:PowerMethod}, is the main program driver and
a simple implementation of the Power Method~\cite{GoluVanL96} for computing the dominant
eigenvalue and associated eigenvector of a given matrix,
respectively.  In the remainder of this section we proceed with a
description of the code in these figures.
\begin{figure}
\begin{center}
\begin{Verbatim}[frame=single,numbers=left,fontfamily=courier,fontsize=\tiny]
// @HEADER
// ***********************************************************************
// 
//        AztecOO: An Object-Oriented Aztec Linear Solver Package 
//                 Copyright (2002) Sandia Corporation
// 
// Under terms of Contract DE-AC04-94AL85000, there is a non-exclusive
// license for use of this work by or on behalf of the U.S. Government.
// 
// This library is free software; you can redistribute it and/or modify
// it under the terms of the GNU Lesser General Public License as
// published by the Free Software Foundation; either version 2.1 of the
// License, or (at your option) any later version.
//  
// This library is distributed in the hope that it will be useful, but
// WITHOUT ANY WARRANTY; without even the implied warranty of
// MERCHANTABILITY or FITNESS FOR A PARTICULAR PURPOSE.  See the GNU
// Lesser General Public License for more details.
//  
// You should have received a copy of the GNU Lesser General Public
// License along with this library; if not, write to the Free Software
// Foundation, Inc., 59 Temple Place, Suite 330, Boston, MA 02111-1307
// USA
// Questions? Contact Michael A. Heroux (maherou@sandia.gov) 
// 
// ***********************************************************************
// @HEADER

#include "AztecOO_config.h"
#include "AztecOO_Version.h"
#ifdef HAVE_MPI
#include "mpi.h"
#include "Epetra_MpiComm.h"
#else
#include "Epetra_SerialComm.h"
#endif
#include "Epetra_Map.h"
#include "Epetra_Vector.h"
#include "Epetra_CrsMatrix.h"
#include "AztecOO.h"

int main(int argc, char *argv[]) {

#ifdef HAVE_MPI
  MPI_Init(&argc,&argv);
  Epetra_MpiComm Comm( MPI_COMM_WORLD );
#else
  Epetra_SerialComm Comm;
#endif
  if (Comm.MyPID()==0)
    cout << AztecOO_Version() << endl << endl;

  cout << Comm <<endl;

  int NumMyElements = 100;
  // Construct a Map that puts same number of equations on each processor
  Epetra_Map Map(-1, NumMyElements, 0, Comm);
  int NumGlobalElements = Map.NumGlobalElements();

  // Create a Epetra_Matrix
  Epetra_CrsMatrix A(Copy, Map, 3);
  
  // Add  rows one-at-a-time
  double negOne = -1.0;
  double posTwo = 2.0;
  for (int i=0; i<NumMyElements; i++) {
    int GlobalRow = A.GRID(i); int RowLess1 = GlobalRow - 1; int RowPlus1 = GlobalRow + 1;

    if (RowLess1!=-1) A.InsertGlobalValues(GlobalRow, 1, &negOne, &RowLess1);
    if (RowPlus1!=NumGlobalElements) A.InsertGlobalValues(GlobalRow, 1, &negOne, &RowPlus1);
    A.InsertGlobalValues(GlobalRow, 1, &posTwo, &GlobalRow);
  }
  
  // Finish up
  A.FillComplete();

  // Create x and b vectors
  Epetra_Vector x(Map);
  Epetra_Vector b(Map);
  b.Random(); // Fill b with random values

  // Create Linear Problem
  Epetra_LinearProblem problem(&A, &x, &b);
  // Create AztecOO instance
  AztecOO solver(problem);

  solver.SetAztecOption(AZ_precond, AZ_Jacobi);
  solver.Iterate(100, 1.0E-8);

  cout << "Solver performed " << solver.NumIters() << " iterations." << endl
       << "Norm of true residual = " << solver.TrueResidual() << endl;
#ifdef HAVE_MPI
  MPI_Finalize() ;
#endif
  return 0;
}

\end{Verbatim}

\end{center}
\caption{\label{Figure:FirstExample} Simple AztecOO/Epetra Example}
\end{figure}

\subsection{Explanation of Figure~\ref{Figure:FirstExample}}

\paragraph{Lines 1--2}
Include system header files needed for screen output and system
functions.
\paragraph{Line 3}
Include the AztecOO\_config.h file.  This file contains macros
definitions that were defined during the configuration process.  In
particular, the macro HAVE\_MPI will be defined or undefined in this
file, depending on whether AztecOO was built in MPI mode or not.  We
will use HAVE\_MPI below to determine if our example code should be
compiled with MPI support or not.
\paragraph{Lines 4--9}
Include the appropriate implementation of the \comm{} class.  If
AztecOO was built in MPI mode, the macro ``HAVE\_MPI'' will be defined
and this example will be built with MPI support.  If not, then the
example will be built in serial mode.  Note that these lines of code
 and lines 18--23 are the only difference between a serial and distributed memory
version of the example. 
\paragraph{Lines 10--14}
Include the other AztecOO header files needed to use AztecOO classes.
It is a good practice to explicitly include header files for all
AztecOO classes you explicitly use and only those header files.
\paragraph{Line 16}
Declare prototype for power\_method function.
\paragraph{Line 17}
Start of main program.  Note that argv[1] will be used to pass in the
problem size.
\paragraph{Lines 18--24}
Depending on whether or not AztecOO was built in MPI
mode\footnote{Whether or not Trilinos, and AztecOO as a Trilinos package, is built
in MPI mode is determined by how the Trilinos (or AztecOO) {\tt
configure} script in invoked.  If no MPI-related arguments are passed
to the configure script, then packages are built in serial mode only.
If one or more MPI options are invoked, then packages are built with
MPI support (in addition to serial support).}, MPI will be initialized
and an \mpicomm{} object will be constructed, or an \serialcomm{}
object will be constructed.  Please note that, in principle, the serial
version of this example would work, even if AztecOO were built in MPI
mode.  Serial mode is always available.

Line 24 prints the Comm object to cout.  Most AztecOO classes have
implemented the ostream << operator so that an instance of an AztecOO
object can be viewed.  Sometime the volume of output can be very
large, so some care is needed when using this feature.
\paragraph{Lines 25--30}
This example accepts a single integer argument
specifying the number of equations that should be used.  These lines
of code capture that number and put it into NumGlobalElements.
\paragraph{Lines 31--33}
Constructs an \map{} object that has NumGlobalElements elements spread
across the parallel (or serial) machine.  The second argument (a
``0'') indicates that our global indices are zero-based.  Fortran
users would typically pass in a ``1'' here.  The third argument is the
Comm object we just built.
\paragraph{Line 35}
Once an \map{} object is constructed, we can query it for how many
elements of the map are on the calling process\footnote{For readers
who are not familiar with a single-program, multiple data (SPMD)
programming model, it may be useful to read a bit about it.  Typing
``SPMD tutorial'' into a web search engine should be a sufficient
starting point.}.  In this way, the remainder of our code can operate
independent of how data is distributed.
\paragraph{Lines 37--57: Preview}
The majority of executable code in this sample driver is devoted to
constructing our sparse matrix.  At the end of this code segment we
will have constructed a diagonal with each processor owning roughly
the same number of rows and columns.  All diagonal values will be set
to 2.0, except for the very first diagonal value, which will be set to
4.0.  Since our matrix is diagonal, its eigenvalues are the diagonal
values and the dominant eigenvalue is 4.0.  
\paragraph{Line 37}
Instantiates (creates) an \crsmatrix{}.  The first argument tells the
constructor whether or not data passed in to this object should be
copied (user values and indices will be copied to internal storage) or
viewed (user values and indices will be pointed to by this object and
the user {\it must} guarantee the integrity of that data).  View mode
is available across many AztecOO classes.  In general, this is a very
dangerous practice.  However, in certain very important situations, it
is essential to have this mode.  This is especially true when using
AztecOO with Fortran, or when accepting matrix data from other parts of
application where it is too expensive to replicate the data storage.

The second argument is the \map{} object we just constructed.  The
third argument is an advisory value telling the constructor
approximately how many nonzero values will be defined for each row of
the matrix\footnote{getting this value wrong does not
affect the correctness of results, but may affect performance and
efficient use of memory}.  We are constructing a diagonal matrix, so the value ``1''
is appropriate.

At this point the matrix is an empty ``bucket'' ready to receive
matrix values and indices.  Also, at this point, most of the methods
in the \crsmatrix{} cannot be called successfully for this object.
\paragraph{Lines 39--41}
Here we create two helper objects that not really necessary for such a
simple example, but which illustrate how examples that are more
complex might
use the \serialdensevector{} and \intserialdensevector{} classes
effectively.
Line 39 allocates a vector of length one to hold the diagonal value
we will insert into each row.  Line 40 sets the value to 2.0, which is
the value we will assign to all the diagonals of our matrix.  We will
later replace the first diagonal value with a value of 4.0.
\paragraph{Lines 42--48}
These lines insert values and indices into the matrix we just
instantiated.  Our matrix is very simple--a single diagonal entry per
row--so these lines may seem too complicated, but we are illustrating
a process that is effective for more realistic settings.  
The \crsmatrix{} class is
designed to accept matrix entries in a variety of ways.  However,
there is a strong bias in the interface to accept multiple entries in
a specified row.  Values for any matrix entry can be submitted one or more times,
with subsequent values being added to any existing value.
Section~\ref{Section:ConstructingCrsMatrices} discusses these issues
in detail.

Line 42 loops over all elements
owned by the calling processor.  Line 43 get the global ID associated
with the current loop index.  This value will be passed in as the
first argument to the 
InsertGlobalValues() method to indicate the global row into which the calling
processor is inserting values.  Line 44 sets the column index to the
same as the row index because we have only a diagonal value.  

Lines 46--47 is a call the the InsertGlobalValues() method on our matrix
object A.  The first argument is the ID of the global row that the
calling processor will add values to.  The second argument indicates
how many values we are passing in on this call.  We are using the
Length() method of the \serialdensevector{} class to provide this
information.  The third argument is a pointer to an array of doubles
that are the values we are passing in,
provided by the Values() method from \serialdensevector{}.  Finally
the last argument is a pointer to an array of ints that are the column
indices we are passing in.  Because we are only passing in a single
value per row, both of the last arguments to InsertGlobalValues() are trivial
uses of arrays, but illustrate a general process.
\paragraph{Lines 49--55}
The last modification we want to make to our matrix is to put a value
of 4.0 in the first diagonal entry.  We accomplish this by having each
processor query the map to see the processor owns GID 0.  Only the
processor that does own that GID will execute the remaining lines of
code in this segment.  Lines 50--52 set up the value and index
information.  Lines 53--54 are a call to ReplaceGlobalValues(), which
will replace existing values in our matrix with the new values we submit.
\paragraph{Line 57}
After all values are submitted to our matrix, we finally call the
method TransformToLocal().  This method performs all of the analysis
on our matrix required for efficient execution on a parallel
computer.  Part of the procces involves changing the index values of
the matrix to a local index system on each processor.  This will be
described in greater detail in
Section~\ref{Section:ConstructingCrsMatrices}.
\paragraph{Lines 59--60}
After constructing our matrix, we call the power method function,
passing in our matrix.  Then we print out the eigenvalue estimate that
was returned.
\paragraph{Lines 61--63}
If our code was compile in MPI mode, we need to call MPI\_Finalize()
for proper clean up.
\paragraph{Line 64} Program exit.

\section{ Advanced topics}
\subsection {Using \crsmatrix{} Objects as Column Matrices}

\clearpage
\bibliographystyle{plain}
\bibliography{AztecOOUserGuide}
\addcontentsline{toc}{section}{References}

\appendix
\section{Troubleshooting}
\label{Section:Troubleshooting}
\subsection{Conditional Code, Incorrect Results and Stalled Programs}
Many methods in AztecOO distributed classes (those derive from the
\distobject{} class) require all processors to
participate in the method call.  For example, to compute the update of
an \vector{}, all processors that own a portion of the vector must
call the Update() method.  When calling a Norm2() method or something
similar, all processors, regardless of whether they own any portion of
a vector, must participate in the Norm2() call.  
Figure~\ref{Figure:HungCode} illustrates
several versions of a code segment that computes the 2-norm of a residual
and prints it from processor 0.  Only the final version is correct.

\end{document}

