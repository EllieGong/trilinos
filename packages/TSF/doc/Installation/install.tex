\documentclass[10pt]{article}
%
\usepackage{html}
\usepackage{heqn}
\usepackage{moreverb}
\usepackage{alltt}
\usepackage{color}

\newcommand{\tsf}{{\bf TSF}}

\sloppy



\makeindex

\begin{document}
%



\title{TSF Installation Guide}

\author{Kevin Long\\Computational Science and Mathematics Research Department\\
Sandia National Laboratories\\Livermore, California\\{\tt krlong@sandia.gov}}

\maketitle


\tableofcontents

\section{Requirements}

At a minimum, you will need 

\begin{itemize}
\item A C++ compiler with support for templates, exceptions, rtti, namespaces, and
the {\tt mutable} keyword. 
\item Gnu make (gmake). On Linux systems, make is gmake by default.
\item The BLAS and LAPACK libraries. Reference fortran
implementations of BLAS and LAPACK are freely available from netlib:
\begin{itemize}
\item http://www.netlib.org/blas/index.html
\item http://www.netlib.org/lapack/index.html
\end{itemize}
Some vendors will have versions of BLAS tuned for a particular system. Alternatively,
you can build an auto-tuned BLAS using ATLAS, avaliable from:
\begin{itemize}
\item http://math-atlas.sourceforge.net/
\end{itemize}
\end{itemize}

\subsection{Optional third-party software}

A minimal build of TSF will have only 
dense, serial linear algebra implemented with LAPACK. Most likely you will want
distributed, sparse linear algebra, for which you will need to download some additional
software.

\begin{itemize}
\item TSF has extensive support for the lower-level components of the Trilinos Project, 
such as EPetra and IFPack. To use these, obtain a Trilinos distribution from 
\begin{itemize}
\item http://www.cs.sandia.gov/~mheroux/Trilinos/TrilinosDetails.htm
\end{itemize}
Note that Trilinos in turn requires several other packages such as MPI. 
\end{itemize}

\section{Configuring TSF}

TSF uses Gnu {\tt autoconf} to customize makefiles for your site. 

\begin{itemize}
\item {\tt --with-arch=}{\it arch}, where {\it arch} is one of {\tt LINUX}, 
{\tt SunWS5.0}, or {\tt SunWS5.0-Purify}.
\item {\tt --with-math=}{\it math libraries}. The default is {\tt -llapack -lblas -lm}.
\item {\tt --enable-mpi=yes|no}. Default is no. 
\item {\tt --disable-trilinos=yes|no}. Default is no. If Trilinos is used, its location
is guessed from various Trilinos-specific environment variables.
\item {\tt --enable-opt=yes|no} toggles compiler optimization flags. Default is
no. 
\item {\tt --enable-debug=yes|no} toggles generation of debugging information. 
Default is yes. 
\item {\tt --enable-boundscheck=yes|no} toggles bounds checking on dense vector 
element access. Default is yes. Bounds checking helps with debugging, but also incurs 
a significant performance penalty. 
\item {\tt --with-mpi-cxx} specifies the C++ compiler to be used with MPI. Default is
{\tt mpiCC}. 
\item {\tt FLIBS} specifies flags and libraries for linking Fortran. By default,
{\tt configure} will attempt to determine these automatically for your system; however,
it may occasionally guess wrong, in which case you much supply this information
manually. 
\item {\tt --with-libs} specifies any additional libraries needed for your build. Default
is none.
\item {\tt --with-include} specifies any additional include directories
needed for your build. Default is none.
\item {\tt CXX=}{\it C++ compiler} specifies the C++ compiler. By default, common choices
such as {\tt g++} and {\tt CC} are tried. 
\end{itemize}

\section{Building TSF}

Once you have configured TSF, building it is simple. Just type
{\tt make} at the command line in the TSF root directory, then get a cup of coffee. 

\section{Building the documentation}

To build the online documentation, you will need {\tt doxygen}. Once you have
installed doxygen, simply run {\tt make manual} in the TSF root directory. 

\section{Cleaning up}

To destroy all objects, dependencies, libraries, and executable files, do
{\tt make clean}. To destroy all that and more (such as the current configuration files
and documentation) do {\tt make spotless}. 

\end{document}
