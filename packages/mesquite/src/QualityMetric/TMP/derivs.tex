\documentclass{report}
\usepackage{amssymb}

\title{ Target Metrics and Derivatives }

%\author{Patrick Knupp \\
%The Sandia National Laboratories \\
%Albuquerque NM USA \\
%pknupp@sandia.gov}
%\author{Jason Kraftcheck \\
%The University of Wisconsin-Madison \\
%Madison, WI \\
%kraftche@cae.wisc.edu}

\date{28 October 2008}

\begin{document}

\maketitle
%\begin{abstract} 
%This document contains notes on TMP metrics and their derivaitves, \\
% including: notes on the first and second derivatives of scalar \\
% functions of matrices, a list of target metrics and their derivatives, \\
% and notes on calculating the derivatives with respect to element node \\
% coordinates given the corresponding deriviatives with of the target \\
% metrics.
%\end{abstract}

\tableofcontents

\chapter{Notation}

\section{Functions of Matrices}

Let $t_{i,j}$ denote the element in row $i$ and column $j$ of the matrix $T$.
Let $f=f(T)$ be a scalar function on the set of $n \times n$ real matrices 
$M_n$. \newline

\noindent {\it In the manipulations below it is assumed that $f$ and all of 
its first and second partial derivatives exist and are continuous. Some of the 
TMP metrics do not satisfy this requirement.} For example, the first partial
derivatives of the norm of $T$ to the first power (i.e. $|T|$) do not exist 
when $T=0$. 
They do exist everywhere else in $M_n$, however.  The first partial 
derivatives of 
the absolute value, $|det(T)|$, do not exist when $T$ is singular. Therefore,
special care must be taken for metrics which include these expressions. 
\newline

\noindent {\bf Outer Products} \newline
Let ${\bf a}=(a_1,...,a_n)$ and ${\bf z}=(z_1,...,z_n)$ be two vectors in 
$R^n$.  The outer product ${\bf a} \otimes {\bf z}$ is a $n \times n$ matrix 
with entries
\begin{eqnarray}
\left( {\bf a} \otimes {\bf z} \right)_{ij} = a_i \, z_j
\end{eqnarray}

\noindent Let $A^t = [{\bf a}_1,...,{\bf a}_n]$ and 
$Z^t = [{\bf z}_1,...,{\bf z}_n]$ be two $n \times n$ matrices,
with ${\bf a}_i$ and ${\bf z}_j$ being the $i^{th}$ and $j^{th}$ 
row vectors of $A$ and $Z$, respectively. Define the 
outer product $A \otimes Z$ of the two matrices to be an $n^2 \times n^2$ 
matrix with sub-matrices
\begin{eqnarray}
\label{matrixouter}
\left( A \otimes Z \right)_{ij} = {\bf a}_i \otimes {\bf z}_j
\end{eqnarray}

\noindent The previous can be expressed as
\begin{eqnarray}
A \otimes Z & = &
\left[ \begin{array}{ccc|c|ccc}
A_{11} Z_{11} &
\ldots &
A_{11} Z_{1n} &
&
A_{11} Z_{n1} &
\ldots &
A_{11} Z_{nn} \\
\vdots & \ddots & \vdots &
\ldots &
\vdots & \ddots & \vdots \\
A_{1n} Z_{11} &
\ldots &
A_{1n} Z_{1n} &
&
A_{1n} Z_{n1} &
\ldots &
A_{1n} Z_{nn} \\
\hline
& \vdots & & \ddots & & \vdots \\
\hline
A_{n1} Z_{11} &
\ldots &
A_{n1} Z_{1n} &
&
A_{n1} Z_{n1} &
\ldots &
A_{n1} Z_{nn} \\
\vdots & \ddots & \vdots &
\ldots &
\vdots & \ddots & \vdots \\
A_{nn} Z_{11} &
\ldots &
A_{nn} Z_{1n} &
&
A_{nn} Z_{n1} &
\ldots &
A_{nn} Z_{nn} 
\end{array} \right] \nonumber \label{matrix-outer} \\
\end{eqnarray}

\section{First Derivatives}

Let $\frac{\partial f}{\partial T}$ be the matrix of partial derivatives of the scalar function $f(T)$ where $T$ is a matrix, such that each element 
$\left(\frac{\partial f}{\partial T}\right)_{i,j} = \frac{\partial f}{\partial \, t_{i,j}}$.

\noindent Let $f=f(T)$ and $g=g(T)$
\begin{eqnarray}
\frac{\partial}{\partial T} \left( f \, g \right) & = & f \frac{\partial g}{\partial T} + g \frac{\partial f}{\partial T} \label{prod1} \\
\frac{\partial}{\partial T} \left( \frac{f}{g} \right) & = & \frac{1}{g^2} \left\{ g \frac{\partial f}{\partial T} - f \frac{\partial g}{\partial T} \right\} \label{quotient1}
\end{eqnarray}


\section{Second Derivatives}

Let $\frac{\partial^2 f}{\partial T_i \partial T_j}$ be the matrix of second partial derivatives with respect to the $i^{th}$ and $j^{th}$ rows of $T$:

\begin{equation}
\frac{\partial^2 f}{\partial T_i \partial T_j} = 
\left[ \begin{array}{cccc}
\frac{\partial^2 f}{\partial t_{i,1} \partial t_{j,1}} &
\frac{\partial^2 f}{\partial t_{i,1} \partial t_{j,2}} &
\ldots &
\frac{\partial^2 f}{\partial t_{i,1} \partial t_{j,n}} \\
\frac{\partial^2 f}{\partial t_{i,2} \partial t_{j,1}} &
\frac{\partial^2 f}{\partial t_{i,2} \partial t_{j,2}} \\
\vdots & & \ddots \\
\frac{\partial^2 f}{\partial t_{i,n} \partial t_{j,1}} & & &
\frac{\partial^2 f}{\partial t_{i,n} \partial t_{j,n}}
\end{array} \right]
\end{equation}

and

\begin{equation}
\label{dsecelem}
\left(\frac{\partial^2 f}{\partial T_i \partial T_j}\right)_{k,l} = 
\frac{\partial^2 f}{\partial t_{i,k} \partial t_{j,l}}
\end{equation}

Thus the symmetric Hessian of $f(T)$, with the elements of $T$ occuring in row-major order, is composed of blocks of the form $\frac{\partial^2 f}{\partial T_i \partial T_j}$:

\begin{eqnarray}
\frac{\partial^2}{\partial T^2} f(T) & = &
\left[ \begin{array}{c|c|c}
\frac{\partial^2 f}{\partial T_1 \partial T_1} &
\ldots &
\frac{\partial^2 f}{\partial T_1 \partial T_n} \\
\hline
\vdots & \ddots \\
\hline
\frac{\partial^2 f}{\partial T_1 \partial T_n} & 
\ldots & 
\frac{\partial^2 f}{\partial T_n \partial T_n} \\
\end{array} \right]
\\ & = &
\left[ \begin{array}{ccc|c|ccc}
\frac{\partial^2 f}{\partial t_{1,1}^2} &
\ldots &
\frac{\partial^2 f}{\partial t_{1,1} \partial t_{1,n}} &
&
\frac{\partial^2 f}{\partial t_{1,1} \partial t_{n,1}} &
\ldots &
\frac{\partial^2 f}{\partial t_{1,1} \partial t_{n,n}} \\
\vdots & \ddots & \vdots &
\ldots &
\vdots & \ddots & \vdots \\
\frac{\partial^2 f}{\partial t_{1,n} \partial t_{1,1}} &
\ldots &
\frac{\partial^2 f}{\partial t_{1,n}^2} &
&
\frac{\partial^2 f}{\partial t_{1,n} \partial t_{n,1}} &
\ldots &
\frac{\partial^2 f}{\partial t_{1,n} \partial t_{n,n}} \\
\hline
& \vdots & & \ddots & & \vdots \\
\hline
\frac{\partial^2 f}{\partial t_{n,1} \partial t_{1,1}} &
\ldots &
\frac{\partial^2 f}{\partial t_{n,1} \partial t_{1,n}} &
&
\frac{\partial^2 f}{\partial t_{n,1}^2} &
\ldots &
\frac{\partial^2 f}{\partial t_{n,1} \partial t_{n,n}} \\
\vdots & \ddots & \vdots &
\ldots &
\vdots & \ddots & \vdots \\
\frac{\partial^2 f}{\partial t_{n,n} \partial t_{1,1}} &
\ldots &
\frac{\partial^2 f}{\partial t_{n,n} \partial t_{1,n}} &
&
\frac{\partial^2 f}{\partial t_{n,n} \partial t_{n,1}} &
\ldots &
\frac{\partial^2 f}{\partial t_{n,n}^2} 
\end{array} \right] \nonumber \label{2nd-derv} \\
\end{eqnarray}

%\noindent The second derivatives can also be expressed in terms of 
%matrix outer products and operators. For $T$ and $Z$ in $M_n$, define the 
%derivative
%\begin{eqnarray}
%\frac{\partial Z}{\partial T} = \left( \frac{\partial}{\partial T} \otimes Z \right)^t
%\end{eqnarray}
%so that 
%\begin{eqnarray}
%\frac{\partial Z}{\partial T} & = &
%\left[ \begin{array}{ccc|c|ccc}
%\frac{\partial z_{1,1}}{\partial t_{1,1}} &
%\ldots &
%\frac{\partial z_{1,1}}{\partial t_{1,n}} &
%&
%\frac{\partial z_{1,1}}{\partial t_{n,1}} &
%\ldots &
%\frac{\partial z_{1,1}}{\partial t_{n,n}} \\
%\vdots & \ddots & \vdots &
%\ldots &
%\vdots & \ddots & \vdots \\
%\frac{\partial z_{1,n}}{\partial t_{1,1}} &
%\ldots &
%\frac{\partial z_{1,n}}{\partial t_{1,n}} &
%&
%\frac{\partial z_{1,n}}{\partial t_{n,1}} &
%\ldots &
%\frac{\partial z_{1,n}}{\partial t_{n,n}} \\
%\hline
%& \vdots & & \ddots & & \vdots \\
%\hline
%\frac{\partial z_{n,1}}{\partial t_{1,1}} &
%\ldots &
%\frac{\partial z_{n,1}}{\partial t_{1,n}} &
%&
%\frac{\partial z_{n,1}}{\partial t_{n,1}} &
%\ldots &
%\frac{\partial z_{n,1}}{\partial t_{n,n}} \\
%\vdots & \ddots & \vdots &
%\ldots &
%\vdots & \ddots & \vdots \\
%\frac{\partial z_{n,n}}{\partial t_{1,1}} &
%\ldots &
%\frac{\partial z_{n,n}}{\partial t_{1,n}} &
%&
%\frac{\partial z_{n,n}}{\partial t_{n,1}} &
%\ldots &
%\frac{\partial z_{n,n}}{\partial t_{n,n}} 
%\end{array} \right] 
%\end{eqnarray}

%\noindent If one lets $Z = \frac{\partial f}{\partial T}$ in the previous,
%one obtains \ref{2nd-derv}. \newline

\noindent {\bf Product and Quotient Rules for the Second Derivative} \newline

\noindent Let $f=f(T)$ and $g=g(T)$.
\begin{eqnarray}
\frac{\partial^2}{\partial T^2} \left( f \, g \right) & = & f \, \frac{\partial^2 g}{\partial T^2} + g \, \frac{\partial^2 f}{\partial T^2} + \frac{\partial f}{\partial T} \otimes \frac{\partial g}{\partial T} +  \frac{\partial g}{\partial T} \otimes \frac{\partial f}{\partial T} \label{prod2} \\
\frac{\partial}{\partial T} \left( f \frac{\partial g}{\partial T} \right) & = & f \, \frac{\partial^2 g}{\partial T^2} + \frac{\partial f}{\partial T} \otimes \frac{\partial g}{\partial T} \label{diffusionopp} \\
\frac{\partial^2}{\partial T^2} \left( \frac{f}{g} \right) & = & \frac{1}{g} \, \frac{\partial^2 f}{\partial T^2} - \frac{1}{g^2} \left\{ f \, \frac{\partial^2 g}{\partial T^2} - \frac{2f}{g} \left( \frac{\partial g}{\partial T} \otimes \frac{\partial g}{\partial T} \right) + \frac{\partial f}{\partial T} \otimes \frac{\partial g}{\partial T} +  \frac{\partial g}{\partial T} \otimes \frac{\partial f}{\partial T} \right\} \nonumber \label{quotient2} \\
\end{eqnarray}

\chapter{Basic Matrix Derivatives}


\section{Chain rule for matrix multiplication}

\subsection{\label{first}First Derivative}

The following demonstrates that for the product of square matrices $T = A Z$,
that $\frac{\partial f(T)}{\partial A} = \frac{\partial f(T)}{\partial T} Z^t$. \newline

\noindent Beginning with the chain rule:

\begin{equation} 
\label{eqn0}
\frac{\partial f}{\partial a_{i,j}} = 
\sum_{l=1}^{n} \sum_{k=1}^{n} 
\frac{\partial f}{\partial t_{l,k}} 
\frac{\partial t_{l,k}}{\partial a_{i,j}}
\end{equation}

\noindent Since $T=AZ$, matrix multiplication gives:

\begin{equation}
\label{eqn1}
t_{l,k} = \sum_{r=1}^n a_{l,r} z_{r,k}
\end{equation}

\noindent Taking the first derivative of Equation \ref{eqn1}:

\begin{eqnarray}
%\label{eqn2}
\frac{\partial t_{l,k}}{\partial a_{i,j}} 
  &=& \sum_{r=1}^n \delta_i^l \delta_j^r z_{r,k} \\
  &=& \delta_i^l z_{j,k} \label{eqn2}
\end{eqnarray}

\noindent Substituting Equation \ref{eqn2} into \ref{eqn0}:

\begin{eqnarray}
\frac{\partial f}{\partial a_{i,j}} &=& \sum_{l=1}^n \sum_{k=1}^n
 \frac{\partial f}{\partial t_{l,k}} \delta_i^l z_{j,k} \\
&=& \sum_{k=1}^{n} \frac{ \partial f}{ \partial t_{i,k}} z_{j,k} \label{eqn4}
\end{eqnarray}

\noindent which is:

\begin{equation}
\frac{\partial f}{\partial A} = \frac{\partial f}{\partial T} Z^t
\end{equation}


\subsection{Second Derivative}

Define:

\begin{equation}
C = \frac{\partial^2 f}{\partial A_i \partial A_j}
\end{equation}

\noindent for some $i$ and $j$ such that from Equation \ref{dsecelem}:

\begin{equation}
\label{chainsecondelem}
c_{k,l} = \frac{\partial^2 f}{\partial a_{i,k} \partial a_{j,l}}
\end{equation}

\noindent Substituting Equation \ref{eqn4} into Equation \ref{chainsecondelem}:

\begin{eqnarray}
c_{k,l} &=& \frac{\partial}{\partial a_{i,k}} \sum_{p=1}^n \frac{ \partial f}{\partial t_{j,p}} z_{l,p} \\
        &=& \sum_{p=1}^n \frac{\partial}{\partial t_{j,p}} \frac{\partial f}{\partial a_{i,k}} z_{l,p} \\
\label{chainsecondr1}
        &=& \sum_{p=1}^n \sum_{q=1}^n \frac{\partial^2 f}{ \partial t_{j,p} \partial t_{i,q}} z_{l,p} z_{k,q} 
\end{eqnarray}

\noindent Let:

\begin{equation}
D = \frac{\partial^2 f}{\partial T_i \partial T_j}
\end{equation}
which, given the definition in Equation \ref{dsecelem}:
\begin{equation}
d_{q,p} = \frac{\partial^2 f}{\partial t_{i,q} \partial t_{j,p}}
\end{equation}

\noindent Define matrices $X$ and $Y$:

\begin{eqnarray}
X &=& Z D \\
Y &=& X Z^t = Z D Z^t
\end{eqnarray}

\noindent The elements of $X$ and $Y$ are then:

\begin{eqnarray}
x_{k,p} & = & \sum_{q=1}^n z_{k,q} d_{q,p} \\
y_{k,l} & = & \sum_{p=1}^n x_{k,p} z_{l,p} \\ 
        & = & \sum_{p=1}^n \sum_{q=1}^n z_{k,q} d_{q,p} z_{l,p} \\
\label{chainsecondr2}
        & = & \sum_{p=1}^n \sum_{q=1}^n \frac{\partial^2 f}{\partial t_{j,p} \partial t_{i,q}} z_{l,p} z_{k,q}
\end{eqnarray}

\noindent Equation \ref{chainsecondr1} is equal to Equation \ref{chainsecondr2}, so:

\begin{eqnarray}
\frac{\partial^2 f}{\partial A_i \partial A_j} &=& C \\
&=& Y \\
&=& Z D Z^t \\
&=& Z \frac{\partial^2 f}{\partial T_i \partial T_j} Z^t 
\end{eqnarray}


\section{Derivatives of Basic TMP Scalar Functions}
All TMP metrics are functions of the matrices $A$, $Z$, or $T=AZ$.
Each such metric in turn is some combination of the basic scalar functions
$|T|^m$, $det(T)$, $tr(T)$, $tr(adj \, T)$, $|T^t T|$, or $|adj T|^2$.  
Therefore, derivatives of these basic TMP functions will be derived first,
to serve as the building blocks for TMP metrics.  \newline

\noindent Throughout this document, let $\tau = det(T)$. Some of the 
metrics include a user-supplied constant, $\gamma$. \newline

\subsection{Derivatives of $tr(T)$}

\begin{eqnarray}
f(T) & = & tr(T) = \sum_{k=1}^n t_{k,k} \\
\frac{\partial f}{\partial t_{i,j}} &=& \delta_i^j \\
\frac{\partial f}{\partial T} &=& I 
\end{eqnarray}

\noindent All of the second derivatives are zero. \newline

\subsection{Derivatives of $tr(adj \, T)$ \label{traj}}

\noindent For $n=2$, $tr(adj \, T) = tr(T)$, so the results of 
the previous section can be used. \newline

\noindent For $n=3$, 
\begin{eqnarray}
tr( adj \, T ) = (t_{11} t_{22} - t_{12} t_{21})+ (t_{22} t_{33} - t_{23} t_{32})+ (t_{33} t_{11} - t_{13} t_{31})
\end{eqnarray}

\begin{eqnarray}
\frac{\partial}{\partial T} \, tr(adj \, T) = tr(T) \, I - T^t
\end{eqnarray}

\begin{equation}
\label{tradj2}
\frac{\partial^2}{\partial T^2} \, tr(adj \, T) =
\left[ \begin{array}{ccccccccc}
 0 & 0 & 0 & 0 & 1 & 0 & 0 & 0 & 1 \\
 0 & 0 & 0 & -1 & 0 & 0 & 0 & 0 & 0 \\
 0 & 0 & 0 & 0 & 0 & 0 & -1 & 0 & 0 \\
 0 & -1 & 0 & 0 & 0 & 0 & 0 & 0 & 0 \\
 1 & 0 & 0 & 0 & 0 & 0 & 0 & 0 & 1 \\
 0 & 0 & 0 & 0 & 0 & 0 & 0 & -1 & 0 \\
 0 & 0 & -1 & 0 & 0 & 0 & 0 & 0 & 0 \\
 0 & 0 & 0 & 0 & 0 & -1 & 0 & 0 & 0 \\
 1 & 0 & 0 & 0 & 1 & 0 & 0 & 0 & 0
\end{array} \right]
\end{equation}


\subsection{Derivatives of $|T|^2$ \label{tsquared}}

\begin{eqnarray}
f(T) &=& |T|^2 = \sum_{i=1}^n \sum_{j=1}^n t_{i,j}^2 \\
\label{tsquaredterm}
\frac{\partial f}{\partial t_{i,j}} &=& 2 \, t_{i,j} \\
\frac{\partial f}{\partial T} &=& 2 \, T
\end{eqnarray}

\noindent Beginning with Equation \ref{tsquaredterm}:
\begin{eqnarray}
\frac{\partial^2 f}{\partial t_{i,j} \partial t_{k,l}} &=& 2 \delta_i^k \delta_j^l \\
\left(\frac{\partial^2 f}{\partial T_i \partial T_j}\right)_{r,s} &=& 2 \delta_i^j \delta_r^s \\
\frac{\partial^2 f}{\partial T_i \partial T_j} &=& 2 \delta_i^j I
\end{eqnarray}

\noindent All non-diagonal blocks of the Hessian are zero due to the $\delta_i^j$.  Thus, if $I_{n^2}$ is the $n^2 \times n^2$ identity matrix, then:

\begin{equation}
\frac{\partial^2}{\partial T^2} \, |T|^2 = 2 \, I_{n^2}
\end{equation}

\subsection{Derivatives of $f(T)=|T|^m$}
Using \ref{tsquaredterm},
\begin{eqnarray}
2 \, T & = & \frac{\partial}{\partial T} |T|^2 \nonumber \\
 & = & 2 \, |T| \frac{\partial}{\partial T} |T| \nonumber
\end{eqnarray}
so that when $T \neq 0$,
\begin{eqnarray}
\frac{\partial}{\partial T} |T| & = & \frac{T}{|T|}
\end{eqnarray}
Let $m=\pm 1, \pm 2,\pm 3,\ldots$  Then
\begin{eqnarray}
\frac{\partial}{\partial T} |T|^m & = & m \, |T|^{m-1} \frac{\partial }{\partial T}|T| \\
 & = & m \, |T|^{m-2} \, T \label{firstdervoftm}
\end{eqnarray}
For $m=2,3,\ldots$, the latter formula is valid even when $T=0$. \newline

\noindent Particularly useful are
\begin{eqnarray}
\label{tcubed}
\frac{\partial}{\partial T} |T|^3 & = & 3 \, |T| \, T \\
\frac{\partial}{\partial T} |T|^4 & = & 4 \, |T|^2 \, T 
\end{eqnarray}

\noindent {\bf Second Partial Derivatives of $f(T)=|T|^m$} \newline

\noindent Write \ref{firstdervoftm} in index form as
\begin{eqnarray}
\frac{\partial}{\partial t_{i,j}} |T|^m & = & m |T|^{m-2} \, t_{i,j}
\end{eqnarray}
Then
\begin{eqnarray}
\frac{\partial^2 f}{\partial t_{i,j} \partial t_{k,l}} = m \, \delta_i^k \delta_j^l |T|^{m-2} + m (m-2) \, t_{i,j} t_{k,l} |T|^{m-4}
\end{eqnarray}
and so we can write
\begin{eqnarray}
\frac{\partial^2}{\partial T^2} \, |T|^m = m \, \left\{ |T|^{m-2} \, I_{n^2} + (m-2) \, |T|^{m-4} \, \left( T \otimes T \right) \, \right\} \label{hesstm}
\end{eqnarray}
%and, for $m \neq 3$,
%\begin{eqnarray}
%\label{hesstm}
%\frac{\partial^2 f}{\partial T_i \partial T_j} = m \,  |T|^{m-4} \, \left\{ \delta_i^j \, |T|^2 \, I + (m-2) S_{ij} \right\} 
%\end{eqnarray}
%with $S_{ij}$ an $n \times n$ matrix with entries
%\begin{eqnarray}
%\left( S_{ij} \right)_{r,s} = t_{i,r} \, t_{j,s} \label{sij}
%\end{eqnarray}
When $m=2$ or $m \geq 4$ the second derivatives exist for all $T \in M_3$.
Otherwise, the second derivative exists everywhere except $T=0$. 
The second derivative for $m=3$ is
\begin{eqnarray}
\label{hesst3}
\frac{\partial^2}{\partial T^2} \, |T|^3 = 3 \, \left\{ |T| \, I_{n^2} + |T|^{-1} \, \left( T \otimes T \right) \, \right\}
%\frac{\partial^2}{\partial T_i \partial T_j} \, |T|^3 = 3 \,  \delta_i^j \, |T| \, I 
\end{eqnarray}
%If $m<2$, the second derivative exists everywhere except $T=0$. \newline

\subsection{Derivatives of $f(T)=|T^t T|^2$}

\noindent The first derivative is
\begin{eqnarray}
\frac{\partial}{\partial T} \, |T^t T|^2 = 4 T \, T^t T \label{dttt}
\end{eqnarray}

\noindent To obtain the second derivatives, write
\begin{eqnarray}
\left( T^t T \right)_{r,s} = \sum_{\ell=1}^n t_{\ell,r} t_{\ell,s} 
\end{eqnarray}
and so
\begin{eqnarray}
\left( T \, T^t T \right)_{p,q} & = & \sum_{k=1}^n (T)_{p,k} \left( T^t T \right)_{k,q} \\
 & = & \sum_{k=1}^n t_{p,k} \left( \sum_{\ell=1}^n t_{\ell,k} t_{\ell,q} \right)
\end{eqnarray}
Then \ref{dttt} can be written
\begin{eqnarray}
\frac{\partial}{\partial t_{p,q}} |T^t T|^2 = 4 \, \sum_{k=1}^n \sum_{\ell=1}^n  t_{p,k} t_{\ell,k} t_{\ell,q} 
\end{eqnarray}
Next,
\begin{eqnarray} 
\frac{\partial^2 f}{\partial t_{p,q} \partial t_{i,j}} & = & 4\, \sum_{k=1}^n \sum_{\ell=1}^n  \left\{ \delta_i^\ell \, \delta_j^q \, t_{p,k} \, t_{\ell,k} + \delta_i^\ell \, \delta_j^k \, t_{p,k} \, t_{\ell,q} + \delta_i^p \, \delta_j^k \, t_{\ell,q} \, t_{\ell,k} \right\} \nonumber \\
 & = & 4 \left\{ t_{p,j} \, t_{i,q} + \sum_{k=1}^n \left( \delta_j^q \, t_{p,k} \, t_{i,k} + \delta_i^p \, t_{k,q} \, t_{k,j} \right) \right\} \\
& = & 4 \left\{ \left( S_{pi} \right)_{j,q} + \delta_j^q (T T^t)_{p,i} + \delta_i^p (T^t T)_{q,j}\right\}
\end{eqnarray}
and finally
\footnote{Beware: the notation on the left-hand-side of this equation has a problem: namely that permuting the indices $i$ and $j$ 
does not give the same result on the left when one would expect it to. In fact
\begin{eqnarray}
\frac{\partial^2 f}{\partial T_j \partial T_i} = \left( \frac{\partial^2 f}{\partial T_i \partial T_j} \right)^t
\end{eqnarray}
whereas in standard partial derivatives, one can perform the switch of the 
indices without needing the transpose.} 
\begin{eqnarray}
\label{hessttt}
\frac{\partial^2 f}{\partial T_i \partial T_j} = 4 \left( S_{ji} + (T T^t)_{i,j} \, I + \delta_j^i \, T^t T \right)
\end{eqnarray}
with $S_{ij}$ an $n \times n$ matrix with entries
\begin{eqnarray}
\left( S_{ij} \right)_{r,s} = t_{i,r} \, t_{j,s} \label{sij}
\end{eqnarray}
or 
\begin{eqnarray}
S &=& T \otimes T
\end{eqnarray}
\newline

\subsection{Derivatives of $f(T)=|adj T|^2$ \label{nrm-adj}}
For $T_{2 \times 2}$, 
\begin{eqnarray}
|adj T|^2 = |T|^2 
\end{eqnarray}
so the first and second derivatives are given in Section \ref{tsquared}. \newline

\noindent For $T_{3 \times 3}$, the following identity holds
\begin{eqnarray}
2 \, |adj T|^2 = |T|^4 - |T^t T|^2
\end{eqnarray}
Taking the partial derivatives of both sides of the identity and using results 
from the previous sections, the first derivative is
\begin{eqnarray}
\frac{\partial}{\partial T} |adj T|^2 = 2 \, \{ |T|^2 T - T T^t T \}
\end{eqnarray}
and, using \ref{hesstm} and \ref{hessttt}, the second derivative is
\begin{eqnarray}
\frac{\partial^2 f}{\partial T_i \partial T_j} & = & \frac{1}{2} \frac{\partial^2 }{\partial T_i \partial T_j} |T|^4 - \frac{1}{2} \frac{\partial^2}{\partial T_i \partial T_j} |T^t T|^2 \nonumber \\
 & = & 2 \, \{ \delta_i^j |T|^2 I + 2 S_{ij} \} - 2 \, \{ S_{ji} + (T T^t)_{i,j} I + \delta_j^i T^t T \} \nonumber \\
 & = & 2 \, \left\{ \left( \delta_i^j \, |T|^2 - (T T^t)_{i,j} \right) I + 2 S_{ij}-S_{ji} - \delta_j^i T^t T \right\} \label{hess-nrm-adj}
\end{eqnarray}
with $S_{ij}$ defined in \ref{sij}. \newline

\subsection{First Partial Derivative of $\tau=det(T)$ \label{tau}}

\begin{eqnarray}
f(T) &=& det(T) \\
     &=& \sum_{k=1}^n t_{l,k} C_{l,k}(T) \quad \forall \quad 1 \le l \le n \\
\label{det1}
     &=& \sum_{k=1}^n (-1)^{l+k} t_{l,k} M_{l,k}(T) \quad \forall \quad 1 \le l \le n
\end{eqnarray}

\noindent For each term of $\frac{\partial f}{\partial T}$, take the partial derivative of of Equation \ref{det1} for $l = i$ with respect to the corresponding element of $T$:

\begin{eqnarray}
\frac{\partial f}{\partial t_{i,j}} &=& 
\sum_{k=1}^n (-1)^{i+k} \left[
 \frac{\partial t_{i,k}}{\partial t_{i,j}} M_{i,k}(T) + 
 \frac{\partial M_{i,k}}{\partial t_{i,k}} t_{i,k} \right] \\
 &=& \sum_{k=1}^n (-1)^{i+k} \left[
 \delta^k_j M_{i,k}(T) + 
 \frac{\partial M_{i,k}}{\partial t_{i,k}} t_{i,k} \right] \\
\label{det2}
 &=& (-1)^{i+j} M_{i,j}(T) + \sum_{k=1}^n (-1)^{i+k} \frac{\partial 
   M_{i,k}}{\partial t_{i,k}} t_{i,k}
\end{eqnarray}

\noindent By definition, the minor for row $i$ and column $j$ contains terms from neither row $i$ nor column $j$. Therefore the partial derivative of a minor with respect to any term from either the corresponding row or column of the original matrix is zero:
\begin{equation}\label{det3}
\frac{\partial M_{i,j}}{\partial t_{i,k}} = 
\frac{\partial M_{i,j}}{\partial t_{k,j}} = 0 \quad \forall \quad k
\end{equation}

\noindent Thus all the terms in the summation in Equation \ref{det2} are zero, leaving:

\begin{equation}\label{det_deriv_term}
\frac{\partial f}{\partial t_{i,j}} = (-1)^{i+j}M_{i,j}(T)
\end{equation}
which is the definition of the cofactor matrix of $T$, and the transpose of the definition of the adjoint of $T$:

\begin{equation}
\label{adjform}
\frac{\partial \tau}{\partial T} = (adj(T))^t = adj(T^t)
\end{equation}

\subsection{Second Partial Derivative of $\tau=det(T)$}

The partial derivative of Equation \ref{det_deriv_term} is:

\begin{equation}
\frac{\partial^2 f}{\partial t_{i,j} \partial t_{k,l}} = (-1)^{i+j}\frac{\partial M_{i,j}(T)}{\partial t_{k,l}}
\end{equation}

\noindent For a $2 \times 2$ matrix, $M_{i,j}(T) = t_{3-i,3-j}$:
\begin{equation}
\frac{\partial^2 f}{\partial t_{i,j} \partial t_{k,l}} = 
(-1)^{i+j} \delta_{3-i}^k \delta_{3-j}^l
\end{equation}

\begin{equation}
\left(\frac{\partial^2 f}{\partial T_i \partial T_j}\right)_{r,s} =
(-1)^{i+r} \delta_{3-i}^j \delta_{3-r}^s
\end{equation}

\begin{equation}
\label{2ndtau2d}
\frac{\partial^2 \, \tau}{\partial T^2} =
\left[ \begin{array}{cccc}
0 & 0 & 0 & 1 \\
0 & 0 & -1 & 0 \\
0 & -1 & 0 & 0 \\
1 & 0 & 0 & 0 
\end{array} \right]
\end{equation}

\noindent For a $3 \times 3$ matrix, the Hessian of the determinant is:

\begin{equation}
\label{2ndtau3d}
\frac{\partial^2 \tau}{\partial T^2} =
\left[ \begin{array}{ccccccccc}
 0 & 0 & 0 & 0 & t_{3,3} & -t_{3,2} & 0 & -t_{2,3} & t_{2,2} \\
 0 & 0 & 0 & -t_{3,3} & 0 & t_{3,1} & t_{2,3} & 0 & -t_{2,1} \\
 0 & 0 & 0 & t_{3,2} & -t_{3,1} & 0 & -t_{2,2} & t_{2,1} & 0 \\
 0 & -t_{3,3} & t_{3,2} & 0 & 0 & 0 & 0 & t_{1,3} & -t_{1,2} \\
 t_{3,3} & 0 & -t_{3,1} & 0 & 0 & 0 & -t_{1,3} & 0 & t_{1,1} \\
 -t_{3,2} & t_{3,1} & 0 & 0 & 0 & 0 & t_{1,2} & -t_{1,1} & 0 \\
 0 & t_{2,3} & -t_{2,2} & 0 & -t_{1,3} & t_{1,2} & 0 & 0 & 0 \\
 -t_{2,3} & 0 & t_{2,1} & t_{1,3} & 0 & -t_{1,1} & 0 & 0 & 0 \\
 t_{2,2} & -t_{2,1} & 0 & -t_{1,2} & t_{1,1} & 0 & 0 & 0 & 0
\end{array} \right]
\end{equation}

\noindent Each non-diagonal $3 \times 3$ block of the $3 \times 3$ result can be expressed as:

\begin{equation}
\label{3dtauhessblock}
\frac{\partial^2 f}{\partial T_i \partial T_j}
 = s\left[\begin{array}{ccc}
0 & t_{k,3} & -t_{k,2} \\
-t_{k,3} & 0 & t_{k,1} \\
t_{k,2} & -t_{k,1} & 0 
\end{array}\right]\end{equation}
where 
\begin{eqnarray}
k &=& (3-i) + (3-j) \\
s &=& (-1)^{i+j}\frac{i-j}{|i-j|}
\end{eqnarray}

\noindent Both the $2 \times 2$ and $3 \times 3$ results are composed of skew-symmetric $n \times n$ submatrices.  For the $3 \times 3$ result, the terms in a submatrix are the values from a single row of $T$.  The diagonal submatrices for both results are zero, and the diagonal terms of each submatrix are zero.

\subsection{Derivatives of other Quantities Involving $\tau$}

\subsubsection{Derivatives of $f(T)=\tau^\gamma$ \label{taugamma}}

For any $\gamma \neq 0$ 
\begin{eqnarray}
\frac{\partial}{\partial T} \, \tau^\gamma = \gamma \, \tau^{\gamma-1} \, (adj T)^t
\end{eqnarray}
If $\gamma<1$, then the formula is only valid for $\tau \neq 0$. 
Otherwise, the derivative at $\tau=0$ does not exist. \newline

\noindent If $\tau \neq 0$, one can also write
\begin{eqnarray}
\frac{\partial}{\partial T} \, \tau^\gamma = \gamma \, \tau^{\gamma} \, T^{-t}
\end{eqnarray}

\noindent The second derivative is
\begin{eqnarray}
\frac{\partial^2}{\partial T^2} \tau^\gamma & = & \frac{\partial}{\partial T} \left( \frac{\partial}{\partial T} \tau^\gamma \right) \nonumber \\
 & = &  \frac{\partial}{\partial T} \left( \gamma \tau^{\gamma-1} (adj T)^t \right) \nonumber \\
 & = & \gamma \,  \frac{\partial}{\partial T} \left( \tau^{\gamma-1} \, \frac{\partial \tau}{\partial T} \right) \nonumber \\
 & = & \gamma \, \left\{ \tau^{\gamma-1} \frac{\partial^2 \tau}{\partial T^2} + (\gamma-1) \, \tau^{\gamma-2} \left( \frac{\partial \tau}{\partial T} \otimes \frac{\partial \tau}{\partial T} \right) \right\}
\end{eqnarray}
To complete this expression, use formulas \ref{adjform} and either 
\ref{2ndtau2d} or \ref{2ndtau3d}. \newline

\noindent If $\gamma < 2$, then the second derivative formula is only valid for $\tau \neq 0$. \newline

\subsubsection{Derivatives of $f(T)=(\tau-1)^2$}

\begin{eqnarray}
\frac{\partial}{\partial T} \, (\tau-1)^2 & = & 2 \, (\tau-1) \, (adj T)^t  \label{tauminus1} \\
\frac{\partial^2}{\partial T^2} \, (\tau-1)^2 & = & 2 \, \left\{ (\tau-1) \frac{\partial^2 \tau}{\partial T^2} + \left( \frac{\partial \tau}{\partial T} \otimes \frac{\partial \tau}{\partial T} \right) \right\}  \label{tm1}
\end{eqnarray}
To complete this expression, use formulas \ref{adjform} and either 
\ref{2ndtau2d} or \ref{2ndtau3d}. \newline

\subsubsection{Derivatives of $f(T)=\ln(\tau)$}

For $\tau>0$
\begin{eqnarray}
\frac{\partial}{\partial T} \, \ln(\tau) & = & \frac{1}{\tau} \, (adj T)^t \\
 & = & T^{-t}
\end{eqnarray}

\begin{eqnarray}
\frac{\partial^2}{\partial T^2} \ln(\tau) & = & \frac{1}{\tau} \frac{\partial^2 \tau}{\partial T^2} - \frac{1}{\tau^2} \left( \frac{\partial \tau}{\partial T} \otimes \frac{\partial \tau}{\partial T} \right)
\end{eqnarray}
To complete this expression, use formulas \ref{adjform} and either 
\ref{2ndtau2d} or \ref{2ndtau3d}. \newline

\subsubsection{Derivatives of $f(T)=|\tau-\gamma|$ \label{abstau}}

\begin{eqnarray}
\frac{\partial}{\partial T} \, |\tau-\gamma| & = & \left\{ \begin{array}{cc} 
+(adj T)^t & \mbox{if $\tau> \gamma$} \\
-(adj T)^t & \mbox{if $\tau< \gamma$}
\end{array} \right.
\end{eqnarray}
The first derivative is undefined if $\tau=\gamma$. \newline

\begin{eqnarray}
\frac{\partial^2}{\partial T^2} |\tau-\gamma| & = & \left\{ \begin{array}{cc} 
+\frac{\partial^2 \tau}{\partial T^2} & \mbox{if $\tau>\gamma$} \\
-\frac{\partial^2 \tau}{\partial T^2} & \mbox{if $\tau<\gamma$} \\
\end{array} \right.
\end{eqnarray}
To complete this expression, use either formula 
\ref{2ndtau2d} or \ref{2ndtau3d}. \newline

\noindent The second derivative is undefined if $\tau=\gamma$. \newline

% \section{Derivatives of $T^p$}
% 
% Define $T^p$ for any positive integer $p$ to be the product of $p$ instances of the matrix $T$.  Define $T^0 = I$.  The derivative of the first power is obvious:
% \begin{eqnarray}
% T^1 &=& T \\
% (T^1)_{i,j} &=& t_{i,j} \\
% \frac{\partial (T^1)_{i,j}}{\partial t_{r,s}} &=& \delta^i_r \delta^j_s \\
% \frac{\partial T^1}{\partial t_{r,s}} &=& E_{r,s}
% \end{eqnarray}
% 
% For the second power of $T$:
% 
% \begin{eqnarray}
% T^2 &=& T T \\
% (T^2)_{i,j} &=& \sum_{k=1}^n t_{i,k} t_{k,j} \\
% \frac{\partial (T^2)_{i,j}}{\partial t_{r,s}} &=& 
%   \sum_{k=1}^n \left( \delta^i_r \delta^k_s t_{k,j} 
%   + t_{i,k} \delta^k_R \delta^j_s 
%   \right) \\
% \frac{\partial T^2}{\partial t_{r,s}} &=& \frac{\partial T}{\partial t_{r,s}} T 
%   + T \frac{\partial T}{\partial t_{r,s}} 
% \end{eqnarray}
% 
% And for the third power of $T$:
% 
% \begin{eqnarray}
% T^3 &=& T T T\\
% (T^3)_{i,j} &=& \sum_{p=1}^n \sum_{k=1}^n t_{i,k} t_{k,p} t_{p,j} \\
% \frac{\partial (T^3)_{i,j}}{\partial t_{r,s}} &=& 
%   \sum_{p=1}^n \sum_{k=1}^n \left( 
%   \delta^i_r \delta^k_s t_{k,p} t_{p,j} 
%   + t_{i,k} \delta^k_r \delta^p_s t_{p,j} 
%   + t_{i,k} t_{k,p} \delta^p_r \delta^j_s 
%   \right)\\
% \frac{\partial T^3}{\partial t_{r,s}} &=& \frac{\partial T}{\partial t_{r,s}} T T + T \frac{\partial T}{\partial t_{r,s}} T + T T \frac{\partial T}{\partial t_{r,s}} 
% \end{eqnarray}
% 
% And for the fourth power of $T$:
% 
% \begin{eqnarray}
% T^4 &=& T T T T\\
% (T^4)_{i,j} &=& \sum_{q=1}^n \sum_{p=1}^n \sum_{k=1}^n t_{i,k} t_{k,p} t_{p,q} t_{q,j} \\
% \nonumber 
% \frac{\partial (T^4)_{i,j}}{\partial t_{r,s}} &=& 
%   \sum_{q=1}^n \sum_{p=1}^n \sum_{k=1}^n ( 
%    \delta^i_r \delta^k_s t_{k,p} t_{p,q} t_{q,j} 
%    + t_{i,k} \delta^k_r \delta^p_s t_{p,q} t_{q,j} \\ & & {} 
%    + t_{i,k} t_{k,p} \delta^p_r \delta^q_s t_{q,j} 
%    + t_{i,k} t_{k,p} t_{p,q} \delta^q_r \delta^j_s)\\
% \frac{\partial T^4}{\partial t_{r,s}} &=& 
%   \frac{\partial T}{\partial t_{r,s}} T T T 
%   + T \frac{\partial T}{\partial t_{r,s}} T T 
%   + T T \frac{\partial T}{\partial t_{r,s}} T 
%   + T T T \frac{\partial T}{\partial t_{r,s}}
% \end{eqnarray}
% 
% I'm not sure how to formalize the next step, but the pattern is clear:
% 
% \begin{equation}
% \frac{\partial T^p}{\partial t_{r,s}}
%  = \sum_{i=1}^p\left( T^{i-1} \frac{\partial T}{\partial t_{r,s}} T^{p-i} 
%    \right) \quad \forall \quad p \in \mathbb{Z}, p > 0
% \end{equation}
% 
% Going back to the formulas for single terms in the derivatives, we can simplify further: several of the summations can be eliminated because all but one 
% of the terms are multiplied by a Kronecker delta equal to zero:
% \begin{eqnarray}
% \frac{\partial (T^1)_{i,j}}{\partial t_{r,s}} &=& \delta^i_r \delta^j_s \\
% \frac{\partial (T^2)_{i,j}}{\partial t_{r,s}} &=& \delta^i_r t_{s,j} + t_{i,r} \delta^j_s \\
% \frac{\partial (T^3)_{i,j}}{\partial t_{r,s}} &=& 
% \delta_r^i \sum_{p=1}^n t_{s,p} t_{p,j} + t_{i,r} t_{s,j} + 
% \delta_s^j \sum_{k=1}^n t_{i,k} t_{k,r} \\
%  & = & \delta_r^i (T^2)^{s,j} + t_{i,r} t_{s,j}
%  + \delta_s^j (T^2)_{i,r} \\
% \nonumber
% \frac{\partial (T^4)_{i,j}}{\partial t_{r,s}} &=& 
% \delta^i_r \sum_{q=1}^n \sum_{p=1}^n t_{s,p} t_{p,q} t_{q,j} +
% t_{i,r} \sum_{q=1}^n t_{s,q} t_{q,j} \\ & & {} +
% t_{s,j} \sum_{k=1}^n t_{i,k} t_{k,r} +
% \delta^j_s \sum_{p=1}^n \sum_{k=1}^n t_{i,k} t_{k,p} t_{p,r} \\
%  & = & \delta^i_r (T^3)_{s,j} +
% t_{i,r} (T^2)_{s,j} + 
% t_{s,j} (T^2)_{i,r} +
% \delta^j_s (T^3)_{i,r}
% \end{eqnarray}
% 
% The pattern isn't evident yet, so for $T^5$:
% 
% \begin{eqnarray}{ccc}
% (T^5)_{i,j} &=& \sum_{u=1}^n \sum_{q=1}^n \sum_{p=1}^n \sum_{k=1}^n
% t_{i,k} t_{k,p} t_{p,q} t_{q,u} t_{u,j} \\
% \nonumber
% \frac{\partial (T^5)_{i,j}}{\partial t_{r,s}} &=&
%   \sum_{u=1}^n \sum_{q=1}^n \sum_{p=1}^n \sum_{k=1}^n (
%   \delta^i_r \delta^k_s t_{k,p} t_{p,q} t_{q,u} t_{u,j} \\ \nonumber & & {} +
%   t_{i,k} \delta^k_r \delta^p_s t_{p,q} t_{q,u} t_{u,j} +
%   t_{i,k} t_{k,p} \delta^p_r \delta^q_s t_{q,u} t_{u,j} \\ & & {} +
%   t_{i,k} t_{k,p} t_{p,q} \delta^q_r \delta^u_s t_{u,j} +
%   t_{i,k} t_{k,p} t_{p,q} t_{q,u} \delta^u_r \delta^j_s )\\
% \nonumber
% &=& \delta^i_r \sum_{u=1}^n \sum_{q=1}^n \sum_{p=1}^n t_{s,p} t_{p,q} t_{q,u} t_{u,j}
%  + t_{i,r} \sum_{u=1}^n \sum_{q=1}^n t_{s,q} t_{q,u} t_{u,j}
%  \\ \nonumber & & {} + 
%  \sum_{u=1}^n \sum_{k=1}^n t_{i,k} t_{k,r} t_{s,u} t_{u,j}
%  + t_{s,j} \sum_{p=1}^n \sum_{k=1}^n t_{i,k} t_{k,p} t_{p,r}
%  \\ & & {} + 
%  \delta^j_s \sum_{q=1}^n \sum_{p=1}^n \sum_{k=1}^n t_{i,k} t_{k,p} t_{p,q} t_{q,r} \\
% \nonumber
% &=& \delta^i_r (T^4)_{s,j} + t_{i,r} (T^3)_{s,j} + (T^2)_{i,r}(T^2)_{s,j}
%   + t_{s,j} (T^3)_{i,r} + \delta^j_s (T^4)_{i,r} \\
% \end{eqnarray}


\chapter{Derivatives of target metrics}

\section{Derivatives of terms $3 \times 2$ $A^{\prime\prime}$ with respect to $2 \times 2$ $A$}

Let $W_{*,i}$ denote the $i^{th}$ column of the matrix $W$.

\begin{eqnarray}
n^\prime_w &=& \frac{W^\prime_{*,1} \times W^\prime_{*,2}}
                   {|W^\prime_{*,1} \times W^\prime_{*,2}|} \\
Z &=& \left[ \frac{W^\prime_{*,1}}{|W^\prime_{*,1}|}, 
              n^\prime_w \times \frac{W^\prime_{*,1}}{|W^\prime_{*,1}|} \right] \\
W &=& Z^t \times W^\prime \\
n^{\prime\prime} &=& \frac{A^{\prime\prime}_{*,1} \times A^{\prime\prime}_{*,2}}
                       {|A^{\prime\prime}_{*,1} \times A^{\prime\prime}_{*,2}|} \\
n_r &=& \left\{ \begin{array}{ll} 
         n^\prime_w & \textrm{if $(n^{\prime\prime} \cdot n^\prime_w) >= 0$} \\
        -n^\prime_w & \textrm{if $(n^{\prime\prime} \cdot n^\prime_w) < 0$}
        \end{array} \right. \\
v &=& \frac{n_r \times n^{\prime\prime}}{|n_r \times n^{\prime\prime}|} \\
R &=& \left[ v, n^{\prime\prime}, v \times n^{\prime\prime} \right] \\
S &=& \left[ v, n_r, v \times n_r \right] \\
T &=& S \times R^t \\
A^\prime &=& T \times A^{\prime\prime} \\
A &=& Z^t \times A^\prime
\end{eqnarray}

\section{Derivatives of Non-Barrier Metrics}

\subsection{Derivatives of $\mu_n (T)=|T-I_n|^2$ \label{tmi}}

\noindent  This metric preserves the shape, size, and orientation of the target and is valid for any $n$. \newline

\begin{equation}
\mu(T) = |T-I_n|^2 = \sum_{i=1}^n \sum_{j=1}^n (t_{i,j} - b_{i,j})^2
\end{equation}

\begin{equation}
\frac{\partial \mu}{\partial t_{i,j}} = 2 (t_{i,j} - \delta_{i,j})
\end{equation}

\begin{equation}
\frac{\partial \mu}{\partial T} = 2(T - I_n)
\end{equation}


\begin{equation}
\frac{\partial^2 \mu}{\partial t_{i,j} \partial t_{k,l}} = 2 \delta^i_k \delta^j_l \end{equation}

\begin{equation}
\frac{\partial^2 \mu}{\partial T^2} = 2 I_{n^2}
\end{equation}
 
\noindent Alternatively, one can write
\begin{eqnarray}
|T-I_n|^2 = |T|^2 - 2 tr(T) + n
\end{eqnarray}
so that 
\begin{eqnarray}
\frac{\partial \mu}{\partial T} & = & \frac{\partial}{\partial T} |T|^2 - 2 \frac{\partial}{\partial T} tr(T) \nonumber \\
 & = & 2 T - 2 I_n \nonumber \\
 & = & 2 (T-I_n)
\end{eqnarray}
and similarly for the second derivatives. \newline

\subsection{Derivatives of $\mu_2(T)=|T|^2 - 2 \, \tau$ \label{2ds}}

This metric preserves the shape of the target and applies to $n=2$ only. \newline

\begin{eqnarray}
\frac{\partial \mu}{\partial T} & = & 2 \, \left\{ T- (adj T)^t \right\}
\end{eqnarray}

\begin{eqnarray}
\frac{\partial^2 \mu}{\partial T_i \partial T_j} & = & 2 \delta_i^j I - 2 \frac{\partial^2 \, \tau}{\partial T_i \partial T_j} \nonumber \\
 & = & 2 \, \left[ \begin{array}{cccc}
1 & 0 & 0 & -1 \\
0 & 1 & 1 & 0 \\
0 & 1 & 1 & 0 \\
-1 & 0 & 0 & 1 
\end{array} \right]
\end{eqnarray}

\subsection{Derivatives of $\mu_2(T)=|T|^2 - 2 \, \psi(T) + 2$ \label{magic}}

This metric preserves the shape and size of the target and applies to $n=2$ only. \newline 

\begin{eqnarray}
\psi(T) = \sqrt{|T|^2 + 2 \, \tau} \label{psi-def}
\end{eqnarray}

\begin{eqnarray}
\frac{\partial}{\partial T} \psi(T) = \frac{T + (adj T)^t}{\psi} \label{psi-1st}
\end{eqnarray}

\noindent The previous formula is valid for all $T$ except those $T$ 
for which $\psi(T)=0$. The set of matices for which $\psi=0$ are those
which satisfy $T^t T + \tau \, I = 0$ (they are the scaled-flips). \newline

\noindent Let $s=t_{1,2}-t_{2,1}$. The following holds for $\psi \neq 0$
\begin{eqnarray}
\label{psi-2nd}
\frac{\partial^2 \psi}{\partial T^2}
 & = & \frac{1}{\psi^3} \, \left[ \begin{array}{cccc}
s^2 & -s \, tr(T) & +s \, tr(T) & s^2 \\
-s \, tr(T) & (tr \, T)^2 & -(tr \, T)^2 & -s \, tr(T) \\
+s \, tr(T) & -(tr \, T)^2 & (tr \, T)^2 & +s \, tr(T) \\
s^2 & -s \, tr(T) & +s \, tr(T) & s^2
\end{array} \right] \nonumber \\
\end{eqnarray}

\noindent The previous formulas give
\begin{eqnarray}
\label{psi1st}
\frac{\partial \mu}{\partial T} = 2 \, \left( T - \frac{T + (adj T)^t}{\psi(T)} \right)
\end{eqnarray}

\begin{eqnarray}
\label{psi2nd}
\frac{\partial^2 \mu}{\partial T^2}  & = & 2 \, \left\{ I_4 - \frac{\partial^2 \psi}{\partial T^2} \right\}
%2 \, \left\{ I_4 - \frac{1}{\psi^3} \, \left[ \begin{array}{cccc}
%s^2 & -s \, tr(T) & +s \, tr(T) & s^2 \\
%-s \, tr(T) & (tr \, T)^2 & -(tr \, T)^2 & -s \, tr(T) \\
%+s \, tr(T) & -(tr \, T)^2 & (tr \, T)^2 & +s \, tr(T) \\
%s^2 & -s \, tr(T) & +s \, tr(T) & s^2
%\end{array} \right] \right\} \nonumber \\
\end{eqnarray}
 
\noindent In terms of a practical software implementation, there should be 
added a check so that, given $T$, if $\psi(T) = 0$, then apply
a random $\epsilon$ perturbation to each component of $T$ so that $\psi$ 
is non-zero.  Then it is valid to apply \ref{psi1st} and \ref{psi2nd}. \newline

\subsection{Derivatives of $\mu_2(T)=\frac{|T|^2 + 2}{2 \, \psi(T)}-1$ \label{magic2}}

\noindent This metric preserves the shape and size of the target and applies
to $n=2$ only. The metric has a 'barrier', not against singular matrices,
but against scaled flips. Recall $T \in M_2$ is a scaled flip if and only if 
$\psi(T)=0$.\footnote{Note that $\psi(T) \geq 0$ for all $T$.} Therefore, the metric is undefined on the set of 
scaled flips, and it should be treated the same as a normal barrier metric 
in that if $T$ is a scaled flip, then the implementation should return 
{\it false} or {\it invalid}. \newline

\noindent For $\psi> 0$,
\begin{eqnarray}
\mu_2(T)=\frac{|T|^2 + 2}{2 \, \psi(T)} - 1
\end{eqnarray}
where $\psi$ is defined in \ref{psi-def}. \newline

\begin{eqnarray}
\frac{\partial \mu_2}{\partial T} & = & \frac{1}{\psi^2} \, \left\{ \psi \, T - \left( \frac{1}{2} |T|^2 + 1 \right) \frac{\partial \psi}{\partial T} \right\} \\
& = & \frac{1}{\psi} \, \left\{ T - (\mu_2+1) \,  \frac{\partial \psi}{\partial T} \right\} 
\end{eqnarray}
with $\frac{\partial \psi}{\partial T}$ given in \ref{psi-1st}. \newline

\begin{eqnarray}
\frac{\partial^2 \mu_2}{\partial T^2} & = & \frac{I_4}{\psi} - \frac{1}{\psi^2} \, \left\{ \psi \, (\mu_2+1) \, \frac{\partial^2 \psi}{\partial T^2} - 2 (\mu_2+1) \left( \frac{\partial \psi}{\partial T} \otimes \frac{\partial \psi}{\partial T} \right) + T \otimes \frac{\partial \psi}{\partial T} + \frac{\partial \psi}{\partial T} \otimes T \right\} \nonumber \\
\end{eqnarray}

\subsection{Derivatives of $\mu_n(T)=|T|^2 - \frac{1}{n} \, (tr \, T) |tr(T)|$ \label{si+}}

This metric preserves the shape and orientation of the target and applies 
to both $n=2$ and $n=3$. \newline

\noindent The metric is
\begin{eqnarray}
\mu_n(T)=|T|^2 - \frac{1}{n} \, (tr \, T) \, |tr(T)|
\end{eqnarray}

\noindent The metric is differentiable on $M_n$, with
\begin{eqnarray}
\frac{\partial \mu_n}{\partial T} = 2 \, \left\{ T - \frac{1}{n} \, |tr(T)| \, I \right\} \label{derv1-si+}
\end{eqnarray}

\noindent The second derivative does not exist when $tr(T)=0$; otherwise 
it is  
\begin{eqnarray}
\frac{\partial^2 \mu_n}{\partial T^2} = 2 \, \left\{ I_{n^2} - \frac{1}{n} \, \left( I \otimes I \right) \right\}
\end{eqnarray}
for $tr(T) > 0$, and 
\begin{eqnarray}
\frac{\partial^2 \mu_n}{\partial T^2} = 2 \, \left\{ I_{n^2} + \frac{1}{n} \, \left( I \otimes I \right) \right\}
\end{eqnarray}
for $tr(T) < 0$. \newline

\subsection{Derivatives of $\mu_n(T) = (\tau-1)^2$ \label{tausqr} }

\noindent This metric preserves the size of the target and is valid for both $n=2$ and $n=3$. \newline

\noindent From section \ref{taugamma}
\begin{eqnarray}
\frac{\partial}{\partial T} \, (\tau-1)^2 = 2 \, (\tau-1) \, (adj T)^t
\end{eqnarray}

\noindent The second derivative is
\begin{eqnarray}
\frac{\partial^2}{\partial T^2} \, (\tau-1)^2 & = & 2 \, \left\{ (\tau-1) \frac{\partial^2 \tau}{\partial T^2} + \left( \frac{\partial \tau}{\partial T} \otimes \frac{\partial \tau}{\partial T} \right) \right\}
\end{eqnarray}
To complete this expression, use formulas \ref{adjform} and either 
\ref{2ndtau2d} or \ref{2ndtau3d}. \newline


\subsection{Derivatives of $\mu_n(T) = \left\{ |\tau-\gamma| - \left( \tau-\gamma \right) \right\}^p$ \label{untangle1}}

\noindent This metric is an untangler and is valid for both $n=2$ and $n=3$. 
\begin{eqnarray}
\mu_n(T) & = & \left\{ |\tau-\gamma| - \left( \tau-\gamma \right) \right\}^p \\
         & = & \left\{ \begin{array}{cc} 
 2^p \, (\tau-\gamma)^p & \mbox{if $\tau<\gamma$} \\
0                       & \mbox{if $\tau \geq \gamma$}
\end{array} \right.
\end{eqnarray}

\noindent The previous formula shows that the metric is twice differentiable everywhere, with
\begin{eqnarray}
\frac{\partial \mu_n}{\partial T} & = & \left\{ \begin{array}{cc} 
p \, 2^p \, (\tau-\gamma)^{p-1} \frac{\partial \tau}{\partial T} & \mbox{if $\tau<\gamma$} \\
0                      & \mbox{if $\tau \geq \gamma$}
\end{array} \right.
\end{eqnarray}

\begin{eqnarray}
\frac{\partial^2 \mu_n}{\partial T^2} & = & \left\{ \begin{array}{cc} 
p \, 2^p \, (\tau-\gamma)^{p-2} \left[ (\tau-\gamma) \frac{\partial^2 \tau}{\partial T^2} + (p-1) \, \left( \frac{\partial \tau}{\partial T} \otimes \frac{\partial \tau}{\partial T} \right) \right] & \mbox{if $\tau<\gamma$} \\
0                      & \mbox{if $\tau \geq \gamma$}
\end{array} \right.
\end{eqnarray}
To complete these expressions, use formulas \ref{adjform} and either 
\ref{2ndtau2d} or \ref{2ndtau3d}. \newline

\subsection{Derivatives of $\mu_n(T) = \frac{1}{2} \left( - \tau + \sqrt{\tau^2 + \left(\epsilon \, \bar{\tau} \right)^2} \right)$ \label{untangle2}}

\noindent This is an untangle metric for both $n=2$ and $n=3$. Both 
$\epsilon$ and $\bar{\tau}$ are parameters. \newline

\begin{eqnarray} 
\mu_n(T) = \frac{1}{2} \left( - \tau + \sqrt{\tau^2 + \left(\epsilon \, \bar{\tau} \right)^2} \right)
\end{eqnarray}

\begin{eqnarray}
\frac{\partial \mu_n}{\partial T} & = & \frac{1}{2} \left( -1 + \frac{\tau}{\sqrt{\tau^2 + \left(\epsilon \, \bar{\tau} \right)^2}} \right) \, (adj \, T)^t 
\end{eqnarray}

\begin{eqnarray}
\frac{\partial^2 \mu_n}{\partial T^2} & = & \frac{1}{2} \left\{ f \, \frac{\partial^2 \tau}{\partial T^2} + \frac{\partial f}{\partial T} \otimes (adj \, T)^t \right\} \\
f & = & -1 +  \frac{\tau}{\sqrt{\tau^2 + \left(\epsilon \, \bar{\tau} \right)^2}} \\
\frac{\partial f}{\partial T} & = & \frac{(\epsilon \, \bar{\tau})^2}{\left(\tau^2 + \left(\epsilon \, \bar{\tau} \right)^2 \right)^{3/2}} \, (adj T)^t
\end{eqnarray}


\subsection{Derivatives of $\mu_3(T) = |T|^3 - 3 \sqrt{3} \, \tau$ \label{shape3d}}

\noindent This metric preserves the shape of the target and applies 
to $n=3$ only. \newline

\begin{eqnarray}
\mu_3 (T) = |T|^3 - 3 \sqrt{3} \, \tau
\end{eqnarray}

\noindent From \ref{tcubed} and \ref{adjform}
\begin{eqnarray}
\frac{\partial \mu_3}{\partial T} & = & 3 |T| T - 3 \sqrt{3} \, (adj T)^t
\end{eqnarray}
The first derivative exists for all $T$. \newline

\noindent From \ref{hesst3},
\begin{eqnarray}
\frac{\partial^2 \mu_3}{\partial T^2} = 3 \, \left\{ |T| \, I_9 + \frac{T \otimes T}{|T|} - \sqrt{3} \, \frac{\partial^2 \tau}{\partial T^2} \, \right\}
\end{eqnarray}
%\begin{eqnarray}
%\frac{\partial^2 \mu_3}{\partial T_i \partial T_j} & = & 3 \, \delta_i^j \, |T| \, I - 3 \sqrt{3} \, \frac{\partial^2 \tau}{\partial T_i \partial T_j}
%\end{eqnarray}
To complete this expression, use formula \ref{2ndtau3d}. \newline

\subsection{Derivatives of $\mu_3(T) = |T|^3 - 3 \sqrt{3} \, \tau + \gamma (\tau-1)^2$ \label{ss3d}}

\noindent This metric preserves the shape and size of the target and applies
to $n=3$ only. \newline

\noindent This is a composite metric for in which the first two terms are the 
same as the metric in the previous section and the last term was discussed
in equations \ref{tauminus1} and \ref{tm1}. \newline 

\begin{eqnarray}
\mu_3 (T) & = & |T|^3 - 3 \sqrt{3} \, \tau + \gamma (\tau-1)^2 \\
\frac{\partial \mu_3}{\partial T} & = & 3 |T| T - \{ 3 \sqrt{3} - 2 \gamma \, (\tau-1) \}  \, (adj T)^t \\ \nonumber
\frac{\partial^2 \mu_3}{\partial T^2} & = & 3 \, \left\{ |T| \, I_9 + \frac{T \otimes T}{|T|} \right\} + 2 \gamma \left( \frac{\partial \tau}{\partial T} \otimes \frac{\partial \tau}{\partial T} \right)- \left\{ 3 \sqrt{3} \, - 2 \gamma (\tau-1) \right\} \, \frac{\partial^2 \tau}{\partial T^2}
%\frac{\partial^2 \mu_3}{\partial T_i \partial T_j} & = & 3 \, \delta_i^j \, |T| \, I - \left\{ 3 \sqrt{3} \, - 2 \gamma (\tau-1) \right\} \, \frac{\partial^2 \tau}{\partial T_i \partial T_j}
\end{eqnarray}
To complete the last expression, use formula \ref{2ndtau3d}. \newline

\subsection{Derivatives of $\mu_2(T)=|T^t T - \tau \, I|^2$ \label{shape2d}}

This metric preserves the shape of the target applies to $n=2$ only. \newline

\begin{eqnarray}
\mu_2 (T) & = & |T^t T - \tau \, I|^2 \\
          & = & |T^t T|^2 - 2 \tau |T|^2 + 2 \tau^2 \nonumber
\end{eqnarray}

\begin{eqnarray}
\frac{\partial \mu_2}{\partial T} & = & 4 \, \left\{ T T^t T - \tau T + \left( \tau - \frac{1}{2} |T|^2 \right) \, (adj T)^t \right\}
\end{eqnarray}

\begin{eqnarray}
\frac{\partial^2 \mu_2}{\partial T^2} & = & \frac{\partial^2}{\partial T^2} |T^t T|^2 - 2 \, \frac{\partial^2}{\partial T^2} \left( \tau |T|^2 \right) + 2 \, \frac{\partial^2}{\partial T^2} \tau^2
\end{eqnarray}
where the first term is given in \ref{hessttt}, and the second a third terms are
\begin{eqnarray}
\frac{\partial^2}{\partial T^2} \left( \tau \, |T|^2 \right) & = & 2 \, \left\{ \tau \, I_4 + \frac{1}{2} \, |T|^2 \, \frac{\partial^2 \tau}{\partial T^2} + (adj \, T)^t \otimes T + T \otimes (adj \, T)^t \right\} \nonumber \\
 \\
\frac{\partial^2}{\partial T^2} \tau^2 & = & 2 \, \left\{ \tau \frac{\partial^2 \tau}{\partial T^2} + \frac{\partial \tau}{\partial T} \otimes \frac{\partial \tau}{\partial T} \right\}
\end{eqnarray}

\subsection{Derivatives of $\mu_2(T)=|T-(adj T)^t|^2 + \gamma (\tau-1)^2$ \label{ss2d} }

This metric preserves the shape and size of the target and applies to $n=2$ only.
\begin{eqnarray}
\mu_2 (T) & = & |T-(adj T)^t|^2 + \gamma (\tau-1)^2 \\
          & = & 2 \, |T|^2 - 4 \, \tau  + \gamma (\tau-1)^2
\end{eqnarray}

\noindent This is a composite metric, the first term being twice the 
metric in Section \ref{2ds} and the second term being $\gamma$ times the
expressions in \ref{tauminus1} and \ref{tm1}. The derivatives are accordingly 
a linear combination of the derivatives of these two terms. \newline

%\subsection{Derivatives of $\mu_3(T)=|T-(adj T)^t|^2 + \gamma (\tau-1)^2$ \label{ss3dalt} }

%This metric preserves the shape and size of the target and applies to $n=3$ only.
%\begin{eqnarray}
%\mu_3 (T) & = & |T-(adj T)^t|^2 + \gamma (\tau-1)^2 \\
%          & = & |T|^2 - 6 \, \tau + |adj T|^2 + \gamma (\tau-1)^2
%\end{eqnarray}

%\noindent This is a composite metric, the second term being $\gamma$ times the
%expressions in \ref{tauminus1} and \ref{tm1}. The derivatives are accordingly 
%a linear combination of the derivatives of these two terms. \newline

%\noindent Let the first term be 
%\begin{eqnarray}
%\nu(T) & = & |T|^2 - 6 \, \tau + |adj T|^2 
%\end{eqnarray}
%The first derivative is
%\begin{eqnarray}
%\frac{\partial \nu}{\partial T} & = &  2 \, \left( 1 + |T|^2 \right) \, T - 6 \, (adj T)^t - 2 \, T \, T^t \, T
%\end{eqnarray}
%while the second derivative is
%\begin{eqnarray}
%\frac{\partial^2 \nu}{\partial T^2} & = & 2 \, I_9 + \frac{\partial^2}{\partial T^2} \, |adj T|^2 - 6\, \frac{\partial^2 \tau}{\partial T^2}
%\end{eqnarray}
%Expressions for the latter two terms are given in (\ref{hess-nrm-adj}) and (\ref{2ndtau3d}), respectively. \newline

\subsection{Derivatives of $\mu_n(T)=|T| - \frac{1}{\sqrt{n}} \, (tr T)$ \label{sho} }

\noindent This metric preserves the shape and orientation of the target and 
applies to both $n=2$ and $n=3$.  \newline

\begin{eqnarray}
\mu(T) & = & |T| - \frac{1}{\sqrt{n}} \, \left( tr \, T \right) \\
\frac{\partial \mu}{\partial T} & = & \frac{T}{|T|} - \frac{1}{\sqrt{n}} \, I \\\frac{\partial^2 \mu}{\partial T^2} & = & \frac{1}{|T|} \, \left\{ I_{n^2} - \frac{(T \otimes T)}{|T|^2}\right\}
\end{eqnarray}
The metric is non-differentiable at $T=0$. \newline

\section{Derivatives of Barrier Metrics}

\subsection{Derivatives of $\mu_n(T)=\frac{|T-I_n|^2}{2 \, \tau}$ \label{sso+}}

\noindent This metric preserves the shape, size, and orientation of the target
and applies to both $n=2$ and $n=3$. It is assumed that $\tau>0$, since 
this is a barrier metric. \newline

\noindent Define $g(T)=2 \tau$ so that
\begin{eqnarray}
\frac{\partial g}{\partial T} & = & 2 \, (adj T)^t \\
\frac{\partial^2 g}{\partial T^2} & = & 2 \, \frac{\partial^2 \tau}{\partial T^2}
\end{eqnarray}
To complete this last expression, use either formula \ref{2ndtau2d} or \ref{2ndtau3d}. \newline

\noindent Next, $f(T)=|T-I|^2$ is the metric in section \ref{tmi}. It's
derivatives are given there as well. Then $\mu_n(T) = \frac{f(T)}{g(T)}$ 
and formulas \ref{quotient1} and \ref{quotient2} may be used to calculate the derivatives.  \newline

\noindent Explicitly, 
\begin{eqnarray}
\frac{\partial \mu_n}{\partial T} = \frac{1}{\tau} \, \left\{ \left( T - I_n \right) - \mu_n(T) \, \left[ adj(T) \right]^t \, \right\}
\end{eqnarray}
and 
\begin{eqnarray}
\frac{\partial^2}{\partial T^2} \, \mu_n & = & \frac{1}{\tau} \, I_{n^2} - \frac{\mu_n}{\tau^2} \, \left\{ \tau \, \frac{\partial^2 \tau}{\partial T^2} - 2 \, \left[ (adj \, T)^t \otimes (adj \, T)^t \right] \right\} \nonumber \\
 & - & \frac{1}{\tau^2} \left\{ \, (T-I_n) \otimes (adj \, T)^t + (adj \, T)^t \otimes (T-I_n) \, \right\}
\end{eqnarray}

\subsection{Derivatives of $\mu_n(T)=|T^{-t}-I|^2$ \label{sso+2}}

\noindent This metric preserves the shape, size, and orientation of the target
and applies to both $n=2$ and $n=3$. \newline

\noindent For $\tau>0$
\begin{eqnarray}
\mu_n (T) & = & |T^{-t}-I|^2 \\
          & = & |T^{-1}-I|^2  \\
          & = & |T^{-1}|^2 - 2 \, tr(T^{-1}) + n \\
          & = & \frac{1}{\tau^2} \, |adj T|^2 - \frac{2}{\tau} \, tr(adj T) + n
\end{eqnarray}

\noindent {\bf The Case $n=2$}. \newline
If $n=2$, the previous simplifies to
\begin{eqnarray}
\mu_2 (T) & = &  \frac{1}{\tau^2} \, |T|^2 - \frac{2}{\tau} \, tr(T) + 2
\end{eqnarray}

\noindent The dervatives for the case $n=2$ are
\begin{eqnarray}
\frac{\partial \mu_2}{\partial T} = \frac{2}{\tau^3} \left\{ \tau \, T - \tau^2 I + \left( \tau \, (tr T) - |T|^2 \right) \, (adj T)^t \right\}
\end{eqnarray}


\begin{eqnarray}
\frac{\partial^2 \mu_2}{\partial T^2} = \frac{\partial^2}{\partial T^2} \frac{|T|^2}{\tau^2} - 2 \frac{\partial^2}{\partial T^2} \frac{tr(T)}{\tau}
\end{eqnarray}
with
\begin{eqnarray}
 \frac{\partial^2}{\partial T^2} \frac{|T|^2}{\tau^2} & = & \frac{2}{\tau^2} \, I_{4} \nonumber \\
 & - & \frac{2}{\tau^4} \left\{ |T|^2 \left( \tau \, \frac{\partial^2 \tau}{\partial T^2} - 3 \frac{\partial \tau}{\partial T} \otimes \frac{\partial \tau}{\partial T}  \right) + 2 \tau \left( T \otimes \frac{\partial \tau}{\partial T} + \frac{\partial \tau}{\partial T} \otimes T \right) \right\} \nonumber \\
\end{eqnarray}
and 
\begin{eqnarray}
 \frac{\partial^2}{\partial T^2} \frac{tr(T)}{\tau} & = & -\frac{1}{\tau^2} \left\{ tr(T) \, \frac{\partial^2 \tau}{\partial T^2} -2 \frac{tr(T)}{\tau} \left( \frac{\partial \tau}{\partial T} \otimes \frac{\partial \tau}{\partial T} \right) + I_2 \otimes \frac{\partial \tau}{\partial T} + \frac{\partial \tau}{\partial T} \otimes I_2 \right\} \nonumber \\
\end{eqnarray}

\noindent {\bf The Case $n=3$}. \newline
\begin{eqnarray}
\mu_3 (T) & = & \frac{1}{\tau^2} \, |adj T|^2 - \frac{2}{\tau} \, tr(adj T) + 3
\end{eqnarray}

\begin{eqnarray}
\frac{\partial \mu_3}{\partial T} &=& 
  \frac{1}{\tau^2}\frac{\partial}{\partial T}|adj T|^2 +
  \frac{2}{\tau^2}\left[ tr(adj T) - \frac{|adj T|^2}{\tau}\right ] \frac{\partial \tau}{\partial T} -
  \frac{2}{\tau}\frac{\partial}{\partial T} tr(adj T) \\
  & = & \nonumber 2 \left( 
  \frac{1}{\tau^2}\left[ |T|^2 T - T T^t T \right] +
  \frac{1}{\tau^2}\left[ tr(adj T) - \frac{|adj T|^2}{\tau}\right ] \frac{\partial \tau}{\partial T} -
  \frac{1}{\tau}\left[ tr(T) I - T^t \right] \right) \\
\frac{\partial^2 \mu_3}{\partial T^2} &=& \frac{\partial^2}{\partial T^2} \frac{|adj T|^2}{\tau^2} - 2 \frac{\partial^2}{\partial T^2} \frac{tr(adj \, T)}{\tau}
\end{eqnarray}
with
\begin{eqnarray}
\frac{\partial^2}{\partial T^2} \frac{|adj \, T|^2}{\tau^2} & = & \frac{1}{\tau^2} \, \frac{\partial^2}{\partial T^2} |adj \, T|^2 - \frac{1}{\tau^4} \, Y \label{adj-frac} \\
Y & = & 2 |adj \, T|^2 \left( \tau \frac{\partial^2 \tau}{\partial T^2} + \frac{\partial \tau}{\partial T} \otimes \frac{\partial \tau}{\partial T} \right) - 8 |adj \, T|^2 \left( (adj \, T)^t \otimes (adj \, T)^t \right) \nonumber \label{yadj-frac} \\
 & + & 2 \tau \, \left( \frac{\partial}{\partial T} |adj \, T|^2 \otimes (adj \, T)^t + (adj \, T)^t \otimes  \frac{\partial}{\partial T} |adj \, T|^2 \right) \\
\frac{\partial^2}{\partial T^2} \frac{tr(adj \, T)}{\tau} & = & \frac{1}{\tau} \, \frac{\partial^2}{\partial T^2} tr(adj \, T) - \frac{1}{\tau^2} \, Z \\
Z & = & tr(adj \, T) \frac{\partial^2 \tau}{\partial T^2} - 2 \frac{tr(adj \, T)}{\tau} \left( \frac{\partial \tau}{\partial T} \otimes  \frac{\partial \tau}{\partial T} \right) \nonumber \\
 & + & \frac{\partial}{\partial T} tr(adj \, T) \otimes \frac{\partial \tau}{\partial T} + \frac{\partial \tau}{\partial T} \otimes  \frac{\partial}{\partial T} tr(adj \, T)
\end{eqnarray}
See Sections \ref{nrm-adj} and \ref{traj} for completion of the derivatives of 
$|adj \, T|^2$ and $tr(adj \, T)$. \newline

\subsection{Derivatives of $\mu_n(T) = \frac{1}{2\tau} \left( |T|^2 - \frac{1}{n} (tr \, T) |tr(T)| \right)$ \label{nu-over-tau}}

\noindent This metric preserves the shape and orientation of the target and
applies to both $n=2$ and $n=3$. \newline

\noindent Let $\nu(T) = |T|^2 - \frac{1}{n} (tr \, T) |tr(T)|$, which is the 
non-barrier metric in Section \ref{si+}. Then
\begin{eqnarray}
\mu_n(T) = \frac{\nu(T)}{2 \tau}
\end{eqnarray}

\begin{eqnarray}
\frac{\partial \mu_n}{\partial T} = \frac{1}{2 \tau^2} \, \left\{ \tau \frac{\partial \nu}{\partial T} - \nu \, (adj \, T)^t \right\}
\end{eqnarray}
with the derivative of $\nu$ given in \ref{derv1-si+}. \newline

\begin{eqnarray}
\frac{\partial^2 \mu_n}{\partial T^2} & = & \frac{1}{2 \tau} \frac{\partial^2 \nu}{\partial T^2} - \frac{1}{2 \tau^2} Z \\
Z & = & \nu \frac{\partial^2 \tau}{\partial T^2} - 2 \frac{\nu}{\tau} \left( \frac{\partial \tau}{\partial T} \otimes \frac{\partial \tau}{\partial T} \right) + \frac{\partial \nu}{\partial T} \otimes \frac{\partial \tau}{\partial T} + \frac{\partial \tau}{\partial T} \otimes \frac{\partial \nu}{\partial T}
\end{eqnarray}

\subsection{Derivatives of $\mu_n(T)=|T-T^{-t}|^2$ \label{ss+1} }

\noindent This metric preserves the shape and size of the target and applies
to both $n=2$ and $n=3$. \newline

\noindent For $\tau>0$
\begin{eqnarray}
\mu_n(T) & = & |T-T^{-t}|^2 \\
         & = & |T|^2 + |T^{-1}|^2 - 2 \, n \\
         & = & |T|^2 + \frac{|adj \, T|^2}{\tau^2} - 2 \, n
\end{eqnarray}

\noindent The first derivative:
\begin{eqnarray}
\frac{\partial \mu_n}{\partial T} & = & 2 \, T + \frac{2}{\tau^3} \left\{ \tau \, |T|^2 T - \tau \, T \, T^t T - |adj \, T|^2 \, (adj \, T)^t \right\}
\end{eqnarray}

\noindent The second derivative:
\begin{eqnarray}
\frac{\partial^2 \mu_n}{\partial T^2} & = & 2 I_{n^2} + \frac{\partial^2}{\partial T^2} \left( \frac{|adj \, T|^2}{\tau^2} \right)
\end{eqnarray}
The last term is given in \ref{adj-frac} and \ref{yadj-frac}. \newline

\noindent Recall that for $n=2$, $|adj \, T|^2=|T|^2$. So for $n=2$, the above formulas simplify to:

\begin{eqnarray}
\mu_2(T) & = & \left(1 + \frac{1}{\tau^2}\right) |T|^2 - 4 \\
\frac{ \partial \mu_2 }{ \partial T } & = & 2 \left\{ 
          \left(1 + \frac{1}{\tau^2}\right) T - \frac{|T|^2}{\tau^3} 
          \frac{\partial \tau}{\partial T} \right\} \\
\nonumber
\frac{\partial^2 \mu_2}{\partial T^2} & = &
          2 \left(1 + \frac{1}{\tau^2}\right) I -
          \frac{4}{\tau^3}\left( T \otimes \frac{\partial \tau}{\partial T} +
                                 \frac{\partial \tau}{\partial T} \otimes T \right) +
          \frac{6 |T|^2}{\tau^4}\left( \frac{\partial \tau}{\partial T} \otimes
                                       \frac{\partial \tau}{\partial T} \right) -
          \frac{2 |T|^2}{\tau^3}\frac{\partial^2 \tau}{\partial T^2}
\end{eqnarray}

\subsection{Derivatives of $\mu_2(T)=|T|^2 - 2 \ln(\tau) - 2$ \label{ss+22d} }

\noindent This metric preserves the shape and size of the target and applies
to $n=2$. \newline

\noindent For $\tau>0$
\begin{eqnarray}
\frac{\partial \mu}{\partial T} & = & 2 \, \left( T - \frac{1}{\tau} (adj T)^t \right) \\
\frac{\partial^2 \mu}{\partial T^2} & = & 2 \, \left\{ I_{n^2}  - \frac{1}{\tau} \frac{\partial^2 \tau}{\partial T^2} + \frac{1}{\tau^2} \left( \frac{\partial \tau}{\partial T} \otimes \frac{\partial \tau}{\partial T} \right) \right\}
\end{eqnarray}
To complete this expression, use formulas \ref{adjform} and either 
\ref{2ndtau2d} or \ref{2ndtau3d}. \newline

\subsection{Derivatives of $\mu_3(T)=|T|^3 - 3 \sqrt{3} \ln(\tau) - 3 \sqrt{3}$ \label{ss+23d}}

\noindent This metric preserves the shape and size of the target and applies
to $n=3$. \newline

\noindent For $\tau>0$
\begin{eqnarray}
\mu_3(T) & = & |T|^3 - 3 \sqrt{3} \ln(\tau) - 3 \sqrt{3} 
\end{eqnarray}

\noindent The first deriviative:
\begin{eqnarray}
\frac{\partial \mu_3}{\partial T} & = & 3 \, |T| \, T - 3 \sqrt{3} \, \frac{1}{\tau} \, (adj \, T)^t \\
 & = & 3 \, \left\{ |T| \, T - \sqrt{3} \, T^{-t} \right\}
\end{eqnarray}

\noindent The second derivative:
\begin{eqnarray}
\frac{\partial^2 \mu_3}{\partial T^2} & = & 3 \left\{ \, |T| \, I_{9} + \frac{T \otimes T}{|T|} \, \right\} - 3 \sqrt{3} \, \left\{ \frac{1}{\tau} \frac{\partial^2 \tau}{\partial T^2} - \frac{1}{\tau^2} \left( \frac{\partial \tau}{\partial T} \otimes \frac{\partial \tau}{\partial T} \right) \right\} \nonumber \\
\end{eqnarray}

\subsection{Derivatives of $\mu_2(T)=\frac{|T|^2-2 \psi(T)+2}{2 \, \tau}$ \label{ss+32d}}

\noindent This metric preserves the shape and size of the target and applies
to $n=2$. \newline

\noindent For $\tau>0$
\begin{eqnarray}
\mu_2(T) & = & \frac{|T|^2-2 \psi(T)+2}{2 \, \tau} \\
         & = & \frac{\nu(T)}{2 \, \tau}
\end{eqnarray}
with $\nu(T) = |T|^2 - 2 \psi(T) + 2$ being the metric in Section \ref{magic}. 
 \newline

\noindent The first and second derivatives of $\mu_2$ in this section are the 
same as the formulas in Section \ref{nu-over-tau}, but with the first and 
second derivatives of $\nu$ from Section \ref{magic}.  \newline

\subsection{Derivatives of $\mu_3(T)=\frac{1}{9} \, \kappa^2(T) - 1$ \label{s+1} }

\noindent This metric preserves the shape of the target and applies
to $n=3$, the $n=2$ case being related to Inverse Mean Ratio. \newline

\noindent The condition number of $T$ is
\begin{eqnarray}
\kappa(T) & = & |T| \, |T^{-1}| \\
          & = & |T| \, |adj(T)| / |\tau|
\end{eqnarray}

\noindent Since $\tau>0$, we can drop the absolute value to obtain
\begin{eqnarray}
\mu_3(T) & = & \frac{|T|^2 \, |adj(T)|^2}{9 \, \tau^2} - 1
\end{eqnarray}

\noindent The derivatives can be obtained from the derivatives of $|T|^2$,
$|adj \, T|^2$, and $\tau^2$, along with the product and quotient rules 
\ref{prod1}, \ref{quotient1}, \ref{prod2}, and \ref{quotient2}. \newline 

\begin{eqnarray}
\frac{ \partial \mu_3 }{ \partial T } & = & \frac{2}{ 9 \tau^2 } \left\{ \left( |adj T|^2 + |T|^4 \right) T - |T|^2 T T^t T - \frac{ |T|^2 |adj T|^2 }{\tau} \frac{\partial \tau}{\partial T} \right\} 
\end{eqnarray}

\begin{eqnarray}
\nonumber
\frac{\partial^2 \mu_3}{\partial T^2} &=& \frac{1}{9 \tau^2} \left\{
   g \frac{\partial^2 f}{\partial T^2} + 
   f \frac{\partial^2 g}{\partial T^2} + 
   \left( \frac{\partial f}{\partial T} \otimes \frac{\partial g}{\partial T} + 
           \frac{\partial g}{\partial T} \otimes \frac{\partial f}{\partial T} \right) \right. \\
 \nonumber
 &-& \frac{2 g}{\tau} \left( \frac{\partial f}{\partial T} \otimes \frac{\partial \tau}{\partial T} + 
           \frac{\partial \tau}{\partial T} \otimes \frac{\partial f}{\partial T} \right) -
    \frac{2 f}{\tau} \left( \frac{\partial g}{\partial T} \otimes \frac{\partial \tau}{\partial T} + 
           \frac{\partial \tau}{\partial T} \otimes \frac{\partial g}{\partial T} \right) \\
 &+& \left. \frac{6 f g}{\tau^2} \left( \frac{\partial \tau}{\partial T} \otimes \frac{\partial \tau}{\partial T} \right) -
    \frac{2 f g}{\tau} \frac{\partial^2 \tau}{\partial T^2} \right\} \\
\nonumber f(T) &=& |T|^2 \\
\nonumber g(T) &=& |adj T|^2 \\
\nonumber \tau(T) &=& det(T)
\end{eqnarray}

\subsection{Derivatives of $\mu_n(T)=\tau + 1/\tau - 2$ \label{z+1} }

\noindent This metric preserves the size of the target and applies
to both $n=2$ and $n=3$. \newline

\noindent For $\tau>0$
\begin{eqnarray}
\frac{\partial \mu}{\partial T} = \left( 1 - \frac{1}{\tau^2} \right) \, (adj T)^t
\end{eqnarray}

\begin{eqnarray}
\frac{\partial^2 \mu}{\partial T^2} = \left( 1 - \frac{1}{\tau^2} \right) \frac{\partial^2 \tau}{\partial T^2} + \frac{2}{\tau^3} \left( \frac{\partial \, \tau}{\partial T} \otimes \frac{\partial \, \tau}{\partial T} \right)
\end{eqnarray}
To complete this expression, use formulas \ref{adjform} and either 
\ref{2ndtau2d} or \ref{2ndtau3d}. \newline

\subsection{Derivatives of $\mu_2(T) = \frac{|T|^2}{2 \tau}-1$ \label{shape+2d} }

\noindent This metric preserves the shape of the target and applies
to $n=2$. It is the 2D Inverse Mean Ratio. It is a special case of the result
in the Inverse Mean Ratio section below, but with an alternate derivation. \newline

\noindent For $\tau>0$,
\begin{eqnarray}
\mu_2(T) = \frac{|T|^2}{2 \tau} - 1
\end{eqnarray}

\begin{eqnarray}
\frac{\partial \mu_2}{\partial T} = \frac{1}{2 \tau^2} \, \left\{ 2 \, \tau \, T - |T|^2 (adj \, T)^t \right\}
\end{eqnarray}

\begin{eqnarray}
\frac{\partial^2 \mu_2}{\partial T^2} & = & \frac{1}{\tau} I_4 - \frac{1}{2 \tau^2} Z \\
Z & = & |T|^2 \frac{\partial^2 \tau}{\partial T^2} - 4 \mu_2 \left(\frac{\partial \tau}{\partial T} \otimes \frac{\partial \tau}{\partial T} \right) + 2 \left( T \otimes \frac{\partial \tau}{\partial T} + \frac{\partial \tau}{\partial T} \otimes T \right) \nonumber \\
\end{eqnarray}

\subsection{Derivatives of $\mu_3(T)=\frac{|T|^3}{3 \sqrt{3} \tau}-1$ \label{shape+3d}}

\noindent This metric preserves the shape of the target and applies
to $n=3$. It is closely related to the 3D Inverse Mean Ratio. \newline

\noindent For $\tau>0$
\begin{eqnarray}
\mu_3(T)=\frac{|T|^3}{3 \sqrt{3} \tau} - 1
\end{eqnarray}

\begin{eqnarray}
\frac{\partial \mu_3}{\partial T} & = & \frac{1}{3 \sqrt{3} \, \tau} \, \left\{ 3 \, |T| \, T - |T|^3 \, T^{-t} \right\}
\end{eqnarray}


\begin{eqnarray}
\frac{\partial^2 \mu_3}{\partial T^2} & = & \frac{1}{\sqrt{3} \, \tau} \, \left\{ \, |T| \, I_9 + \frac{T \otimes T}{|T|} - \frac{1}{3 \, \tau} \, Z \right\} \\
Z & = & |T|^3 \frac{\partial^2 \tau}{\partial T^2} - 2 \frac{|T|^3}{\tau} \left( \frac{\partial \tau}{\partial T} \otimes \frac{\partial \tau}{\partial T} \right) \nonumber \\
 & + & 3 \, |T| \left( T \otimes \frac{\partial \tau}{\partial T} + \frac{\partial \tau}{\partial T} \otimes T \right)
\end{eqnarray}

\subsection{Derivatives of $\mu_3(T)=\frac{|T|^2 + |adj T|^2}{6 \tau} - 1$ \label{ss+3d} }

\noindent This metric preserves the shape and size of the target and applies
to $n=3$. \newline

\noindent For $\tau>0$
\begin{eqnarray}
\mu_3 (T) & = & \frac{|T|^2 + |adj T|^2}{6 \tau} - 1
\end{eqnarray}

\begin{eqnarray}
\frac{\partial \mu_3}{\partial T} & = & \frac{1}{\tau} \, \left\{ \frac{T}{3} \left[ \, (1+|T|^2) \, I - T^t T \right] - (\mu_3+1) \, (adj T)^t \right\}
\end{eqnarray}

\noindent To find the second derivative, let $f(T)=|T|^2 + |adj \, T|^2$ and
$g(T)=6 \tau$.  Then the relation \ref{quotient2} can be applied, with
\begin{eqnarray}
\frac{\partial f}{\partial T} & = & 2 \, T \, \left\{ \left( 1 + |T|^2 \right) \, I - T^t T \right\} \\
\frac{\partial^2 f}{\partial T^2} & = & 2 \, I_9 + \frac{\partial^2}{\partial T^2} \, |adj \, T|^2
\end{eqnarray}
The last derivative is basically given in \ref{hess-nrm-adj}. \newline

\subsection{\label{sho_b}Derivatives of $\mu_n(T) = \frac{1}{2 \tau}\left[|T| - \frac{1}{\sqrt{n}}(tr T)\right] $ }

\noindent This metric preserves the shape and orientation of the target and 
applies to both $n=2$ and $n=3$.  It is the barrier form of the metric in Section \ref{sho_b}\newline

\noindent Given:
\begin{eqnarray}
\frac{\partial}{\partial T}\left(\frac{1}{2 \tau}\right) &=& 
  -\frac{1}{2 \tau^2}\frac{\partial \tau}{\partial T} \\
\frac{\partial^2}{\partial T^2}\left(\frac{1}{2 \tau}\right) &=& \frac{1}{\tau^3}
  \left[\frac{\partial \tau}{\partial T} \otimes \frac{\partial \tau}{\partial T}\right]
  - \frac{1}{2 \tau^2} \frac{\partial^2 \tau}{\partial T^2} 
\end{eqnarray}
Then by applying Section \ref{first} to the equations in Section \ref{sho}, we get:
\begin{eqnarray}
\mu(T) &=& \frac{1}{2 \tau}\left[|T| - \frac{1}{\sqrt{n}}(tr T)\right] \\
\frac{\partial \mu}{\partial T} &=& \frac{1}{2 \tau^2}
        \left[\frac{1}{\sqrt{n}}\left(tr T\right) - |T|\right]\frac{\partial \tau}{\partial T}
        + \frac{1}{2 \tau}\left[\frac{T}{|T|} - \frac{1}{\sqrt{n}}I\right] \\
      &=& \frac{1}{\tau}\left[\frac{1}{2}\left(\frac{T}{|T|} - \frac{1}{\sqrt{n}}I\right)
        - \mu(T) \frac{\partial \tau}{\partial T}\right] \\
\nonumber
\frac{\partial^2 \mu}{\partial T^2} &=& 
  \left(|T|-\frac{1}{\sqrt{n}}(tr T)\right)
  \left[\frac{1}{\tau^3}\left(\frac{\partial \tau}{\partial T} \otimes \frac{\partial \tau}{\partial T}\right)
      - \frac{1}{2\tau^2}\frac{\partial^2 \tau}{\partial T^2}\right]
  + \frac{1}{2 \tau |T|}\left[I_n - \frac{1}{|T|^2}\left(T \otimes T\right)\right] \\
 &-&\frac{1}{2\tau^2|T|}\left(T \otimes \frac{\partial \tau}{\partial T} + \frac{\partial \tau}{\partial T} \otimes T\right)
  + \frac{1}{2\tau^2\sqrt{n}}\left(I \otimes \frac{\partial \tau}{\partial T} + \frac{\partial \tau}{\partial T} \otimes I\right)
\end{eqnarray}

\subsection{Derivatives of Inverse Mean Ratio, $\mu_n = \frac{|T|^2}{n \, det^{2/n}(T)}-1$ \label{imr}}

\begin{eqnarray}
f(T) & = & \frac{|T|^2}{n \, det^{2/n}(T)} \\
g(T) & = & |T|^2 \\
h(T) & = & det(T) \\
e(T) & = & n \cdot h(T)^{2/n} \\
f(T) & = & \frac{g(T)}{e(T)} 
\end{eqnarray}
 
\begin{eqnarray}
\frac{\partial f}{\partial T} &=& \frac{
\frac{\partial g}{\partial T} e(T) - 
\frac{\partial e}{\partial h} \frac{\partial h}{\partial t} g(T)}
{e^2(T)} \\
&=& 
\frac{2 T e(T) - 
2 h(T)^\frac{2-n}{n} adj(T^t) g(T)}
{e^2(T)} \\
& = &  
\frac{2 T}{e(T)} -
\frac{2 h(T)^{\frac{2-n}{n}} adj^t(T) |T|^2}{e^2(T)} \\
& = &  
\frac{2 T}{n \cdot det(T)^{2/n}} -
\frac{2 det(T)^{\frac{2-n}{n}} adj^t(T) |T|^2}{n^2 \cdot det(T)^{4/n}} \\
& = &  
\frac{2}{n \cdot det^{2/n}(T)}\left[ T - \frac{|T|^2}{n \cdot det(T)} adj(T^t) \right]
\end{eqnarray}

\noindent Begin the second derivative with the  expression for individual elements of the first derivative:

\begin{eqnarray}
\frac{\partial f}{\partial t_{i,j}} & = & 
\frac{2}{n \cdot det^{2/n}(T)}\left[ t_{i,j} - \frac{|T|^2}{n \cdot det(T)} adj(T^t)_{i,j} \right] \\
 & = & 
\frac{2}{n}det^{-2/n}(T)\left[ t_{i,j} - \frac{1}{n}|T|^2 det^{-1}(T) adj(T^t)_{i,j}\right] \\
 & = &
\frac{2}{n} t_{i,j} det^{-2/n}(T) - \frac{2}{n^2} |T|^2 det^{\frac{-2}{n} - 1}(T) adj(T^t)_{i,j} 
\end{eqnarray}

\noindent And differentiate with respect to an element of $T$:

\begin{eqnarray}
\nonumber
\frac{\partial^2 f}{\partial t_{i,j} \partial t_{k,l}} & = &
\frac{2}{n} \delta_i^k \delta_j^l det^{-2/n}(T) 
  - \frac{4}{n^2} t_{i,j} det^{\frac{-2}{n}-1}(T) adj(T^t)_{k,l} \\
\nonumber
  & & {} - \frac{4}{n^2}t_{k,l} det^{\frac{-2}{n} - 1}(T) adj(T^t)_{i,j} \\
\nonumber
  & & {} + \frac{4+2n}{n^3} |T|^2 det^{\frac{-2}{n} - 2}(T) adj(T^t)_{i,j} adj(T^t)_{k,l} \\
  & & {} - \frac{2}{n^2} |T|^2 det^{\frac{-2}{n} - 1}(T) \frac{\partial adj(T^t)_{i,j}}{\partial t_{k,l}} \\
 & = & \frac{2}{n \cdot det^{2/n}(T)} \delta_i^k \delta_j^l
     + \frac{4 + 2n}{n^3 det^{\frac{2}{n}+2}(T)} |T|^2 adj(T^t)_{i,j} adj(T^t)_{k,l} \\
\nonumber
& & {} - \frac{2}{n^2 det^{\frac{2}{n}+1}} \left[ 
     2 t_{k,l} adj(T^t)_{i,j} + 2 t_{i,j} adj(T^t)_{k,l}
     + |T|^2 \frac{\partial adj(T^t)_{i,j}}{\partial t_{k,l}} \right] 
\end{eqnarray}

\noindent Returning to the Hessian notation for the second derivatives from Equation \ref{dsecelem}:


\begin{eqnarray}
\left(\frac{\partial^2 f}{\partial T_i \partial T_j}\right)_{r,s} &=&
\frac{2}{n \cdot det^{2/n}(T)} \delta_i^j \delta_r^s
     + \frac{4 + 2n}{n^3 det^{\frac{2}{n}+2}(T)} |T|^2 adj(T^t)_{i,r} adj(T^t)_{j,s} \\
\nonumber
& & {} - \frac{2}{n^2 det^{\frac{2}{n}+1}} \left[ 
     2 t_{j,s} adj(T^t)_{i,r} + 2 t_{i,r} adj(T^t)_{j,s}
     + |T|^2 \frac{\partial adj(T^t)_{i,r}}{\partial t_{j,s}} \right] 
\end{eqnarray}

\noindent Define $A = adj(T^t)$ and $A_i$ equal to the $i^th$ row of $A$.

\begin{eqnarray}
\frac{\partial^2 f}{\partial T_i \partial T_j}
&=& \frac{2 \delta_i^j}{n \cdot det^{2/n}(T)} I
+ \frac{4 + 2n}{n^3 det^{\frac{2}{n}+2}(T)} |T|^2 A_i^t A_j \\ \nonumber
& & {} - \frac{2}{n^2 det^{\frac{2}{n}+1}} \left[ 
2 A_i^t T_j + 2 T_i^t A_j + |T|^2 \frac{\partial^2 det(T)}{\partial T_i \partial T_j} \right]
\end{eqnarray}

\noindent Or if, instead, $A = adj(T)$, then

\begin{eqnarray} \label{imrblockhess}
\frac{\partial^2 f}{\partial T_i \partial T_j} 
&=& \frac{2 \delta_i^j}{n \cdot det^{2/n}(T)} I
+ \frac{4 + 2n}{n^3 det^{\frac{2}{n}+2}(T)} |T|^2 A_j^t A_i \\ \nonumber
& & {} - \frac{2}{n^2 det^{\frac{2}{n}+1}} \left[ 
2 T_j^t A_i + 2 A_j^t T_i + |T|^2 \frac{\partial^2 det(T)}{\partial T_i \partial T_j} \right]
\end{eqnarray}



\noindent For $2 \times 2$ target matrices, Equation \ref{imrblockhess} simplifies to:
\begin{eqnarray}
\frac{\partial^2 f}{\partial T_i \partial T_j}
&=& \frac{\delta^i_j}{det(T)}I + \frac{|T|^2}{det^3(T)} A_j^t A_i \\ \nonumber
& & {} - \frac{1}{det^2(T)} \left[ T_j^t A_i + A_j^t T_i 
+ \frac{|T|^2}{2} \frac{\partial^2 det(T)}{\partial T_i \partial T_j} \right]
\end{eqnarray}

\chapter{Derivatives of Functions of other Metrics}

\section{Untanlge Metrics}

\subsection{\label{franks} Derivatives of $\mu^\prime(T)=\{|d(T)| - d(T)\}^2$ }
\begin{eqnarray}
\mu^\prime(T) &=& \{|d(T)| - d(T)\}^2 \\
 d(T)        &=& (\sigma - \epsilon) - \mu(T) \\
 \frac{\partial \mu^\prime}{\partial T} 
    &=& 2 (|d| - d)\left(\frac{|d|}{d}-1\right)\frac{\partial d}{\partial T} \\
    &=& 4 (d - |d|)\frac{\partial d}{\partial T} \\
    &=& -4 (d - |d|)\frac{\partial \mu}{\partial T} \\
    &=& \left\{\begin{array}{ll}
         -8 d \frac{\partial \mu}{\partial T} & \textrm{if } d<0 \\
         0                                    & \textrm{if } d \ge 0 \\
         \end{array}\right. \\
 \frac{\partial^2 \mu^\prime}{\partial T^2} 
    &=& \left\{\begin{array}{ll}
         8 \left( \frac{\partial \mu}{\partial T} \otimes \frac{\partial \mu}{\partial T} 
             - d \frac{\partial^2 \mu}{\partial T^2} \right) & \textrm{if } d<0 \\
         0                                    & \textrm{if } d>0 \\
         \end{array}\right. 
\end{eqnarray}
         

\chapter{Derivatives of Metrics that are Explicit Functions of $A$}

\section{Derivatives of Non-Barrier Metrics}

\subsection{\label{diff} Derivatives of $\mu_n(A)=|A-W|^2$}

\begin{eqnarray}
\mu_n (A) & = & |A-W|^2 \\
          & = & |A|^2 - 2 \, tr(A^t W) + |W|^2 \nonumber
\end{eqnarray}

\begin{eqnarray}
\frac{\partial \mu_n}{\partial A} = 2 \, (A-W)
\end{eqnarray}
 
\begin{eqnarray}
\frac{\partial^2 \mu_n}{\partial A^2} = 2 \, I_{n^2}
\end{eqnarray}

\chapter{Derivatives of Element Quality Metrics}

\section{Volume Case}

\noindent {\bf The Map.} \newline
Given a master element $M$ and a physical element 
$E$ in $\Re^3$, 
let there be a one-to-one onto map from $M$ to $E$ given by 
\begin{eqnarray}
x_m & = & x_m ( \xi_n )
\end{eqnarray}
where $m = 1, 2, 3$ and $n = 1, 2, 3$.  The map is usually a polynomial
in the coordinates $\xi_n$.  Let the physical element contain K control 
points $X_m^k$, with $k=1,\ldots,K$. The map is then generally written as 
\begin{eqnarray}
x_m & = & \sum_{k=1}^K N_k (\xi_n) X_m^k
\end{eqnarray}
with $N_k$ being the 'basis' polynomial in the master coordinates associated
with the $k^{th}$ control point. \newline

\noindent {\bf The Jacobian.} \newline
The Jacobian of the map is a $3 \times 3$ matrix
$A$, whose elements are
\begin{eqnarray}
A_{mn} = \frac{\partial x_m}{\partial \xi_n} 
\end{eqnarray}
Assuming the map has the form given in (2), we have
\begin{eqnarray}
A_{mn} = \sum_{k=1}^K \left( \frac{\partial N_k}{\partial \xi_n} \right) X_m^k 
\end{eqnarray}
The Jacobian is linear in the control points $X_m^k$. \newline

\noindent The derivative of $A_{mn}$ with respect to the control point 
coordinates is, with $\ell=1,\ldots,K$ and $p=1,2,3$
\begin{eqnarray}
\frac{\partial A_{mn}}{\partial X_p^\ell} & = &  \sum_{k=1}^K \left( \frac{\partial N_k}{\partial \xi_n} \right) \frac{\partial X_m^k}{\partial X_p^\ell} \\
  & = &  \left( \frac{\partial N_\ell}{\partial \xi_n} \right) \frac{\partial X_m^\ell}{\partial X_p^\ell} \\
  & = &  \left( \frac{\partial N_\ell}{\partial \xi_n} \right) \delta_{m p}
\end{eqnarray}

\noindent {\bf The Function $f(A)$ and its derivatives.} \newline
Let $f$ be a multi-variable function from $\Re^9$ to $\Re$, given by 
$f(A) = f(A_{11},A_{12},...,A_{33})$.  The derivative of $f$ with respect 
to $A$ is defined as the matrix $df/dA$ with elements
\begin{eqnarray}
\left( \frac{df}{dA} \right)_{ij} & = & \frac{\partial f}{\partial A_{ij}}
\end{eqnarray}
The Hessian of $f$ with respect to $A$ is a tensor $d^2 f/ dA^2$ with elements
\begin{eqnarray}
\left( \frac{d^2 f}{d A^2} \right)_{rsij} = \frac{\partial^2 f}{\partial A_{rs} \partial A_{ij}}
\end{eqnarray}

\noindent {\bf The Gradient of $f$ with respect to the Control Points.} 
\newline The gradient of $f$ with respect to the control points within an
element is defined as a vector in $\Re^{3K}$ with components 
$\frac{\partial f}{\partial X_m^k}$. We have
\begin{eqnarray}
\frac{\partial f}{\partial X_m^k} & = & \sum_{i=1}^3 \sum_{j=1}^3 \frac{\partial f}{\partial A_{ij}} \frac{\partial A_{ij}}{\partial X_m^k} \\
& = & \sum_{i=1}^3 \sum_{j=1}^3 \frac{\partial f}{\partial A_{ij}} \frac{\partial N_k}{\partial \xi_j} \delta_{im} \\
% & = & \sum_{i=1}^3 \sum_{j=1}^3 \frac{\partial f}{\partial A_{ij}} \frac{\partial}{\partial X_m^k} \left( \sum_{\ell=1}^K \frac{\partial N_\ell}{\partial \xi_j} X_i^\ell \right) \\ 
% & = & \sum_{i=1}^3 \sum_{j=1}^3 \frac{\partial f}{\partial A_{ij}} \sum_{\ell=1}^K \left( \frac{\partial N_\ell}{\partial \xi_j} \right) \frac{\partial}{\partial X_m^k} X_i^\ell  \\
 & = & \sum_{j=1}^3 \frac{\partial f}{\partial A_{mj}} \frac{\partial N_k}{\partial \xi_j}
\end{eqnarray}

\noindent {\bf The Hessian of $f$ with respect to the Control Points.} \newline
The Hessian of $f$ with respect to the control points within an element is 
defined as a $3K \times 3K$ matrix with components $\frac{\partial^2 f}{\partial X_n^\ell \partial X_m^k}$, with $\ell=1,\ldots,K$.  We have
\begin{eqnarray}
\frac{\partial^2 f}{\partial X_n^\ell \partial X_m^k} & = & \frac{\partial}{\partial X_n^\ell} \left( \sum_{j=1}^3 \frac{\partial f}{\partial A_{mj}} \frac{\partial N_k}{\partial \xi_j} \right) \\
& = & \sum_{j=1}^3  \frac{\partial}{\partial X_n^\ell} \left( \frac{\partial f}{\partial A_{mj}} \frac{\partial N_k}{\partial \xi_j} \right) \\
& = & \sum_{j=1}^3  \left\{ \frac{\partial}{\partial X_n^\ell} \left( \frac{\partial f}{\partial A_{mj}} \right) \right\} \frac{\partial N_k}{\partial \xi_j} \\
& = & \sum_{j=1}^3  \left\{ \sum_{r=1}^3 \sum_{s=1}^3 \frac{\partial}{\partial A_{rs}} \left( \frac{\partial f}{\partial A_{mj}} \right) \frac{\partial A_{rs}}{\partial X_n^\ell} \right\} \frac{\partial N_k}{\partial \xi_j} \\
& = & \sum_{j=1}^3  \left\{ \sum_{r=1}^3 \sum_{s=1}^3 \left( \frac{\partial^2 f}{\partial A_{rs} \partial A_{mj}} \right) \frac{\partial N_\ell}{\partial \xi_s} \delta_{rn} \right\} \frac{\partial N_k}{\partial \xi_j} \\
& = & \label{eqn:vol-hess} \sum_{j=1}^3  \sum_{s=1}^3 \left( \frac{\partial^2 f}{\partial A_{ns} \partial A_{mj}} \right) \frac{\partial N_\ell}{\partial \xi_s} \frac{\partial N_k}{\partial \xi_j}
\end{eqnarray}
For simple $f(A)$'s the Hessian will be sparse. \newline

\noindent Equation (\ref{eqn:vol-hess}) can be expressed using matrix multiplication as follows:

\begin{equation} \label{eqn:vol-hess-mat}
\frac{\partial^2 f}{\partial X_n^\ell \partial X_m^k} =
(\nabla N_l)^t \frac{\partial^2 f}{\partial A_n \partial A_m} \nabla N_k
\end{equation}
where $\nabla N_k = \left[\begin{array}{c} 
\frac{\partial N_k}{\partial \xi_1} \\
\frac{\partial N_k}{\partial \xi_2} \\
\frac{\partial N_k}{\partial \xi_3} \end{array}\right]$ 
and $\frac{\partial^2 f}{\partial A_n \partial A_m} = \left[\begin{array}{ccc}
\frac{\partial^2 f}{\partial A_{n,1} \partial A_{m,1}} &
\frac{\partial^2 f}{\partial A_{n,1} \partial A_{m,2}} &
\frac{\partial^2 f}{\partial A_{n,1} \partial A_{m,3}} \\
\frac{\partial^2 f}{\partial A_{n,2} \partial A_{m,1}} &
\frac{\partial^2 f}{\partial A_{n,2} \partial A_{m,2}} &
\frac{\partial^2 f}{\partial A_{n,2} \partial A_{m,3}} \\
\frac{\partial^2 f}{\partial A_{n,3} \partial A_{m,1}} &
\frac{\partial^2 f}{\partial A_{n,3} \partial A_{m,2}} &
\frac{\partial^2 f}{\partial A_{n,3} \partial A_{m,3}} \end{array}\right]$.\newline


\noindent Note that all these derivatives are local quantities, say at a sample point.  There would also be a global gradient and Hessian that can be assembled from these local pieces. \newline

\noindent {\bf The function $f(T)$ and it's derivatives} \newline
For the case $T = A W^{-1}$, some of the metrics are of the form $f=f(T)$,
which is, indirectly, still a function of $A$.  For convennience, let
$Z = W^{-1}$.  Then $T = A Z$, so that 
\begin{eqnarray}
T_{rs} = \sum_{k=1}^3 A_{rk} Z_{ks}
\end{eqnarray}
and 
\begin{eqnarray}
\frac{\partial T_{rs}}{\partial A_{ij}} & = & \sum_{k=1}^3 \frac{\partial A_{rk}}{\partial A_{ij}} Z_{ks} \\
 & = & \sum_{k=1}^3 \delta_{ri} \delta_{kj} Z_{ks} \\
 & = & \delta_{ri} \sum_{k=1}^3 \delta_{kj} Z_{ks} \\
 & = & \delta_{ri} Z_{js}
\end{eqnarray}
Then we have
\begin{eqnarray}
\frac{\partial f}{\partial A_{ij}} & = & \sum_{r=1}^3 \sum_{s=1}^3 \frac{\partial f}{\partial T_{rs}} \frac{\partial T_{rs}}{\partial A_{ij}} \\
 & = & \sum_{r=1}^3 \sum_{s=1}^3 \frac{\partial f}{\partial T_{rs}} \delta_{ri} Z_{js} \\
 & = & \sum_{r=1}^3 \delta_{ri} \sum_{s=1}^3 \frac{\partial f}{\partial T_{rs}} Z_{sj}^t \\
 & = & \sum_{s=1}^3 \frac{\partial f}{\partial T_{is}} Z_{sj}^t 
\end{eqnarray}
or, more compactly,
\begin{eqnarray}
\frac{\partial f}{\partial A} = \frac{\partial f}{\partial T} W^{-t}
\end{eqnarray}
Similarly, we have for the Hessian of $f(T)$
\begin{eqnarray}
\frac{\partial^2 f}{\partial A_{pq} \partial A_{ij}} & = & \sum_{s=1}^3 \frac{\partial}{\partial A_{pq}} \left( \frac{\partial f}{\partial T_{is}} \right) Z_{sj}^t \\
 & = & \sum_{s=1}^3 \sum_{p=1}^3 \sum_{q=1}^3  Z_{sj}^t \frac{\partial T_{pq}}{\partial A_{pq}}  \frac{\partial}{\partial T_{pq}} \left( \frac{\partial f}{\partial T_{is}} \right) \\
 & = & \sum_{s=1}^3 \sum_{p=1}^3 \sum_{q=1}^3  Z_{sj}^t Z_{qq} \left( \frac{\partial^2 f}{\partial T_{pq} \partial T_{is}} \right)
\end{eqnarray}

\section{Surface Case}

\noindent {\bf The Map.} \newline
Given a 2D master element $M$ and a physical surface element 
$E$ in $\Re^3$, let there be a one-to-one onto map from $M$ to $E$ given by 
\begin{eqnarray}
x_m & = & x_m ( \xi_n )
\end{eqnarray}
where $m = 1, 2, 3$ and $n = 1, 2$.  The map is usually a polynomial
in the coordinates $\xi_n$.  Let the physical element contain K control 
points $X_m^k$, with $k=1,\ldots,K$. The map is then generally written as 
\begin{eqnarray}
x_m & = & \sum_{k=1}^K N_k (\xi_n) X_m^k
\end{eqnarray}
with $N_k$ being the 'basis' polynomial in the master coordinates associated
with the $k^{th}$ control point. \newline

\noindent {\bf The Jacobian.} \newline
The Jacobian of the map is a $3 \times 2$ matrix
$J$, whose elements are
\begin{eqnarray}
J_{mn} = \frac{\partial x_m}{\partial \xi_n} 
\end{eqnarray}
Assuming the map has the form given in (33), we have
\begin{eqnarray}
J_{mn} = \sum_{k=1}^K \left( \frac{\partial N_k}{\partial \xi_n} \right) X_m^k 
\end{eqnarray}
The Jacobian is linear in the control points $X_m^k$. \newline

\noindent The derivative of $J_{mn}$ with respect to the control point 
coordinates is, with $\ell=1,\ldots,K$ and $p=1,2,3$
\begin{eqnarray}
\frac{\partial J_{mn}}{\partial X_p^\ell} & = &  \sum_{k=1}^K \left( \frac{\partial N_k}{\partial \xi_n} \right) \frac{\partial X_m^k}{\partial X_p^\ell} \\
  & = &  \left( \frac{\partial N_\ell}{\partial \xi_n} \right) \frac{\partial X_m^\ell}{\partial X_p^\ell} \\
  & = &  \left( \frac{\partial N_\ell}{\partial \xi_n} \right) \delta_{m p}
\end{eqnarray}

\noindent {\bf Dependence of $A_{2 \times 2}$ on $J$} \newline
The surface mesh optimization algorithm requires that $J_{3 \times 2}$ be 
converted to a matrix $A_{2 \times 2}$ so that the 2D TMP metrics can be 
used.\footnote{In previous writings we used the notation $A^{\prime\prime}$ for $J$.} Similarly, the target matrix is $W^\prime_{3 \times 2}$, which is converted to the matrix $W_{2 \times 2}$. Elements of $A_{2 \times 2}=(RZ)^t J$ 
are given by
\begin{eqnarray}
A_{ij} = \sum_{r=1}^3 \sum_{s=1}^3 \left( Z^t \right)_{ir} \left( R^t \right)_{rs} \left( J \right)_{sj}
\end{eqnarray}
for $i=1,2$ and $j=1,2$. We seek the derivative of the elements of $A$ with
respect to the control point coordinates. 
First, recall that $Z$ is independent of the control
point coordinates $X_m^k$ because the matrix is constructed from the 
$3 \times 2$ target matrix.  Second, the derivatives of $J$ with respect to
these coordinates have already been given in (38) above.  Third, the matrix $R$ is 
defined as $R = [{\bf v},{\bf n}_J,{\bf v} \times {\bf n}_J][{\bf v},{\bf n}_r,{\bf v} \times {\bf n}_r]^t$, where ${\bf N}_J = {\bf j}_1 \times {\bf j}_2$, 
with $J= [{\bf j}_1,{\bf j}_2]$ and ${\bf n}_J={\bf N}_J/|{\bf N}_J|$.
Additionally, let ${\bf N}_w^\prime = {\bf w}_1^\prime \times {\bf w}_2^\prime$, ${\bf n}_w^\prime = {\bf N}_w^\prime/|{\bf N}_w^\prime|$, so that 
${\bf n}_r = {\bf n}_w^\prime$ if ${\bf n}_J \cdot {\bf n}_w^\prime>0$ and 
-${\bf n}_w^\prime$ otherwise. Thus, ${\bf v}$ and ${\bf n}_J$ depend on 
$X_m^k$ while the other vectors do not. The dependence of $R$ on the control
point coordinates is thus complex. \newline

\noindent Small changes in the control point coordinates can result in large 
changes to $R$ if the surface curvature is large compared to the size
of the mesh elements. When the mesh under-resolves the surface curvature, 
one expects any mesh optimization algorithm that uses 'snap-to' 
functionality to fare poorly, so accuracy of the derivative of $R$ is  
immaterial.  \newline
 
\noindent For meshes that resolve the 
surface well, (i.e., have elements that are small compared to the radius of 
curvature), $R$ is a slowly-varying function of the 
control point coordinates.  In other words, when the surface is well resolved 
by the mesh elements, ${\bf n}_J$ is close to ${\bf n}_r$ and small
changes in the control point coordinates will result in small changes to $R$. 
The special case of a planar mesh is an example of this since then 
$R=I_{3 \times 3}$ is constant. Therefore, {\it we assume that the derivative 
of $R$ with respect to the control point coordinates is zero.} \newline

\noindent {\bf The Function $f(A)$ and its derivatives.} \newline
Let $f$ be a multi-variable function from $\Re^4$ to $\Re$, given by 
$f(A) = f(A_{11},A_{12},...,A_{22})$.  The derivative of $f$ with respect 
to $A$ is defined as the matrix $df/dA$ with elements
\begin{eqnarray}
\left( \frac{df}{dA} \right)_{ij} & = & \frac{\partial f}{\partial A_{ij}}
\end{eqnarray}
The Hessian of $f$ with respect to $A$ is a tensor $d^2 f/ dA^2$ with elements
\begin{eqnarray}
\left( \frac{d^2 f}{d A^2} \right)_{rsij} = \frac{\partial^2 f}{\partial A_{rs} \partial A_{ij}}
\end{eqnarray}

\noindent {\bf The Gradient of $f$ with respect to the Control Points.} 
\newline The gradient of $f$ with respect to the control points within an
element is defined as a vector in $\Re^{3K}$ with components 
$\frac{\partial f}{\partial X_m^k}$. We have
\begin{eqnarray}
\frac{\partial f}{\partial X_m^k} & = & \sum_{i=1}^2 \sum_{j=1}^2 \frac{\partial f}{\partial A_{ij}} \frac{\partial A_{ij}}{\partial X_m^k} \nonumber \\
 & = & \sum_{i=1}^2 \sum_{j=1}^2 \frac{\partial f}{\partial A_{ij}} \frac{\partial}{\partial X_m^k} \left\{ \left( RZ \right)^t J \right\}_{ij} \nonumber \\
 & = & \sum_{i=1}^2 \sum_{j=1}^2 \frac{\partial f}{\partial A_{ij}} \frac{\partial}{\partial X_m^k} \left\{ \sum_{r=1}^3  \left[ (RZ)^t \right]_{ir} \left[ J \right]_{rj} \right\} \nonumber \\
 & \approx & \sum_{i=1}^2 \sum_{j=1}^2 \frac{\partial f}{\partial A_{ij}} \left\{ \sum_{r=1}^3  \left[ (RZ)^t \right]_{ir}  \frac{\partial}{\partial X_m^k} \left[ J \right]_{rj} \right\} \nonumber \\
 & = & \sum_{i=1}^2 \sum_{j=1}^2 \frac{\partial f}{\partial A_{ij}} \left\{ \sum_{r=1}^3  \left[ (RZ)^t \right]_{ir}  \left( \frac{\partial N_{k}}{\partial \xi_j} \right) \delta_{rm} \right\} \nonumber \\
 & = & \sum_{i=1}^2 \sum_{j=1}^2 \frac{\partial f}{\partial A_{ij}}  \left( \frac{\partial N_{k}}{\partial \xi_j} \right) \left\{ \sum_{r=1}^3  \left[ (RZ)^t \right]_{ir} \delta_{rm} \right\} \nonumber \\
 & = & \sum_{i=1}^2 \sum_{j=1}^2 \frac{\partial f}{\partial A_{ij}} \frac{\partial N_k}{\partial \xi_j} \left( RZ \right)_{mi} \nonumber \\
 \label{eqn:surfgrad} & = & \sum_{i=1}^2  \left( RZ \right)_{mi} \sum_{j=1}^2 \frac{\partial f}{\partial A_{ij}} \frac{\partial N_k}{\partial \xi_j} 
\end{eqnarray}
Note the similarities and difference with equation (12). If 
$RZ = I$, as in the planar case, then this reduces to equation (12),
except the sums go from 1 to 2. \newline

\noindent Equation (\ref{eqn:surfgrad}) can be expressed using matrix operations:

\begin{equation}
\frac{\partial f}{\partial X^k} = R Z \frac{\partial f}{\partial A} \nabla N_k
\end{equation}

where $\frac{\partial f}{\partial X^k} = \left[ \begin{array}{c} 
\frac{\partial f}{\partial X^k_1} \\
\frac{\partial f}{\partial X^k_2} \\
\frac{\partial f}{\partial X^k_3} \end{array} \right]$
and $\nabla N_k = \left[ \begin{array}{c} 
\frac{\partial N_k}{\partial \xi_1} \\
\frac{\partial N_k}{\partial \xi_2} \end{array} \right]$.


\noindent {\bf The Hessian of $f$ with respect to the Control Points.} \newline
The Hessian of $f$ with respect to the control points within an element is 
defined as a $3K \times 3K$ matrix with components $\frac{\partial^2 f}{\partial X_n^\ell \partial X_m^k}$, with $\ell=1,\ldots,K$.  We have
\begin{eqnarray}
\frac{\partial^2 f}{\partial X_n^\ell \partial X_m^k} & = & \frac{\partial}{\partial X_n^\ell} \left( \sum_{i=1}^2  \left( RZ \right)_{mi} \sum_{j=1}^2 \frac{\partial f}{\partial A_{ij}} \frac{\partial N_k}{\partial \xi_j} \right) \nonumber \\
& \approx &\sum_{i=1}^2 \left( RZ \right)_{mi} \sum_{j=1}^2  \frac{\partial}{\partial X_n^\ell}  \left( \frac{\partial f}{\partial A_{ij}} \right) \frac{\partial N_k}{\partial \xi_j} \nonumber \\
& = & \sum_{i=1}^2  \left( RZ \right)_{mi} \sum_{j=1}^2  \frac{\partial N_k}{\partial \xi_j} \left\{ \sum_{r=1}^2 \sum_{s=1}^2 \frac{\partial A_{rs}}{\partial X_{n}^\ell} \frac{\partial}{\partial A_{rs}}  \left(  \frac{\partial f}{\partial A_{ij}}  \right) \right\}  \nonumber \\
& = & \sum_{i=1}^2  \left( RZ \right)_{mi} \sum_{j=1}^2  \frac{\partial N_k}{\partial \xi_j} \left\{ \sum_{r=1}^2 \sum_{s=1}^2 \frac{\partial A_{rs}}{\partial X_{n}^\ell} \left(  \frac{\partial^2 f}{\partial A_{rs} \partial A_{ij}}  \right) \right\}  \nonumber \\
& \approx & \sum_{i=1}^2  \left( RZ \right)_{mi} \sum_{j=1}^2 \frac{\partial N_k}{\partial \xi_j} \left\{ \sum_{r=1}^2 \sum_{s=1}^2 \left( RZ \right)_{rn}^t \frac{\partial N_\ell}{\partial \xi_s} \left( \frac{\partial^2 f}{\partial A_{rs} \partial A_{ij}}  \right)  \right\}  \nonumber \\
& = & \label{eqn:surf-hess-sum} \sum_{i=1}^2  \left( RZ \right)_{mi} \sum_{j=1}^2 \frac{\partial N_k}{\partial \xi_j} \left\{ \sum_{r=1}^2  \left( RZ \right)_{rn}^t \sum_{s=1}^2 \frac{\partial N_\ell}{\partial \xi_s} \left( \frac{\partial^2 f}{\partial A_{rs} \partial A_{ij}}  \right)  \right\} 
\end{eqnarray}
Note the similarities and difference with equation (18). If 
$RZ = I$, as in the planar case, then the Hessian reduces to equation (18),
except the sums go from 1 to 2.\newline

\noindent Equation (\ref{eqn:surf-hess-sum}) can be reordered as:

\begin{eqnarray}
\frac{\partial^2 f}{\partial X_n^\ell \partial X_m^k} 
&=& \sum_{i=1}^2 \sum_{r=1}^2 (RZ)_{m,i} (RZ)^t_{r,n}
    \sum_{j=1}^2 \sum_{s=1}^2 \frac{\partial N_k}{\partial \xi_j}
    \frac{\partial N_\ell}{\partial \xi_s}
    \left( \frac{\partial^2 f}{\partial A_i \partial A_r} \right)_{j,s} 
    \nonumber \\
&=& \label{eqn:surf-hess-summat} 
    \sum_{i=1}^2 \sum_{r=1}^2 (RZ)_{m,i} (RZ)^t_{r,n} 
    \left[ (\nabla N_\ell)^t \frac{\partial^2 f}{\partial A_i \partial A_r}
            \nabla N_k \right] 
\end{eqnarray}

\noindent For a purely 2-dimensional case, a block of the Hessian containing 
derivatives with respect to the coordinates of two vertices $Y^\ell$ and $Y^k$ is:
\begin{equation} \label{eqn:hess-2d}
\frac{\partial^2 g}{\partial Y^\ell \partial Y^k} =
\left[ \begin{array}{cc}
\frac{\partial^2 g}{\partial Y^\ell_1 \partial Y^k_1} &
\frac{\partial^2 g}{\partial Y^\ell_1 \partial Y^k_2} \\
\frac{\partial^2 g}{\partial Y^\ell_2 \partial Y^k_1} &
\frac{\partial^2 g}{\partial Y^\ell_2 \partial Y^k_2} 
\end{array} \right] \end{equation}
\noindent And analogous to the 3-dimensional case from equation:
(\ref{eqn:vol-hess-mat}):
\begin{equation}
\left(\frac{\partial^2 g}{\partial Y^\ell \partial Y^k}\right)_{i,r}
= \frac{\partial^2 g}{\partial Y^\ell_i \partial Y^k_r}
= (\nabla N_\ell)^t \frac{\partial^2 f}{\partial A_i \partial A_r} \nabla N_k
\end{equation}
\noindent A block of the Hessian for the surface case can be expressed in terms of the corresonding 2D Hessian:
\begin{equation} \label{eqn:surf-hess-mat}
\frac{\partial^2 f}{\partial X^\ell \partial X^k}
= (RZ) \left(\frac{\partial^2 g}{\partial Y^\ell \partial Y^k}\right)^t (RZ)^t
\end{equation}
\noindent A single element $n,m$ of equation (\ref{eqn:surf-hess-mat}) is:
\begin{eqnarray}
\left(\frac{\partial^2 f}{\partial X^\ell \partial X^k}\right)_{n,m}
&=& \frac{\partial^2 f}{\partial X^\ell_n \partial X^k_m} \nonumber \\
&=& \left[(RZ) \left(\frac{\partial^2 g}{\partial Y^\ell \partial Y^k} \right)^t  (RZ)^t\right]_{n,m} 
    \nonumber \\
&=& \sum_{r=1}^2 (RZ)_{n,r} \left[ \left(\frac{\partial^2 g}{\partial Y^\ell \partial Y^k} \right)^t
    (RZ)^t \right]_{r,m} \nonumber \\
&=& \sum_{r=1}^2 (RZ)_{n,r} \sum_{i=1}^2 
    \left(\frac{\partial^2 g}{\partial Y^\ell \partial Y^k}\right)^t_{r,i} (RZ)^t_{i,m} 
    \nonumber \\
&=& \sum_{r=1}^2 (RZ)^t_{r,n} \sum_{i=1}^2 
    \left(\frac{\partial^2 g}{\partial Y^\ell \partial Y^k}\right)_{i,r} (RZ)_{m,i} 
    \nonumber \\
&=& \sum_{r=1}^2 \sum_{i=1}^2 (RZ)^t_{r,n} (RZ)_{m,i} \left[
    (\nabla N_\ell)^t \frac{\partial^2 f}{\partial A_i \partial A_r} \nabla N_k
    \right] \nonumber
\end{eqnarray}
\noindent which is equation (\ref{eqn:surf-hess-summat}). \newline

\noindent In the planer case ($RZ = I$), equation (\ref{eqn:surf-hess-mat}) results in 
a $3 \times 3$  matrix where the upper left $2 \times 2$ sub-block is exactly the 
corresponding Hessian block from 
the two-dimensional case and all terms corresponding to derivatives with respect to the 
coordinate of the third dimension are zero.  Each vertex-block of both the first and second derivatives is a transform using $RZ$ applied to the corresponding vertex-block from the 
two-dimensional derivatives.


\chapter{Implementation Names}

The table in this section is a map from class names in the Mesquite implementation to sections in this document describing the corresponding
metric and its derivatives.  Section numbers in parantheses indicate
metrics that are not yet implemented.  Entries with question marks indicate
implementations for which there is no corresponding section in this document.\newline

\begin{tabular}{|c|c|c|}
\hline
Name & 2D & 3D \\
\hline
Shape & \ref{2ds} & \ref{shape3d} \\
ShapeAlt1 & \ref{shape2d} &  \\
\hline
ShapeBarrier & \ref{shape+2d} & \ref{shape+3d} \\
ShapeBarrierAlt1 &  & \ref{s+1} \\
InverseMeanRatio & \ref{imr} & \ref{imr} \\
\hline
ShapeOrient & \ref{sho} & \ref{sho} \\
ShapeOrientAlt1 & \ref{si+} & \ref{si+} \\
\hline
ShapeOrientBarrier & \ref{sho_b} & \ref{sho_b} \\
ShapeOrientBarrierAlt1 & \ref{nu-over-tau} & \ref{nu-over-tau} \\
\hline
ShapeSize & \ref{magic} & \ref{ss3d} \\
ShapeSizeAlt1 & \ref{magic2} &  \\
ShapeSizeAlt2 & \ref{ss2d} &  \\
\hline
ShapeSizeBarrier & \ref{ss+1} & \ref{ss+1} \\
ShapeSizeBarrierAlt1 & \ref{ss+32d} & \ref{ss+3d} \\
ShapeSizeBarrierAlt2 & \ref{ss+22d} & \ref{ss+23d} \\
\hline
ShapeSizeOrient & \ref{tmi} & \ref{tmi} \\
ShapeSizeOrientAlt1 & \ref{diff} & \ref{diff} \\
\hline
ShapeSizeOrientBarrier & \ref{sso+} & \ref{sso+} \\
ShapeSizeOrientBarrierAlt1 & \ref{sso+2} & \ref{sso+2} \\
\hline
Size & \ref{tausqr} & \ref{tausqr} \\
Tau & \ref{tau} & \ref{tau} \\
\hline
SizeBarrier & \ref{z+1} & \ref{z+1} \\
\hline
Squared & \ref{tsquared} & \ref{tsquared} \\
\hline
Untangle & \ref{untangle1} & \ref{untangle1} \\
UntangleAlt1 & \ref{untangle2} & \ref{untangle2} \\
\hline
\end{tabular}

\end{document}
