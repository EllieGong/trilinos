\noindent {\bf Remarks:}

\begin{itemize}
  \item To compile this code, you must tell the compiler where to find
the header files that are included at the top.  In one machine that we
use, the \tri\ header files are located in \$TRILINOS\_HOME/include, so
the command 
\bdcode
cpp ...
\edcode
will work.
  \item This is still a very simple version of the conjugate gradient
method, but is more efficient than the elementary version shown in the
introduction.
  \item In the \lcode{run} method we have put in a \lcode{double}
variable for the convergence tolerance (\lcode{tol}) and an
\lcode{int} variable for the maximum number of iterations allowed.
  \item The \lcode{run} method returns an \lcode{enum} value.  This is
a common practice in \cpp\ to make the return values easier to read.
%thing to do, and so there are tools in \teuchos\ that make
%this easier.  We'll show how to do this later.
  \item Note that in setting up the vector \lcode{p} we used the
\lcode{copy()} function to ensure that we get a real copy of \lcode{r}
and not just another pointer to \lcode{r} that we would get from the
statement \lcode{Vector<double> p = r;}\ .
  \item As we mentioned in the introduction, we compute \lcode{A*p}
just once and use it in several places.  We also compute \lcode{p *
  Ap} once and use it in two places as well.  Both of these improve
the efficiency of the code.
%  \item The convergence criterion, 
%\bdcode
%if (r.norm2() <= tol) return CONVERGENCE;
%\edcode 
%is not a very good choice.  Experts in numerical
%linear algebra often use a scaled norm, e.g., 
%\bdcode
%if(r.norm2()/b.norm2() <= tol) return CONVERGENCE; 
%\edcode 
%which is the
%relative error with respect to the norm of the right hand side.
  \item Note that we have used \lcode{norm2()}, which is the standard
Euclidean norm, i.e., the square root of the sum of the squares of the
components of the vector. \thyra\ has other choices for the norm,
e.g., \lcode{norm1()} is the sum of the absolute values of the
elements of the vector and \lcode{normInf()} is the ``infinity'' norm
which is the maximum of the absolute values of the elements.  These
are all useful and standard in linear algebra.
\end{itemize}


%more sophisticated codes:
% derive this from solver so that we can attach it to A
 
