\documentclass[10pt]{amsart}

\usepackage{fullpage}
%\usepackage{amsmath}
%\usepackage{amsthm}
%\usepackage{graphicx}
%\usepackage{url}
%\usepackage{subfigure}
%\usepackage[foot]{amsaddr}

\date{}
\title{Robust Parallel Hybrid Solvers for Sparse Linear Systems}

\pagestyle{empty}
\setlength{\topmargin}{-0.35in}
\setlength{\headheight}{0in}
\setlength{\headsep}{0in}
\advance\textheight by 1.2in 
\advance\oddsidemargin by -0.25in
\advance\textwidth by 0.25in
%\advance\textwidth by 0.6in 
%\advance\oddsidemargin by -0.3in
%\advance\evensidemargin by -0.3in

\begin{document}

\maketitle

%\vspace{-8mm}
%\paragraph{\bf Introduction:}
We propose to develop parallel algorithms and software for robust solvers
of sparse linear systems. We seek to address two issues that
are not adequately addressed by current solver packages:
\begin{enumerate}
\item high performance on manycore and future exascale architectures, and
\item robustness for ill-conditioned problems.
\end{enumerate}

As parallelism in a single node increases, NW applications
have to adapt to a hybrid system where each compute node by itself
is a shared memory system (likely NUMA) with dozens of cores. 
In order to address the new challenges and opportunities 
for robust sparse linear solvers on the node,
we have developed a hybrid sparse solver ShyLU (initially funded by 
Office of Science). ShyLU is hybrid in the
mathematical sense (direct and iterative methods) making it more
robust than other preconditioners, while being less expensive
than a direct factorization.  ShyLU is also hybrid in the parallel
computing sense (MPI and threads). We envision ShyLU as both
a standalone black-box solver for medium sized problems,
and as a subdomain solver/preconditioner within a larger 
distributed-memory framework. 

ShyLU is based on a general Schur complement framework, with different options
for partitioning the matrices, block-diagonal solves and Schur complement
approximations.  Our hybrid implementation scales better than
a flat MPI implementation as the problem size per subdomain gets smaller, 
which is important for strong scaling in applications.
%The MPI + threads implementation helps ShyLU to scale well for up to $384$ cores.
Our first target application is circuit simulation (Xyce, POC: H. Thornquist). 
There is an acute lack of robust iterative methods in this area, so sparse
direct methods are often used but they often require a lot of memory.
The most successful iterative methods
for circuits are based on Schur complements but no such parallel solver
is currently available in Trilinos or other DOE solver libraries.
%
Other potential applications are structural mechanics, where ShyLU
may be used as a subdomain solver within FETI (POC: K. Pierson) or
domain decomposition methods such as GDSW (POC: G. Reese, C. Dohrmann). 
We also anticipate that ShyLU
will be useful as a smoother within ML/MueLu (POC: J. Hu, C. Siefert).
%Tramonto?

\paragraph{\bf Proposed Work:}
ShyLU is currently based on Trilinos, in particular Epetra.
We will use Trilinos as a development platform and delivery vehicle.
ShyLU's robustness and scalability depends on good
approximations to the Schur complement. Currently, there
are two algebraic approaches in ShyLU: 
structure-based probing and value-based dropping. The dropping approximation
was added for circuit simulation problems. However, for problems
with nice Schur complement structure, like PDEs where there is decay
property in the Schur complement as you move away from the diagonal,
a probing-based method is more appropriate. We will develop other
approximations to Schur complement based on the needs of our applications. For
example, we know indefinite matrices require a better approximation than either
of the two existing approaches in ShyLU.

For unsymmetric matrices, ShyLU
uses the permutation from a hypergraph partitioning as a symmetric
permutation to find the interface. However, we can improve the robustness
of ShyLU by
using an unsymmetric permutation. We will develop algorithms that can handle
the unsymmetric case and implement it in ShyLU.

Xyce requires capabilities to efficiently
reuse components within the ShyLU framework for a sequence of linear systems that
have a static graph and/or become increasingly difficult to solve. We will
support these capabilities in our software.
We expect much of our development to be driven by application demands.

\begin{description}
\item[Year 1] {\bf Release Epetra-based implementation and evaluate with NW Applications}
  \begin{itemize} 
  \item Release of ShyLU as a stable code in Trilinos.
  \item Make the solver available as a smoother for ML.
  \item Evaluate performance on solid mechanics test problems from 1500 collaborators.
  \item Integrate as one of the solvers within Xyce.
  \end{itemize}
\item[Year 2] {\bf Improve implementation / algorithms for applications of interest}
  \begin{itemize}
  \item Improve ShyLU algorithm for structurally nonsymmetric problems.
  \item Improve robustness for circuit and mechanics problems, including indefinite problems.
  \end{itemize}
\item[Year 3] {\bf Extend ShyLU compatability to 2nd generation Trilinos}
  \begin{itemize}
  \item Tpetra-based version of ShyLU, compatible with 2nd generation Trilinos.
  \item Support for complex arithmetic (to be useful for other solvers like GDSW).
  \item Make the solver available as a smoother for MueLu.
  \end{itemize}
\end{description}


\paragraph{\bf Personnel and Budget:}
Erik Boman, PI (1426): 0.3 FTE\\
David Day (1442): 0.2 FTE\\
Siva Rajamanickam (1426): 0.4 FTE\\
Heidi Thornquist (1445): 0.3 FTE\\
Summer student.\\
%
\textbf{Total budget request:} \$400K FY12, \$410 FY13, \$420 FY14.

\end{document}
