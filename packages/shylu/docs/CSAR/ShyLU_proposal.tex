\documentclass[10pt]{amsart}

\usepackage{fullpage}
%\usepackage{amsmath}
%\usepackage{amsthm}
%\usepackage{graphicx}
%\usepackage{url}
%\usepackage{subfigure}
%\usepackage[foot]{amsaddr}

\date{}
\title{Robust Parallel Hybrid Solvers for Sparse Linear Systems
from Circuit Simulation and Solid Mechanics}

\setlength{\topmargin}{-0.75in}
\setlength{\headheight}{0in}
\setlength{\headsep}{0in}
\advance\textheight by 1.2in 
\advance\oddsidemargin by -0.25in
\advance\textwidth by 0.5in
%\advance\textwidth by 0.6in 
%\advance\oddsidemargin by -0.3in
%\advance\evensidemargin by -0.3in
\pagestyle{empty}

\begin{document}

\maketitle

\vspace{-4mm}
%\paragraph{\bf Introduction:}
We propose to develop parallel algorithms and software for robust sparse linear solvers,
which will address two issues that are not adequately regarded by current solver packages:
\begin{enumerate}
\item high performance on manycore and future exascale architectures, and
\item robustness for ill-conditioned problems.
\end{enumerate}

As parallelism in a single node increases, NW applications
have to adapt to a hybrid system where each compute node 
is a shared memory system (likely NUMA) with dozens of cores. 
To address the new challenges and opportunities 
for robust sparse linear solvers on the node,
we have developed ShyLU, a hybrid sparse solver. 
ShyLU is hybrid in the
mathematical sense (direct and iterative methods) making it more
robust than other preconditioners, while being less expensive
than a direct factorization.  It is also hybrid in the parallel
computing sense (MPI and threads).  ShyLU can serve as both
a standalone black-box solver for medium-sized problems,
and as a subdomain solver/preconditioner within a larger 
distributed-memory framework. 
An experimental implementation of
this solver was initially funded by Office of Science and is unreleased. 

ShyLU is based on a general Schur complement framework, with different options
for partitioning the matrices, block-diagonal solves and Schur complement
approximations.  This solver can improve strong scaling for important NW 
applications, as preliminary results show that our hybrid implementation scales better than
a flat MPI implementation as the problem size per subdomain gets smaller.
%The MPI + threads implementation helps ShyLU to scale well for up to $384$ cores.
Our first target application is circuit simulation (Xyce, POC: H. Thornquist), 
where there is an acute need for robust iterative methods to perform full chip simulations
for the W88 LEP.  These simulations generate linear systems that are too large for
direct methods.
Iterative methods, based on Schur complements, have proven effective for circuits, but
are not currently available in Trilinos or any other DOE solver library.
%
Another potential application is structural mechanics, where ShyLU
can be used as a subdomain solver within FETI (POC: K. Pierson) or
domain decomposition methods such as GDSW (POC: G. Reese, C. Dohrmann). 
We also anticipate that ShyLU will be useful as a smoother within 
ML/MueLu (POC: J. Hu, C. Siefert) for solid mechanics applications.
%Tramonto?
\paragraph{\bf Proposed Work:}
Application-focused research will be performed on matrix partitioning and
Schur complement approximations, using Trilinos as a development platform
and delivery vehicle.
ShyLU's robustness and scalability depends on good
approximations to the Schur complement. Currently, there
are two algebraic approaches in ShyLU: 
structure-based probing and value-based dropping. The dropping approximation
was added for circuit simulation problems. However, for problems
with nice Schur complement structure, like PDEs,
% where there is decay property in the Schur complement as you move away from the diagonal
a probing-based method is more appropriate. 
For structural dynamics, indefinite matrices require a better 
approximation than either of the two existing approaches in ShyLU provide.
This will be an area of research for this project.

ShyLU currently uses symmetric permutations but we can improve robustness on
nonsymmetric problems by using an unsymmetric permutation.
We will develop and implement algorithms that better handle
the unsymmetric case.
Xyce requires the capability to efficiently
reuse components within the ShyLU framework for sequences of linear systems that
have a static graph and/or become increasingly difficult to solve.
These requirements will be incorporated into ShyLU's software design.
We propose the following research and development tasks:

\begin{description}
\item[Year 1] {\bf Release Epetra-based implementation and evaluate with NW applications}
  \begin{itemize} 
  \item Release of ShyLU as a stable code in Trilinos.
  \item Make the solver available as a smoother for ML.
  \item Evaluate performance on test problems from 1500 collaborators.
  \item Integrate as one of the solvers within Xyce.
  \end{itemize}
\item[Year 2] {\bf Improve implementation / algorithms for NW applications}
  \begin{itemize}
  \item Improve ShyLU algorithm for structurally nonsymmetric problems.
  \item Improve robustness for NW applications, including indefinite linear systems.
  \end{itemize}
\item[Year 3] {\bf Extend ShyLU compatability to 2nd generation Trilinos}
  \begin{itemize}
  \item Tpetra-based version of ShyLU, compatible with 2nd generation Trilinos.
  \item Support for complex arithmetic (to be useful in Helmholtz problems).
  \item Make the solver available as a smoother for MueLu.
  \end{itemize}
\end{description}


\paragraph{\bf Personnel and Budget:}
Erik Boman, PI (1426): 0.3 FTE\\
David Day (1442): 0.2 FTE\\
Siva Rajamanickam (1426): 0.3 FTE\\
Heidi Thornquist (1445): 0.3 FTE\\
1500 consulting (Pierson, Dohrmann): 0.1 FTE\\
Summer student\\
%
\textbf{Total budget request:} \$400K FY12, \$410 FY13, \$420 FY14.

\end{document}
