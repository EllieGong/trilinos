\documentclass[12pt,relax]{SANDreport}
\usepackage{fancyvrb}
\usepackage{fullpage}
\usepackage{latexsym}
\usepackage{fancyheadings}
\usepackage{epsf,psfrag}    % Include Postscript files
\usepackage{times}
\usepackage{amsmath}
\usepackage{amssymb}
\usepackage{moreverb}
\usepackage{epsfig}


\setcounter{secnumdepth}{4}



\title{TSFExtended User's Guide}
\author{Kevin Long\\
Roscoe A. Bartlett\\
Paul T. Boggs\\
Michael A. Heroux\\
Patricia A. Howard\\
Victoria E. Howle\\
Jill P. Reese
}

    % There is a "Printed" date on the title page of a SAND report, so
    % the generic \date should generally be empty.
    \date{} % Remove ``\today'' in final version 


\SANDnum{SAND2004-ZZZZ}
\SANDprintDate{Printed August 2004} 
\SANDauthor{Kevin Long\\ 
Paul T. Boggs\\ 
Patricia A. Howard\\
Victoria E. Howle\\
Jill P. Reese\\
\\
Computational Science and Mathematics Research Department \\
Sandia National Laboratories \\
Livermore CA, 94553\\
\\
Roscoe A. Bartlett\\
\\
Optimization and Uncertainty Quantification Department\\
Sandia National Laboratories\\
Albuquerque, NM\\
\\
Michael A. Heroux\\
\\
Computational Mathematics and Algorithms Department\\
Sandia National Laboratories\\
Albuquerque, NM}


\SANDreleaseType{Unlimited Release}  


% New commands
\newcommand{\Az}  {{\bf Aztec}}


%
% define boxes for describing variables, etc
%

\def\optionbox#1#2{\noindent$\hphantom{hix}${\parbox[t]{2.10in}{\it #1}}{\parbox[t]{3.9in}{#2}} \\[1.1em]}

\def\choicebox#1#2{\noindent$\hphantom{hixthere}$\parbox[t]{2.10in}{\sf #1}\parbox[t]{3.5in}{#2}\\[0.8em]}

\def\structbox#1#2{\noindent$\hphantom{hix}${\parbox[t]{2.10in}{\it #1}}{\parbox[t]{3.9in}{#2}} \\[.02cm]}



\begin{document}
\maketitle

\begin{abstract}
TSFExtended
\end{abstract}


\section*{Acknowledgement}


\clearpage
\tableofcontents
\listoffigures
%\listoftables

\clearpage

\SANDmain
\chapter{Introduction}
\label{Section:Introduction}

\section{Creating Vectors}

\subsection{VectorSpace and VectorType}

\section{Vector Operations}

\subsection{Multiplication by Scalars}

The \verb+scale()+ method multiplies a vector by a scalar. For example,
\begin{verbatim}
x.scale(3.14159);
\end{verbatim}
replaces $x$ by $3.14159 x$. The operation is done in place, so that 
no additional memory is used beyond that required to store $x$. 
Alternatively, the overloaded multiplication operator can be used,
\begin{verbatim}
y = 3.14159*x;
\end{verbatim}
or
\begin{verbatim}
y = x*3.14159;
\end{verbatim}


\subsection{Updates/Axpys}

The various \verb+update()+ methods carry out generalized axpy operations.
\begin{itemize}
\item The operation
\begin{equation}
y = y + \alpha x
\end{equation}
is accomplished with
\begin{verbatim}
y.update(alpha, x);
\end{verbatim}
\item The operation
\begin{equation}
y = \gamma y + \alpha x
\end{equation}
is accomplished with
\begin{verbatim}
y.update(alpha, x, gamma);
\end{verbatim}
\item The operation
\begin{equation}
z = \gamma z + \alpha x + \beta y
\end{equation}
is accomplished with
\begin{verbatim}
z.update(alpha, x, beta, y, gamma);
\end{verbatim}
\end{itemize}
Any of these can be written using overloaded operations, for example,
\begin{verbatim}
z = z + alpha*x + beta*y;
\end{verbatim}

\subsection{Element-by-element Operations}

Element-by-element operations are rarely used in linear solvers, but
appear frequently in nonlinear solvers and optimizers. TSFExtended provides
methods for several common element-by-element operations; however, if an
arbitrary element-by-element operation is required it can be provided
through the RTOp mechanism.

The built-in element-by-element methods are multiplication, division, 
reciprocal, and absolute value. All these methods come in two versions, a
self-modifying method that replaces \verb+*this+ with the result of the 
operation, and a method that creates a new vector into which the result is
written. 
The methods are
\begin{itemize}
\item Element-by-element multiplication $z_i = x_i y_i$ 
\begin{verbatim}
z = x.dotStar(y);
\end{verbatim}
\item In-place element-by-element multiplication $x_i \leftarrow x_i y_i$ 
\begin{verbatim}
x.dotStar(y);
\end{verbatim}
\item Element-by-element division $z_i = x_i / y_i$ 
\begin{verbatim}
z = x.dotSlash(y);
\end{verbatim}
\item In-place element-by-element division $x_i \leftarrow x_i / y_i$ 
\begin{verbatim}
x.dotSlash(y);
\end{verbatim}
\item Element-by-element reciprocal $z_i = 1 / x_i$ 
\begin{verbatim}
z = x.reciprocal();
\end{verbatim}
\item In-place element-by-element reciprocal $x_i \leftarrow 1 / x_i$ 
\begin{verbatim}
x.reciprocal();
\end{verbatim}
\item Element-by-element absolute value $z_i = \vert x_i \vert$ 
\begin{verbatim}
z = x.abs();
\end{verbatim}
\item In-place element-by-element reciprocal $x_i \leftarrow \vert x_i \vert$ 
\begin{verbatim}
x.reciprocal();
\end{verbatim}
\end{itemize}


\subsection{Dot Product}

The dot product can be computed with either the \verb+dot()+ method or
the overloaded multiplication operator,
\begin{verbatim}
double xDotY = x.dot(y);
\end{verbatim}
or
\begin{verbatim}
double xDotY = x*y;
\end{verbatim}

\subsection{Vector Norms}

Methods exist for the 1, 2, and $\infty$ norms of vectors and for the
weighted two norm:
\begin{itemize}
\item The \verb+norm1()+ method computes the 1-norm
\begin{equation}
\| x \|_1 = \sum_{i=1}^N \vert x_i \vert
\end{equation}
\item The \verb+norm2()+ method computes the 2-norm
\begin{equation}
\| x \|_2 = \left[\sum_{i=1}^N x_i^2\right]^{1/2}
\end{equation}
\item The \verb+normInf()+ method computes the $\infty$-norm
\begin{equation}
\| x \|_\infty = \max x_i
\end{equation}
\end{itemize}


\section{Linear Operators}

\subsection{Creating Matrix Operators}

\subsection{Implicit Operator Composition}

The overloaded multiplication operator is used to create compositions of
operators without actually forming the product of the two operators. 
\begin{verbatim}
// Form a composed operator using the * operator
LinearOperator<double> A = B*C;
// Apply the composed operator
Vector<double> y = A*x;
\end{verbatim}
The application of a composed operator first applies the right operand
\verb+C+ to \verb+x+, and then applies the left operand \verb+B+
to the result.  


\subsection{Implicit Transpose Operators}

The \verb+transpose()+ method is used to create an implicit transpose 
operator, which when applied will do a transpose apply on the original
operator. For example,
\begin{verbatim}
// Form an implicit transpose operator
LinearOperator<double> At = A.transpose();
// Apply the transposed operator
Vector<double> y = At*x;
\end{verbatim}

\subsection{Forming Transposes}

In some cases it may be useful to form a transpose explicitly. This is done
using the \verb+formTranspose()+ method. For example,
\begin{verbatim}
// Form an implicit transpose operator
LinearOperator<double> At = A.formTranspose();
// Apply the transposed operator
Vector<double> y = At*x;
\end{verbatim}
It is not possible to form explicit transpose operators for arbitrary
operators; explicit transposition works only for operators that implement
the \verb+ExplicitlyTransposeableOp+ interface. If called on an operator
that does not implement that interface, an exception will be thrown.



\subsection{Implicit Inverse Operators}

\subsection{Diagonal Operators}

\subsection{Identity and Zero Operators}





\chapter{Solver Options and Parameters\label{Section:OptionsAndParams}}

\section{TSF Solver Options\label{optionI}}

\vspace{2em}
{\flushleft{\bf Specifications} \hrulefill}
\optionbox{``Method''}{Specifies solution
  algorithm. DEFAULT: ``GMRES''}

\choicebox{``GMRES''}{Restarted generalized minimal residual.}
\choicebox{``BICGSTAB''}{Bi-conjugate gradient with
  stabilization.}

\optionbox{``Precond''}{Specifies preconditioner. DEFAULT: ``ILUK''}

\choicebox{``ILUK''}{Incomplete LU}

\section{Aztec Solver Options\label{optionII}}

\vspace{2em}
{\flushleft{\bf Specifications} \hrulefill}
\nopagebreak \\[0.5em]
%
\optionbox{``Method''}{Specifies solution
  algorithm. DEFAULT: ``GMRES''}
\choicebox{``CG''}{Conjugate gradient (only
  applicable to symmetric positive definite matrices).}
\choicebox{``GMRES''}{Restarted generalized minimal residual.}
\choicebox{``CGS''}{Conjugate gradient squared.}
\choicebox{``TFQMR''}{Transpose-free quasi-minimal residual.}
\choicebox{``BICGSTAB''}{Bi-conjugate gradient with
  stabilization.}
\choicebox{``Direct''}{Sparse direct solver (single processor only).
{\bf Note: This option is available only when --enable-aztecoo-azlu is
specified on the AztecOO configure script invocation command}}

\begin{Verbatim}
`configure' configures aztecoo 3.3d to adapt to many kinds of systems.

Usage: ../../../packages/aztecoo/configure [OPTION]... [VAR=VALUE]...

To assign environment variables (e.g., CC, CFLAGS...), specify them as
VAR=VALUE.  See below for descriptions of some of the useful variables.

Defaults for the options are specified in brackets.

Configuration:
  -h, --help              display this help and exit
      --help=short        display options specific to this package
      --help=recursive    display the short help of all the included packages
  -V, --version           display version information and exit
  -q, --quiet, --silent   do not print `checking...' messages
      --cache-file=FILE   cache test results in FILE [disabled]
  -C, --config-cache      alias for `--cache-file=config.cache'
  -n, --no-create         do not create output files
      --srcdir=DIR        find the sources in DIR [configure dir or `..']

Installation directories:
  --prefix=PREFIX         install architecture-independent files in PREFIX
                          [/usr/local]
  --exec-prefix=EPREFIX   install architecture-dependent files in EPREFIX
                          [PREFIX]

By default, `make install' will install all the files in
`/usr/local/bin', `/usr/local/lib' etc.  You can specify
an installation prefix other than `/usr/local' using `--prefix',
for instance `--prefix=$HOME'.

For better control, use the options below.

Fine tuning of the installation directories:
  --bindir=DIR           user executables [EPREFIX/bin]
  --sbindir=DIR          system admin executables [EPREFIX/sbin]
  --libexecdir=DIR       program executables [EPREFIX/libexec]
  --datadir=DIR          read-only architecture-independent data [PREFIX/share]
  --sysconfdir=DIR       read-only single-machine data [PREFIX/etc]
  --sharedstatedir=DIR   modifiable architecture-independent data [PREFIX/com]
  --localstatedir=DIR    modifiable single-machine data [PREFIX/var]
  --libdir=DIR           object code libraries [EPREFIX/lib]
  --includedir=DIR       C header files [PREFIX/include]
  --oldincludedir=DIR    C header files for non-gcc [/usr/include]
  --infodir=DIR          info documentation [PREFIX/info]
  --mandir=DIR           man documentation [PREFIX/man]

Program names:
  --program-prefix=PREFIX            prepend PREFIX to installed program names
  --program-suffix=SUFFIX            append SUFFIX to installed program names
  --program-transform-name=PROGRAM   run sed PROGRAM on installed program names

System types:
  --build=BUILD     configure for building on BUILD [guessed]
  --host=HOST       cross-compile to build programs to run on HOST [BUILD]
  --target=TARGET   configure for building compilers for TARGET [HOST]

Optional Features:
  --disable-FEATURE       do not include FEATURE (same as --enable-FEATURE=no)
  --enable-FEATURE[=ARG]  include FEATURE [ARG=yes]
  --enable-maintainer-mode  enable make rules and dependencies not useful
                          (and sometimes confusing) to the casual installer
  --enable-mpi            MPI support
  --disable-dependency-tracking  speeds up one-time build
  --enable-dependency-tracking   do not reject slow dependency extractors
  --enable-aztecoo-azlu   Enable az-lu preconditioner. Default is no. Requires
                          y12m.
  --enable-aztecoo-teuchos
                          Enable AztecOO Teuchos::ParameterList interface.
                          Default is no. Requires Teuchos package.
  --enable-tests          Build tests for all Trilinos packages (not all
                          packages are sensitive to this option) (default is
                          yes)

  --enable-aztecoo-tests  Build AztecOO tests (default is yes if
                          --disable-tests is not specified)
  --enable-examples       Build examples for all Trilinos packages (not all
                          packages are sensitive to this option) (default is
                          yes)
  --enable-aztecoo-examples
                          Build AztecOO examples (default is yes if
                          --disable-examples is not specified)
  --enable-libcheck       Check for some third-party libraries including BLAS
                          and LAPACK. (Cannot be disabled unless tests and
                          examples are also disabled.) (default is yes)

Optional Packages:
  --with-PACKAGE[=ARG]    use PACKAGE [ARG=yes]
  --without-PACKAGE       do not use PACKAGE (same as --with-PACKAGE=no)
  --with-install=INSTALL_PROGRAM
                          Use the installation program INSTALL_PROGRAM rather
                          the default that is provided. For example
                          --with-install="/path/install -p"
  --with-mpi-compilers=PATH
                          use MPI compilers mpicc, mpif77, and mpicxx, mpic++
                          or mpiCC in the specified path or in the default
                          path if no path is specified. Enables MPI
  --with-mpi=MPIROOT      use MPI root directory (enables MPI)
  --with-mpi-libs="LIBS"  MPI libraries ["-lmpi"]
  --with-mpi-incdir=DIR   MPI include directory [MPIROOT/include] Do not
                          use -I
  --with-mpi-libdir=DIR   MPI library directory [MPIROOT/lib] Do not use
                          -L
  --with-ccflags          additional CCFLAGS flags to be added: will prepend
                          to CCFLAGS
  --with-cxxflags         additional CXXFLAGS flags to be added: will
                          prepend to CXXFLAGS
  --with-cflags           additional CFLAGS flags to be added: will prepend
                          to CFLAGS
  --with-fflags           additional FFLAGS flags to be added: will prepend
                          to FFLAGS
  --with-libs             List additional libraries here. For example,
                          --with-libs=-lsuperlu or
                          --with-libs=/path/libsuperlu.a
  --with-ldflags          additional LDFLAGS flags to be added: will prepend
                          to LDFLAGS
  --with-ar               override archiver command (default is "ar cru")
  --with-libdirs          OBSOLETE use --with-ldflags instead. (ex.
                          --with-ldflags="-L<DIR> -L<DIR2>")
  --with-incdirs          additional directories containing include files:
                          will prepend to search here for includes, use -Idir
                          format
  --with-lapacklib        name of library containing LAPACK: will search lib
                          directories for -lname
  --with-blaslib          name of library containing BLAS: will search lib
                          directories for -lname
  --with-blas=<lib>       use BLAS library <lib>
  --with-lapack=<lib>     use LAPACK library <lib>

Some influential environment variables:
  CC          C compiler command
  CFLAGS      C compiler flags
  LDFLAGS     linker flags, e.g. -L<lib dir> if you have libraries in a
              nonstandard directory <lib dir>
  CPPFLAGS    C/C++ preprocessor flags, e.g. -I<include dir> if you have
              headers in a nonstandard directory <include dir>
  CXX         C++ compiler command
  CXXFLAGS    C++ compiler flags
  F77         Fortran 77 compiler command
  FFLAGS      Fortran 77 compiler flags
  CXXCPP      C++ preprocessor

Use these variables to override the choices made by `configure' or to help
it to find libraries and programs with nonstandard names/locations.

Report bugs to <maherou@sandia.gov>.
\end{Verbatim}

%\section{Troubleshooting}
%\label{Section:Troubleshooting}
%\subsection{Conditional Code, Incorrect Results and Stalled Programs}
%Many methods in AztecOO distributed classes (those derive from the
%\distobject{} class) require all processors to
%participate in the method call.  For example, to compute the update of
%an \vector{}, all processors that own a portion of the vector must
%call the Update() method.  When calling a Norm2() method or something
%similar, all processors, regardless of whether they own any portion of
%a vector, must participate in the Norm2() call.  
%Figure~\ref{Figure:HungCode} illustrates
%several versions of a code segment that computes the 2-norm of a residual
%and prints it from processor 0.  Only the final version is correct.

\end{document}

