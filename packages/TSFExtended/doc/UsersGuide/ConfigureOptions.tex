\begin{Verbatim}
`configure' configures aztecoo 3.3d to adapt to many kinds of systems.

Usage: configure [OPTION]... [VAR=VALUE]...

To assign environment variables (e.g., CC, CFLAGS...), specify them as
VAR=VALUE.  See below for descriptions of some of the useful variables.

Defaults for the options are specified in brackets.

Configuration:
  -h, --help              display this help and exit
      --help=short        display options specific to this package
      --help=recursive    display the short help of all the included packages
  -V, --version           display version information and exit
  -q, --quiet, --silent   do not print `checking...' messages
      --cache-file=FILE   cache test results in FILE [disabled]
  -C, --config-cache      alias for `--cache-file=config.cache'
  -n, --no-create         do not create output files
      --srcdir=DIR        find the sources in DIR [configure dir or `..']

Installation directories:
  --prefix=PREFIX         install architecture-independent files in PREFIX
			  [/usr/local]
  --exec-prefix=EPREFIX   install architecture-dependent files in EPREFIX
			  [PREFIX]

By default, `make install' will install all the files in
`/usr/local/bin', `/usr/local/lib' etc.  You can specify
an installation prefix other than `/usr/local' using `--prefix',
for instance `--prefix=$HOME'.

For better control, use the options below.

Fine tuning of the installation directories:
  --bindir=DIR           user executables [EPREFIX/bin]
  --sbindir=DIR          system admin executables [EPREFIX/sbin]
  --libexecdir=DIR       program executables [EPREFIX/libexec]
  --datadir=DIR          read-only architecture-independent data [PREFIX/share]
  --sysconfdir=DIR       read-only single-machine data [PREFIX/etc]
  --sharedstatedir=DIR   modifiable architecture-independent data [PREFIX/com]
  --localstatedir=DIR    modifiable single-machine data [PREFIX/var]
  --libdir=DIR           object code libraries [EPREFIX/lib]
  --includedir=DIR       C header files [PREFIX/include]
  --oldincludedir=DIR    C header files for non-gcc [/usr/include]
  --infodir=DIR          info documentation [PREFIX/info]
  --mandir=DIR           man documentation [PREFIX/man]

Program names:
  --program-prefix=PREFIX            prepend PREFIX to installed program names
  --program-suffix=SUFFIX            append SUFFIX to installed program names
  --program-transform-name=PROGRAM   run sed PROGRAM on installed program names

System types:
  --build=BUILD     configure for building on BUILD [guessed]
  --host=HOST       cross-compile to build programs to run on HOST [BUILD]
  --target=TARGET   configure for building compilers for TARGET [HOST]

Optional Features:
  --disable-FEATURE       do not include FEATURE (same as --enable-FEATURE=no)
  --enable-FEATURE[=ARG]  include FEATURE [ARG=yes]
  --enable-maintainer-mode  enable make rules and dependencies not useful
			  (and sometimes confusing) to the casual installer
  --enable-mpi            MPI support
  --disable-dependency-tracking  speeds up one-time build
  --enable-dependency-tracking   do not reject slow dependency extractors
  --enable-aztecoo-azlu   Enable az-lu preconditioner. Default is no. Requires
                          y12m.
  --enable-tests          Build tests for all Trilinos packages (not all
                          packages are sensitive to this option) (default is
                          yes)
  --enable-examples       Build examples for all Trilinos packages (not all
                          packages are sensitive to this option) (default is
                          yes)

Optional Packages:
  --with-PACKAGE[=ARG]    use PACKAGE [ARG=yes]
  --without-PACKAGE       do not use PACKAGE (same as --with-PACKAGE=no)
  --with-install=INSTALL_PROGRAM
                          Use the installation program INSTALL_PROGRAM rather
                          the default that is provided. For example
                          --with-install="/path/install -p"
  --with-mpi-compilers=PATH
                          use MPI compilers mpicc, mpif77, and mpicxx, mpic++
                          or mpiCC in the specified path or in the default
                          path if no path is specified. Enables MPI
  --with-mpi=MPIROOT      use MPI root directory (enables MPI)
  --with-mpi-libs="LIBS"  MPI libraries ["-lmpi"]
  --with-mpi-incdir=DIR   MPI include directory [MPIROOT/include] Do not
                          use -I
  --with-mpi-libdir=DIR   MPI library directory [MPIROOT/lib] Do not use
                          -L
  --with-ccflags          additional CCFLAGS flags to be added: will prepend
                          to CCFLAGS
  --with-cxxflags         additional CXXFLAGS flags to be added: will
                          prepend to CXXFLAGS
  --with-cflags           additional CFLAGS flags to be added: will prepend
                          to CFLAGS
  --with-fflags           additional FFLAGS flags to be added: will prepend
                          to FFLAGS
  --with-libs             List additional libraries here. For example,
                          --with-libs=-lsuperlu or
                          --with-libs=/path/libsuperlu.a
  --with-ldflags          additional LDFLAGS flags to be added: will prepend
                          to LDFLAGS
  --with-ar               override archiver command (default is "ar cru")
  --with-libdirs          OBSOLETE use --with-ldflags instead. (ex.
                          --with-ldflags="-L<DIR> -L<DIR2>")
  --with-incdirs          additional directories containing include files:
                          will prepend to search here for includes, use -Idir
                          format
  --with-lapacklib        name of library containing LAPACK: will search lib
                          directories for -lname
  --with-blaslib          name of library containing BLAS: will search lib
                          directories for -lname
  --with-blas=<lib>       use BLAS library <lib>
  --with-lapack=<lib>     use LAPACK library <lib>

Some influential environment variables:
  CC          C compiler command
  CFLAGS      C compiler flags
  LDFLAGS     linker flags, e.g. -L<lib dir> if you have libraries in a
              nonstandard directory <lib dir>
  CPPFLAGS    C/C++ preprocessor flags, e.g. -I<include dir> if you have
              headers in a nonstandard directory <include dir>
  CXX         C++ compiler command
  CXXFLAGS    C++ compiler flags
  F77         Fortran 77 compiler command
  FFLAGS      Fortran 77 compiler flags
  CXXCPP      C++ preprocessor

Use these variables to override the choices made by `configure' or to help
it to find libraries and programs with nonstandard names/locations.

Report bugs to <maherou@sandia.gov>.
\end{Verbatim}
