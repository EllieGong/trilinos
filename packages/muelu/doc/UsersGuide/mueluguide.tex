%
% $Id: SANDExampleReportNotstrict.tex,v 1.26 2009-05-01 20:59:19 rolf Exp $
%
% This is an example LaTeX file which uses the SANDreport class file.
% It shows how a SAND report should be formatted, what sections and
% elements it should contain, and how to use the SANDreport class.
% It uses the LaTeX report class, but not the strict option.
%
% Get the latest version of the class file and more at
%    http://www.cs.sandia.gov/~rolf/SANDreport
%
% This file and the SANDreport.cls file are based on information
% contained in "Guide to Preparing {SAND} Reports", Sand98-0730, edited
% by Tamara K. Locke, and the newer "Guide to Preparing SAND Reports and
% Other Communication Products", SAND2002-2068P.
% Please send corrections and suggestions for improvements to
% Rolf Riesen, Org. 9223, MS 1110, rolf@cs.sandia.gov
%
\documentclass[pdf,12pt,report]{SANDreport}
\usepackage{algpseudocode}
\usepackage{amsthm}
\usepackage{booktabs}
\usepackage{calc}
\usepackage{color}
\usepackage{eso-pic}
\usepackage{fancyhdr}
\usepackage{ifthen}
\usepackage{indentfirst}
\usepackage{geometry}
\usepackage{graphicx}
\usepackage[colorlinks, bookmarksopen, %pagebackref=true, backref=page,
             linkcolor={blue},
             anchorcolor={black},
             citecolor={blue},
             filecolor={magenta},
             menucolor={blue},
             pagecolor={red},
             plainpages=false,pdfpagelabels,
             pdfauthor={Jonathan J. Hu, Andrey Prokopenko, Tobias Wiesner, Chris Siefert, Ray Tuminaro},
             pdftitle={MueLu User's Guide},
             pdfkeywords={MueLu,AMG,multigrid,guide,user},
             urlcolor={blue}]{hyperref}
\usepackage{listings}
\usepackage{mathptmx}	% Use the Postscript Times font
\usepackage{multirow}
\usepackage{pifont}
\usepackage[FIGBOTCAP,normal,bf,tight]{subfigure}
\usepackage{tabularx}
\usepackage{verbatim}
\usepackage{xspace}
\usepackage{flowchart} % also loads tikz
\usepackage{algorithm}
\usetikzlibrary{arrows}

%\usepackage{draftwatermark}
%\SetWatermarkScale{.5}

\algrenewcommand{\algorithmiccomment}[1]{\hskip3em // #1}


% If you want to relax some of the SAND98-0730 requirements, use the "relax"
% option. It adds spaces and boldface in the table of contents, and does not
% force the page layout sizes.
% e.g. \documentclass[relax,12pt]{SANDreport}
%
% You can also use the "strict" option, which applies even more of the
% SAND98-0730 guidelines. It gets rid of section numbers which are often
% useful; e.g. \documentclass[strict]{SANDreport}



% ---------------------------------------------------------------------------- %
%
% Set the title, author, and date
%
\title{MueLu User's Guide 1.0 \\
(Trilinos version 11.12)}

\author{Andrey Prokopenko \\
  Scalable Algorithms \\
  Sandia National Laboratories\\
  Mailstop 1318 \\
  P.O.~Box 5800 \\
  Albuquerque, NM 87185-1318\\
  aprokop@sandia.gov\\
  \and
  Tobias Wiesner \\
  Institute for Computational Mechanics \\
  Technische Universit\"at M\"unchen\\
  Boltzmanstra\ss e 15 \\
  85747 Garching, Germany\\
  wiesner@lnm.mw.tum.de\\
  \and
  Jonathan J. Hu \\
  Scalable Algorithms \\
  Sandia National Laboratories\\
  Mailstop 9159 \\
  P.O.~Box 0969 \\
  Livermore, CA 94551-0969\\
  jhu@sandia.gov
  \and
  Christopher M. Siefert\\
  Computational Multiphysics\\
  Sandia National Laboratories\\
  Mailstop 1322 \\
  P.O.~Box 5800 \\
  Albuquerque, NM 87185-1322\\
  csiefer@sandia.gov
  \and
  Raymond S. Tuminaro\\
  Computational Mathematics\\
  Sandia National Laboratories\\
  Mailstop 9159 \\
  P.O.~Box 0969 \\
  Livermore, CA 94551-0969\\
  rstumin@sandia.gov
}

% There is a "Printed" date on the title page of a SAND report, so
% the generic \date should generally be empty.
\date{}

\def\choicebox#1#2{\noindent$\hphantom{th}$\parbox[t]{2.2in}{\sf
#1}\ \ \ \ \parbox[t]{3.65in}{#2}\\[0.8em]}

\newcommand{\amesos}       {\textsc{Amesos}\xspace}
\newcommand{\amesostwo}    {\textsc{Amesos2}\xspace}
\newcommand{\anasazi}      {\textsc{Anasazi}\xspace}
\newcommand{\aztecoo}      {\textsc{AztecOO}\xspace}
\newcommand{\belos}        {\textsc{Belos}\xspace}
\newcommand{\epetra}       {\textsc{Epetra}\xspace}
\newcommand{\epetraext}    {\textsc{EpetraExt}\xspace}
\newcommand{\galeri}       {\textsc{Galeri}\xspace}
\newcommand{\ifpack}       {\textsc{Ifpack}\xspace}
\newcommand{\ifpacktwo}    {\textsc{Ifpack2}\xspace}
\newcommand{\isorropia}    {\textsc{Isorropia}\xspace}
\newcommand{\kokkosclassic}{\textsc{KokkosClassic}\xspace}
\newcommand{\loca}         {\textsc{Loca}\xspace}
\newcommand{\ml}           {\textsc{ML}\xspace}
\newcommand{\muelu}        {\textsc{MueLu}\xspace}
\newcommand{\nox}          {\textsc{NOX}\xspace}
\newcommand{\stratimikos}  {\textsc{Stratimikos}\xspace}
\newcommand{\teuchos}      {\textsc{Teuchos}\xspace}
\newcommand{\tpetra}       {\textsc{Tpetra}\xspace}
\newcommand{\trilinos}     {\textsc{Trilinos}\xspace}
\newcommand{\xpetra}       {\textsc{Xpetra}\xspace}
\newcommand{\zoltan}       {\textsc{Zoltan}\xspace}
\newcommand{\zoltantwo}    {\textsc{Zoltan2}\xspace}

\newcommand{\klu}          {\textsc{Klu}\xspace}
\newcommand{\klutwo}       {\textsc{Klu2}\xspace}
\newcommand{\metis}        {\textsc{Metis}\xspace}
\newcommand{\mumps}        {\textsc{Mumps}\xspace}
\newcommand{\umfpack}      {\textsc{Umfpack}\xspace}
\newcommand{\superlu}      {\textsc{SuperLU}\xspace}
\newcommand{\superludist}  {\textsc{SuperLU\_dist}\xspace}
\newcommand{\parmetis}     {\textsc{ParMetis}\xspace}
\newcommand{\paraview}     {\textsc{ParaView}\xspace}

\newcommand{\parameterlist}{\texttt{ParameterList}\xspace}

\newcommand \trilinosWeb   {trilinos.sandia.gov\xspace}

\newcommand{\be}  {\begin{enumerate}}
\newcommand{\ee}  {\end{enumerate}}
\newcommand{\cba}[3]{\choicebox{\texttt{#1}}{[{\texttt #2}] #3}}
\newcommand{\cbb}[4]{\choicebox{\texttt{#1}}{[{\texttt #2}] #4 {\bf Default:~}#3.}}
\newcommand{\cbc}[4]{\choicebox{\texttt{\color{red}#1}}{[{\texttt #2}] #4 {\bf Default:~}#3.}}

\newcommand{\comm}[2]{\bigskip
                      \begin{tabular}{|p{4.5in}|}\hline
                      \multicolumn{1}{|c|}{{\bf Comment by #1}}\\ \hline
                      #2\\ \hline
                      \end{tabular}\\
                      \bigskip
                     }


\newtheorem*{mycomment}{\ding{42}}
\newtheoremstyle{plain}
  {\topsep}   % ABOVESPACE
  {\topsep}   % BELOWSPACE
  {\normalfont}  % BODYFONT
  {0pt}       % INDENT (empty value is the same as 0pt)
  {\bfseries} % HEADFONT
  {}         % HEADPUNCT
  {5pt plus 1pt minus 1pt} % HEADSPACE
  {}          % CUSTOM-HEAD-SPEC

% further declarations and additional commands
\definecolor{hellgelb}{rgb}{1,1,0.8}   % background color for C++ listings
%\definecolor{hellrot}{HTML}{FFA4C2}    % background color for xml files

% settings for listings package
\lstset{%
    float=hbp,%
    basicstyle=\ttfamily\small, %
    identifierstyle=\color{black}, %
    keywordstyle=\color{blue}, %
    stringstyle=\color{purple}, %
    commentstyle=\color{darkgray}, %
    columns=flexible, %
    tabsize=2, %
    frame=single, %
    extendedchars=true, %
    showspaces=false, %
    showstringspaces=false, %
    numbers=none, %
    numberstyle=\tiny, %
    breaklines=true, %
    backgroundcolor=\color{hellgelb}, %
    breakautoindent=true, %
    captionpos=b%
}


% ---------------------------------------------------------------------------- %
% Set some things we need for SAND reports. These are mandatory
%
\SANDnum{SAND2014-18874}
\SANDprintDate{October 2014}
\SANDauthor{Andrey Prokopenko, Jonathan J. Hu, Tobias A. Wiesner, Christopher M. Siefert, Raymond S. Tuminaro}


% ---------------------------------------------------------------------------- %
% Include the markings required for your SAND report. The default is "Unlimited
% Release". You may have to edit the file included here, or create your own
% (see the examples provided).
%
% \include{MarkUR} % Not needed for unlimted release reports


% ---------------------------------------------------------------------------- %
% The following definition does not have a default value and will not
% print anything, if not defined
%
%\SANDsupersed{SAND1901-0001}{January 1901}
%\input{MarkOUO}


% ---------------------------------------------------------------------------- %
%
% Start the document
%
\begin{document}
    \maketitle

    % ------------------------------------------------------------------------ %
    % An Abstract is required for SAND reports
    %
    \begin{abstract}
	%This is the definitive user guide for the \muelu{} library in Trilinos version XX.YY.
%\muelu{} is a C++ multigrid framework that can work with either the \epetra or \tpetra linear
%algebra libraries.
%\muelu{} provides a variety of aggregation-based multigrid algorithms,
%including smoothed aggregation algebraic multigrid (AMG), Petrov-Galerkin AMG, and AMG for
%Maxwell's equations, as well as many different types of smoothers.
%\muelu{} is templated on the index, scalar, and compute node types.
%Thus it is possible to use \muelu{} on problems with scalar types other than double, on very
%large problems, and to exploit node-level parallelism.

This is the official user guide for \muelu{} multigrid library in Trilinos
version 11.12.  This guide provides an overview of \muelu, its capabilities, and
instructions for new users who want to start using \muelu{} with a minimum of
effort. Detailed information is given on how to drive \muelu{} through its XML
interface. Links to more advanced use cases are given. This guide gives
information on how to achieve good parallel performance, as well as how to
introduce new algorithms. Finally, readers will find a comprehensive listing of
available \muelu{} options.  {\em Any options not documented in this manual
should be considered strictly experimental.}

    \end{abstract}


    % ------------------------------------------------------------------------ %
    % An Acknowledgement section is optional but important, if someone made
    % contributions or helped beyond the normal part of a work assignment.
    % Use \section* since we don't want it in the table of context
    %
    \clearpage
    \chapter*{Acknowledgment}
	Many people have helped develop \muelu{} and/or provided valuable feedback, and
we would like to acknowledge their contributions here: Tom Benson, Julian
Cortial, Eric Cyr, Stefan Domino, Travis Fisher, Jeremie Gaidamour, Axel
Gerstenberger, Chetan Jhurani, Mark Hoemmen, Paul Lin, Eric Phipps, Siva
Rajamanickam, Nico Schl{\"o}mer, and Paul Tsuji.



    % ------------------------------------------------------------------------ %
    % The table of contents and list of figures and tables
    % Comment out \listoffigures and \listoftables if there are no
    % figures or tables. Make sure this starts on an odd numbered page
    %
    \cleardoublepage		% TOC needs to start on an odd page
    \tableofcontents
    \listoffigures
    \listoftables


    % ---------------------------------------------------------------------- %
    % An optional preface or Foreword
    %\clearpage
    %\chapter*{Preface}
    %\addcontentsline{toc}{chapter}{Preface}
	%    Although staff members usually have only limited experience
    with dry-erase markers, and many even dispute their existence,
    it is worthwhile to be open minded and explore the possibilities.



    % ---------------------------------------------------------------------- %
    % An optional executive summary
    %\clearpage
    %\chapter*{Summary}
    %\addcontentsline{toc}{chapter}{Summary}
	%    Once a certain level of mistrust and skepticism has been
    overcome, dry-erase markers find many uses in todays science and
    engineering. In this report we explain some of the fundamental
    properties, dangers, and benefits of dry-erase markers. We then
    conclude with a few examples on how they can be used in daily
    activities at national Laboratories.



    % ---------------------------------------------------------------------- %
    % An optional glossary. We don't want it to be numbered
    %\clearpage
    %\chapter*{Nomenclature}
    %\addcontentsline{toc}{chapter}{Nomenclature}
    %\begin{description}
	%\item[dry spell]
	%    using a dry erase marker to spell words
	%\item[dry wall]
	%    the writing on the wall
	%\item[dry humor]
	%    when people just do not understand
	%\item[DRY]
	%    Don't Repeat Yourself
    %\end{description}


    % ---------------------------------------------------------------------- %
    % This is where the body of the report begins; usually with an Introduction
    %
    \SANDmain		% Start the main part of the report

    %-----------------------------%
    \chapter{Introduction}\label{sec:introduction}
    %-----------------------------%
    % @HEADER
% ***********************************************************************
%
%                      Didasko Tutorial Package
%                 Copyright (2005) Sandia Corporation
%
% Under terms of Contract DE-AC04-94AL85000, there is a non-exclusive
% license for use of this work by or on behalf of the U.S. Government.
%
% This library is free software; you can redistribute it and/or modify
% it under the terms of the GNU Lesser General Public License as
% published by the Free Software Foundation; either version 2.1 of the
% License, or (at your option) any later version.
%
% This library is distributed in the hope that it will be useful, but
% WITHOUT ANY WARRANTY; without even the implied warranty of
% MERCHANTABILITY or FITNESS FOR A PARTICULAR PURPOSE.  See the GNU
% Lesser General Public License for more details.
%
% You should have received a copy of the GNU Lesser General Public
% License along with this library; if not, write to the Free Software
% Foundation, Inc., 59 Temple Place, Suite 330, Boston, MA 02111-1307
% USA
% Questions? Contact Michael A. Heroux (maherou@sandia.gov)
%
% ***********************************************************************
% @HEADER

\chapter{Introduction}

\ChapterAuthors{Marzio Sala, Michael Heroux, David Day, James Willenbring}
%%%
%%%
%%%

\section{Getting Started}
\label{sec:getting}

The Trilinos framework uses a two level software structure that connects
a system of {\sl packages}. A Trilinos package is an integral unit,
usually developed to solve a specific task, by a (relatively) small
group of experts.  Packages exist beneath the Trilinos top level,
which provides a common look-and-feel. Each package has its own
structure, documentation and set of examples, and it is possibly
available independently of Trilinos. However, each package is even more
valuable when combined with other Trilinos packages.

\smallskip

Trilinos is a large software project, and currently about forty
packages are included.  The entire set of packages covers a wide range
of algorithms and enabling technologies for the solution of large-scale,
complex multi-physics engineering and scientific problems, as well as a
large set of utilities to improve the development of software for scientific
computing.

Clearly, a full understanding all the functionalities of the Trilinos
packages requires time.  Each package offers sophisticated features,
difficult to ``unleash'' at a first sight.  Besides that, a detailed
description of each Trilinos package is beyond the scope of this
document. For these reasons, the goal of this tutorial is to ensure that
users have the background to make good use of the extensive
documentation contained in each package.
%For a
%successive fine-tuning phase, users still must look through individual
%package's documentation and examples.

\medskip

We will describe the following subset of the Trilinos packages.
\begin{itemize}
\item {\bf Epetra}. The package defines the basic classes for
  distributed matrices and vectors, linear operators and linear
  problems. Epetra classes are the common language spoken by all the
  Trilinos packages (even if some packages can ``speak'' other
  languages). Each Trilinos package accepts as input Epetra objects.
  This allows powerful combinations among the various Trilinos
  functionalities.
\item {\bf Triutils}. This is a collection of utilities that are useful
  in software development. Here, we present a command line parser and a
  matrix generator, that are used throughout this document to define
  example matrices.
\item {\bf AztecOO}. This is a linear solver package based on
  preconditioned Krylov methods. Aztec users will find that AztecOO
  supports all the Aztec interfaces and functionality, and also provides
  significant new functionality.
\item {\bf Belos}. Provides next-generation iterative linear solvers and a
powerful linear solver developer framework.
\item {\bf IFPACK}. The package performs various incomplete
  factorizations, and is here used with AztecOO.
\item {\bf Teuchos}. This is a collection of classes that can be
  essential for advanced code development.
\item {\bf ML}. The algebraic multilevel and domain decomposition
  preconditioner package provides scalable preconditioning capabilities
  for a variety of problems. It is here used as a preconditioner
  for AztecOO solvers.
\item {\bf Amesos}. The package provides a common interface to certain
  sparse direct linear solvers (generally available outside the Trilinos
  framework), both sequential and parallel.
\item {\bf Anasazi}. The package provides a common interface to parallel eigenvalue
  and eigenvector solvers, for both symmetric and non-symmetric linear problems.
\item {\bf NOX}. This is a collection of nonlinear solvers, designed to
  be easily integrated into an application and used with many different
  linear solvers.
\item {\bf Zoltan}. A toolkit of parallel services for dynamic, unstructured,
and/or adaptive simulations. Zoltan provides parallel dynamic load balancing
and related services for a wide variety of applications, including finite
element methods, matrix operations, particle methods, and crash simulations. 
\item {\bf Tpetra}. Next-generation, templated version of Petra, taking
advantage of the newer advanced features of C++. 
\item {\bf Didasko}. This package contains all the examples reported in this
tutorial. The sources of the examples can be found in the subdirectory\\
\verb!<your-trilinos-home>/packages/didasko/examples!.
\end{itemize}

Table~\ref{tab:tripackages} gives a partial overview of what can be
accomplished using Trilinos.
\begin{table}[htbp]
  \centering
  \begin{tabular}{| p{8cm} | p{2.5cm} | p{3cm} |}
    \hline
    {\bf Service provided/Task performed} & {\bf Package} & {\bf Tutorial}\\
    \hline
    \hline
    Advanced serial dense or sparse matrices: & Epetra
    & Chapter \ref{chap:epetra_mat} \\
    Advanced utilities for Epetra vectors and sparse matrices: &
    EpetraExt & --
    \\
    \hline
    Templated distributed vectors and sparse matrices: & Tpetra
    & Chapter \ref{chap:tpetra} \\
    \hline
    Distributed sparse matrices:& Epetra & -- \\
    \hline
    Solve a linear system with preconditioned Krylov accelerators,
    CG, GMRES, Bi-CGSTAB, TFQMR:& AztecOO, Belos &
    Chapters \ref{chap:aztecoo}, \ref{chap:belos} \\
    \hline
    Incomplete Factorizations:& AztecOO, \newline IFPACK &
    Chapter \ref{chap:ifpack} \\
    \hline
    Multilevel  preconditioners: & ML & Chapter \ref{chap:ml} \\
    \hline
    ``Black-box'' smoothed aggregation preconditioners:& ML & Section
    \ref{sec:ml:preconditioner} \\
    \hline
    One-level Schwarz preconditioner (overlapping domain
    decomposition):& AztecOO, \newline IFPACK & Chapter
    \ref{chap:ifpack} \\
    \hline
    Two-level Schwarz preconditioner, with coarse matrix based on
    aggregation:& AztecOO+ML & Section \ref{sec:ml_DD} \\
    \hline
    Systems of nonlinear equations:& NOX & Chapter \ref{chap:nox} \\
    \hline
    Interface with various direct solvers, as UMFPACK, MUMPS, SuperLU\_DIST
    and ScaLAPACK :& Amesos & Chapter \ref{chap:amesos} \\
    \hline
    Eigenvalue problems for sparse matrices:& Anasazi & Chapter
    \ref{chap:anasazi} \\
    \hline
    Complex linear systems (using equivalent real formulation):&
    Komplex$^\star$ & -- \\
    \hline
    Segregated and block preconditioners (e.g., incompressible
    Navier-Stokes equations):&
    Meros$^\star$ & -- \\
    \hline
    Light-weight interface to BLAS and LAPACK: & Epetra
    & Chapter \ref{chap:epetra_mat} \\
    \hline
    Templated interface to BLAS and LAPACK, arbitrary-precision
    arithmetic, parameters' list, smart pointers:& Teuchos &
    Section \ref{sec:teuchos:LAPACK} \\
    \hline
    Definition of abstract interfaces to vectors, linear operators, and
    solvers:& Thyra$^\star$ & --
    \\
    \hline
    Generation of test matrices & Triutils & Section~\ref{sec:triutils:gallery} \\
    \hline
  \end{tabular}
  \caption{Partial overview of intended uses of Trilinos. $\star$:
    not covered in this tutorial.}
  \label{tab:tripackages}
\end{table}

\begin{remark}
  As already pointed out, Epetra objects are meant to be the ``common
  language'' spoken by all the Trilinos packages, and are a natural
  starting point. For new users, Chapters
  \ref{chap:epetra_vec}-\ref{chap:epetra_others} are a prerequisite to
  the later chapters. Chapters~\ref{chap:triutils} is not essential to
  understand Trilinos, but the functionalities there presented are used
  in this document as a starting point for many examples.  One of the
  classes described in Chapter~\ref{chap:teuchos}, the
  Teuchos::ParameterList, is later used in Chapters~\ref{chap:ml} and
  \ref{chap:amesos}.  Chapter~\ref{chap:aztecoo} should be read before
  Chapters~\ref{chap:ifpack} and~\ref{chap:ml} (even if both IFPACK and
  ML can be compiled and run without AztecOO).
\end{remark}

The only prerequisites assumed in this tutorial are some familiarities
with numerical methods for PDEs, and with iterative linear and nonlinear
solvers. Although not strictly necessary, the reader is assumed to have
some familiarity with distributed memory computing and, to a lesser
extent, with MPI\footnote{Although almost no explicit MPI instructions
  are required in a Trilinos code, the reader should be aware of the
  basic concepts of message passing, like the definition of a
  communicator.}.

\smallskip

Note that this tutorial is not a substitute for individual packages'
documentation. Also, for an overview of all the Trilinos packages, the
Trilinos philosophy, and a description of the packages provided by
Trilinos, the reader is referred to \cite{Trilinos-Overview}.
Developers should also consider the Trilinos Developers' Guide, which
addresses many topics, including the development tools used by Trilinos'
developers, and a description of how to include a new package.

%%%
%%%
%%%

\section{Installation}
\label{sec:installing}

To obtain Trilinos, please follow the instructions at the Trilinos download
page:
\begin{verbatim}
http://trilinos.sandia.gov/download
\end{verbatim}

Trilinos has been compiled on a variety of architectures, including
various flavors of Linux, Sun Solaris, SGI Irix, DEC, Mac OSX, IBM
AIX, ASC Red Storm, and others. Trilinos has been designed to support parallel
applications.  However, it also compiles and runs on serial computers.
An introduction to Trilinos and a list of FAQs
may be found at the web pages:
\begin{verbatim}
http://trilinos.sandia.gov/getting_started.html
http://trilinos.sandia.gov/faq.html
\end{verbatim}

After obtaining Trilinos, the next step is its compilation.  Instructions for
building Trilinos are available online:

\begin{verbatim}
http://trilinos.sandia.gov/build_instructions.html
\end{verbatim}

%%%
%%%
%%%

\section{Copyright and Licensing of Trilinos}
\label{sec:copyright}

Trilinos is released under the Lesser GPL GNU Licence.

Trilinos is copyrighted by Sandia Corporation. Under the terms of
Contract DE-AC04-94AL85000, there is a non-exclusive license for use of
this work by or on behalf of the U.S. Government.  Export of this
program may require a license from the United States Government.

NOTICE: The United States Government is granted for itself and others
acting on its behalf a paid-up, nonexclusive, irrevocable worldwide
license in ths data to reproduce, prepare derivative works, and perform
publicly and display publicly.  Beginning five (5) years from July 25,
2001, the United States Government is granted for itself and others
acting on its behalf a paid-up, nonexclusive, irrevocable worldwide
license in this data to reproduce, prepare derivative works, distribute
copies to the public, perform publicly and display publicly, and to
permit others to do so.

NEITHER THE UNITED STATES GOVERNMENT, NOR THE UNITED STATES DEPARTMENT
OF ENERGY, NOR SANDIA CORPORATION, NOR ANY OF THEIR EMPLOYEES, MAKES ANY
WARRANTY, EXPRESS OR IMPLIED, OR ASSUMES ANY LEGAL LIABILITY OR
RESPONSIBILITY FOR THE ACCURACY, COMPLETENESS, OR USEFULNESS OF ANY
INFORMATION, APPARATUS, PRODUCT, OR PROCESS DISCLOSED, OR REPRESENTS
THAT ITS USE WOULD NOT INFRINGE PRIVATELY OWNED RIGHTS.

\medskip

Some parts of Trilinos are dependent on a third party code. Each third
party code comes with its own copyright and/or licensing requirements.
It is responsibility of the user to understand these requirements.

%%%
%%%
%%%

\section{Programming Language Used in this Tutorial}
\label{sec:language}

Trilinos is written in C++ (for most packages), and in C. Some
interfaces are provided to FORTRAN codes (mainly BLAS and LAPACK
routines). Even if limited support is included for C programs (and a
more limited for FORTRAN code), to unleash the full power of Trilinos we
recommend C++. All the example programs contained in this tutorial are
in C++; some packages (like ML) contain examples in C.

%%%
%%%
%%%

\section{Referencing Trilinos}
\label{sec:referencing}

The Trilinos project can be referenced by using the following BiBTeX
citation information:
\begin{verbatim}
@techreport{Trilinos-Overview,
title = "{An Overview of Trilinos}",
author = "Michael Heroux and Roscoe Bartlett and Vicki Howle
Robert Hoekstra and Jonathan Hu and Tamara Kolda and
Richard Lehoucq and Kevin Long and Roger Pawlowski and
Eric Phipps and Andrew Salinger and Heidi Thornquist and
Ray Tuminaro and James Willenbring and Alan Williams ",
institution = "Sandia National Laboratories",
number = "SAND2003-2927",
year = 2003}

@techreport{Trilinos-Dev-Guide,
title = "{Trilinos Developers Guide}",
author = "Michael A. Heroux and James M. Willenbring and Robert Heaphy",
institution = "Sandia National Laboratories",
number = "SAND2003-1898",
year = 2003}

@techreport{Trilinos-Dev-Guide-II,
title = "{Trilinos Developers Guide Part II: ASCI Software Quality
Engineering Practices Version 1.0}",
author = "Michael A. Heroux and James M. Willenbring and Robert Heaphy",
institution = "Sandia National Laboratories",
number = "SAND2003-1899",
year = 2003}

@techreport{Trilinos-Users-Guide,
title = "{Trilinos Users Guide}",
author = "Michael A. Heroux and James M. Willenbring",
institution = "Sandia National Laboratories",
number = "SAND2003-2952",
year = 2003}

@techreport{Trilinos-Tutorial-5.0,
title = "{Trilinos Tutorial}",
author = "Marzio Sala and Michael A. Heroux and David M. Day",
institution = "Sandia National Laboratories",
number = "SAND2004-2189",
year = 2004}
\end{verbatim}
The BiBTeX information is available at the web page
\begin{verbatim}
http://trilinos.sandia.gov/citing.html
\end{verbatim}

%%%
%%%
%%%

\section{A Note on the Directory Structure}
\label{sec:into_note}

Each Trilinos package in contained in the subdirectory
\begin{verbatim}
<your-trilinos-directory>/packages
\end{verbatim}
Each package generally contains sources, examples, tests and documentation
subdirectories:
\begin{verbatim}
<your-trilinos-directory>/packages/<package-name>/src
<your-trilinos-directory>/packages/<package-name>/examples
<your-trilinos-directory>/packages/<package-name>/test
<your-trilinos-directory>/packages/<package-name>/doc
\end{verbatim}
Developers' documentation is written using Doxygen\footnote{Copyright
  \copyright 1997-2003 by Dimitri van Heesch. More information can by
  found at the web address {\tt
    http://www.stack.nl/~dimitri/doxygen/}.}.  The Doxygen documentation, and
other available documentation, for each package is available online via 

\begin{verbatim}
http://trilinos.sandia.gov/packages/<package-name>/documentation.html
\end{verbatim}

The Doxygen documentation can also be generated from the source code directly
when accessing a version-controlled copy of the Trilinos source code.  For
some packages, Doxygen documentation can also be generated from the 
distribution tarball.

 For example, to create
the documentation for Epetra, use the following commands:
\begin{verbatim}
$ cd <your-trilinos-home>/packages/epetra/doc
$ doxygen
\end{verbatim}
Generally, both HTML and \LaTeX~documentation are created by Doxygen.
The browser of choice can be used to walk through the HTML
documentation.  To compile the \LaTeX~sources, the commands are:
\begin{verbatim}
$ cd <your-trilinos-home>/packages/epetra/doc/latex
$ make
\end{verbatim}

%%%
%%%
%%%

\section{List of Trilinos Developers}
\label{sec:intro_incomplete}

A list of past and present Trilinos developers is available online at:
\begin{verbatim}
http://trilinos.sandia.gov/team.html
\end{verbatim}



    %-----------------------------%
    \chapter{Multigrid background}\label{sec:multigrid}
    %-----------------------------%
    \label{sec:multigrid intro}
Here we provide a brief multigrid introduction (see~\cite{MGTutorial}
or~\cite{OwlBook} for more information). A multigrid solver tries to approximate
the original problem of interest with a sequence of smaller (\textit{coarser})
problems. The solutions from the coarser problems are combined in order to
accelerate convergence of the original (\textit{fine}) problem on the finest
grid. A simple multilevel iteration is illustrated in
Algorithm~\ref{multigrid_code}.

\begin{algorithm}
\centering
\begin{algorithmic}[0]
  \State{$A_0 = A$}
  \Function{Multilevel}{$A_k$, $b$, $u$, $k$}
    \State{// Solve $A_k$ u = b (k is current grid level)}
    \State $ u = S^{1}_m (A_k, b, u)$
      \If{$(k \ne {\bf N-1})$}
        \State{$P_k = $ determine\_interpolant( $A_k$ )}
        \State{$R_k = $ determine\_restrictor( $A_k$ )}
        \State{$\widehat{r}_{k+1} = R_k (b - A_k u )$}
        \State{$A_{k+1} = R_k A_k P_k$}
        \State{$v = 0$}
        \State{}\Call{Multilevel}{$\widehat{A}_{k+1}$, $\widehat{r}_{k+1}$, $v$, $k+1$}
        \State{$ u = u + P_{k} v$}
        \State{$ u = S^{2}_m (A_k, b, u )$}
      \EndIf
  \EndFunction
\end{algorithmic}
\caption{V-cycle multigrid with $N$ levels to solve $Ax=b$.}
\label{multigrid_code}
\end{algorithm}

In the multigrid iteration in Algorithm~\ref{multigrid_code}, the $S^{1}_m()$'s
and $S^{2}_m()$'s are called \textit{pre-smoothers} and \textit{post-smoothers}.
They are approximate solvers (e.g., symmetric Gauss-Seidel), with the subscript
$m$ denoting the number of applications of the approximate solution method. The
purpose of a smoother is to quickly reduce certain error modes in the
approximate solution on a level $k$. For symmetric problems, the pre-
and post-smoothers should be chosen to maintain symmetry (e.g., forward
Gauss-Seidel for the pre-smoother and backward Gauss-Seidel for the
post-smoother). The $P_k$'s are \textit{interpolation} matrices that transfer
solutions from coarse levels to finer levels. The $R_k$'s are
\textit{restriction} matrices that restrict a fine level solution to a coarser
level. In a geometric multigrid, $P_k$'s and $R_k$'s are determined
by the application, whereas in an algebraic multigrid they are automatically
generated. For symmetric problems, typically $R_k=P_k^T$. For nonsymmetric
problems, this is not necessarily true. The $A_k$'s are the coarse level
problems, and are generated through a Galerkin (triple matrix) product.

Please note that the algebraic multigrid algorithms implemented in \muelu{}
generate the grid transfers $P_k$ automatically and the coarse problems $A_k$
via a sparse triple matrix product. \trilinos{} provides a wide selection of
smoothers and direct solvers for use in \muelu through the \ifpack,
\ifpacktwo, \amesos, and \amesostwo packages (see \S\ref{sec:options}).



    %-----------------------------%
    \chapter{Getting Started}\label{sec:getting started}
    %-----------------------------%
    This section is meant to get you using \muelu{} as quickly as possible.  \S\ref{sec:overview} gives a
summary of \muelu's design.  \S\ref{sec:configuration and build} lists \muelu's dependencies on other
\trilinos libraries and provides a sample cmake configuration line.  Finally, code examples using the XML
interface are given in \S\ref{sec:examples in code}.

\label{sec:overview}
\section{Overview of \muelu}
%algorithm types
%problems types
\muelu{} is an extensible algebraic multigrid (AMG) library that is part of the
\trilinos{} project. \muelu{} works with \epetra (32-bit version
\footnote{Support for the Epetra 64-bit version is planned.}) and
\tpetra matrix types. The library is written in C++ and allows for different
ordinal (index) and scalar types.  \muelu{} is designed to be efficient on many
different computer architectures, from workstations to supercomputers, relying
on ``MPI+X" principle, where ``X" can be threading or CUDA.

\muelu{} provides a number of different multigrid algorithms:
\be
  \item smoothed aggregation AMG (for Poisson-like and elasticity problems);
  \item Petrov-Galerkin aggregation AMG (for convection-diffusion problems);
  \item energy-minimizing AMG;
  \item aggregation-based AMG for problems arising from the eddy current
    formulation of Maxwell's equations.
\ee
\muelu's software design allows for the rapid introduction of new multigrid algorithms.
The most important features of \muelu{} can be summarized as:
\begin{description}
  \item \textbf{Easy-to-use interface}

    \muelu{} has a user-friendly parameter input deck which covers
    most important use cases.  Reasonable defaults are provided for common problem types
    (see Table \ref{t:problem_types}).

  \item \textbf{Modern object-oriented software architecture}

    \muelu{} is written completely in C++ as a modular object-oriented multigrid
    framework, which provides flexibility to combine and reuse existing
    components to develop novel multigrid methods.

  \item \textbf{Extensibility}

    Due to its flexible design, \muelu{} is an excellent toolkit for
    research on novel multigrid concepts. Experienced multigrid users have full
    access to the underlying framework through an advanced XML based interface.
    Expert users may use and extend the C++ API directly.

  \item \textbf{Integration with \trilinos{} library}

    As a package of \trilinos, \muelu{} is well integrated into the \trilinos
    environment. \muelu{} can be used with either the \tpetra{} or \epetra{}
    (32-bit) linear algebra stack. It is templated on the local index, global
    index, scalar, and compute node types. This makes \muelu{} ready for
    future developments.

  \item \textbf{Broad range of supported platforms}

    \muelu{} runs on wide variety of architectures, from desktop workstations to
    parallel Linux clusters and supercomputers (~\cite{lin2014}).

  \item \textbf{Open source}

    \muelu{} is freely available through a simplified BSD license (see Appendix~\ref{sec:license}).
\end{description}

\section{Configuration and Build}
\label{sec:configuration and build}

\muelu{} has been compiled successfully under Linux with the following C++
compilers: GNU versions 4.1 and later, Intel versions 12.1/13.1, and clang versions 3.2 and later.
In the future, we recommend using compilers supporting C++11 standard.

\subsection{Dependencies}

\noindent{\bf Required Dependencies}

\muelu{} requires that \teuchos{} and either \epetra/\ifpack or \tpetra/\ifpacktwo
are enabled.

\noindent{\bf Recommended Dependencies}

We strongly recommend that you enable the following \trilinos libraries along with \muelu:

\begin{itemize}
  \item \epetra stack: \aztecoo, \epetra, \amesos, \ifpack, \isorropia, \galeri,
    \zoltan;
  \item \tpetra stack: \amesostwo, \belos, \galeri, \ifpacktwo, \tpetra,
    \zoltantwo.
\end{itemize}

\noindent{\bf Tutorial Dependencies}

In order to run the \muelu{} Tutorial \cite{MueLuTutorial} located in \verb!muelu/doc/Tutorial!, \muelu{} must be configured with the following
dependencies enabled:

  \aztecoo, \amesos, \amesostwo, \belos, \epetra, \ifpack, \ifpacktwo, \isorropia,
  \galeri, \tpetra, \zoltan, \zoltantwo.

\begin{mycomment}
Note that the \muelu{} tutorial \cite{MueLuTutorial} comes with a VirtualBox image with a pre-installed
Linux and \trilinos{}.   In this way, a user can immediately begin experimenting with \muelu{} without
having to install the \trilinos{} libraries. Therefore, it is an ideal starting point to get in touch with \muelu{}.
\end{mycomment}

\noindent{\bf Complete List of Direct Dependencies}

\begin{table}[ht]
  \centering
  \begin{tabular}{p{3.5cm} c c c c}
    \toprule
    \multirow{2}{*}{Dependency} & \multicolumn{2}{c}{Required} & \multicolumn{2}{c}{Optional} \\
    \cmidrule(r){2-3} \cmidrule(l){4-5}
                   & Library  & Testing  & Library  & Testing        \\
    \hline
    \amesos        &          &          & $\times$ & $\times$  \\
    \amesostwo     &          &          & $\times$ & $\times$  \\
    \aztecoo       &          &          &          & $\times$  \\
    \belos         &          &          &          & $\times$  \\
    \epetra        &          &          & $\times$ & $\times$  \\
    \ifpack        &          &          & $\times$ & $\times$  \\
    \ifpacktwo     &          &          & $\times$ & $\times$  \\
    \isorropia     &          &          & $\times$ & $\times$  \\
    \galeri        &          &          &          & $\times$  \\
    \kokkosclassic &          &          & $\times$ & \\
    \teuchos{}     & $\times$ & $\times$ &          & \\
    \tpetra        &          &          & $\times$ & $\times$  \\
    \xpetra        & $\times$ & $\times$ &          & \\
    \zoltan        &          &          & $\times$ & $\times$  \\
    \zoltantwo     &          &          & $\times$ & $\times$  \\
    \midrule
    Boost          &          &          & $\times$ & \\
    BLAS           & $\times$ & $\times$ &          & \\
    LAPACK         & $\times$ & $\times$ &          & \\
    MPI            &          &          & $\times$ & $\times$  \\
    \bottomrule
  \end{tabular}
  \caption{\label{tab:dependencies}\muelu's required and optional dependencies,
    subdivided by whether a dependency is that of the \muelu{} library itself
    (\textit{Library}), or of some \muelu{} test (\textit{Testing}). }
\end{table}

Table~\ref{tab:dependencies} lists the dependencies of \muelu, both required and
optional. If an optional dependency is not present, the tests requiring that
dependency are not built.

\begin{mycomment}
\amesos/\amesostwo are necessary if one wants to use a sparse direct solve on the coarsest level.
\zoltan/\zoltantwo are necessary if one wants to use matrix rebalancing in parallel runs (see~\S\ref{sec:performance}).
\aztecoo/\belos are necessary if one wants to test \muelu{} as a preconditioner instead of a solver.
\end{mycomment}

\begin{mycomment}
\muelu{} has also been successfully tested with SuperLU 4.1 and SuperLU 4.2.
\end{mycomment}
\begin{mycomment}
Some packages that \muelu{} depends on may come with additional requirements for
third party libraries, which are not listed here as explicit dependencies of \muelu{}.
It is highly recommended to install ParMetis 3.1.1 or newer for \zoltan{},
ParMetis 4.0.x for \zoltantwo{}, and SuperLU 4.1 or newer for
\amesos{}/\amesostwo{}.
\end{mycomment}

\subsection{Configuration}
The preferred way to configure and build \muelu{} is to do that outside of the source directory.
Here we provide a sample configure script that will enable \muelu{} and all of its optional dependencies:
\begin{lstlisting}
  export TRILINOS_HOME=/path/to/your/Trilinos/source/directory
  cmake -D BUILD_SHARED_LIBS:BOOL=ON \
        -D CMAKE_BUILD_TYPE:STRING="RELEASE" \
        -D CMAKE_CXX_FLAGS:STRING="-g" \
        -D Trilinos_ENABLE_EXPLICIT_INSTANTIATION:BOOL=ON \
        -D Trilinos_ENABLE_TESTS:BOOL=OFF \
        -D Trilinos_ENABLE_EXAMPLES:BOOL=OFF \
        -D Trilinos_ENABLE_MueLu:BOOL=ON \
        -D MueLu_ENABLE_TESTS:STRING=ON \
        -D MueLu_ENABLE_EXAMPLES:STRING=ON \
        -D TPL_ENABLE_BLAS:BOOL=ON \
        -D TPL_ENABLE_MPI:BOOL=ON \
        ${TRILINOS_HOME}
\end{lstlisting}

\noindent
More configure examples can be found in \texttt{Trilinos/sampleScripts}.
For more information on configuring, see the \trilinos Cmake Quickstart guide \cite{TrilinosCmakeQuickStart}.

\section{Examples in code}
\label{sec:examples in code}
% simple scaling test
%   galeri
%   XML input
%   belos/aztecoo or stand-alone solver
%   look @ tutorial or elsewhere for more advanced usage

The most commonly used scenario involving \muelu{} is using a multigrid
preconditioner inside an iterative linear solver. In \trilinos{}, a user has a
choice between \epetra and \tpetra for the underlying linear algebra library.
Important Krylov subspace methods (such as preconditioned CG and GMRES) are
provided in \trilinos{} packages \aztecoo (\epetra{}) and \belos
(\epetra{}/\tpetra{}).

At this point, we assume that the reader is comfortable with \teuchos{} referenced-counted
pointers (RCPs) for memory management (an introduction to RCPs can be found
in~\cite{RCP2010}) and the \texttt{Teuchos::ParameterList} class~\cite{TeuchosURL}.

\subsection{\muelu{} as a preconditioner within \belos}
\label{sec:tpetraexample}
The following code shows the basic steps of how to use a \muelu{}
multigrid preconditioner with \tpetra{} linear algebra library and with a linear
solver from \belos{}. To keep the example short and clear, we skip the template
parameters and focus on the algorithmic outline of setting up
a linear solver. For further details, a user may refer to the \texttt{examples} and
\texttt{test} directories.

First, we create the \muelu{} multigrid preconditioner. It can be done in a few
ways. For instance, multigrid parameters can be read from an XML file
(e.g., \textit{mueluOptions.xml} in the example below).
\begin{lstlisting}
    Teuchos::RCP<Tpetra::CrsMatrix<> > A;
    // create A here ...
    std::string optionsFile = "mueluOptions.xml";
    Teuchos::RCP<MueLu::TpetraOperator> mueLuPreconditioner =
       MueLu::CreateTpetraPreconditioner(A, optionsFile);
\end{lstlisting}
The XML file contains multigrid options. A typical file with \muelu{} parameters
looks like the following.
\begin{lstlisting}
<ParameterList name="MueLu">

  <Parameter name="verbosity" type="string" value="low"/>

  <Parameter name="max levels" type="int" value="3"/>
  <Parameter name="coarse: max size" type="int" value="10"/>

  <Parameter name="multigrid algorithm" type="string" value="sa"/>

  <!-- Damped Jacobi smoothing -->
  <Parameter name="smoother: type" type="string" value="RELAXATION"/>
  <ParameterList name="smoother: params">
    <Parameter name="relaxation: type"  type="string" value="Jacobi"/>
    <Parameter name="relaxation: sweeps" type="int" value="1"/>
    <Parameter name="relaxation: damping factor" type="double" value="0.9"/>
  </ParameterList>

  <!-- Aggregation -->
  <Parameter name="aggregation: type" type="string" value="uncoupled"/>
  <Parameter name="aggregation: min agg size" type="int" value="3"/>
  <Parameter name="aggregation: max agg size" type="int" value="9"/>

</ParameterList>
\end{lstlisting}
It defines a three level smoothed aggregation multigrid algorithm. The
aggregation size is between three and nine(2D)/27(3D) nodes.  One sweep with a
damped Jacobi method is used as a level smoother. By default, a direct solver is
applied on the coarsest level. A complete list of available parameters and valid
parameter choices is given in \S\ref{sec:muelu_options} of this User's Guide.

Users can also construct a multigrid preconditioner using a provided \parameterlist
without accessing any files in the following manner.
\begin{lstlisting}
  Teuchos::RCP<Tpetra::CrsMatrix<> > A;
  // create A here ...
  Teuchos::ParameterList paramList;
  paramList.set("verbosity", "low");
  paramList.set("max levels", 3);
  paramList.set("coarse: max size", 10);
  paramList.set("multigrid algorithm", "sa");
  // ...
  Teuchos::RCP<MueLu::TpetraOperator> mueLuPreconditioner =
     MueLu::CreateTpetraPreconditioner(A, paramList);
\end{lstlisting}

Besides the linear operator $A$, we also need an initial guess vector for the
solution $X$ and a right hand side vector $B$ for solving a linear system.
\begin{lstlisting}
    Teuchos::RCP<const Tpetra::Map<> > map = A->getDomainMap();

    // Create initial vectors
    Teuchos::RCP<Tpetra::MultiVector<> > B, X;
    X = Teuchos::rcp( new Tpetra::MultiVector<>(map,numrhs) );
    Belos::MultiVecTraits<>::MvRandom( *X );
    B = Teuchos::rcp( new Tpetra::MultiVector<>(map,numrhs) );
    Belos::OperatorTraits<>::Apply( *A, *X, *B );
    Belos::MultiVecTraits<>::MvInit( *X, 0.0 );
\end{lstlisting}
To generate a dummy example, the above code first declares two vectors. Then, a
right hand side vector is calculated as the matrix-vector product of a random vector
with the operator $A$. Finally, an initial guess is initialized with zeros.

Then, one can define a \texttt{Belos::LinearProblem} object where the
\texttt{mueLuPreconditioner} is used for left preconditioning
\begin{lstlisting}
    Belos::LinearProblem<> problem( A, X, B );
    problem->setLeftPrec(mueLuPreconditioner);
    bool set = problem.setProblem();
\end{lstlisting}

Next, we set up a \belos{} solver using some basic parameters
\begin{lstlisting}
    Teuchos::ParameterList belosList;
    belosList.set( "Block Size", 1 );
    belosList.set( "Maximum Iterations", 100 );
    belosList.set( "Convergence Tolerance", 1e-10 );
    belosList.set( "Output Frequency", 1 );
    belosList.set( "Verbosity", Belos::TimingDetails + Belos::FinalSummary );

    Belos::BlockCGSolMgr<> solver( rcp(&problem,false), rcp(&belosList,false) );
\end{lstlisting}

Finally, we solve the system.
\begin{lstlisting}
    Belos::ReturnType ret = solver.solve();
\end{lstlisting}

\subsection{\muelu{} as a preconditioner for \aztecoo}

For \epetra, users have two library options: \belos{} (recommended) and \aztecoo{}.
\aztecoo{} and \belos both provide fast and mature implementations of common iterative Krylov linear solvers.
\belos has additional capabilities, such as Krylov subspace recycling and ``tall skinny QR".

Constructing a \muelu{} preconditioner for Epetra operators is done in a similar
manner to Tpetra.
\begin{lstlisting}
    Teuchos::RCP<Epetra_CrsMatrix> A;
    // create A here ...
    Teuchos::RCP<MueLu::EpetraOperator> mueLuPreconditioner;
    std::string optionsFile = "mueluOptions.xml";
    mueLuPreconditioner = MueLu::CreateEpetraPreconditioner(A, optionsFile);
\end{lstlisting}
\muelu{} parameters are generally Epetra/Tpetra agnostic, thus the XML parameter file
could be the same as~\S\ref{sec:tpetraexample}.

Furthermore, we assume that a right hand side vector and a solution vector with
the initial guess are defined.
\begin{lstlisting}
    Teuchos::RCP<const Epetra_Map> map = A->DomainMap();
    Teuchos::RCP<Epetra_Vector> B = Teuchos::rcp(new Epetra_Vector(map));
    Teuchos::RCP<Epetra_Vector> X = Teuchos::rcp(new Epetra_Vector(map));
    X->PutScalar(0.0);
\end{lstlisting}

Then, an \texttt{Epetra\_LinearProblem} can be defined.
\begin{lstlisting}
    Epetra_LinearProblem epetraProblem(A.get(), X.get(), B.get());
\end{lstlisting}

The following code constructs an \aztecoo{} CG solver.
\begin{lstlisting}
    AztecOO aztecSolver(epetraProblem);
    aztecSolver.SetAztecOption(AZ_solver, AZ_cg);
    aztecSolver.SetPrecOperator(mueLuPreconditioner.get());
\end{lstlisting}

Finally, the linear system is solved.
\begin{lstlisting}
    int maxIts = 100;
    double tol = 1e-10;
    aztecSolver.Iterate(maxIts, tol);
\end{lstlisting}

\subsection{Further remarks}

This section is only meant to give a brief introduction on how to use \muelu{}
as a preconditioner within the \trilinos{} packages for iterative solvers. There
are other, more complicated, ways to use \muelu{} as a preconditioner for \belos
and \aztecoo through the \xpetra interface. Of course, \muelu{} can also work as
standalone multigrid solver. For more information on these topics, the reader
may refer to the examples and tests in the \muelu{} source folder
(\texttt{Trilinos/packages/muelu}), well as to the \muelu{} tutorial~\cite{MueLuTutorial}.
For in-depth details of \muelu applied to multiphysics problems, please see~\cite{Wiesner2014}.


    %-----------------------------%
    \chapter{Performance tips}\label{sec:performance}
    %-----------------------------%
    In practice, it can be very challenging to find an appropriate set of multigrid
parameters for a specific problem, especially if few details are known about the
underlying linear system. In this Chapter, we provide some advice for improving
multigrid performance.

\begin{mycomment}
For optimizing multigrid parameters, it is highly recommended to set the
verbosity to \verb|high| or \verb|extreme| for \muelu{} to output more
information (e.g., for the effect of the chosen parameters to the aggregation
and coarsening process).
\end{mycomment}

Some general advice:
\begin{itemize}
  \item
    Choose appropriate iterative linear solver (e.g., GMRES for non-symmetric problems).

  \item
    Start with the recommended settings for particular problem types. See
    Table~\ref{t:problem_types}.

  \item
    Choose reasonable basic multigrid parameters
    (see~\S\ref{sec:options_general}), including maximum number of multigrid
    levels (\texttt{max levels}) and maximum allowed coarse size of the problem
    (\texttt{coarse: max size}). Take fine level problem size and sparsity
    pattern into account for a reasonable choice of these parameters.

  \item
    Select an appropriate transfer operator strategy
    (see~\S\ref{sec:options_mg}). For symmetric problems, you should start with smoothed
    aggregation multigrid. For non-symmetric problems, a Petrov-Galerkin smoothed
    aggregation method is a good starting point, though smoothed aggregation may
    also perform well.

  \item
    Enable implicit restrictor construction (\texttt{transpose:} \texttt{use implicit}) for symmetric
    problems.

  \item
    Find good level smoothers (see~\S\ref{sec:options_smoothing}). If a problem
    is symmetric positive definite, choose a smoother with a matrix-vector
    computational kernel, such as the Chebyshev polynomial smoother. If you are
    using relaxation smoothers, we recommend starting with optimizing the
    damping parameter. Once you have found a good damping parameter for your
    problem, you can increase the number of smoothing iterations.

  \item
    Adjust aggregation parameters if you experience bad coarsening ratios
    (see~\S\ref{sec:options_aggregation}). Particularly, try adjusting the
    minimum (\texttt{aggregation:} \texttt{min agg size}) and maximum
    (\texttt{aggregation:} \texttt{max agg size}) aggregation parameters. For a
    2D (3D) isotropic problem on a regular mesh, the aggregate size should be
    about 9 (27) nodes per aggregate.

  \item
    Replace a direct solver with an iterative method (\texttt{coarse: type}) if
    your coarse level solution becomes too expensive (see~\S\ref{sec:options_smoothing}).
\end{itemize}

Some advice for parallel runs include:
\begin{enumerate}
  \item
    Enable matrix rebalancing when running in parallel (\texttt{repartition:}
    \texttt{enable}).

  \item
    Use smoothers invariant to the number of processors, such as
    polynomial smoothing, if possible.

  \item
    Use \texttt{uncoupled} aggregation instead of \texttt{coupled}, as later
    requires significantly more communication.

  \item
    Adjust rebalancing parameters (see~\S\ref{sec:options_rebalancing}). Try
    choosing rebalancing parameters so that you end up with one processor on the
    coarsest level for the direct solver (this avoids additional communication).

  \item
    Enable implicit rebalancing of prolongators and restrictors
    (\texttt{repartition: rebalance P and R}).
\end{enumerate}


    %-----------------------------%
    \chapter{\muelu{} options} \label{sec:options}
    %-----------------------------%
    \label{sec:muelu_options}

In this section, we report the complete list of \muelu{} input parameters.  It
is important to notice, however, that \muelu{} relies on other \trilinos{}
packages to provide support for some of its algorithms. For instance,
\ifpack{}/\ifpacktwo{} provide standard smoothers like Jacobi, Gauss-Seidel or
Chebyshev, while \amesos{}/\amesostwo{} provide access to direct solvers. The
parameters affecting the behavior of the algorithms in those packages are
simply passed by \muelu{} to a routine from the corresponding library. Please
consult corresponding packages' documentation for a full list of supported
algorithms and corresponding parameters.

\section{Using parameters on individual levels}
Some of the parameters that affect the preconditioner can in principle be
different from level to level. By default, parameters affect all levels in
a multigrid hierarchy.

The settings on a particular levels can be changed by using level sublists.
A level sublist is a \parameterlist{} sublist with the name ``level XX'', where XX is the level number. The
parameter names in the sublist do not require any modifications. For example,
the following fragment of code
\begin{lstlisting}[language=XML]
  <ParameterList name="level 2">
    <Parameter name="smoother: type" type="string" value="CHEBYSHEV"/>
  </ParameterList>
\end{lstlisting}
changes the smoother for level 2 to be a polynomial smoother.

\section{Parameter validation}
By default, \muelu{} validates the input parameter list. A parameter that is
misspelled, is unknown, or has an incorrect value type will cause an exception to be
thrown and execution to halt.

\begin{mycomment}
Spaces are important within a parameter's name. Please separate words
by just one space, and make sure there are no leading or trailing spaces.
\end{mycomment}

The option \verb|print initial parameters| prints the initial list given to the
interpreter. The option \verb|print unused parameters| prints the list of unused
parameters.

% ==================== GENERAL ====================
\section{General options}
\label{sec:options_general}

\begin{table}[h!]
  \begin{center}
    \begin{tabular}{p{4.3cm} p{4.3cm} c p{4.5cm}}
      \toprule
      Problem type               & Multigrid algorithm    & Block size  & Smoother \\
      \midrule
      \verb!unknown!             & --                     & --          & -- \\
      \verb!Poisson-2D!          & Smoothed aggregation   & 1           & Chebyshev \\
      \verb!Poisson-3D!          & Smoothed aggregation   & 1           & Chebyshev \\
      \verb!Elasticity-2D!       & Smoothed aggregation   & 2           & Chebyshev \\
      \verb!Elasticity-3D!       & Smoothed aggregation   & 3           & Chebyshev \\
      \verb!ConvectionDiffusion! & Petrov-Galerkin  AMG   & 1           & Gauss-Seidel \\
      \verb!MHD!                 & Unsmoothed aggregation & --          & Additive Schwarz method with one level of overlap and ILU(0) as a subdomain solver \\
      \bottomrule
    \end{tabular}
    \caption{Supported problem types (``--'' means not set).}
\label{t:problem_types}
  \end{center}
\end{table}


\cbb{problem: type}{string}{"unknown"}{Type of problem to be solved.  Possible values: see Table~\ref{t:problem_types}.}

\cbb{verbosity}{string}{"high"}{Control of the amount of printed information. Possible values: "none", "low", "medium", "high", "extreme".}

\cbb{number of equations}{int}{1}{Number of PDE equations at each grid node. Only constant block size is considered.}

\cbb{max levels}{int}{10}{Maximum number of levels in a hierarchy.}

\cbb{cycle type}{string}{"V"}{Multigrid cycle type. Possible values: "V", "W".}

\cbb{problem: symmetric}{bool}{true}{Symmetry of a problem. This setting affects the construction of a restrictor. If set to true, the restrictor is set to be the transpose of a prolongator. If set to false, underlying multigrid algorithm makes the decision.}


% ==================== SMOOTHERS ====================
\section{Smoothing and coarse solver options}
\label{sec:options_smoothing}

\muelu{} relies on other \trilinos{} packages to provide level smoothers and
coarse solvers. \ifpack{} and \ifpacktwo{} provide standard smoothers (see
Table~\ref{tab:smoothers}), and \amesos{} and \amesostwo{} provide direct
solvers (see Table~\ref{tab:coarse}). Note that it is completely possible to use
any level smoother as a direct solver.

\muelu{} checks parameters \verb|smoother: * type| and \verb|coarse: type| to
determine:
\begin{itemize}
  \item what package to use (i.e., is it a smoother or a direct solver);
  \item (possibly) transform a smoother type

    \ding{42} \ifpack{} and \ifpacktwo{} use different smoother type names,
    e.g., ``point relaxation stand-alone'' vs ``RELAXATION''.  \muelu{} tries to follow
    \ifpacktwo{} notation for smoother types. Please consult \ifpacktwo{}
    documentation~\cite{Ifpack2URL} for more information.
\end{itemize}
The parameter lists \verb|smoother: * params| and \verb|coarse: params| are
passed directly to the corresponding package without any examination of their
content. Please consult the documentation of the corresponding packages for a list of
possible values.

By default, \muelu{} uses one sweep of symmetric Gauss-Seidel for both pre- and
post-smoothing, and SuperLU for coarse system solver.

\begin{table}[tbh]
  \begin{center}
    \begin{tabular}{p{4.0cm} p{10cm}}
      \toprule
      \texttt{smoother: type}           & \\
      \midrule
      \verb|RELAXATION|                 & Point relaxation smoothers, including
                                          Jacobi, Gauss-Seidel, symmetric Gauss-Seidel, etc. The exact
                                          smoother is chosen by specifying \texttt{relaxation: type} parameter in
                                          the \texttt{smoother: params} sublist. \\
      \verb|CHEBYSHEV|                  & Chebyshev polynomial smoother. \\
      \verb|ILUT|, \verb|RILUK|         & Local (processor-based) incomplete factorization methods. \\
      \bottomrule
    \end{tabular}
    \caption{Commonly used smoothers provided by \ifpack{}/\ifpacktwo{}. Note
    that these smoothers can also be used as coarse grid solvers.}
\label{tab:smoothers}
  \end{center}
\end{table}

\begin{table}[tbh]
  \begin{center}
    \begin{tabular}{p{4.0cm} c c p{7cm}}
      \toprule
      \texttt{coarse: type}             & \amesos{} & \amesostwo{} &  \\
      \midrule
      \verb|KLU|                        & x & & Default \amesos{} solver~\cite{klu}. \\
      \verb|KLU2|                       & & x & Default \amesostwo{} solver~\cite{amesos2_belos}. \\
      \verb|SuperLU|                    & x & x & Third-party serial sparse direct solver~\cite{Li2011}. \\
      \verb|SuperLU_dist|               & x & x & Third-party parallel sparse direct solver~\cite{Li2011}. \\
      \verb|Umfpack|                    & x & & Third-party solver~\cite{umfpack}. \\
      \verb|Mumps|                      & x & & Third-party solver~\cite{mumps}. \\
      \bottomrule
    \end{tabular}
    \caption{Commonly used direct solvers provided by \amesos{}/\amesostwo{}}
\label{tab:coarse}
  \end{center}
\end{table}


\cbb{smoother: pre or post}{string}{"both"}{Pre- and post-smoother combination. Possible values: "pre" (only pre-smoother), "post" (only post-smoother), "both" (both pre-and post-smoothers), "none" (no smoothing).}
          
\cba{smoother: type}{string}{Smoother type. Possible values: see Table~\ref{tab:smoothers}.}
          
\cba{smoother: pre type}{string}{Pre-smoother type. Possible values: see Table~\ref{tab:smoothers}.}
          
\cba{smoother: post type}{string}{Post-smoother type. Possible values: see Table~\ref{tab:smoothers}.}
          
\cba{smoother: params}{\parameterlist}{Smoother parameters. For standard smoothers, \muelu passes them directly to the appropriate package library.}
          
\cba{smoother: pre params}{\parameterlist}{Pre-smoother parameters. For standard smoothers, \muelu passes them directly to the appropriate package library.}
          
\cba{smoother: post params}{\parameterlist}{Post-smoother parameters. For standard smoothers, \muelu passes them directly to the appropriate package library.}
          
\cbb{smoother: overlap}{int}{0}{Smoother subdomain overlap.}
          
\cbb{smoother: pre overlap}{int}{0}{Pre-smoother subdomain overlap.}
          
\cbb{smoother: post overlap}{int}{0}{Post-smoother subdomain overlap.}
          
\cbb{coarse: max size}{int}{2000}{Maximum dimension of a coarse grid. \muelu will stop coarsening once it is achieved.}
          
\cbb{coarse: type}{string}{"SuperLU"}{Coarse solver. Possible values: see Table~\ref{tab:coarse_solvers}.}
          
\cba{coarse: params}{\parameterlist}{Coarse solver parameters. \muelu passes them directly to the appropriate package library.}
          
\cbb{coarse: overlap}{int}{0}{Coarse solver subdomain overlap.}
          

% ==================== AGGREGATION ====================
\section{Aggregation options}
\label{sec:options_aggregation}

\begin{table}[h!]
  \begin{center}
    \begin{tabular}{p{5.0cm} p{10cm}}
      \toprule
      \verb!uncoupled! & Attempts to construct aggregates of optimal size ($3^d$
                         nodes in $d$ dimensions). Each process works independently, and
                         aggregates cannot span multiple processes.\\
      \verb!coupled!   & Attempts to construct aggregates of optimal size ($3^d$
                         nodes in $d$ dimensions). Aggregates are allowed to
                         cross processor boundaries. \textbf{Use carefully}. If
                         unsure, use \verb!uncoupled! instead.\\
      %\verb!METIS!     & Use graph partitioning algorithm to create aggregates,
      %                   working process-wise. Number of nodes in each aggregate
      %                   is specified with option \texttt{aggregation: max agg
      %                   size}. \\
      % \verb!ParMETIS!  & As \verb!METIS!, but partition global graph. Aggregates
                         % can span arbitrary number of processes. Specify global
                         % number of aggregates with {\tt aggregation: global
                         % number}. \\
      \bottomrule
    \end{tabular}
    \caption{Available coarsening schemes. }
\label{t:aggregation}
  \end{center}
\end{table}


\cbb{aggregation: type}{string}{"uncoupled"}{Aggregation scheme. Possible values: see Table~\ref{t:aggregation}.}
          
\cbb{aggregation: ordering}{string}{"natural"}{Node ordering strategy. Possible values: "natural" (local index order), "graph" (filtered graph breadth-first order), "random" (random local index order).}
          
\cbb{aggregation: drop scheme}{string}{"classical"}{Connectivity dropping scheme for a graph used in aggregation. Possible values: "classical", "distance laplacian".}
          
\cbb{aggregation: drop tol}{double}{0.0}{Connectivity dropping threshold for a graph used in aggregation.}
          
\cbb{aggregation: min agg size}{int}{2}{Minimum size of an aggregate.}
          
\cbb{aggregation: max agg size}{int}{-1}{Maximum size of an aggregate (-1 means unlimited).}
          
\cbb{aggregation: Dirichlet threshold}{double}{0.0}{Threshold for determining whether entries are zero during Dirichlet row detection.}
          
\cbb{aggregation: export visualization data}{bool}{false}{Export data for visualization post-processing.}
          

% ==================== REBALANCING ====================
\section{Rebalancing options}
\label{sec:options_rebalancing}


\cbb{repartition: enable}{bool}{false}{Rebalancing on/off switch.}
          
\cbb{repartition: partitioner}{string}{"zoltan2"}{Partitioning package to use. Possible values: "zoltan" (\zoltan{} library), "zoltan2" (\zoltantwo{} library).}
          
\cba{repartition: params}{\parameterlist}{Partitioner parameters. \muelu passes them directly to the appropriate package library.}
          
\cbb{repartition: start level}{int}{2}{Minimum level to run partitioner. \muelu does not rebalance levels finer than this one.}
          
\cbb{repartition: min rows per proc}{int}{800}{Minimum number of rows per processor. If the actual number if smaller, then rebalancing occurs.}
          
\cbb{repartition: max imbalance}{double}{1.2}{Maximum nonzero imbalance ratio. If the actual number is larger, the rebalancing occurs.}
          
\cbb{repartition: remap parts}{bool}{true}{Postprocessing for partitioning to reduce data migration.}
          
\cbb{repartition: rebalance P and R}{bool}{false}{Explicit rebalancing of R and P during the setup. This speeds up the solve, but slows down the setup phases.}
          

% ==================== MULTIGRID ====================
\section{Multigrid algorithm options}
\label{sec:options_mg}

\begin{table}[h!]
  \begin{center}
    \begin{tabular}{p{3.5cm} p{11cm}}
      \toprule
      \verb!sa!         & Classic smoothed aggregation~\cite{VMB1996} \\
      \verb!unsmoothed! & Aggregation-based, same as \verb!sa! but without damped Jacobi prolongator improvement step \\
      \verb!pg!         & Prolongator smoothing using $A$, restriction smoothing using $A^T$, local damping factors~\cite{ST2008} \\
      \verb!emin!       & Constrained minimization of energy in basis functions of grid transfer operator~\cite{WTWG2014,OST2011} \\
      \bottomrule
    \end{tabular}
    \caption{Available multigrid algorithms for generating grid transfer matrices. }
\label{t:mgs}
  \end{center}
\end{table}


\cbb{multigrid algorithm}{string}{"sa"}{Multigrid method. Possible values: see Table~\ref{t:mgs}.}

\cbb{semicoarsen: coarsen rate}{int}{3}{Rate at which to coarsen unknowns in the z direction.}

\cbb{sa: damping factor}{double}{1.33}{Damping factor for smoothed aggregation.}

\cbb{sa: use filtered matrix}{bool}{true}{Matrix to use for smoothing the tentative prolongator. The two options are: to use the original matrix, and to use the filtered matrix with filtering based on filtered graph used for aggregation.}

\cbb{filtered matrix: use lumping}{bool}{true}{Lump (add to diagonal) dropped entries during the construction of a filtered matrix. This allows user to preserve constant nullspace.}

\cbb{filtered matrix: reuse eigenvalue}{bool}{true}{Skip eigenvalue calculation during the construction of a filtered matrix by reusing eigenvalue estimate from the original matrix. This allows us to skip heavy computation, but may lead to poorer convergence.}

\cbb{emin: iterative method}{string}{"cg"}{Iterative method to use for energy minimization of initial prolongator in energy-minimization. Possible values: "cg" (conjugate gradient), "sd" (steepest descent).}

\cbb{emin: num iterations}{int}{2}{Number of iterations to minimize initial prolongator energy in energy-minimization.}

\cbb{emin: num reuse iterations}{int}{1}{Number of iterations to minimize the reused prolongator energy in energy-minimization.}

\cbb{emin: pattern}{string}{"AkPtent"}{Sparsity pattern to use for energy minimization. Possible values: "AkPtent".}

\cbb{emin: pattern order}{int}{1}{Matrix order for the "AkPtent" pattern.}


% ==================== REUSE ====================
\section{Reuse options}
\label{sec:options_reuse}

Reuse options are currently only used with \verb!sa! multigrid algorithm. We
also assume that the matrix preserves graph structure, and only matrix values
change.

\begin{table}[h!]
  \begin{center}
    \begin{tabular}{p{3.0cm} p{12cm}}
      \toprule
      \verb!none!   & No reuse \\
      \verb!RP!     & Reuse smoothed prolongator and restrictor. This also reuses matrix graphs during the construction
                      of the Galerkin product. Smoothers are recomputed. \\
      % \verb!tP!     & Reuse tentative prolongator. The graphs of smoothed prolongator and matrices in Galerkin product are reused only
                      % if filtering is not being used ({\it i.e.}, either \verb!sa: use filtered matrix! or \verb!aggregation: drop tol! is
                      % false) \\
      \verb!RAP!    & Recompute only the finest level smoothers, reuse all other operators \\
      \verb!full!   & Reuse everything \\
      \bottomrule
    \end{tabular}
    \caption{Available coarsening schemes. }
\label{t:reuse_types}
  \end{center}
\end{table}


\cbb{reuse: type}{string}{"none"}{Reuse options for consecutive hierarchy construction. This speeds up the setup phase, but may lead to poorer convergence. Possible values: see Table~\ref{t:reuse_types}.}


% ==================== MISCELLANEOUS ====================
\section{Miscellaneous options}


\cba{export data}{\parameterlist}{Exporting a subset of the hierarchy data in a file. Currently, the list can contain any of three parameter names ("A", ``P'', ``R'') of type \texttt{string} and value ``\{levels separated by commas\}''. A matrix with a name ``X'' is saved in three MatrixMarket files: a) data is saved in \textit{X\_level.mm}; b) its row map is saved in \textit{rowmap\_X\_level.mm}; c) its column map is saved in \textit{colmap\_X\_level.mm}.}

\cbb{print initial parameters}{bool}{true}{Print parameters provided for a hierarchy construction.}

\cbb{print unused parameters}{bool}{true}{Print parameters unused during a hierarchy construction.}

\cbb{transpose: use implicit}{bool}{false}{Use implicit transpose for the restriction operator.}



    %\nocite{*}

    % ---------------------------------------------------------------------- %
    % References
    %
    \clearpage
    % If hyperref is included, then \phantomsection is already defined.
    % If not, we need to define it.
    \providecommand*{\phantomsection}{}
    \phantomsection
    \addcontentsline{toc}{chapter}{References}
    \bibliographystyle{plain}
    \bibliography{mueluguide}


    % ---------------------------------------------------------------------- %
    %
    \appendix
    \chapter{Copyright and License}
    \label{sec:license}
\begin{center}
MueLu: A package for multigrid based preconditioning

Copyright 2012 Sandia Corporation
\end{center}

\noindent
Under the terms of Contract DE--AC04--94AL85000 with Sandia Corporation,
the U.S. Government retains certain rights in this software.

\noindent
Redistribution and use in source and binary forms, with or without
modification, are permitted provided that the following conditions are
met:

\begin{enumerate}
  \item Redistributions of source code must retain the above copyright
    notice, this list of conditions and the following disclaimer.

\item Redistributions in binary form must reproduce the above copyright
  notice, this list of conditions and the following disclaimer in the
  documentation and/or other materials provided with the distribution.

\item Neither the name of the Corporation nor the names of the
  contributors may be used to endorse or promote products derived from
  this software without specific prior written permission.
\end{enumerate}

\noindent
THIS SOFTWARE IS PROVIDED BY SANDIA CORPORATION ``AS IS'' AND ANY
EXPRESS OR IMPLIED WARRANTIES, INCLUDING, BUT NOT LIMITED TO, THE
IMPLIED WARRANTIES OF MERCHANTABILITY AND FITNESS FOR A PARTICULAR
PURPOSE ARE DISCLAIMED\@. IN NO EVENT SHALL SANDIA CORPORATION OR THE
CONTRIBUTORS BE LIABLE FOR ANY DIRECT, INDIRECT, INCIDENTAL, SPECIAL,
EXEMPLARY, OR CONSEQUENTIAL DAMAGES (INCLUDING, BUT NOT LIMITED TO,
PROCUREMENT OF SUBSTITUTE GOODS OR SERVICES\@; LOSS OF USE, DATA, OR
PROFITS\@; OR BUSINESS INTERRUPTION) HOWEVER CAUSED AND ON ANY THEORY OF
LIABILITY, \\WHETHER IN CONTRACT, STRICT LIABILITY, OR TORT (INCLUDING
NEGLIGENCE OR OTHERWISE) ARISING IN ANY WAY OUT OF THE USE OF THIS
SOFTWARE, EVEN IF ADVISED OF THE POSSIBILITY OF SUCH DAMAGE\@.

    %\chapter{Historical Perspective}
	%    This is an example of an appendix.

    If we follow~\cite{Sand98-0730} strictly, we would have to have
    a separate bibliography section for each appendix.  The style
    file doesn't provide that, but it can be done using the {\tt
    bibunits} and {\tt chapterbib} packages.

    If there are many subsections in an appendix, it should also
    have its own table of contents. Again, the SAND report class
    file does not provide that functionality.

    \ifthenelse{\boolean{reportSAND}}   {
	\section{The Past a Long Time Ago}
    }{
	\subsection{The Past a Long Time Ago}
    }
	This is where we talk about things so old nobody can verify
	them. We are safe.

    \ifthenelse{\boolean{reportSAND}}   {
	\section{The Past More Recently}
    }{
	\subsection{The Past More Recently}
    }
	Now we have to be a little bit more careful, since records
	exist from that time, and some people still alive actually
	lived back then.



    %\chapter{Some Other Appendix}
	%    Just to show what a second Appendix would look like. It contains
    a table. Each appendix is supposed to be self-contained, so
    tables and figures are not supposed to show up in the main
    table of contents. There can be a separate table of contents
    for each appendix.

    \begin{table}[ht]
	\centering
	\caption{A small table}
	\bigskip

	\begin{tabular}{|c|c|}
	    \hline
		A & B  \\ \hline
		C & D  \\ \hline
	\end{tabular}
	\label{tab3}
    \end{table}

    \begin{figure}[ht]
	\centering
	\begin{picture}(50,50)(0,0)
	    \put(25,25){\circle{1}}
	    \put(25,25){\circle{5}}
	    \put(25,25){\circle{10}}
	    \put(25,25){\circle{15}}
	    \put(25,25){\circle{20}}
	    \put(25,25){\circle{25}}
	    \put(25,25){\circle{30}}
	    \put(25,25){\circle{35}}
	    \put(25,25){\circle{40}}
	    \put(25,25){\circle{45}}
	    \put(25,25){\circle{50}}
	\end{picture}
	\caption{Dizzy yet?}
	\label{fig4}
    \end{figure}


    % \printindex

    %
% This is an example of how to create the distribution page. Some
% distributions are required by Sandia; e.g. the housekeeping copies.
% Depending on the type of report; e.g. CRADA, Patent Caution, etc.
% additional distribution lines may have to be added. See the
% "Guide for Preparing SAND Reports"
%
% SANDdistribution takes CA or NM as an optional argument. If given,
% the approrpiate housekeeping copies are inserted autmatically.
% Inside the SANDdistribution environment, several commands can be used
% insert the distributions for CRADA, LDRD, etc. See example below.
%
% You can leave the CA or NM option off and not use any of the SANDdist*
% commands. This will allow you to create a distribution list manually.
%
\begin{SANDdistribution}[NM]
    % Housekeeping copies necessary for every unclassified report:
    % \SANDdistCRADA	% If this report is about CRADA work
    % \SANDdistPatent	% If this report has a Patent Caution or Patent Interest
    % \SANDdistLDRD	% If this report is about LDRD work

    % Some external Addresses

    \SANDdistExternal{1}{Roscoe A. Bartlett\\ Oak Ridge National Laboratory \\ P.O. Box 2008 \\ Oak Ridge, Tennessee 37831 \\ United States }
    % \SANDdistExternal{3}{Some Address\\ and street\\City, State}
    % \SANDdistExternal{12}{Another Address\\ On a street\\City, State\\U.S.A.}
    %\bigskip

    % The following MUST BE between the external and internal distributions!
    % \SANDdistClassified % If this report is classified

    % Internal Addresses
    \SANDdistInternal{1}{9018}{Central Technical Files}{8945-1}
    \SANDdistInternal{2}{0899}{Technical Library}{9610}
    \SANDdistInternal{2}{0612}{Review \& Approval Desk}{4916}
    \SANDdistInternal{1}{0897}{Todd S. Coffey}{1543}
    \SANDdistInternal{1}{1320}{Curtis C. Ober}{1442}
    \SANDdistInternal{1}{1318}{Roger P. Pawlowski}{1444}
    
    % Example of a mail channel use (instead of a mail stop)
    % \SANDdistInternalM{1}{M9999}{Someone}{01234}

\end{SANDdistribution}


\end{document}
