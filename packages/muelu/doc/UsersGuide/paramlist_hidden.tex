
\cbb{verbosity}{string}{"high"}{Control of the amount of printed information. Possible values: "none", "low", "medium", "high", "extreme".}

\cbb{number of equations}{int}{1}{Number of PDE equations at each gride node. Only constant block size is considered.}

\cbb{max levels}{int}{10}{Maximum number of levels.}

\cbb{cycle type}{string}{"V"}{Multigrid cycle type. Possible values: "V", "W".}

\cbb{problem: symmetric}{bool}{true}{Symmetry of a problem.}

\cba{print}{\parameterlist}{Saving a subset of the hierarchy data in a file. Currently, the list can contain any of three parameter names ("A", "P", "R") of type "string" and value "\{levels separated by commas\}". A matrix is saved in two files: a) data is saved in the MatrixMarket format in a file called "A\_level.mm", or similar; b) row map is saved in the MatrixMarket format in a file called "rowmap\_A\_level.mm", or similar.}

\cbb{multigrid algorithm}{string}{"sa"}{Multigrid method. Possible values: "unsmoothed", "sa", "emin", "pg".}

\cbb{transpose: use implicit}{bool}{false}{Use implicit transpose for the restriction operator.}

\cbb{aggregation: type}{string}{"uncoupled"}{Aggregation scheme. Possible values: "uncoupled", "coupled".}

\cbb{aggregation: ordering}{string}{"natural"}{Ordering strategy. Possible values: "natrual", "graph", "random".}

\cbb{aggregation: drop scheme}{string}{"classical"}{Aggregation connectivity dropping scheme. Possible values: "classical", "laplacian".}

\cbb{aggregation: drop tol}{double}{0.0}{Aggregation dropping threshold.}

\cbb{aggregation: min agg size}{int}{2}{Minimum size of an aggregate.}

\cbb{aggregation: max agg size}{int}{2147483647}{Maximum size of an aggregate.}

\cbb{aggregation: max selected neighbors}{int}{0}{Number of maximum neighbour nodes that are already aggregated already. If a new aggregate has some neighbours that are already aggregated, this node probably can be added to one of these aggregates. We don't need a new one.}

\cbb{aggregation: Dirichlet threshold}{double}{0.0}{Threshold for determining whether entries are zero during Dirichlet row detection.}

\cbb{aggregation: enable phase 1}{bool}{true}{Turn on/off phase 1 of aggregation}

\cbb{aggregation: enable phase 2a}{bool}{true}{Turn on/off phase 2a of aggregation}

\cbb{aggregation: enable phase 2b}{bool}{true}{Turn on/off phase 2b of aggregation}

\cbb{aggregation: enable phase 3}{bool}{true}{Turn on/off phase 3 of aggregation}

\cbb{aggregation: preserve Dirichlet points}{bool}{false}{Turn on/off Dirichlet points preservation}

\cbb{aggregation: visualize}{bool}{false}{Visualize aggregates.}

\cbb{aggregation: mode}{string}{"old"}{old = Tobias, new = ML}

\cbb{smoother: pre or post}{string}{"both"}{Smoother combination. Possible values: "pre", "post", "both", "none".}

\cbb{smoother: type}{string}{"gs"}{Smoother type. Possible values: see Table~\ref{t:smoothers}.}

\cbb{smoother: pre type}{string}{"gs"}{Pre-smoother type. Possible values: see Table~\ref{t:smoothers}.}

\cbb{smoother: post type}{string}{"gs"}{Post-smoother type. Possible values: see Table~\ref{t:smoothers}.}

\cba{smoother: params}{\parameterlist}{Smoother parameters. For standard smoothers, \muelu passes them directly to \stratimikos.}

\cba{smoother: pre params}{\parameterlist}{Pre-smoother parameters. For standard smoothers, \muelu passes them directly to \stratimikos.}

\cba{smoother: post params}{\parameterlist}{Post-smoother parameters. For standard smoothers, \muelu passes them directly to \stratimikos.}

\cbb{smoother: overlap}{int}{0}{Smoother subdomain overlap.}

\cbb{smoother: pre overlap}{int}{0}{Pre-smoother subdomain overlap.}

\cbb{smoother: post overlap}{int}{0}{Post-smoother subdomain overlap.}

\cbb{coarse: max size}{int}{2000}{Maximum dimension of the coarse grid. \muelu will stop coarsening once it is achieved.}

\cbb{coarse: type}{string}{"SuperLU"}{Coarse solver. Possible values: see Table~\ref{t:coarse_solvers}.}

\cba{coarse: params}{\parameterlist}{Coarse solver parameters. \muelu passes them directly to coarse solver.}

\cbb{coarse: overlap}{int}{0}{Coarse solver subdomain overlap.}

\cbb{repartition: enable}{bool}{false}{Repartitioning on/off switch.}

\cbb{repartition: partitioner}{string}{"zoltan2"}{Partitioning package to use. Possible values: "zoltan", "zoltan2".}

\cba{repartition: params}{\parameterlist}{Partitioner parameters. \muelu passes them directly to partitioner.}

\cbb{repartition: start level}{int}{2}{Minimum level to run partitioner. \muelu does not repartition for finer levels.}

\cbb{repartition: min rows per proc}{int}{800}{Desired minimum number of rows per processor. If actual number if smaller, then repartitioning occurs.}

\cbb{repartition: max imbalance}{double}{1.2}{Desired maximum nonzero imbalance ratio.}

\cbb{repartition: remap parts}{bool}{true}{Postprocessing for partitioning to reduce data migration.}

\cbb{repartition: remap num values}{int}{4}{Number of maximum components from each processor used to construct partial bipartite graph.}

\cbb{repartition: keep proc 0}{bool}{true}{Postprocessing for partitioning to keep processor 0 from dropping out. The goal is to keep processor 0 of the original fine level communication even when we use level subcommunicators.}

\cbb{repartition: print partition distribution}{bool}{false}{Print partition distribution with '+' and '.'}

\cbb{repartition: rebalance P and R}{bool}{true}{Do rebalancing of R and P during the setup. This speeds up the solve, but slows down the setup phases.}

\cbb{sa: damping factor}{double}{1.33333333}{Damping factor for smoothed aggregation.}

\cbb{sa: use filtered matrix}{bool}{true}{Matrix to use for smoothing the tentative prolongator. The two options are: to use the original matrix, and to use the filtered matrix with filtering based on filtered graph used for aggregation.}

\cbb{filtered matrix: use lumping}{bool}{true}{During construction of a filtered matrix, we have an option to add dropped entries to the diagonal. This is useful for preserving constant nullspace for the Laplacian type matrix.}

\cbb{filtered matrix: reuse eigenvalue}{bool}{true}{During construction of a filtered matrix, we have an option to get the eigenvalue estimate from the original matrix. This allows us to skip heavy computation.}

\cbb{filtered matrix: reuse graph}{bool}{true}{Set values to zero instead of constructing a new graph.}

\cbb{emin: iterative method}{string}{"cg"}{Iterative method to use for energy minimization of intial prolongator in energy-minimization. Possible values: "cg", "sd".}

\cbb{emin: num iterations}{int}{2}{Number of iterations to minimize initial prolongator energy in energy-minimization.}

\cbb{emin: num reuse iterations}{int}{1}{Number of iterations to minimize the reused prolongator energy in energy-minimization.}

\cbb{emin: pattern}{string}{"AkPtent"}{Sparsity pattern to use for energy minization. Possible values: "AkPtent".}

\cbb{emin: pattern order}{int}{1}{Matrix order for the "AkPtent" pattern.}
