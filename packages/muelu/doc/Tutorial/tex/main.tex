 
\documentclass{article}
\usepackage[latin1]{inputenc}
\usepackage[T1]{fontenc}
\usepackage{listings}
\title{A MueLu hands-on tutorial}
\author{Tobias Wiesner  \\
	Technische Universit\"at M\"unchen  \\
	Boltzmannstr. 15 \\
	85747 Garching \\
	\and 
  Jonathan Hu
  \and
  Andrey Prokopenko
	}

\date{\today}
% Hinweis: \title{um was auch immer es geht}, \author{wer es auch immer 
% geschrieben hat} und  \date{wann auch immer das war} k\"onnen vor 
% oder nach dem  Kommando \begin{document} stehen 
% Aber der \maketitle Befehl mu\ss{} nach dem \begin{document} Kommando stehen! 
\begin{document}

\maketitle


\begin{abstract}
This is a MueLu tutorial.
\end{abstract}

\section{Introduction}
Ziel der Arbeit ist es m\"oglichsten vielen oder wenn m\"oglichen es allen 
zu erm\"oglichen, Dokumente mit \LaTeX{} zu erstellen!


\section{Minimal example}

Generate an input file containing the following solver parameters: 
\small
\lstinputlisting{../src/xml/s1_easy.xml}
\normalsize

The parameters are self-explaining and for an overview of all options the reader may refer to the MueLu user guide where an extensive list of all options and parameter choices is given.
Beside the multigrid parameter the user can choose different types of multigrid algorithms. For a symmetric problem a smoothed aggregation multigrid method is the proper choice (\texttt{multigrid algorithm = sa}). Furthermore, one can define the level smoother. In above case a damped Jacobi smoother is used. Last but not least, some basic aggregation parameters can be chosen by the user. 

Assuming that above xml parameters are stored in the xml file \texttt{s1\_easy.xml} one can run the test program using
\begin{center}
\texttt{MueLu\_tutorial\_laplace2d.exe ----xml=s1\_easy.xml}.
\end{center}
This should produce the following output on screen.

\small
\lstinputlisting{s1_easy.txt_3.fragment_1.fragment}
\normalsize

It helps to learn about the setup process and gives basic information about aggregation and level smoothers. In the end an overview of the multigrid setup is printed

\small
\lstinputlisting{s1_easy.txt_3.fragment_3.fragment}
\normalsize

The multigrid overview just gives some basic information about the different multigrid levels. One can see that a three level multigrid method is used with a direct solver on the coarsest level and Jacobi level smoothers on the fine and intermedium level. Furthermore some basic information is printed such as the operator complexity.

In the end the CG convergence is printed when applying the generated multigrid method as preconditioner within a CG solver from the Aztec package in Trilinos.
The numbers give the relative residual after the corresponding number of iterations as well as the solution time in seconds.
\small
\lstinputlisting{s1_easy.txt_5.fragment}
\normalsize


\section{Dokumentenklassen} \label{documentclasses}

\begin{itemize}
\item article
\item book 
\item report 
\item letter 
\end{itemize}


\begin{enumerate}
\item article
\item book 
\item report 
\item letter 
\end{enumerate}

\begin{description}
\item[article\label{article}]{Article ist \ldots}
\item[book\label{book}]{Die book Klasse ist \ldots}
\item[report\label{report}]{Die Klasse report erm\"oglicht es  \ldots}
\item[letter\label{letter}]{Wenn man einen Breif schreiben sollte man eine 
	andere Klasse nutzen, da diese f\"ur ein anderes als das deutsche 
	Briefformat ausgelegt ist.}
\end{description}


\section{Fazit}\label{conclusions}
Nach langer Suche hat sich herausgestellt, dass es kein l\"angeres 
\LaTeX{} Beispiel, als das von \cite{doe} geschriebene gibt. 

\begin{thebibliography}{9}
\bibitem[Doe]{doe} \emph{Erstes und letztes \LaTeX{} Beispiel.},
John Doe 50 v.Chr.  
\end{thebibliography}

\end{document}