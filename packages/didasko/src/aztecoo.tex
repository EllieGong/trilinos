% @HEADER
% ***********************************************************************
% 
%            Trilinos: An Object-Oriented Solver Framework
%                 Copyright (2001) Sandia Corporation
% 
% Under terms of Contract DE-AC04-94AL85000, there is a non-exclusive
% license for use of this work by or on behalf of the U.S. Government.
% 
% This library is free software; you can redistribute it and/or modify
% it under the terms of the GNU Lesser General Public License as
% published by the Free Software Foundation; either version 2.1 of the
% License, or (at your option) any later version.
%  
% This library is distributed in the hope that it will be useful, but
% WITHOUT ANY WARRANTY; without even the implied warranty of
% MERCHANTABILITY or FITNESS FOR A PARTICULAR PURPOSE.  See the GNU
% Lesser General Public License for more details.
%  
% You should have received a copy of the GNU Lesser General Public
% License along with this library; if not, write to the Free Software
% Foundation, Inc., 59 Temple Place, Suite 330, Boston, MA 02111-1307
% USA
% Questions? Contact Michael A. Heroux (maherou@sandia.gov) 
% 
% ***********************************************************************
% @HEADER

\chapter{Iterative Solution of Linear Systems with AztecOO}
\label{chap:aztecoo}

\ChapterAuthors{Marzio Sala, Michael Heroux, David Day}

\begin{introchapter}
The AztecOO \cite{AztecOO-Users-Guide} package extends the Aztec library~\cite{Aztec2.1}.
Aztec is the legacy iterative solver at the Sandia National
Laboratories.  It has been extracted from the MPSalsa reacting flow
code~\cite{MPSalsa-Theory,MPSalsa-User-Guide}, and it is currently
installed in dozens of Sandia's applications. AztecOO extends this
package, using C++ classes to enable more sophisticated uses.

AztecOO is intended for the iterative solution of linear systems of the form
\begin{equation}
  \label{eq:linear_sys}
  A \; X = B ,
\end{equation}
when $A \in \mathbb{R}^{n \times n}$ is the linear system matrix, $X$
the solution, and $B$ the right-hand side. Although AztecOO can live
independently of Epetra, in this tutorial it is supposed that $A$ is an
Epetra\_RowMatrix, and both $X$ and $B$ are Epetra\_Vector or
Epetra\_MultiVector objects.

The Chapter reviews the following topics:
\begin{itemize}
\item Outline the basic issued of the iterative solution of linear
  systems (in \S~\ref{aztecoo:theoretical});
\item Present the basic usage of AztecOO (in
  \S~\ref{sec:basic_aztecoo});
\item Define one-level domain decomposition preconditioners (in
  \S~\ref{sec:aztecoo_dd});
\item Use of AztecOO problems as preconditioners to other AztecOO
  problems (in \S~\ref{sec:prec_aztecoo}).
\end{itemize}
\end{introchapter}

%%%
%%%
%%%

\section{Theoretical Background}
\label{aztecoo:theoretical}

Our aim is to briefly present the vocabulary and notation 
required to define the main features available to users.
The Section is neither exhaustive, nor complete. Readers are
referred to the existing literature for a comprehensive presentation (see,
for instance, \cite{temp94,axelsson94iterative,saad96iterative}).

\medskip

One can distinguish between two different aspects of the iterative
solution of a linear system. The first one in the particular
acceleration technique for a sequence of iterations vectors, that is a
technique used to construct a new approximation for the solution, with
information from previous approximations.  This leads to specific
iteration methods, generally of Krylov type, such as conjugate gradient
or GMRES. The second aspect is the transformation of the given system to
one that is solved more efficiently by a particular iteration
method. This is called {\em preconditioning}.  A good preconditioner
improves the convergence of the iterative method, sufficiently to
overcome the extra cost of its construction and application. Indeed,
without a preconditioner the iterative method may even fail to converge
in practice.

The convergence of iterative methods depends on the spectral properties
of the linear system matrix. The basic idea is to replace the original
system~(\ref{eq:linear_sys}) by
the left preconditioned system,
\[
P^{-1} A ~ X = P^{-1} B
\]
or the right preconditioned system
\[
A P^{-1} ~ P X = B
\]
using the linear transformation $P^{-1}$,
called preconditioner, in order to improve the spectral properties of
the linear system matrix. In general terms, a preconditioner is any
kind of transformation applied to the original system which makes it
easier to solve.

From a modern perspective, the general problem of finding an efficient
preconditioner is to identify a linear operator $P$ with the following
properties:
\begin{enumerate}
\item $P^{-1} A$ is {\em near} to the identity matrix 
\item The cost of applying the preconditioner is relatively low.
\item Preconditioning is scalable.
\end{enumerate}

The role of the $P$ in the iterative method is simple. At each
iteration, it is necessary to solve an auxiliary linear system
$P z_m = r_m$ for $z_m$ given $r_m$.
It is unnecessary to explicitly invert $P$.

It should be stressed that computing the inverse of $P$ is not
mandatory; actually, the role of $P$ is to ``preconditioning'' the
residual at step $m$, $r_m$, through the solution of the additional
system $P z_m = r_m$. This system $P z_m = {r}_m$ should be much easier
to solve than the original system.

\smallskip

The choice of $P$ varies from ``black-box'' algebraic techniques
for general matrices to ``problem dependent''
preconditioners that exploit special features of a particular class
of problems. Although problem dependent preconditioners may be very
powerful, there is still a practical need for efficient
preconditioning techniques for large classes of problems. 
Even the ``black-box'' preconditioners require parameters, and suitable
parameter settings do vary.  Between these two extrema, 
there is a class of preconditioners which are
``general-purpose'' for a particular -- although large -- class of
problems.  These preconditioners are sometimes called ``gray-box''
preconditioners; users supply a little information about
the matrix and the problem to be solved.

The AztecOO preconditioners include Jacobi, Gauss-Seidel, polynomial and
domain decomposition based methods \cite{smbg:96}. 
Furthermore preconditioners can be
given to an AztecOO Krylov accelerator, by using the Trilinos packages
IFPACK and ML, covered in \S~\ref{chap:ifpack} and \ref{chap:ml},
respectively.

%%%
%%%
%%%

\section{Basic Usage}
\label{sec:basic_aztecoo}

First we delineate the steps required to apply 
AztecOO to a linear system.
To solve a linear system with AztecOO, one must create an
\verb!Epetra_LinearProblem!  object (see
\S~\ref{sec:linear_problem}) with the command
\begin{verbatim}
Epetra_LinearProblem Problem(&A,&x,&b);
\end{verbatim}
\verb!A! is an Epetra matrix, and both \verb!x! and \verb!b! are Epetra
vectors\footnote{At the current stage of development, AztecOO does not
  take advantage of the Epetra\_MultiVectors. It accepts Multi\_Vectors,
  but it will solve the linear system corresponding to the first
  multivector only.
  The Belos package implements block Krylov subspace methods that are
  significantly more efficient for linear systems with multiple simultaneous right-hand sides.
}.  Second create an AztecOO object,
\begin{verbatim}
AztecOO Solver(Problem);
\end{verbatim}
Next specify how to solve the linear system. All AztecOO options are set
using two vectors, one of integers and the other of doubles, as detailed
in the Aztec's User Guide \cite{Aztec2.1}.

To choose among the different AztecOO parameters, the user can create
two vectors, usually called \verb!options! and \verb!params!, set them
to the default values, and then override with the desired parameters.
Here's how to set default values: 
\begin{verbatim}
int    options[AZ_OPTIONS_SIZE];
double params[AZ_PARAMS_SIZE];
AZ_defaults(options, params);
\end{verbatim}
\begin{verbatim}
Solver.SetAllAztecOptions( options );
Solver.SetAllAztecParams( params );
\end{verbatim}
The latter two functions copy the values of \verb!options! and
\verb!params! 
to variables internal to the AztecOO object.

Alternatively, it is possible to set specific parameters without
creating \verb!options! and \verb!params!, using the AztecOO methods
\verb!SetAztecOption()! and \verb!SetAztecParams()!. 
For instance,
\begin{verbatim}
Solver.SetAztecOption( AZ_precond, AZ_Jacobi );
\end{verbatim}
to specify a point Jacobi preconditioner.  (Please see to the Aztec
documentation \cite{AztecOO-Users-Guide} for more details about the various Aztec settings.)

Finally solve the linear system, say with a maximum of $1550$ iterations and
the residual error norm threshold $10^{-9}$:
\begin{verbatim}
Solver.Iterate(1550,1E-9);
\end{verbatim}
The complete code is in \TriExe{aztec/ex1.cpp}.

Note that the matrix must be in local form (that is, the command
\verb!A.FillComplete()! {\em must} have been invoked before solving the
linear system).  The same AztecOO linear system solution procedure 
applies in serial and in parallel.  However for some preconditioners,
the convergence rate (and the number of iterations) 
depends on the number of processor.

When \verb!Iterate()! returns, one can query for the number of
iterations performed by the linear solver using
\verb!Solver.NumIters()!, while \verb!Solver.TrueResidual()! gives the
(unscaled) norm the residual.

%%%
%%%
%%%
\section{Overlapping Domain Decomposition Preconditioners}
\label{sec:aztecoo_dd}
A one level overlapping domain decomposition preconditioners $P$ takes the form
\begin{equation}
  \label{eq:prec_dd}
  P^{-1} = \sum_{i=1}^M R_i^T \tilde{A}_i^{-1} R_i,
\end{equation}
The number of subdomains is $M$, 
$R_i$ is a rectangular Boolean matrix that restricts
a global vector to the subspace defined by the interior of the $i$th
subdomain, and $\tilde{A}_i$ approximates
\begin{equation}
  \label{eq:aztecoo_tilde_a}
  A_i = R_i A R_i^T .
\end{equation}
($\tilde{A}_i$ may equal $A_i$). Typically, $\tilde{A}_i$ differs
from $A_i$ when incomplete factorizations are used in (\ref{eq:prec_dd})
to apply $\tilde{A}_i^{-1}$, or when a matrix different from $A$ is used
in (\ref{eq:aztecoo_tilde_a}).

(\ref{eq:aztecoo_tilde_a}),
The specification starts with
\begin{verbatim}
Solver.SetAztecOption( AZ_precond, AZ_dom_decomp );
\end{verbatim}
Next if an incomplete factorization of $A_i$ will be used, then specify its parameters:
\begin{verbatim}
Solver.SetAztecOption( AZ_subdomain_solve, AZ_ilu );
Solver.SetAztecOption( AZ_graph_fill, 1 );
\end{verbatim}
On the other hand, exact subdomain solves 
\footnote{AztecOO must be
  configured with the option {\tt --enable-aztecoo-azlu}, and the
  package Y12M is required.}
are specified like this:
\begin{verbatim}
Solver.SetAztecOption( AZ_subdomain_solve, AZ_lu );
\end{verbatim}

The default amount of overlap is $0$; this is equivalent to
``block" Jacobi preconditioning with block size equal to the size of the subdomain. 
If amount of overlap is one ($k$),
\begin{verbatim}
Solver.SetAztecOption( AZ_overlap, 1 );
\end{verbatim}
then the ``block" is augmented by all distance one ($k$),
neighbors in the sparse matrix graph.

\begin{remark} 
  Two-level domain decomposition schemes \cite{smbg:96}
are available through AztecOO in conjunction with ML.
Please see \S~\ref{sec:ml_DD}.
\end{remark}

\begin{remark} The IFPACK Trilinos package (see \S~\ref{chap:ifpack})
computes different incomplete factorizations. 
\end{remark}

Consider for example a Laplace equation descretized with the $5$-point stencil 
over a regular Cartesian grid on a square.
If $h$ is the mesh size, then the condition number of the unpreconditioned system
is $\mathcal{O}(h)^{-2}$.
In theory \cite{smbg:96} if $H$ is the size of each (square) subdomain,
a one level Schwarz preconditioner with minimal overlap
yields a preconditioned linear system with condition number
$\mathcal{O}(hH)^{-1}$.
The AztecOO preconditioner delivers the 
Table~\ref{tab:aztecoo:dd} gives the estimated condition
numbers for the corresponding AztecOO preconditioner.

The corresponding source code is contained in
\newline
\TriExe{aztecoo/ex3.cpp} 
\newline
The code uses the Trilinos\_Util\_CrsMatrixGallery class to
create the matrix. The command to estimate the condition number with
$h=1/60$ and $H=1/3$ is 
\begin{verbatim}
mpirun -np 9 ex3.exe -problem_type=laplace_2d \
       -problem_size=900 -map_type=box
\end{verbatim}

\begin{table}[htbp]
  \centering
  \begin{tabular}{| c | c c c |}
    \hline
    & $h=1/30$ & $h=1/60$ & $h=1/120$ \\
    \hline
    $H=1/3$ & 40.01 & 83.80 & 183.41 \\
    $H=1/4$ & 51.46 & 106.01 & 223.22 \\
    $H=1/6$ & 79.19 & 150.40 & 311.26 \\
    $H=1/8$ & -     & 191.06 & 403.29 \\
    \hline
  \end{tabular}
  \caption{Condition number for one-level domain decomposition preconditioner on a square.}
  \label{tab:aztecoo:dd}
\end{table}

%%%
%%%
%%%

\section{AztecOO Problems as Preconditioners}
\label{sec:prec_aztecoo}

Preconditioners fall into two categories.
In the first category 
are preconditioners that are a function of the entires of the coefficient matrix.
Examples include Jacobi, Gauss-Seidel, incomplete factorizations, and domain
decomposition preconditioners.
The second category contains preconditioners 
such as polynomial preconditioners
that are defined by the action of
the matrix on some set of vectors.
For example only category two preconditioners apply in matrix free mode.
The topic of this section is another type of category two preconditioner, 
preconditioning one Krylov subspace method by another Krylov subspace method, 

AztecOO accepts Epetra\_Operator objects as preconditioners. That is
any class, derived from an Epetra\_Operator implementing the
method \verb!ApplyInverse()! may also be used as a preconditioner,
using \verb!SetPreOperator()!. AztecOO itself can be used to define a
preconditioner for AztecOO; the class AztecOO\_Operator (which takes an
AztecOO object in the construction phase) is derived from
Epetra\_Operator.

File \TriExe{aztecoo/ex2.cpp} shows how to use an AztecOO solver in the
preconditioning phase.  The main steps are sketched here.

First, we have to specify the linear problem to be solved (set the
linear operator, the solution and the right-hand side), and create an
AztecOO object:
\begin{verbatim}
Epetra_LinearProblem A_Problem(&A, &x, &b);
AztecOO A_Solver(A_Problem);
\end{verbatim}
Now, we have to define the preconditioner. For the sake of clarity,
we use the same Epetra\_Matrix \verb!A! in the
preconditioning phase. However, the two matrices may differ.
\begin{verbatim}
Epetra_CrsMatrix P(A);
\end{verbatim}
(This operation is in general expensive as it involves the copy
constructor. It is used here for the sake of clarity.)  Then, we
create the linear problem that will be used as a preconditioner.  This
takes a few steps to explain.  Note that all the \verb!P! prefix identifies
preconditioner' objects.
\begin{enumerate}
\item Create the linear system solve at each preconditioning step, and and we
  assign the linear operator (in this case, the matrix A itself)
\begin{verbatim}
Epetra_LinearProblem P_Problem;
P_Problem.SetOperator(&P);
\end{verbatim}
\item As we wish to use AztecOO to solve the preconditioner step recursively, 
we have to define an AztecOO object:
\begin{verbatim}
AztecOO P_Solver(P_Problem);  
\end{verbatim}
\item Specify a particular preconditioner:
\begin{verbatim}
P_Solver.SetAztecOption(AZ_precond, AZ_Jacobi);
P_Solver.SetAztecOption(AZ_output, AZ_none);
P_Solver.SetAztecOption(AZ_solver, AZ_cg);
\end{verbatim}
\item Create an AztecOO\_Operator, to set
  the Aztec's preconditioner with and set the users defined preconditioners:
\begin{verbatim}
AztecOO_Operator
P_Operator(&P_Solver, 10);  
A_Solver.SetPrecOperator(&P_Operator);
\end{verbatim}
(Here 10 is the maximum number of iterations of the AztecOO solver in
the preconditioning phase.)
\item Solve the linear system:
\begin{verbatim}
int Niters=100;
A_Solver.SetAztecOption(AZ_kspace, Niters);
A_Solver.SetAztecOption(AZ_solver, AZ_gmres);
A_Solver.Iterate(Niters, 1.0E-12);  
\end{verbatim}
\end{enumerate}


\begin{table}
\begin{center}
\begin{tabular}{ | p{5cm} | p{10cm} | } \hline
\verb!AZ_cg!         & Conjugate Gradient method (intended for symmetric positive definite matrices). \\
\verb!AZ_cg_condnum! & \verb!AZ_cg! that also estimates the condition number of the preconditioned linear system
                       (returned in \verb!params[AZ_condnum]!)\\ 
\verb!AZ_gmres!      & restarted Generalized Minimal Residual method \\
\verb!AZ_gmres_condnum! & \verb!AZ_gmres! that also estimates the condition number of the preconditioned linear operator
                          (returned in \verb!params[AZ_condnum]!)\\ 
\verb!AZ_cgs! &  Conjugate Gradient Squared method\\
\verb!AZ_tfqmr! & Transpose-Free Quasi Minimal Residual method \\
\verb!AZ_bicgstab! &  Bi-Conjugate Gradient with Stabilization method\\
\verb!AZ_lu! & Serial sparse direct linear solver available if
               AztecOO is configured with {\tt --enable-aztecoo-azlu}\\
\hline \end{tabular}
\caption{ {\tt options[AZ\_solver]} Choices}
\label{tab:aztec:solver}
\end{center}
\end{table}

%%%
%%%
%%%

\section{Concluding Remarks on AztecOO}

The following methods are often used:
\begin{itemize}
\item \verb!NumIters()! returns the total number of iterations performed
  on {\sl this} problem;
\item \verb!TrueResidal()! returns the true unscaled residual;
\item \verb!ScaledResidual()! returns the unscaled residual;
\item \verb!SetAztecDefaults()! can be used to restore default values in
  the options and params vectors.
\end{itemize}
\verb!NumIters()! is useful in performance optimization.
A less costly preconditioner is viable if \verb!NumIters()! is ``small", 
and a more costly preconditioner may be worthwhile if \verb!NumIters()! is ``large".
Many iterative methods with right preconditioning 
apply the residual error norm threshold to the preconditioned residual,
$P^{-1}r$, instead of $r=b-Ax$.
\verb!TrueResidal()! will always return the unscaled residual norm.
