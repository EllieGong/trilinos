%@HEADER
% ************************************************************************
% 
%          Trilinos: An Object-Oriented Solver Framework
%              Copyright (2001) Sandia Corporation
% 
% Under terms of Contract DE-AC04-94AL85000, there is a non-exclusive
% license for use of this work by or on behalf of the U.S. Government.
% 
% This program is free software; you can redistribute it and/or modify
% it under the terms of the GNU General Public License as published by
% the Free Software Foundation; either version 2, or (at your option)
% any later version.
%   
% This program is distributed in the hope that it will be useful, but
% WITHOUT ANY WARRANTY; without even the implied warranty of
% MERCHANTABILITY or FITNESS FOR A PARTICULAR PURPOSE.  See the GNU
% General Public License for more details.
%   
% You should have received a copy of the GNU General Public License
% along with this program; if not, write to the Free Software
% Foundation, Inc., 675 Mass Ave, Cambridge, MA 02139, USA.
% 
% Questions? Contact Michael A. Heroux (maherou@sandia.gov)
% 
% ************************************************************************
%@HEADER

\chapter{Iterative Solution of Linear Systems with Belos}
\label{chap:belos}

\ChapterAuthors{Chris Baker, Michael Heroux, Heidi Thornquist}

\begin{introchapter}
Several goals motivated the development of the Belos linear solver package: interoperability,
extensibility, and the capability to use and develop next-generation linear solvers.  The intention 
of \emph{interoperability} is to ease the use of Belos in a wide range of application environments. 
To this end, the algorithms written in Belos utilize an abstract interface for operators and vectors, 
allowing the user to leverage existing linear algebra libraries. The concept of \emph{extensibility} 
drives development of Belos to allow users to make efficient use of available solvers while
simultaneously enabling them to easily develop their own code in the Belos framework.
This is encouraged by promoting code modularization and multiple levels of access to
solvers and their data.

Another motivation for Belos was the \emph{capability to use and develop
next-generation linear solvers}.  These \emph{next-generation linear solvers} should
enable the efficient solution of
\begin{itemize}
  \item Single right-hand side systems: $Ax=b$
  \item Simultaneously solved, multiple right-hand side systems: $AX=B$
  \item Sequentially solved, multiple right-hand side systems: $AX_i = B_i$, for $i=1,\ldots,t$
  \item Sequences of multiple right-hand side systems: $A_iX_i=B_i$, for $i=1,\ldots,t$
\end{itemize}
where $A \in \mathbb{R}^{n \times n}$, $x,b \in \mathbb{R}^n$, and $X,B \in \mathbb{R}^{n \times k}$.

In this Chapter, we outline the Belos linear solver framework and motivate the design.
In particular, we present
\begin{itemize}
  \item the Belos operator/vector interface (Section~\ref{sec:belos:opvec});
  \item the Belos linear solver framework (Section~\ref{sec:belos:solver_framework});
  \item a description of Belos classes (Section~\ref{sec:belos:classes});
  \item the interface to the Epetra linear algebra package (Section~\ref{sec:belos:epetra});
  \item an example using Belos for the solution of a linear problem (Section~\ref{sec:belos:example}).
\end{itemize}
\end{introchapter}

\section{The Belos Operator/Vector Interface}
%%%%%%%%%%%%%%%%%%%%%%%%%%%%%%%%%%%%%%%%%%%%%%%%%%%%%%
\label{sec:belos:opvec}

The Belos linear solver package utilizes abstract interfaces for operators and
multivectors. This allows users to leverage existing linear algebra libraries and to
protect previous software investment. Algorithms in Belos are developed at a high-level,
where the underlying linear algebra objects are opaque. The choice in linear algebra is
made through templating, and access to the functionality of the underlying objects is
provided via the traits classes Belos::MultiVecTraits and Belos::OperatorTraits.
These classes define opaque interfaces, specifying the operations that multivector and
operator classes must support in order to be used in Belos without exposing low-level
details of the underlying data structures.

The benefit of using a templated traits class over inheritance is that the latter requires
the user to derive multivectors and operators from Belos-defined abstract
base classes. The former, however, defines only local requirements: Belos-defined traits
classes implemented as user-developed adapters for the chosen multivector and operator
classes.

Belos::MultiVecTraits provides routines for the creation of multivectors, as well as
their manipulation. In order to use a specific scalar type and multivector type with
Belos, there must exist a template specialization of Belos::MultiVecTraits for this
pair of classes. A full list of methods required by Belos::MultiVecTraits is given in
Table~\ref{tab:belos:mvt}.

\begin{table}
\begin{center}
\begin{tabular}{| p{3.5cm} |  p{2.2cm} || p{8cm} |}
\hline
Method name & Args & Description \\
\hline\hline
Clone           & $X$, $n$    
                & Creates a new empty multivector from $X$ containing $n$ columns.  \\\hline
CloneCopy       & $X$   
                & Creates a new multivector from $X$ with a copy of all the vectors in $X$ (deep copy). \\\hline
CloneCopy       & $X$, $index$   
                & Creates a new multivector from $X$ with only a copy of the $index$ vectors (deep copy). \\\hline
CloneView       & $X$, $index$ 
                & Creates a new multivector from $X$ that shares the $index$ vectors from $X$ (shallow copy).  \\\hline
GetVecLength    & $X$ 
                & Returns the vector length of the multivector $X$.  \\\hline
GetNumberVecs   & $X$
                & Returns the number of vectors in the multivector $X$.  \\\hline
MvTimesMatAddMv & $X$, $Y$, $\alpha$, $\beta$
                & Apply a SerialDenseMatrix $M$ to another multivector $X$, $Y \leftarrow \alpha X M + \beta Y$.  \\\hline
MvAddMv         & $X$, $Y$, $Z$, $\alpha$, $\beta$
                & Perform $Z \leftarrow \alpha X + \beta Y$.  \\\hline
MvTransMv       & $X$, $Y$, $\alpha$, $C$
                & Compute the matrix $C \leftarrow \alpha X^H Y$.  \\\hline
MvDot           & $X$, $Y$, $d$
                & Compute the vector $d$ where the components are the individual dot-products of the $i$-th columns of $X$ and $Y$, i.e. $d[i] = X[i]^H Y[i]$.  \\\hline
MvScale         & $X$, $\alpha$
                & Scale the columns of the multivector $X$ by $\alpha$. \\\hline
MvNorm          & $X$, $norm$
                & Compute the 2-norm of each individual vector of $X$, $norm[i]=\|X[i]\|_2$. \\\hline
SetBlock        & $X$, $Y$, $index$
                & Copy the vectors in $X$ to the subset of vectors in $Y$ indicated by $index$. \\\hline
MvRandom        & $X$
                & Replace the vectors in $X$ with random vectors.  \\\hline
MvInit          & $X$, $\alpha$
                & Replace each element of $X$ with $\alpha$.  \\\hline
MvPrint         & $X$, $os$
                & Print the multivector $X$ to an output stream $os$.  \\\hline
\hline
\end{tabular}
\caption{Methods required by Belos::MultiVecTraits interface.}
\label{tab:belos:mvt}
\end{center}
\end{table}

Just as Belos::MultiVecTraits defined the interface required to use a
multivector class with Belos, Belos::OperatorTraits defines the
interface required to use the combination of a specific operator class with a
specific multivector class. This interface defines a single method:
\begin{verbatim}
OperatorTraits<ScalarType,MV,OP>::Apply(const OP &Op, const MV &X, MV &Y)
\end{verbatim}
This method performs the operation $Y = Op(X)$, where $Op$ is an operator of type
\verb!OP! and $X$ and $Y$ are multivectors of type \verb!MV!. In order to use the
combination of \verb!OP! and \verb!MV!, there must be a specialization of
Belos::OperatorTraits for \verb!ScalarType!, \verb!OP! and \verb!MV!. 

Calling methods of MultiVecTraits and OperatorTraits requires that specializations of
these traits classes have been implemented for given template arguments.  
Belos provides the following specializations of these traits classes:
\begin{itemize}
  \item Epetra\_MultiVector and Epetra\_Operator (with scalar type double)    
  \item Thyra::MultiVectorBase and Thyra::LinearOpBase (with arbitrary scalar type) \\
        This allows Belos to be used with any classes that implement the abstract interfaces provided by the Thyra package.    
  \item Belos::MultiVec and Belos::Operator (with arbitrary scalar type) \\
        This allows Belos to be used with any classes that implement the abstract base
        classes Belos::MultiVec and Belos::Operator.
\end{itemize}

For user-specified classes that do not match one of the above, specializations of
MultiVecTraits and OperatorTraits will need to be created by the user for use by Belos.
Test routines \verb!Belos::TestMultiVecTraits()! and
\verb!Belos::TestOperatorTraits()! are provided by Belos to help in testing
user-developed adapters.


\section{The Belos Linear Solver Framework}
%%%%%%%%%%%%%%%%%%%%%%%%%%%%%%%%%%%%%%%%%%%%%%%%%%%%%%
\label{sec:belos:solver_framework}

The goals of flexibility and efficiency can interfere with the goals of simplicity and
ease of use. For example, efficient memory use and low-level data access required in
scientific codes can lead to complicated interfaces and violations of standard
object-oriented development practices.  In Belos, this problem is addressed by providing a 
multi-tiered access strategy for linear solver algorithms. Belos users have the choice of 
interfacing at one of two levels: either working at a high-level with a linear solver 
manager or working at a low-level directly with an iteration.

Consider as an example the conjugate gradient (CG) iteration. The essence of this iteration 
can be distilled into the following steps:
\begin{enumerate}
  \item apply operator $A$ to current direction vector $p_i$: $Ap_i = A*p_i$
  \item use $Ap_i$ to compute step length $\alpha_i$: $\alpha_i=\frac{<r_i,r_i>}{<p_i,Ap_i>}$
  \item use $\alpha_i$ and $p_i$ to compute approximate solution $x_{i+1}$: $x_{i+1}=x_i+\alpha_i*p_i$
  \item compute new residual $r_{i+1}$ using step length and $Ap_i$: $r_{i+1}=r_i-\alpha_i*Ap_i$
  \item compute improvement in the step $\beta_i$ using $r_i$ and $r_{i+1}$: $\beta_i=\frac{<r_{i+1},r_{i+1}>}{<r_i,r_i>}$
  \item use $\beta_i$ and $r_{i+1}$ to compute the new direction vector $p_{i+1}$: $p_i=r_{i+1}+\beta_i*p_i$
\end{enumerate}

In implementing a CG solver, this iteration repeats until some stopping criterion has 
been satisfied.  Many valid stopping criteria exist, however, they are distinct from 
the essential iteration as described above. A user wanting to perform CG iterations could 
ask the solver to perform these iterations until a user-specified stopping criterion, like
maximum number of iterations or residual norm, was 
satisfied.  This allows the user complete control over the stopping criteria and leaves 
the iteration responsible for a relatively simple bit of state and behavior.

This is the way that Belos has been designed. The iterations (encapsulating an
iteration kernel and the associated state) are derived classes of the abstract base class
Belos::Iteration. The goals of this class are three-fold:
\begin{itemize}
  \item to define an interface used for checking the status of an iteration by a status test;
  \item to contain the iteration kernel associated with a particular linear solver;
  \item to contain the state associated with that iteration.
\end{itemize}

The status tests, assembled to describe a specific stopping criterion and queried by the 
iteration, are represented as subclasses of Belos::StatusTest. The
communication between status test and iteration occurs
inside of the \verb!iterate()! method provided by each Belos::Iteration. This code
generally takes the form:
\begin{verbatim}
SomeIteration::iterate() {
  while ( statustest.checkStatus(this) != Passed ) {
    //
    // perform iteration kernel
    //
  }
  return;  // return back to caller
}
\end{verbatim}

Each Belos::StatusTest provides a method, \verb!checkStatus()!, which through queries to
the methods provided by Belos::Iteration, determines whether the solver meets the
criteria defined by that particular status test. After an iteration returns from
\verb!iterate()!, the caller has the option of accessing the state associated with the
iteration and re-initializing it with a new state.

While this method of interfacing with the iteration is powerful, it can be tedious.
This method requires that a user construct a number of support classes, in addition to
managing calls to \verb!Iteration::iterate()!.  The Belos::SolverManager class was developed to
address this need. A solver manager is a class that wraps around an iteration,
providing additional functionality while also handling lower-level interaction with the
iteration that a user may not wish to handle. Solver managers are intended to be 
easy to use, while still providing the features and flexibility needed to solve real-world
linear problems. For example, the Belos::BlockCGSolMgr takes only two
arguments in its constructor: a Belos::LinearProblem specifying the linear problem
to be solved and a Teuchos::ParameterList of options specific to this solver manager. The
solver manager instantiates an iteration, along with the status tests and other support
classes needed by the iteration. To solve the linear problem, the user simply calls
the \verb!solve()! method of the solver manager. The solver manager performs repeated
calls to the \verb!iterate()! method, performs restarts or recycling, and
places the final solution into the linear problem.

Users therefore have a number of options for computing solutions to linear problems with Belos:
\begin{itemize}
  \item use an existing solver manager;\\
        In this case, the user is limited to the functionality provided by the current linear solvers.
  \item develop a new solver manager around an existing iteration;\\
        The user can extend the functionality provided by the iteration, specifying 
        custom configurations for status tests, orthogonalization, restarting, recycling,
        etc.
  \item develop a new iteration/solver manager;\\
        The user can write an iteration that is not represented in
        Belos. The user still has the benefit of the support classes provided by 
        Belos, and the knowledge that the new iteration / solver manager can be easily
        used by anyone already familiar with Belos.
\end{itemize}


\section{Belos Classes}
%%%%%%%%%%%%%%%%%%%%%%%%%%%%%%%%%%%%%%%%%%%%%%%%%%%%%%
\label{sec:belos:classes}

Belos is designed with extensibility in mind, so that users can augment the package with
any special functionality that they need. However, the released version of Belos
provides all functionality necessary for solving a wide variety of problems. This section
lists and briefly describes the current abstract classes found in Belos.  Each subsection 
also includes information on the derived classes provided by Belos.

\begin{remark}
Belos makes extensive use of the Teuchos utility classes, especially
Teuchos::RCP (Section~\ref{sec:teuchos:RCP}) and
Teuchos::ParameterList (Section~\ref{sec:teuchos:ParameterList}). Users
are encouraged to become familiar with these classes and their correct
usage.
\end{remark}

\subsection{Belos::LinearProblem}
%%%%%%%%%%%%%%%%%%%%%%%%%%%
\label{sec:belos:linearproblem}

Belos::LinearProblem is a templated container for the components of a linear problem, 
as well as the solutions.  Both the linear problem and the linear solver in Belos are 
templated on the scalar type, the multivector type and the operator type. Before
declaring a linear problem, users must choose classes to represent these
entities. Having done so, they can begin to specify the parameters of the
linear problem using the Belos::LinearProblem \textbf{set} methods:
\begin{itemize}
\item \verb!setOperator! - set the operator $A$ for which the solutions will be computed
\item \verb!setLHS! - set the left-hand side (solution) vector $X$ for the linear problem $A X = B$
\item \verb!setRHS! - set the right-hand side vector $B$ for the linear problem $A X = B$
\item \verb!setLeftPrec! - set the left preconditioner for the linear problem
\item \verb!setRightPrec! - set the right preconditioner for the linear problem
\item \verb!setHermitian! - specify whether the problem is Hermitian
\item \verb!setLabel! - specify the label prefix used by the timers in this object
\end{itemize}
In addition to these \textbf{set} methods, Belos::LinearProblem defines
a method \verb!setProblem()! that gives the class the opportunity to perform
any initialization that may be necessary before the problem is handed off 
to a linear solver, in addition to verifying that the problem has been adequately defined. 

For each of the \textbf{set} methods listed above, there is a corresponding
\textbf{get} function. These are the functions used the iterations and solver managers to 
get necessary information from the linear problem. 

\subsection{Belos::Iteration}
%%%%%%%%%%%%%%%%%%%%%%%%%%%
\label{sec:belos:iterativesolver}

The Belos::Iteration class defines the basic interface that must be
met by any iteration class in Belos. The specific iterations are
implemented as derived classes of Belos::Iteration.
Table~\ref{tab:belos:solvers} lists the linear solver currently implemented in
Belos.  

\begin{table}[htp]
\begin{center}
\begin{tabular}{| p{4cm} p{10cm} |}
\hline
Solver & Description \\
\hline
{\tt CG}               & A basic single-vector CG iteration for SPD linear problems.\\
{\tt BlockCG}          & A block CG iteration for SPD linear problems. \\
{\tt BlockGmres}       & A block GMRES iteration for non-Hermitian linear problems. \\
{\tt BlockFGmres}      & A block flexible GMRES iteration for non-Hermitian linear problems. \\
{\tt PseudoBlockGmres} & A simultaneous single-vector GMRES iteration for non-Hermitian linear problems. \\
{\tt GCRODR }          & A deflation, restarted GCRO iteration for non-Hermitian linear problems. \\
\hline
\end{tabular}
\caption{Iterations currently implemented in Belos.}
\label{tab:belos:solvers}
\end{center}
\end{table}

The iteration interface provides two significant types of methods: status methods and
solver-specific state methods. The status methods are defined by the Belos::Iteration
abstract base class and represent the information that a generic status test can request
from any linear solver. A list of these methods is given in
Table~\ref{tab:belos:genstatusmethods}.

\begin{table}[htp]
\begin{center}
\begin{tabular}{| p{4cm} p{10cm} |}
\hline
Method & Description \\
\hline
{\tt getNumIters}       & Get the current number of iterations. \\
{\tt resetNumIters}     & Reset the number of iterations. \\
{\tt getNativeResiduals}& Get the most recent residual norms native to the iteration. \\
{\tt getCurrentUpdate}  & Get the most recent solution update computed by the iteration. \\
\hline
\end{tabular}
\caption{A list of generic status methods provided by Belos::Iteration.}
\label{tab:belos:genstatusmethods}
\end{center}
\end{table}

The class Belos::Iteration, like Belos::LinearProblem, is templated on the scalar
type, multivector type and operator type. The options for the iteration are passed in
through the constructor, defined by Belos::Iteration to have the following form:
\begin{verbatim}
Iteration( 
   const Teuchos::RCP< LinearProblem<ST,MV,OP> > &problem, 
   const Teuchos::RCP< OutputManager<ST>       > &printer,
   const Teuchos::RCP< StatusTest<ST,MV,OP>    > &tester,
   const Teuchos::RCP< OrthoManager<ST,OP>     > &ortho,
   ParameterList                                 &params  
 );
\end{verbatim}

These classes are used as follows:
\begin{itemize}
  \item \verb!problem! - the linear problem to be solved; the solver will
    get the operator and vectors from here; see Section~\ref{sec:belos:linearproblem}.
  \item \verb!printer! - the output manager dictates verbosity level in addition to 
    processing output streams; see Section~\ref{sec:belos:printer}.
  \item \verb!tester! - the status tester dictates when the solver should quit
    \verb!iterate()! and return to the caller; see Section~\ref{sec:belos:tester}.
  \item \verb!ortho! - the orthogonalization manager defines the inner product and other
    concepts related to orthogonality, in addition to performing these computations for 
    the solver; see Section~\ref{sec:belos:ortho}.
  \item \verb!params! - the parameter list specifies linear solver-specific options; see the
    documentation for a list of options support by individual solvers.
\end{itemize}

An iteration class also specifies a concept of state,
i.e. the current data associated with the iteration. After declaring an iteration
object, it is in an uninitialized state. For most iterations, to be
initialized means to be in a valid state, containing all of the information necessary for
performing an iteration step. Belos::Iteration provides two methods concerning
initialization: \verb!isInitialized()! indicates whether the iteration is initialized or not,
and \verb!initialize()! (with no arguments) instructs the iteration to initialize itself
using the linear problem.

To ensure that iterations can be used as efficiently as possible, 
the user needs access to their state. To this end, each iteration provides 
low-level methods for getting and setting their
state:
\begin{itemize}
  \item \verb!getState()! - returns an iteration-specific structure with read-only pointers to
    the current state of the iteration.
  \item \verb!initialize(...)! - accepts an iteration-specific structure enabling the user to
    initialize the iteration object with a particular state.
\end{itemize}

The combination of these two methods, along with the flexibility provided by status tests,
allows the user almost total control over linear solver iterations.


\subsection{Belos::SolverManager}
%%%%%%%%%%%%%%%%%%%%%%%%%%%
\label{sec:belos:solvermanager}

Using Belos by interfacing directly with the linear solver iterations is extremely powerful, 
but can be more entailed than necessary. Solver managers provide a way for users to encapsulate 
specific solving strategies inside of an easy-to-use class. Novice users may prefer to use 
existing solver managers, while advanced users may prefer to write custom solver managers.

The Belos::SolverManager class provides a parameter list driven interface for solving linear
systems and, as such, has very few methods.  Only two constructors are supported: a default constructor
and a constructor accepting a Belos::LinearProblem and a parameter list of solver manager-specific 
options.  If the default constructor is used, the Belos::LinearProblem and parameter list can be passed 
in, post-construction, using the methods \verb!setProblem(...)! and \verb!setParameters(...)!, 
respectively.  It is also possible to get the valid parameters and current parameters from the 
solver manager by using the methods \verb!getValidParameters()! and \verb!getCurrentParameters()!, 
respectively.  Most importantly, there is a \verb!solve()! method that takes no arguments and 
returns either Belos::Converged or Belos::Unconverged.
Consider the following simple example code:
\begin{verbatim}
// create a linear problem
Teuchos::RCP< Belos::LinearProblem<ScalarType,MV,OP> > problem = ...;
// create a parameter list
Teuchos::RCP<Teuchos::ParameterList> params;
params->set(...);
// create a solver manager
Belos::BlockCGSolMgr<ScalarType,MV,OP> CGsolver( problem, params );
// solve the linear problem
Belos::ReturnType ret = CGsolver.solve();
// get the solution from the problem
Teuchos::RCP< MV > sol = problem->getLHS();
\end{verbatim}

\begin{remark}
  Errors in Belos are communicated via exceptions. This is outside the scope of this
  tutorial; see the Belos documentation for more information.
\end{remark}

As has been stated before, the goal of the solver manager is to create a linear solver iteration
object, along with the necessary support objects.  Another purpose of many
solver managers is to manage and initiate the repeated calls to the underlying iteration's
\verb!iterate()! method. For linear solver iterations that build a Krylov subspace to some 
maximum dimension (e.g., BlockGmres, BlockFGmres, etc.), the solver manager will also employ 
a strategy for restarting the solver when the subspace is full. This is something for which
multiple approaches are possible. Also, there may be substantial flexibility in creating
the support classes (e.g., sort manager, status tests) for the solver. An aggressive
solver manager could even go so far as to construct a preconditioner for the linear
problem, or switch its solution technique based on convergence behavior. 

These examples are meant to illustrate the flexibility that specific solver managers may
have in implementing the \verb!solve()! routine. Some of these options might best be
incorporated into a single solver manager, which takes orders from the user via the
parameter list. Some of these options may better be contained in
multiple solver managers, for the sake of code simplicity. It is even possible to write
solver managers that contain other solvers managers; motivation for something like this
would be to select the optimal solver manager at runtime based on some expert knowledge,
or to create a hybrid method which uses the output from one solver manager to
initialize another one.

\subsection{Belos::StatusTest}
%%%%%%%%%%%%%%%%%%%%%%%%%%%
\label{sec:belos:tester}

By this point in the tutorial, the purpose of the Belos::StatusTest should be clear: to
give the user or solver manager flexibility in stopping the linear solver iterations in
order to interact directly with the iteration.

Many reasons exist for why a user would want to stop the iteration from continuing:
\begin{itemize}
  \item some convergence criterion has been satisfied and it is time to quit;
  \item some part of the current solution has reached a sufficient accuracy to removed
    from the iteration;
  \item the solver has performed a sufficient or excessive number of iterations.
\end{itemize}
These are just some commonly seen reasons for ceasing the iteration, and each of these can
be so varied in implementation/parametrization as to require some abstract mechanism
controlling the iteration.

The following is a list of Belos-provided status tests:
\begin{itemize}
  \item Belos::StatusTestCombo - this status test allows for the boolean combination of
    other status tests, creating near unlimited potential for complex status tests.
  \item Belos::StatusTestOutput - this status test acts as a wrapper around another
    status test, allowing for printing of status information on a call to
    \verb!checkStatus()!
  \item Belos::StatusTestMaxIters - this status test monitors the number of iterations
    performed by the solver; it can be used to halt the solver at some maximum number of iterations
    or even to require some minimum number of iterations.
  \item Belos::StatusTestGenResNorm - this status test allows the user to construct one of
    a family of residual tests to monitor the residual norms of the current iterate.
  \item Belos::StatusTestImpResNorm - this status test monitors the implicit residual norm
    (e.g., native residual available through GMRES) and checks for loss of accuracy.
\end{itemize}

\subsection{Belos::OrthoManager}
%%%%%%%%%%%%%%%%%%%%%%%%%%%
\label{sec:belos:ortho}

Orthogonalization and orthonormalization are commonly performed computations in iterative
linear solvers; in fact, for some linear solvers, they represent the dominant cost.  Different
scenarios may require different approaches (e.g., Euclidean inner product versus a weighted inner
product, full versus partial orthogonalization).  Combined with the plethora
of available methods for performing these computations, Belos has left as much leeway to
the users as possible.

Orthogonalization of multivectors in Belos is performed by derived classes of
the abstract class Belos::OrthoManager. This class provides five methods:
\begin{itemize}
  \item \verb!innerProd(X,Y,Z)! - performs the inner product defined by the manager.
  \item \verb!norm(X)! - computes the norm induced by \verb!innerProd()!.
  \item \verb!project(X,C,Q)! - given an orthonormal basis $Q$, projects $X$ onto to the space perpindicular to
    $colspan(Q)$, optionally returning the coefficients of $X$ in $Q$.
  \item \verb!normalize(X,B)! - returns an orthonormal basis for $colspan(X)$, optionally
    returning the coefficients of $X$ in the computed basis.
  \item \verb!projectAndNormalize(X,C,B,Q)! - computes an orthonormal basis for subspace
    \newline
    $colspan(X) - colspan(Q)$, optionally returning the coefficients of
    $X$ in $Q$ and the new basis.
\end{itemize}

It should be noted that a call to \verb!projectAndNormalize()! is not necessarily
equivalent to a call to \verb!project()! followed by \verb!normalize()!. This follows from
the fact that, for some orthogonalization managers, a call to \verb!normalize()! may
augment the column span of a rank-deficient multivector in order to create an orthonormal
basis with the same number of columns as the input multivector. In this case, the code
\begin{verbatim}
orthoMgr.project(X,C,Q);
orthoMgr.normalize(X,B);
\end{verbatim}
\noindent could result in an orthonormal basis $X$ that is not orthogonal to the basis in $Q$.

Belos provides three orthogonalization managers:
\begin{itemize}
  \item Belos::DGKSOrthoManager - performs orthogonalization using classical Gram-Schmidt with
    a possible correction step.
  \item Belos::ICGSOrthoManager - performs orthogonalization using iterated classical Gram-Schmidt.
  \item Belos::IMGSOrthoManager - performs orthogonalization using iterated modified Gram-Schmidt.
\end{itemize}

More information on these orthogonalization managers is available in the Belos
documentation.

\subsection{Belos::OutputManager}
%%%%%%%%%%%%%%%%%%%%%%%%%%%
\label{sec:belos:printer}

The output manager is a concrete class in Belos and exists to provide
flexibility with regard to the verbosity of the linear solver. The output manager has
two primary concerns: what output is printed and where the output is printed to.
When working with the output manager, output is classified into one of the 
message types from Table~\ref{tab:belos:om}.

\begin{table}
\begin{center}
  \begin{tabular}{| p{4cm} p{8cm} |}
\hline
Message type & Description \\
\hline
{\tt Errors           } & Errors (always printed)  \\
{\tt Warnings         } & Warning messages   \\
{\tt IterationDetails } & Approximate eigenvalues, errors   \\
{\tt OrthoDetails     } & Orthogonalization/orthonormalization checking \\
{\tt FinalSummary     } & Final computational summary (usually from SolverManager::solve())  \\
{\tt TimingDetails    } & Timing details  \\
{\tt StatusTestDetails} & Status test details   \\
{\tt Debug            } & Debugging information \\
\hline
\end{tabular}
\caption{Message types used by Belos::OutputManager.}
\label{tab:belos:om}
\end{center}
\end{table}

The output manager in Belos provides the following output-related methods:
\begin{itemize}
  \item {\tt bool setVerbosity (MsgType type)} - 
  Set the type of messages we need to print out information for.
  \item {\tt bool setOStream ( const Teuchos::RCP<std::ostream> \&os )} - 
  Set the output stream where information should be sent.
  \item {\tt bool isVerbosity (MsgType type)} - 
  Find out whether we need to print out information for this message type.
\item {\tt void  print (MsgType type, const string output)} - 
  Send output to the output manager.
\item {\tt ostream \& stream (MsgType type)} - 
  Create a stream for outputting to.
\end{itemize}

The output manager is meant to ease some of the difficulty associated with I/O in a
distributed programming environment. For example, consider some debugging output requiring
optional computation. For reasons of efficiency, we may want to perform the computation
only if debugging is requested; i.e., \verb!isVerbosity(Belos::Debug) == true!. However,
while we need all nodes to enter the code block to perform the computation, we probably
want only one of them to print the output. For the Belos::OutputManager, the output 
corresponding to the verbosity level of the manager is sent to the output stream only on 
the master node; the output for other nodes is neglected.


\section{Using the Belos adapter to Epetra}
%%%%%%%%%%%%%%%%%%%%%%%%%%%%%%%%%%%%%%%%%%%%%%%%%%%%%%
\label{sec:belos:epetra}

The Epetra package provides the underlying linear algebra foundation for many
Trilinos solvers.  By using the Belos adapter to Epetra, users not only
avoid the trouble of implementing their own multivector and operator classes, but
they also gain the ability to utilize any other Trilinos package which
recognizes Epetra classes (such as AztecOO, IFPACK, ML, and others).

In order to use the Belos adapter to Epetra, users must include the following
file:
\begin{verbatim}
#include "BelosEpetraAdapter.hpp"
\end{verbatim}
This file simply defines specializations of the Belos::MultiVecTraits
and Belos::Operator\-Traits classes, while also including the Epetra
header files defining the multivector and operator classes.

Because Epetra makes exclusive use of double precision arithmetic, 
Epetra\_Operator and Epetra\_MultiVector are used only with 
scalar type \verb!double!. For brevity, it is useful to declare type definitions
for these classes:
\begin{verbatim}
typedef double ST;
typedef Epetra_MultiVector MV;
typedef Epetra_Operator OP;
\end{verbatim}

\noindent Multivectors will be of type \verb!MV!:
\begin{verbatim}
Teuchos::RCP<MV> X 
   = Teuchos::rcp( new MV(...) );
\end{verbatim}

\noindent Operators can be any subclass of \verb!OP!, for example, an Epetra\_CrsMatrix:
\begin{verbatim}
Teuchos::RCP<OP> A 
   = Teuchos::rcp( new Epetra_CrsMatrix(...) );
\end{verbatim}

The Belos interface to Epetra defines a specialization of
Belos::MultiVecTraits for Epetra\_MultiVector and a
specialization of Belos::OperatorTraits for Epetra\_Operator
applied to Epetra\_MultiVector. Therefore, we can now specify a
linear solver and preconditioner utilizing these computational classes. 
An example of defining and solving a linear problem using a Belos
linear solver is given in the next section.


\section{Defining and Solving a Linear Problem}
%%%%%%%%%%%%%%%%%%%%%%%%%%%%%%%%%%%%%%%%%%%%%%%%%%%%%%
\label{sec:belos:example}

This section gives sample code for solving a symmetric positive definite (SPD)
linear problem using the block CG solver manager, Belos::BlockCGSolMgr. 
The example code in this section comes from the Didasko example \TriExe{belos/ex1.cpp}.

The first step in solving a linear problem is to define the problem using the
Belos::LinearProblem class.  Assume we have chosen classes to represent our scalars, 
multivectors and operators as \verb!ST!, \verb!MV! and \verb!OP!, respectively. Given an
operator \verb!A!, a multivector \verb!X! containing the initial guess, and a multivector
\verb!B! containing the right-hand side, all wrapped in Teuchos::RCP, 
we would define the linear problem as follows:
\begin{verbatim}
Teuchos::RCP< LinearProblem<ST,MV,OP> > myProblem 
  = Teuchos::rcp( new LinearProblem<ST,MV,OP>(A,X,B) );
myProblem->setHermitian();
bool ret = myProblem->setProblem();
if (ret != true) {
   // there should be no error in this example :)
}
\end{verbatim}

The first line creates a Belos::LinearProblem object and wraps it in a
Teuchos::\-RCP (Section~\ref{sec:teuchos:RCP}). The second line
specifies the symmetry of the linear problem. The third line signals that we have
finished setting up the linear problem. This step must be completed before
attempting to solve the problem; failure to do so will result in the solver manager
throwing an exception.

Next, we create a parameter list wrapped in a Teuchos::\-RCP to specify the 
parameters for the solver manager:
\begin{verbatim}
int verb = Belos::Warnings + Belos::Errors 
         + Belos::FinalSummary + Belos::TimingDetails;
Teuchos::RCP<Teuchos::ParameterList> myPL;
myPL->set( "Verbosity", verb );
myPL->set( "Block Size", 4 );
myPL->set( "Maximum Iterations", 100 );
myPL->set( "Convergence Tolerance", 1.0e-8 );
\end{verbatim}

Here, we have asked for the linear solver to output information regarding errors and
warnings, as well as to provide a final summary after completing all iterations and
print the timing information collected during the solve. We have also specified the
tolerance for convergence testing (used to construct a status test); the block size; 
and the maximum number of iterations (used to construct a status test).
This solver manager permits other options as well, affecting the
block size size as well as the output; see the Belos documentation.

We now have all of the information needed to declare the solver manager and solve the
problem:
\begin{verbatim}
Belos::BlockCGSolMgr<ST,MV,OP> mySolver( myProblem, myPL );
\end{verbatim}
The linear problem is solved with the instruction
\begin{verbatim}
Belos::ReturnType solverRet = mySolver.solve();
\end{verbatim}

The return value of the solver indicates whether the algorithm succeeded or not; i.e.,
whether the requested residual tolerance was achieved in the alotted number of iterations.
Output from \verb!solve()! routine in this example might look as follows:
\begin{verbatim}
Passed.......OR Combination ->
  OK...........Number of Iterations = 21 < 100
  Converged....(2-Norm Imp Res Vec) / (2-Norm Res0)
               residual [ 0 ] = 6.38815e-09 < 1e-08
               residual [ 1 ] = 9.08579e-09 < 1e-08
               residual [ 2 ] = 9.26932e-09 < 1e-08
               residual [ 3 ] = 4.43278e-09 < 1e-08

================================================================================

                              TimeMonitor Results

Timer Name                                Local time (num calls)
--------------------------------------------------------------------------------
Belos: Operation Op*x                     0.000583 (22)
Belos: Operation Prec*x                   0 (0)
Epetra_CrsMatrix::Multiply(TransA,X,Y)    0.000453 (22)
Belos: Orthogonalization                  0.006996 (23)
Belos: BlockCGSolMgr total solve time     0.01208 (1)
================================================================================
\end{verbatim}

The solution vector can be retrieved from the linear problem (where they were
stored by the solver manager) as follows:
\begin{verbatim}
  Teuchos::RCP<MV> sol = myProblem->getLHS();
\end{verbatim}
Then the residual can be checked using the computed solution, resulting in output
that should look like:
\begin{verbatim}
******************************************************
           Results (outside of linear solver)
------------------------------------------------------
  Linear System         Relative Residual
------------------------------------------------------
  1                     6.388147e-09
  2                     9.085794e-09
  3                     9.269324e-09
  4                     4.432775e-09
******************************************************
\end{verbatim}

%Four examples are provided with the tutorial:
%\begin{itemize}
%\item \TriExe{belos/ex1.cpp}: compute the eigenvectors
%corresponding to the smallest eigenvalues for a 2D Laplace problem using the block
%Krylov Schur solver
%\item \TriExe{belos/ex2.cpp}: solves the problem from \verb!ex1! 
%using instead the block Davidson linear solver
%\item \TriExe{belos/ex3.cpp}: uses the block Krylov Schur solver to solve a
%  non-Hermitian convection-diffusion problem
%\item \TriExe{belos/ex4.cpp}: uses the LOBPCG solver to solve the 2D Laplacian
%problem from \verb!ex1!
%\end{itemize}

