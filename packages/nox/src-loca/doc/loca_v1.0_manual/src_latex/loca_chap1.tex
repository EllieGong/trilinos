\chapter{Introduction}

\section{LOCA Overview}

This document is the theory and implementation manual for version 1.0 of LOCA, the Library Of Continuation Algorithms. When implemented with an application code, LOCA enables the tracking of solution branches as a function of a system parameter and the direct tracking of bifurcation points. LOCA is designed to drive application codes that use Newton's method to locate steady-state solutions to nonlinear problems. The algorithms are chosen to work for large problems, such as those that arise from discretizations of partial differential equations, and to run on distributed memory parallel machines.

The approach in LOCA for locating and tracking bifurcations begins with augmenting the residual equations defining a steady state with additional equations that describe the bifurcation. A Newton method is then formulated for this augmented system; however, instead of loading up the Jacobian matrix for the entire augmented system (a task that involved second derivatives and dense matrix rows), bordering algorithms are used to decompose the linear solve into several solves with smaller matrices. Almost all of the algorithms just require multiple solves of the Jacobian matrix for the steady state problem to calculate the Newton updates for the augmented system. This greatly simplifies the implementation, since this is the same linear system that an application code using Newton's method will already have invested in. Only the Hopf tracking algorithm requires the solution of a larger matrix, which is the complex matrix involving the Jacobian matrix and an imaginary multiple of the mass matrix.

The following algorithms are available in this version of LOCA:
\begin{description} 
\item[1.] \textsc{zero\_order\_continuation }
\item[2.] \textsc{first\_order\_continuation }
\item[3.] \textsc{arc\_length\_continuation }
\item[4.] \textsc{turning\_point\_continuation } (fold points)
\item[5.] \textsc{pitchfork\_continuation }
\item[6.] \textsc{hopf\_continuation }
\item[7.] \textsc{phase\_transition\_continuation }
\end{description}
The LOCA library relies heavily on a robust linear stability analysis capability. This is needed for detecting the first instance of Hopf and Pitchfork bifurcations and generating initial guesses for the null vectors for these routines. We have used the P\_ARPACK package \cite{lehoucq98b}\cite{maschhoff96} to indentify a few select eigenvalues and eigenmodes. With this release of LOCA, we are including routines that drive P\_ARPACK to perform the Cayley transformation, which we have found to work well for large scale problems \cite{lehoucq01}\cite{Burroughs01}.

(In addition, the LOCA interface has been linked to an rSQP optimization code. The
\textsc{rsqp\_optimization} option will be discussed in the implementation section, but details on how to use this feature are outside the scope of this document.)

The rest of the document is organized as follows. Some background information on bifurcations follows in Section \ref{sec:background}. Chapter \ref{ch:algorithms}
is the theory manual, and presents the algorithms that are implemented in LOCA.
Chapter \ref{ch:implementation} shows the directory structure of the Loca code and contains instructions on how to implement LOCA around a new application code, including a recipe to follow. Chapter \ref{ch:wrappers} gives details on each of the wrapper routines that give LOCA access to routines in the application code. Chapter \ref{ch:structures} goes through each of the elements of the two structures that control the continuation algorithms. Chapter \ref{ch:strategies} explains the typical strategies employed to track bifurcations. This document does not contain demonstrations of the LOCA library being used with application codes.

Publications and presentations that demonstrate the use of the LOCA library for analysis of applications can be obtained from the LOCA web page:
\begin{itemize}
\item \textbf{www.cs.sandia.gov/LOCA}
\end{itemize}
This web page contains links to the download site for the source code and
this manual. 

The LOCA code is being licensed under the GNU Lesser General Public License,  
a copy of which is available with the code and can also be found at www.gnu.org.
We welcome users to submit suggestions, bug reports, bug fixes, extensions, and results and publications generated using LOCA.

\section{A Survey of Bifurcation Theory}
\label{sec:background}
This section is meant to give the user a brief introduction to bifurcation theory.  For more information we refer users to the following references \cite{tavener01,guckenheimer83,seydel88,iooss81,golubitsky85}. The Seydel book in particular has an excellent list of references.

\subsection{Complex Behavior in Nonlinear System}
The existence of multiple steady state solutions, or of 
unstable steady solutions, can lead to 
interesting behavior in nonlinear systems.  
This interesting behavior can lead to undesirable effects
such as a system operating in one way on certain occasions, and
in another way on other occasions.  On the other hand, this
interesting behavior can also be desirable, as  when  one is 
interested in making switches or  oscillators.

LOCA, and bifurcation theory in general, is aimed at 
helping scientists and engineers efficiently map out 
different regions of parameter space that have 
qualitatively different behavior.  Although one could in principle
map out these regions using only  transient calculations,
 this would be  inefficient and  unreliable\footnote{We should mention that transient calculations
could yield information not obtained in a bifurcation
analysis.  Ideally one would use a transient analysis  in
conjunction with 
a  bifurcation analysis.}.   For
example, suppose we wanted to know for what values of parameters
 a system has two stable steady state solutions.  For
 a given value of the parameters we could determine that there are 
multiple  stable steady states by solving many different initial value
problems.  If some converge towards one solution, and others
converge towards another, then we have multiple stable 
 steady states.
However, each transient calculation will be time consuming, 
and it is not clear  how many trials we need to do in order to 
be convinced that only one stable  steady solution exists.   Furthermore,
once we have convinced ourselves how many stable steady states
 there are for one set of parameters,  this information does
not greatly reduce the search when we change the parameters 
slightly.

Using tools like LOCA and P\_ARPACK, one only solves for steady state
solutions, but takes into account the  transient behavior by 
looking at the eigenvalues of the linearized system. 
The bifurcation approach has two advantages over the transient
approach.  
\begin{itemize}
\item We can solve steady equations, which is much more 
efficient computationally then integrating until a steady state is reached.  
\item  We can append equations to our steady state solution that
allow us to find the  value of a  parameter where the 
system changes its behavior.  When we have more than one
 parameter in our system, we can then map out parameter
space very efficiently. 
\end{itemize}

\subsection{Generic Behavior} 
Suppose that we have a dynamical system of the form
\begin{equation}
 \bfB \dotbfx = \bfR(\bfx,\lambda) 
\end{equation} 
where \bfx\ is a vector (possibly of infinite dimension)
that determines the state of our system, and $\lambda$ is a 
parameter.  Steady state solutions can be found by solving
\begin{equation}
 \bfR(\bfx,\lambda)=0
\end{equation}  
The stability of a steady 
state solution \bfx$_0$ is determined by 
finding the eigenvalues of the system
\begin{equation}
 \sigma \bfB \bfw = \bfJ(\bfx_0) \bfw
\end{equation}
where \bfJ(\bfx$_0$) is the Jacobian of the function \bfR\
evaluated at the steady solution \bfx$_0$.  If all of the
eigenvalues of this system have real parts less than 
zero, the steady state solution is stable.  If any of the
eigenvalues have a real part bigger than zero, the 
system is unstable.  

It is possible to find many 
different ways for a solution to lose its stability.
For example, we can mathematically construct
  systems that have 10 modes all go
 unstable at the same value of the parameter.  It can be 
exceedingly difficult to analyze such problems, but fortunately
problems of this sort are unlikely to occur in practice. 
 One of the
 basic ideas in bifurcation
theory is that of generic behavior.  We only concern
 ourselves with behavior that is likely to occur in a system.  Behavior is 
said to be non-generic if when we perturb our equations, the
behavior no longer exists.  For example, it is generic for
two curves in the plane to intersect with non zero slope.
If we slightly change the equations of the two curves, they will
still intersect at some other point with non-zero slope.  
However, it is not generic for two curves in three dimensional 
space to intersect.  Although it is not difficult to find examples
of curves that intersect.  If we apply almost any
arbitrarily small perturbation to the equations, they
 will not intersect.

We now apply this notion of a generic property to nonlinear
stability problems.
Suppose we have a steady state solution \bfx$_0$($\lambda$) that
depends on a parameter $\lambda$.  Suppose that we analyze the
stability of this solution, and find that at 
some value $\lambda_0$ this solution goes unstable.  This
implies that there is a critical eigenvalue that has a 
zero real part.
Elementary bifurcation theory shows that there are only two 
generic ways that  a solution can lose its stability.   

\begin{itemize}
\item  If the  critical eigenvalue  is zero, the system 
has a turning point.
\item If the critical eigenvalue is imaginary, the system
undergoes a Hopf bifurcation.
\end{itemize}

A turning point is a point in parameter space where   two
solution branches merge and then disappear.  For example
the equation $x^2 + \lambda=0$ has two solutions if $\lambda<0$, but
no solutions if $\lambda>0$.  The point $x=0$, $\lambda=0$ is 
a turning point (Figure \ref{fig:turningpoint}a).
\begin{figure}[tb]
 \begin{minipage}{0.5\linewidth}
  \centering\epsfig{figure=figures/tecgraphone.eps,width=0.8\linewidth,angle=0} \\
  (a) 
 \end{minipage} \hfill
 \begin{minipage}{0.5\linewidth}
  \centering\epsfig{figure=figures/tecgraphtwo.eps,width=0.8\linewidth,angle=0} \\
  (b)
 \end{minipage} \\
  \caption{Bifurcation diagrams of a turning point: (a) one turning point, (b) multiple turning points (leading to hysteresis).}
 \label{fig:turningpoint}
\end{figure} Some physical examples of turning points include:
\begin{itemize}
\item  The buckling of a shallow arch under symmetrical loading.
\item  The breaking away of a drop from a tube when the volume
is too large.
\item  The bursting of a balloon when a critical volume is reached.
\item Ignition of an explosion.
\end{itemize}

Elementary bifurcation theory shows that the stability of 
one mode changes as we go around a turning point.  This implies
that if one of the solutions is stable, then the  other solution 
will be unstable.  Frequently we encounter more than one
turning point in a system.  In this case we can get an \textsf{S} shaped 
solution branch (Figure \ref{fig:turningpoint}b).   For sufficiently large or small values of the
parameters we will have only one solution, but for intermediate
values we will have three steady solutions.  The upper and
lower branches will be stable, the middle branch unstable.  
This \textsf{S} shaped solution curve can lead to nonlinear
hysteresis, where as we increase our parameter value
the solution suddenly jumps from the lower to the upper
solution. If we then slightly decrease the value of the
parameter the solution does not jump back to the lower solution,
but will only do this if we greatly reduce the value of 
the parameter.  

The other generic form of loss of stability is the
 Hopf bifurcation.   After  a Hopf bifurcation  the system 
no longer settles down to a steady state, but will begin to
oscillate periodically.  Examples of Hopf bifurcations include
the onset of :

\begin{itemize}
\item  Vertex shedding behind bluff bodies.
\item  Flutter in airplane wings.
\item   Oscillations in electrical circuits 
\item  Shimmying of wheels.
\end{itemize}

Note that when  we have located either one of these 
bifurcations, we are at very interesting points in
parameter space.  
The turning point and Hopf bifurcation are generically the 
only bifurcations we expect to see in a one parameter system.
For this reason they are called co-dimension one bifurcations.
If we have more than one parameter in our system we can get 
more degenerate bifurcations.  These more degenerate bifurcations
tell us even more about our system than a co-dimension 
one bifurcation.  For example suppose our system has two
 parameters 
$\lambda$ and $\mu$. Suppose that for some value of $\mu$ 
our solution curves as a function of $\lambda $ look like those
in Figure \ref{fig:pitchfork}a.
\begin{figure}[tb]
 \begin{minipage}{0.5\linewidth}  
  \centering\epsfig{figure=figures/tecgraphfour.eps,width=0.8\linewidth,angle=0} \\
  (a)
 \end{minipage} \hfill
 \begin{minipage}{0.5\linewidth}
  \centering\epsfig{figure=figures/tecgraphthree.eps,width=0.8\linewidth,angle=0} \\
  (b)
 \end{minipage}
 \caption{Bifurcation diagrams of a symmetry breaking bifurcation: (a) broken pitchfork bifurcation (b) pitchfork bifurcation.}
 \label{fig:pitchfork}
\end{figure}  If we know about the  turning point
$T$, we know that our solution has at least two 
solutions for some values of the parameters.  However, knowing 
about the turning point does not tell us that the 
upper disconnected branch exists.  Suppose that as we change the
parameter $\mu$ we find that for some value of 
$\mu$  our solution curves look like those in Figure \ref{fig:pitchfork}b.  The
particular value of $\lambda$, $\mu$ and $\bfx$ where we
have these solution branches intersecting is an example of a 
co-dimension two bifurcation. (This particular co-dimension 
two bifurcation is called a pitchfork bifurcation.) If we know about the
existence of a co-dimension two bifurcation we know that
when we unfold it, we will get behavior like we see in 
Figure \ref{fig:pitchfork}a (and other behavior as well).  This example
illustrates the point that
there is more information in a co-dimension two bifurcation.

\subsection{Symmetry Breaking}
When analyzing the stability of steady state solutions it is 
very common to find non-generic behavior.  For example,
if we take a beam and load it symmetrically we find that
 the simple solution where it is straight up and down is stable 
if the loading is not large.  When the loading gets to be
large enough the beam will buckle by bending either to 
the left or right.  Note that this is not a turning point, or a 
Hopf bifurcation.  It is not a Hopf bifurcation because the
beam does not begin oscillating, It is not a turning point because
we have two solution branches intersecting each other, rather than
a single solution branch turning around on itself.  

At first sight this appears to cast some doubt on the usefulness of the concept of generic behavior.  However, we can note that
if the beam was not built perfectly symmetrical 
we would find that the pitchfork bifurcation gets split up 
into two branches that do not intersect.  We only get the
non-generic behavior in the symmetrical case.   

The best way out of this dilemma is to note that
scientists and engineers  often analyze symmetrical situations.
We then ask  what type of generic behavior we expect to 
encounter under the assumption that our system has some symmetry.
It should be emphasized that these additional types of 
bifurcations are co-dimension one bifurcations
 in the presence of symmetry, but they would be
 co-dimension two (or
 higher) bifurcations if no symmetry were present.  For this
reason we see that the scientist or engineer is actually 
being wise in choosing to analyze a system with symmetry.
This allows them to get more information out of a one 
parameter system.

Certain sorts of symmetry can ensure the existence of multiple
eigenvalues in our system.  This can lead to very interesting and
complex behavior at bifurcation points.  In LOCA we limit
ourselves to  the simplest sort of bifurcations that can occur
when a system has symmetry.   These bifurcations assume that
the critical mode has a simple eigenvalue.

As in the previous section we assume that we are analyzing the
stability of a steady solution $\bfx_0(\lambda)$.  However, we
now assume that the governing equations (and our
solution $\bfx_0(\lambda)$)  are invariant under
some group of symmetry transformations.  We now suppose that
at some value of $\lambda$, our system goes unstable.  
We assume that the critical mode is a simple eigenvalue (this
is not generically requited in all systems with symmetry).
In this case if the critical eigenvalue has a
 nonzero imaginary part
we will once again get a Hopf bifurcation.  This bifurcation will
have some interesting spatio-temporal symmetry.  But, it is
still a Hopf bifurcation.  On the other hand, if the critical
eigenvalue is zero, there are two possibilities.
In order to understand these possibilities we need to realize that
when we analyze a system with symmetry, then the eigenvector 
associated with a simple real  eigenvalue must either be symmetric
or anti-symmetric with respect to any of the symmetries of the
group.   If the eigenvector is symmetric
 with respect to all of the transformations in the group, it is 
said to be a fully symmetric eigenvector.  If it is anti-symmetric
with respect to some transformations, it 
is said to be a symmetry breaking eigenvector.

When a symmetric system loses its stability at a simple real
eigenvalue, there are two possibilities


\begin{itemize}
\item  If the critical eigenvector is fully symmetric, then the
system encounters a turning point.  The symmetry of the 
solution does not change as we go around the turning point. 
\item  If the critical eigenvector is symmetry breaking, then 
we encounter a pitchfork bifurcation.  In this case the 
symmetrical solution $\bfx_0(\lambda)$ continues to exist on 
both sides of the critical value of $\lambda$.  There is a
second solution branch that intersects this symmetrical 
branch.  This second branch only exists locally on one side of the 
bifurcation point, and the solutions on this branch are not
symmetric with respect to all elements of the group.  
Furthermore the elements on the upper and lower branches are
transformed into each other by the elements of the
group.

\end{itemize}

\subsection{Tranversality Conditions}

We used the example of two curves intersecting with non-zero
slope in two 
dimensional space as an example of generic behavior.  
If we were a bit sloppy we could merely say that it is generic
for two curves to intersect in two dimensional space, without 
mentioning the non-zero slope condition.  The condition that
they have non-zero slope is known as a transversality 
condition, and is included to guarantee that we do not have a 
degenerate situation, and that we can prove that for any 
small enough perturbation of our equations, we can find a new
point of intersection.

In bifurcation theory the conditions stating that we are not at
a degenerate point in parameter space are also known as 
transversality conditions.  For example, suppose we 
have a system that has a turning point at $\bfx=\bfx_0$, 
$\lambda = \lambda_0$.   If this system is not degenerate, if 
we perturb it slightly,  the perturbed system should 
also have a turning point nearby  the original turning point.
If this is not the case, then we are at a degenerate point
in parameter space, and we have a situation similar to 
two two dimensional curves intersecting with zero slope.
There are 3 transversality conditions which guarantee that we
are not at a degenerate point.
Let $\bfr$ and $\bfl$ be the right and left eigenvectors of the
Jacobian matrix.   The transversality conditions can be stated as 
follows:

\begin{itemize}
\item  The zero eigenvalue is a simple eigenvalue.
\item  The quantity $\bfl\trans \pdone{F(\bfx_0,\lambda_0)}{\lambda}$
must not vanish.
\item The quantity $\bfl\trans \bfR_{\bfx \bfx}(\bfx_0,\lambda_0) 
\bfr \bfr $ must not vanish.
\end{itemize}

For a one dimensional problem these conditions can be 
stated as $\pdone{F}{\lambda} \neq 0 $, and $\pdtwo{F}{x} \neq0$.
For higher dimensional problems the  term $\bfl\trans \bfR_{\bfx \bfx}
\bfr \bfr $ is the projection of the quadratic terms onto the
center manifold.  

 For a Hopf bifurcation, the transversality conditions 
guarantee that if we have located a Hopf bifurcation, and then
we perturb our system, we can locate a nearby Hopf bifurcation.
The transversality conditions are

\begin{itemize}
\item   The Jacobian matrix does not have a zero eigenvalue, 
and there is one and only one complex conjugate pair of 
eigenvalues that has a zero real part.
\item  The complex conjugate pair of eigenvalues are simple 
eigenvalues.
\item  Let $\sigma(\lambda)$ be the critical eigenvalue.
The real part of the critical eigenvalue must pass through 
zero with non zero slope.  That is 
$ Re \odone{\sigma(\lambda_0)}{\lambda} \neq 0 $. 
\end{itemize}

The transversality conditions for a pitchfork bifurcation
(in the presence of symmetry) are :

\begin{itemize}
\item The critical eigenvalue is simple.
\item The critical eigenvalue passes through zero with 
nonzero slope, $\odone{\sigma(\lambda_0}{\lambda} \neq 0 $.
\end{itemize}

