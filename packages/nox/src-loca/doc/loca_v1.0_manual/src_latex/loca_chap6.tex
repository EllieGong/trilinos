%###################################################################
% Nonlinear Analysis Strategies
%###################################################################
\chapter{Nonlinear Analysis Strategies}
\label{ch:strategies}
This chapter describes procedures for performing nonlinear analysis on a system of equations using the algorithms provided by LOCA. These procedures are strategies for dealing with the fact that Newton's method is only locally convergent. For example, a Hopf bifurcation tracking requires highly accurate guesses for the eigenvectors of the complex pair of eigenvalues associated with the Hopf bifurcation. This section will discuss what the user is required to supply prior to running the tracking algorithms.  A typical nonlinear analysis would follow these steps:
\begin{description}
\item[1.] Calculate a steady state solution for the set of nonlinear equations from a trivial solution. This may require changing a parameter value so that the problem is more linear (e.g. decreasing the Reynolds number).
\item[2.] Use parameter continuation (zero order, first order, and arc-length) and eigenvalue analysis (linear stability analysis) to continue to parameter regions of interest and locate bifurcation points on the the steady state branch.
\item[3.] Using solution and eigenvalue/eigenvector information calculated near a bifurcation point, the tracking algorithms are used to ``lock-on'' to the solution and value of the bifurcation parameter at the bifurcation point, and track it as a function of a second parameter.
\end{description}

\section{Steady State Solutions}
We assume that the code was designed for steady state calculations and that this can be achieved trivially.  If the initial steady state can not be found using pure Newton's method, a globalized Newton method can be applied \cite{dennis83,nocedal99,more92,more83} (including integration in in time) or continuation in real or contrived parameters (Homotopy) can be used to get to the initial steady state of interest.  For parameter continuation the idea is to converge to a solution at simplified conditions (where Newton's method converges with a trivial initial guess). Then repeatedly adjust the parameters, while using previous solution information to generate a good initial guess for the solution vector, until you reach the steady state conditions of interest.  For example, this can be critical in reacting flow applications comprised of coupled sets of descretized PDEs. Here one would converge to a solution at isothermal conditions and slowly increase (take continuation steps) in temperature until the desired conditions are achieved.

\section{Parameter Continuation} \label{sec:parametercontinuation}
Once a steady state has been calculated, one can search along the steady state branch by stepping in the continuation parameter, $\lambda$.  LOCA provides algorithms for performing zero order, first order and arc-length parameter continuation, set by the flags \textsc{zero\_order\_continuation, first\_order\_continuation, and arc\_length\_continuation}, respectively. 

When a zero order or first order continuation run repeatedly has failed steps that results in cuts in the step size, this often indicates that the algorithm is approaching a turning point. This is where the solution branch doubles back and no solution exists locally for the next parameter step. Near this point one should restart the algorithm using the initial guess at the last converged solution but switch to arc-length parameter continuation.  If the parameter doubles back on a new branch, a turning point has been located.

\subsection{Locating a Bifurcation Point} \label{sec:locate}
A turning point can be identified directly by using the arc-length continuation algorithm as discussed in the previous section.  To identify any other bifurcation we must use the eigenvalue capabilities in LOCA to locate bifurcation points by monitoring the rightmost eigenvalues.  Section \ref{sec:background} gives a detailed description of bifurcation phenomena which we summarize here.  A bifurcation point occurs when the real part of an eigenvalue of the system under study is zero.  If the eigenvalue is purely real the bifurcation may be a turning point or a higher codimension bifurcation such as a pitchfork bifurcation.  If the eigenvalue has an imaginary component (i.e. is a complex conjugate pair), the bifurcation is a Hopf bifurcation.  Eigenvalue calculations can also ascertain branch stability. If all the eigenvalues are on the left side of the complex plane (i.e. all the eigenvalues have negative real parts) then the solution is a stable steady state.

To locate a bifurcation point, step in the continuation parameter while monitoring the eigenvalues. When the real part of an eigenvalue passes zero, you have evidence of a bifurcation point. Once a bifurcation point is identified, the solution and corresponding eigenvalues and eigenvectors should be calculated and saved for initial guesses for the bifurcation tracking algorithms discussed in the next section. Often the bifurcation tracking algorithms will converge using a solution and the appropriate eigenvectors from somewhere near the bifurcation point, and it is not necessary to zero in on the bifurcation using parameter continuation and eigenvalue calculations before starting the tracking algorithm . However, we have sometimes found that this extra work is needed to launch the tracking algorithms.

\section{Bifurcation Tracking} \label{sec:track}
The bifurcation tracking algorithms calculate the parameter value (referred to as the bifurcation parameter) that corresponds to a bifurcation point while doing zero order continuation in another parameter (referred to as the continuation parameter).  In order to perform bifurcation tracking, certain steps must be taken beforehand.  First, a bifurcation point must be located as discussed in section \ref{sec:parametercontinuation}.  The solution nearest the bifurcation should be used as the initial guess for the solution vector of the bifurcation tracking run.  The tracking algorithms calculate the steady state solution, eigenvector(s) (also called the NULL vector(s)), and a bifurcation parameter value.  Some tracking routines require accurate initial guesses for the NULL vectors while others can use a trivial guess.  We will discuss the typical ways to converge the initial step of a tracking algorithm individually.

Since LOCA was written for massively parallel distributed memory systems, we assume an iterative linear solver is used.  The bordering algorithms force us to solve nearly-singular linear systems so we typically tighten the linear solver tolerances and loosen the nonlinear solver tolerances during bifurcation tracking. Convergence is determined by driving the scaled update norm for the solution vector, NULL vector, bifurcation parameter and any other algorithm specific unknowns to \textbf{less than one}:
\begin{equation}
\mbox{Scaled Vector Update Norm} = \sqrt{\frac{1}{N}\sum_{i=1}^{N}\left[ \frac{|x_i|}{|x_i|*RTol + ATol} \right]^2}
\label{eqn:convergence}
\end{equation}
where $RTol$ and $ATol$ are the relative and absolute tolerances for the nonlinear solver, $N$ is the number of unknowns, and $x_i$ is the $i$th unknown value in the vector.  We often run with tolerances around 1.0E-2 and 1.0E-5 for the relative and absolute tolerances in the nonlinear solver.  Linear solver tolerances are usually kept in the range 1.0e-4 to 1.0E-8 and can greatly effect the convergence of the Newton method.

\subsection{Turning Point Tracking} \label{sec:TPT}
The turning point tracking only requires initial guesses for the solution vector and bifurcation parameter. Section \ref{sec:locate} describes the process for identifying the turning point with arclength continuation. Starting with the extreme parameter value and associated solution vector from the arclength continuation run will usually be adequate starting points for the turning point algorithm. No auxiliary variables are required by the code.

\subsection{Pitchfork Tracking} \label{sec:PT}
For pitchfork tracking, the code requires an initial guess for the solution vector, the antisymmetric vector ($\psi$), and the bifurcation parameter. We use the $\psi$ vector as an initial guess for $\bfy$. This vector is calculated by first detecting the Pitchfork bifurcation with an eigensolver. By choosing as initial guesses the solution and parameter value at the point where the eigenvalue was closest to zero, and choosing the associated eigenvector as $\psi$, the pitchfork tracking algorithm will usually converge. For problems that have multiple pitchfork bifurcations in the same region of parameter space, which is often the case when the system can go unstable to different modes, the pitchfork algorithm can be started multiple times with different $\psi$ vectors to track each pitchfork separately.

\subsection{Hopf Tracking} \label{sec:HT}
The Hopf tracking algorithm requires an initial guess for the solution vector, the bifurcation parameter, the value of the imaginary part of the eigenvalue, $\omega$, and the real and complex parts of the corresponding eigenvector, $\bfy$ and $\bfz$. The procedure is the same as that for the pitchfork bifurcation, except that a complex pair of eigenvalues crosses the imaginary axis. By finding the point on a continuation run where the real part is closest to zero, all the information is available to create initial guesses to start the Hopf tracking algorithm. Since the eigenvectors given by the eigensolver will not generally satisfy the normalization conditions (Eq.s \ref{eq:hpnorm} and \ref{eq:hpnorm2}), the initial guesses for the eigenvectors are initially rotated and scaled so that these conditions are met. We have found that excess Newton iterations are taken after the solution vector and bifurcation parameter are converged if we keep the same tolerances for both the solution vector and the eigenvectors. Therefore, while we require the scaled update norm to be less than one for the solution, imaginary eigenvalue, and bifurcation parameter, we only require the eigenvectors to have a scaled update norm of less than 100.

\subsection{Phase Transition Tracking} \label{sec:PTT}
Phase transition tracking is initiated after first identifying one instance of a phase transition. When arclength continuation locates a region of hysteresis, the solution branch can be plotted using the free energy as a measure and a phase transition can be visually picked off. By using other solution on each branch nearest the visible crossing as initial guesses for $\bfx_1$ and $\bfx_2$, and the parameter value where they appear to cross, we have found the phase transition algorithm to be very robust. The system of governing equations does become singular at a critical point, so there is a limit to how close the algorithm can come to this point.
%################################################################### 
