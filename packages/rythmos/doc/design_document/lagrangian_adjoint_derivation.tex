\section{Derivation of adjoint sensitivities using the reduced Lagrangian}

In this section we consider the derivation of adjoint approaches for computing
sensitivities of the reduced response function $\hat{d}(p,v(t_0,t_f))$ with
respect to $p$ and $v(t)$ for $t\in[t_0,t_f]$ using a Lagrangian approach.  The
primary devise that that we use here is the Lagrangian which is a special
function which is the weighted composition of the response function and the
DAE constraints.  This derivation follows in a similar manner from the one in
[???].  The advantage of the Lagrangian approach is that the algebraic
manipulations are more straightforward but the disadvantage is that it lacks
the more comfortable approaches used in the above direct sensitivity weak-form
derivations for the adjoint.

The Lagrangian is
%
\begin{eqnarray}
L(y,p,v,\Lambda)
& = & \int_{t_0}^{t_f} g(y(t),p,v(t),t) dt + h(y(t_f),p) \nonumber \\
& & + \int_{t_0}^{t_f} \Lambda(t)^H c(\dot{y}(t),y(t),p,v(t),t) dt,
\label{rythmos:apdx:eqn:sens:L}
\end{eqnarray}
\begin{tabbing}
\hspace{4ex}where\hspace{1ex}\= \\
\>	$\Lambda(t) \in \mathcal{C}|\mathcal{G}$ are the Lagrange multipliers defined on $t\in[t_0,t_f]$,
\end{tabbing}
%
and $\Lambda(t)^H\in\mathcal{G}|\mathcal{C}$ is the adjoint operator
of $\Lambda(t)$.

A few remarks about the Lagrangian function in (\ref{rythmos:apdx:eqn:sens:L})
are in order before we get into the detailed derivations.  While the
Lagrangian function can have a significant theoretical interpretation in a
variety of contexts, it is usually primarily just a convenient means to derive
sensitivities of constrained problems.  For instance, it is well known that
the variations of the Lagrangian set to zero represent the first-order
optimality conditions for the constrained optimization problem of minimizing
(or maximizing) (\ref{rythmos:apdx:eqn:sens:d}) (with $n_g=1$ of course)
subject to
(\ref{rythmos:apdx:eqn:sens:c})--(\ref{rythmos:apdx:eqn:sens:c:ic}).

In the following derivations, we will plug the implicit state function
$y(p,v,t)$ into (\ref{rythmos:apdx:eqn:sens:L}), forming a reduced
Lagrangian $\hat{L}(p,v)$ and then take variations with
respect to $p$ and $v$.  From these variations, followed by
some other manipulations, the adjoint equation will fall out as well
as the other derivative computations.

Note: I am a little very uncomfortable with this approach of plugging
$y(p,v,t)$ into the Lagrangian and then taking variations.  However,
this approach works correctly for steady-state problems and therefore should
also work for transient problems.  Therefore, for now, I am just going to
accept that I don't fully appreciate the foundation for this general approach
but I will just accept that it is bullet proof and gives the right result
every time.

\subsection{Sensitivities for steady-state parameters and the adjoint equation}

In this section we will derive the expressions for $\jac{\hat{d}}{p}$ and in
the process will derive the adjoint equation and boundary conditions that
solves for $\Lambda$.  We start by taking the variation of
(\ref{rythmos:apdx:eqn:sens:L}) (through the implicit function $y(p,v,t)$)
with respect to $p$ and we claim that this will be equal to
$\jac{\hat{d}}{p}$.  This variation is
%
\begin{eqnarray}
\Jac{\hat{d}}{p}
& = & \Jac{\hat{L}}{p}
\nonumber \\
& = & \int_{t_0}^{t_f} \left( \Jac{g}{y} \Jac{y}{p} +  \Jac{g}{p} \right) dt
+ \left. \left(  \Jac{h}{y} \Jac{y}{p} + \Jac{h}{p} \right) \right|_{t=t_f}
\nonumber \\
& & + \int_{t_0}^{t_f} \Lambda^H \left( \Jac{c}{\dot{y}} \frac{d}{dt}\left( \Jac{y}{p} \right)
   + \Jac{c}{y} \Jac{y}{p} +  \Jac{c}{p} \right) dt.
\label{rythmos:apdx:eqn:sens:d_L_hat_d_p}
\end{eqnarray}
%
Substituting the integration by parts
%
\begin{equation}
\int_{t_0}^{t_f} \left[ \left( \Lambda^H \Jac{c}{\dot{y}} \right) \frac{d}{dt}\left( \Jac{y}{p} \right) \right] dt
= \left. \left( \Lambda^H \Jac{c}{\dot{y}} \Jac{y}{p} \right) \right|_{t_0}^{t_f}
- \int_{t_0}^{t_f} \left[ \frac{d}{dt}\left( \Lambda^H \Jac{c}{\dot{y}} \right) \Jac{y}{p} \right] dt
\end{equation}
%
into (\ref{rythmos:apdx:eqn:sens:d_L_hat_d_p}) and rearranging yields
%
\begin{eqnarray}
\Jac{\hat{d}}{p}
& = & \int_{t_0}^{t_f} \left(
    \Jac{g}{p}
    + \Lambda^H \Jac{c}{p}
  \right) dt
  + \left. \left( \Jac{h}{p} \right) \right|_{t=t_f}
  - \left. \left( \Lambda^H \Jac{c}{\dot{y}} \Jac{y}{p} \right) \right|_{t=t_0}
\nonumber \\
& & + \int_{t_0}^{t_f} \left(
    - \frac{d}{dt}\left( \Lambda^H \Jac{c}{\dot{y}} \right)
    +  \Lambda^H \Jac{c}{y} +  \Jac{g}{y}
  \right) \Jac{y}{p} dt
\nonumber \\
& & + \left. \left[ \left(
    \Lambda^H \Jac{c}{\dot{y}}
    +  \Jac{h}{y}
  \right)  \Jac{y}{p} \right] \right|_{t=t_f}
\label{rythmos:apdx:eqn:sens:d_L_hat_d_p_2}.
\end{eqnarray}
%
We will now identify the integrand in the second integral in
(\ref{rythmos:apdx:eqn:sens:d_L_hat_d_p_2}) as the adjoint equation and we
will force it to be satisfied as
%
\begin{equation}
- \frac{d}{dt}\left( \Lambda^H \Jac{c}{\dot{y}} \right)
+  \Lambda^H \Jac{c}{y} +  \Jac{g}{y} = 0, \; t \in \left[ t_0, t_f \right].
\label{rythmos:apdx:eqn:sens:adj-trans-2}
\end{equation}
%
We must also specify boundary conditions for the adjoint equation in
(\ref{rythmos:apdx:eqn:sens:adj-trans-2}).  As stated in [???] for index-0 and
index-1 DAEs, we can simply specify the endpoint condition
%
\begin{equation}
\left. \left(
  \Lambda^H \Jac{c}{\dot{y}}
  +  \Jac{h}{y}
\right) \right|_{t=t_f}
 = 0,
\label{rythmos:apdx:eqn:sens:adj-trans:fc-2}
\end{equation}
%
which drops out the last term in (\ref{rythmos:apdx:eqn:sens:d_L_hat_d_p_2})
and, together with (\ref{rythmos:apdx:eqn:sens:adj-trans-2}) and substituting
$\jac{y_0}{p}$ for $\jac{y}{p}$ in the second expression at $t=t_0$, leaves
%
\begin{equation}
\Jac{\hat{d}}{p} =
\int_{t_0}^{t_f} \left(
    \Jac{g}{p}
    + \Lambda^H \Jac{c}{p}
  \right) dt
  + \left. \left( \Jac{h}{p} \right) \right|_{t=t_f}
  - \left. \left( \Lambda^H \Jac{c}{\dot{y}} \Jac{y_0}{p} \right) \right|_{t=t_0}
\label{rythmos:apdx:eqn:sens:d_d_hat_d_p_final-2}
\end{equation}
%
which is the same as (\ref{rythmos:apdx:eqn:sens:d_d_hat_d_p_final}) derived
in another way.

\subsection{Adjoint sensitivities for transient parameters}

Here we consider the derivation of the sensitivity
%
\begin{equation}
\Jac{\hat{d}}{v(t)} \in \mathcal{G}|\mathcal{V}, \; \mbox{for} \; t\in[t_0,t_f].
\end{equation}
%
Our derivation for these sensitivities for the transient parameters $v(t)$
will be a little different than for the steady-state parameters $p$.  The
primary reason for using a different derivation is that the
infinite-dimensional transient parameters $v$ are fundamentally different than
finite-dimensional steady-state parameters $p$.  Because of the
infinite-dimensional nature of $v(t_0,t_f)$, we use a different tool of
functional analysis.  The tool we use is the application of the
infinite-dimensional operator $\jac{\hat{L}}{v}$ to the infinite-dimensional
perturbation $\delta v$ where we will then identify $\jac{\hat{d}}{v(t)} =
{}\jac{\hat{L}}{v(t)}$ in the integrand of
%
\begin{equation}
\left( \Jac{\hat{d}}{v} \right) \delta v
= \left( \Jac{\hat{L}}{v} \right) \delta v
= \int_{t_0}^{t_f} \Jac{\hat{L}}{v(t)} \delta v(t) dt
= \int_{t_0}^{t_f} \Jac{\hat{d}}{v(t)} \delta v(t) dt.
\label{rythmos:apdx:eqn:sens:d_d_hat_d_v_delta_v}
\end{equation}
%
The integral in (\ref{rythmos:apdx:eqn:sens:d_d_hat_d_v_delta_v}) is the
definition of the inner product of two functions in an
infinite-dimensional space.
   
We begin our derivation of $\jac{\hat{d}}{v(t)}$ by linearizing $\hat{L}(p,v)$
with respect to $\delta v$ as
%
\begin{eqnarray}
\left( \Jac{\hat{d}}{v} \right) \delta v
& = & \left( \Jac{\hat{L}}{v} \right) \delta v
\nonumber \\
& = & \int_{t_0}^{t_f} \left( \Jac{g}{y} \Jac{y}{v} \delta v +  \Jac{g}{v} \delta v \right) dt
+ \left. \left(  \Jac{h}{y} \Jac{y}{v} \delta v \right) \right|_{t=t_f}
\nonumber \\
& & + \int_{t_0}^{t_f} \Lambda^H \left( \Jac{c}{\dot{y}} \frac{d}{dt}\left( \Jac{y}{v} \right) \delta v
   + \Jac{c}{y} \Jac{y}{v} \delta v +  \Jac{c}{v} \delta v \right) dt.
\label{rythmos:apdx:eqn:sens:d_L_hat_d_v_delta_v}
\end{eqnarray}
%
Substituting the integration by parts
%
\begin{equation}
\int_{t_0}^{t_f} \left[ \left( \Lambda^H \Jac{c}{\dot{y}} \right) \frac{d}{dt}\left( \Jac{y}{v} \right) \delta v \right] dt
= \left. \left( \Lambda^H \Jac{c}{\dot{y}} \Jac{y}{v} \delta v \right) \right|_{t_0}^{t_f}
- \int_{t_0}^{t_f} \left[ \frac{d}{dt}\left( \Lambda^H \Jac{c}{\dot{y}} \right) \Jac{y}{v} \delta v \right] dt
\end{equation}
%
into (\ref{rythmos:apdx:eqn:sens:d_L_hat_d_v_delta_v}) and rearranging
yields
%
\begin{eqnarray}
\left( \Jac{\hat{d}}{v} \right) \delta v
& = & \int_{t_0}^{t_f} \left[ \left(
    \Jac{g}{v}
    + \Lambda^H \Jac{c}{v}
  \right) \delta v \right] dt
  - \left. \left( \Lambda^H \Jac{c}{\dot{y}} \Jac{y}{v} \delta v \right) \right|_{t=t_0}
\nonumber \\
& & + \int_{t_0}^{t_f} \left[ \left(
    - \frac{d}{dt}\left( \Lambda^H \Jac{c}{\dot{y}} \right)
    +  \Lambda^H \Jac{c}{y} +  \Jac{g}{y}
  \right) \Jac{y}{v} \delta v \right] dt
\nonumber \\
& & + \left. \left[ \left(
    \Lambda^H \Jac{c}{\dot{y}}
    +  \Jac{h}{y}
  \right)  \Jac{y}{v} \delta v \right] \right|_{t=t_f}
\label{rythmos:apdx:eqn:sens:d_L_hat_d_v_delta_v_2}.
\end{eqnarray}
%
By applying our choice for the adjoint
(\ref{rythmos:apdx:eqn:sens:adj-trans-2}) and its final condition
(\ref{rythmos:apdx:eqn:sens:adj-trans:fc-2}) and noting that $y(t_0) = y_0(p)$
is not a function so that of $v$ and therefore $\jac{y}{v(t)} = 0$ at $t=t_0$,
then what is left of (\ref{rythmos:apdx:eqn:sens:d_L_hat_d_v_delta_v_2})
becomes
%
\begin{eqnarray}
\left( \Jac{\hat{d}}{v} \right) \delta v
& = & \int_{t_0}^{t_f} \left[ \left(
    \Jac{g}{v}
    + \Lambda^H \Jac{c}{v}
  \right) \delta v \right] dt
\label{rythmos:apdx:eqn:sens:d_L_hat_d_v_delta_v_3}.
\end{eqnarray}

By comparing (\ref{rythmos:apdx:eqn:sens:d_L_hat_d_v_delta_v_3}) to
(\ref{rythmos:apdx:eqn:sens:d_d_hat_d_v_delta_v}) it is clear that
%
\begin{equation}
\Jac{\hat{d}}{v(t)}^H
= \Jac{g}{v(t)}^H + \Jac{c}{v(t)}^H \Lambda(t)
\label{rythmos:apdx:eqn:sens:d_d_hat_d_v_t}
\end{equation}
%
is the reduced gradient that we are seeking.
